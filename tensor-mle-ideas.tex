\documentclass{article}
\usepackage[utf8]{inputenc}
\usepackage{amsmath}
\usepackage{amsthm,mathtools,braket}
\usepackage{amssymb,bm, hyperref}
\usepackage{xcolor, float}
\usepackage[capitalize]{cleveref}

\newtheorem{theorem}{Theorem}
\newtheorem{corollary}[theorem]{Corollary}
\newtheorem{prop}[theorem]{Proposition}
\newtheorem{lemma}[theorem]{Lemma}
\newtheorem{corol}[theorem]{Corollary}
\newtheorem{fact}[theorem]{Fact}
\newtheorem{claim}[theorem]{Claim}
\newtheorem{remark}{Remark}
\newtheorem{definition}{Definition}
\newtheorem{example}{Example}
\floatstyle{boxed}\newfloat{Algorithm}{ht}{alg}

\DeclareMathOperator{\Lie}{Lie}
\DeclareMathOperator{\Lin}{L}
\DeclareMathOperator{\ope}{op}
\DeclareMathOperator{\poly}{poly}

\DeclarePairedDelimiter{\abs}{\lvert}{\rvert}
\DeclarePairedDelimiter{\norm}{\lVert}{\rVert}
\DeclarePairedDelimiter{\ip}{\langle}{\rangle}


\newcommand{\junk}[1]{}


\newcommand{\R}{{\mathbb{R}}}
\newcommand{\C}{{\mathbb{C}}}
\newcommand{\ot}{\otimes}
\newcommand{\mat}{\operatorname{Mat}}
\newcommand{\op}{\operatorname{op}}
\newcommand{\ch}{\operatorname{ch}}
\newcommand{\cut}{\operatorname{cut}}
\newcommand{\vol}{\operatorname{vol}}
\newcommand{\Q}{{\mathbb{Q}}}
\newcommand{\N}{{\mathbb{N}}}
\renewcommand{\vec}{\bm}
\newcommand{\Z}{{\mathbb{Z}}}
\renewcommand{\S}{\mathbb{S}}

\newcommand{\E}{\mathbb{E}}
\renewcommand{\Pr}{\mathbb{P}}

\newcommand\eps{\varepsilon}

\newcommand\cN{\mathcal{N}}
\newcommand\FF{\mathcal{F}}
\newcommand\HH{\mathcal{H}}
\newcommand\GG{\mathcal{G}}
\newcommand\SL{\operatorname{SL}}
\newcommand\PD{\operatorname{PD}}
\newcommand\Herm{\operatorname{Herm}}
\newcommand\Sym{\operatorname{Sym}}
\newcommand\GL{\operatorname{GL}}
\newcommand\TT{\mathcal{T}}
\newcommand\PP{\mathcal{P}}
\newcommand\tr{\operatorname{Tr}}
\newcommand\rk{\operatorname{rk}}
\newcommand\RR{\mathcal{R}}
\newcommand\CC{\mathcal{C}}
\newcommand\BB{\mathcal{B}}
\newcommand\II{\mathcal{I}}
\renewcommand\AA{\mathcal{A}}
\newcommand{\MM}{\mathcal{M}}
\newcommand\AP{\mathcal{AP}}

\newcommand{\eqdef}{:=}
\newcommand{\maps}{\colon}
\newcommand{\ale}{\lesssim}
\newcommand{\age}{\gtrsim}
\newcommand{\CF}[1]{{\color{purple}[CF: #1]}}
\newcommand{\AR}[1]{{\color{orange}[AR: #1]}}
\newcommand{\RMO}[1]{{\color{red}[RMO: #1]}}
\newcommand{\MW}[1]{{\color{red}[MW: #1]}}
\newcommand{\TODO}[1]{{\color{blue}[TODO: #1]}}

\crefname{Algorithm}{Algorithm}{Algorithms}


\title{Ideas towards operator norm bounds}
\author{Cole Franks, Rafael Oliveira, Akshay Ramachandran, Michael Walter}
\date{February 2020}

\begin{document}

\maketitle
\tableofcontents


\section{Tensor Scaling}
We will maintain similar notation. We have $n$ samples of $X \sim \mathcal{N}(0,\frac{1}{n} \otimes_{a} \frac{1}{d_{a}} I_{a})$ with $D := \prod_{a} d_{a}$. We don't have a KLR style analysis at the moment, but strong convexity is enough by the FM analysis, and this can be proven by just controlling each bipartite piece. So the operator scaling analysis does give us very good bounds for $\|\mu\|_{op}$ and expansion with $nD \gg \max_{a} d_{a}^{2} \log^{c}(D)$. These bounds are not enough though, so in this section we will follow the FM analysis to give the requirements, then show the required strong convexity, and show how to maintain this under perturbation.

\subsection{FM Analysis}
Recall that $\forall a: Q^{a} \approx \frac{1}{d_{a}} I_{a}$, so if we can show $\forall a \neq b: \langle Q^{ab}, Z \otimes Y \rangle \lesssim \frac{\|Z\|_{F} \|Y\|_{F}}{\sqrt{d_{a} d_{b}}}$ then we have strong convexity with $\langle Z, \nabla^{2} Z \rangle \gtrsim \sum_{a} \frac{\|Z_{a}\|_{F}^{2}}{d_{a}}$, i.e. the Hessian is strongly diagonally dominant. We will derive our requirements on strong convexity, perturbation bounds, and initial error. Assume we have the following strong convexity
\[ \forall Z: \sum_{a} \frac{\|Z_{a}\|_{F}^{2}}{d_{a}} \leq \kappa^{2}: \langle Y, \nabla^{2}_{e^{Z}}, Y \rangle \geq \lambda \sum_{a} \frac{\|Y_{a}\|_{F}^{2}}{d_{a}}  \]

Choose $X$ to be the geodesic towards the optimum and $g(t) := f(e^{tZ})$ with the opt at $t=1$:
\[ g'(1) = \int_{0}^{1} g''(t) + g'(0) \geq \lambda \sum_{a} \frac{\|Z\|_{F}^{2}}{d_{a}} - |\langle \nabla_{a}, Z_{a} \rangle| \]
\[ \geq \sum_{a} \frac{\|Z_{a}\|_{F}}{\sqrt{d_{a}}} \left( \lambda \frac{\|Z_{a}\|_{F}}{\sqrt{d_{a}}} - \sqrt{d_{a}} \|\nabla_{a}\|_{F}  \right)  \]
\[ \geq \sqrt{\sum_{a} \frac{\|Z_{a}\|_{F}^{2}}{d_{a}}} \left( \lambda \sqrt{\sum_{a} \frac{\|Z_{a}\|_{F}^{2}}{d_{a}}} - \sqrt{\sum_{a} d_{a} \|\nabla_{a}\|_{F}^{2}} \right)  \]
This is $> 0$ if $\forall a: \lambda > d_{a} \frac{\|\nabla_{a}\|_{F}}{\|Z_{a}\|_{F}}$ or $\lambda^{2} > \frac{ \sum_{a} d_{a} \|\nabla_{a}\|_{F}^{2} }{\sum_{a} d_{a}^{-1} \|Z_{a}\|_{F}^{2}} $.

Since standard perturbation bounds ($e^{Z} \approx I + Z$) only work for small $Z$, we will require
\[ \forall a: \|\nabla_{a}\|_{F} \ll \frac{1}{d_{a}} \]
\[ \forall a: \|Z_{a}\|_{F}^{2} \ll 1 \implies \langle Y, \nabla_{e^{Z}}^{2}, Y \rangle \geq \Omega(1) \sum_{a} \frac{\|Y_{a}\|_{F}^{2}}{d_{a}}   \]


\subsection{Moment Map}
For $g \sim \frac{1}{nD} \otimes_{a} I_{a}$, we want to bound $\|Q^{a} - sI_{a}\|_{F}$ using a net:
\[ \E \langle \sum_{t} g_{t} g_{t}^{*}, X_{a} \rangle = \sum_{i} x_{i} \chi(\frac{1}{d_{a}}, \frac{TD}{d_{a}}) = \langle \frac{1}{d_{a}} I_{a}, X \rangle = 0 \]
\[ \log \E \exp \theta \langle \sum_{t} g_{t} g_{t}^{*}, X_{a} \rangle = \log \E \exp \theta \sum_{i} x_{i} \chi(\frac{1}{d_{a}}, \frac{TD}{d_{a}})  \]
\[ = \sum_{i} \frac{-TD}{2 d_{a}} \log \left( 1 - 2 \theta \frac{x_{i}}{TD} \right)   \]
\[ \lesssim \theta^{2} \frac{\|X\|_{F}^{2}}{2 d_{a} TD} \hspace{10mm} \forall \theta < \left( \frac{\|X\|_{op}}{TD} \right)^{-1}  \]
\[ \implies \Pr[ \langle \sum_{t} g_{t} g_{t}^{*}, X_{a} \rangle \geq \epsilon \|X\|_{F} ] \leq
\begin{cases}
\exp( - \Omega(\epsilon^{2} TD d_{a}) ) & \epsilon  < \frac{\|X\|_{F}}{d_{a} \|X\|_{op}}
\\ \exp ( - \Omega(\epsilon TD) \frac{\|X\|_{F}}{\|X\|_{op}} ) & \epsilon \geq \frac{\|X\|_{F}}{d_{a} \|X\|_{op}}
\end{cases}
\]
We will need the following settings of $\epsilon$ in future:
\[ \epsilon \approx \frac{1}{\sqrt{d_{a}}} \implies \Pr [ d_{a} \|Q^{a} - sI_{a}\|_{F}^{2} \gtrsim c ] \leq \exp(d_{a}^{2} \log d_{a} - c \frac{TD}{\sqrt{d_{a}}})  \]
For which we need $TD \gtrsim \max_{a} d_{a}^{5/2} \log d_{a}$.
\[ \epsilon \approx \frac{1}{d_{a}} \implies \Pr [ d_{a} \|Q^{a} - sI_{a}\|_{F}^{2} \gtrsim \frac{c}{d_{a}} ] \leq \exp(d_{a}^{2} \log d_{a} - c^{2} \frac{TD}{d_{a}})   \]
For which we need $TD \gtrsim \max_{a} d_{a}^{3} \log d_{a}$.

\begin{remark}
Note we lose out on the subgaussian part of the bound only when $\frac{d_{a} \|X\|_{op}^{2}}{\|X\|_{F}^{2}}$ is large. It is quite possible that for our setting, we can bound e.g. the condition number or stable rank of is small w.h.p. In particular if we can show the only relevant part of the net has $s \|X\|_{F}^{2} \geq d \|X\|_{op}$ then we only incur a $\sqrt{s}$ loss in required samples.
\end{remark}

Actually there seems to be a simpler way to prove these statements using the $\|\cdot\|_{op}$ bounds derived earlier.
\[ \|Q^{a} - sI_{a}\|_{op} \leq \frac{\sqrt{f(d)}}{d} \implies d_{a} \|Q^{a} - sI_{a}\|_{F}^{2} \leq f(d)  \]
So this means in the first case, we need $TD \gtrsim \max_{a} d_{a}^{2} \log D$ and the second case we need $TD \gtrsim \max_{a} d_{a}^{3} \log D$. But by this analysis we only get $1/poly$ failure probability.


\subsection{Net proof of Expansion}
Recall again we have $Q := \sum_{i \in [n]} X_{i} X_{i}^{*}$ with i.i.d $X \sim \mathcal{N}(0,\frac{1}{n} \otimes_{a} \frac{1}{d_{a}} I_{a})$. For symmetric test matrices $Z,Y$ with eigenvalues $\{z_{i}\},\{y_{j}\}$ respectively:
\[ \E \langle Q, Z_{a} \otimes Y_{b} \rangle = \sum_{ij} z_{i} y_{j} \chi(\frac{1}{d_{a} d_{b}}, \frac{nD}{d_{a} d_{b}}) = \langle \frac{1}{d_{a}} I_{a}, Z \rangle \langle \frac{1}{d_{b}} I_{b}, Y \rangle = 0 \]

\[ \log \E \exp \theta \langle Q, Z_{a} \otimes Y_{b} \rangle = \log \E \exp \theta \sum_{ij} z_{i} y_{j} \chi(\frac{1}{d_{a} d_{b}}, \frac{nD}{d_{a} d_{b}})  \]
\[ = \sum_{ij} \frac{-nD}{2 d_{a} d_{b}} \log \left( 1 - 2 \theta \frac{z_{i} y_{j}}{nD} \right)   \]
\[ \lesssim \theta^{2} \frac{\|Z\|_{F}^{2} \|Y\|_{F}^{2}}{2 d_{a} d_{b} nD} \hspace{10mm} \forall \theta < \left( \frac{\|Z\|_{op} \|Y\|_{op}}{TD} \right)^{-1}  \]
\[ \Pr[ \langle Q, X \otimes Y \rangle \geq \lambda \frac{\|Z\|_{F} \|Y\|_{F}}{\sqrt{d_{a} d_{b}}} ] \leq
\begin{cases}
\exp( - \lambda^{2} nD ) \hspace{7mm} \lambda  < \frac{\|Z\|_{F} \|Y\|_{F}}{\sqrt{d_{a} d_{b}} \|Z\|_{op} \|Y\|_{op}}
\\ \exp ( - \lambda nD \frac{\|Z\|_{F} \|Y\|_{F}}{ \sqrt{d_{a} d_{b}} \|Z\|_{op} \|Y\|_{op}} )
\end{cases}
\]
So for $\lambda \approx \frac{1}{k}$ we need $nD \gtrsim \max_{a} d_{a}^{3} \log D > \max_{a,b} \sqrt{d_{a} d_{b}} (d_{a}^{2} + d_{b}^{2}) \log D$

\begin{remark}
Note again we lose out when $\frac{d \|X\|_{op}^{2}}{\|X\|_{F}^{2}}$ is large. So we would like to show that w.h.p. the singular vectors of our bipartite operator have e.g. small condition number or large stable rank. Again if we can show the relevant part of the net has $s \|X\|_{F}^{2} \geq d \|X\|_{op}^{2}$ then we incur an $s$ factor loss in samples. This is reminiscent of the fact that eigenvectors for random matrices have many delocalization properties, so will look into that.
\end{remark}

\subsection{Delocalization}
I have a few claims which, if true, would give another proof (along with Pisier's) of constant expansion at the start with $1/poly$ failure probability. Unfortunately to make this robust we would either need a robust form of delocalization, which I think is false in general, or exponential failure probability, which again may be false in general.

Recall $M^{ab}$ is the operator defined by the off-diagonal block of the Hessian. For every choice of bases (torus) $U,V$ we have a matrix $M^{ab}_{UV}$ where each entry is $\chi(\frac{1}{d_{a}d_{b}}, \frac{TD}{d_{a}d_{b}})$. In particular each entry is of exponential type.

\begin{claim}
Let $U,V$ be the (random) basis for the optimizers of $\|M\|_{op}$. Then $(U,V)$ and $M_{UV}$ are distributionally independent.
\end{claim}

\begin{theorem} [Informal]
For $M \in \R^{n \times n}$ populated by iid variables $X$ such that $\E X = 0, \E X^{2} = 1$ and $X$ is of exponential type, we have delocalization of eigenvectors with $1/poly$ failure probability:
\[ v \in S^{n-1}, \|v\|_{\infty} \gtrsim \frac{\log^{c} n}{\sqrt{n}} \implies \forall \lambda \in \C: \|Mv - \lambda v\|_{2} \gtrsim \frac{1}{\sqrt{n}}   \]
\end{theorem}

%https://arxiv.org/pdf/1306.2887.pdf
%https://arxiv.org/pdf/1306.3099.pdf
%https://arxiv.org/pdf/1601.03678.pdf

\begin{claim}
The above informal claim can be extended to singular vectors.
\end{claim}

\begin{corollary}
Conditioned on the above delocalization event, our net only has to cover $\{X \mid d \|X\|_{op}^{2} \lesssim \log^{c}(d) \|X\|_{F}^{2}$, and so we only require $TD \gtrsim d_{a}^{2} \log^{c} D$.
\end{corollary}


\subsection{Robust proof of Expansion}
In this section we will show expansion under perturbations of the form $\otimes_{a} e^{\delta_{a}}$. Note if $\|\delta\|_{op} \ll 1$ then we can approximate $e^{\delta} - I \approx O(\delta)$.
\[ \mu := \E \langle Q, e^{\delta_{a}} Z e^{\delta_{a}} \otimes e^{\delta_{b}} Y e^{\delta_{b}} \otimes_{c} e^{2\delta_{c}} \rangle = \langle \frac{e^{2\delta_{a}}}{d_{a}} , Z \rangle \langle \frac{e^{2\delta_{b}}}{d_{b}}, Y \rangle \prod_{c \neq a,b} Tr[\frac{e^{2\delta_{c}}}{d_{c}} ]  \]
\[ \log \E \exp \theta \langle Q, e^{\delta_{a}} Z e^{\delta_{a}} \otimes e^{\delta_{b}} Y e^{\delta_{b}} \otimes_{c \neq a,b} e^{2\delta_{c}} \rangle = \log \E \exp \theta \sum_{\vec{i}} x^{\vec{i}} \chi(\frac{1}{D},T)   \]
\[ \leq \theta \mu + \theta^{2} \frac{\|e^{\delta_{a}} Z e^{\delta_{a}}\|_{F}^{2} \|e^{\delta_{b}} Y e^{\delta_{b}}\|_{F}^{2} \prod_{c} \|e^{2\delta_{c}}\|_{F}^{2}}{2 TD^{2}} \]
\[ \forall \theta < \left( \frac{\|e^{\delta_{a}} Z e^{\delta_{a}}\|_{op} \|e^{\delta_{b}} Y e^{\delta_{b}}\|_{op} \prod_{c} \|e^{2\delta_{c}}\|_{op}}{TD} \right)^{-1}    \]

\begin{lemma}
For $\|\delta\|_{op} \ll 1$
\[ \frac{1}{d} Tr[e^{2\delta}] \leq 1 + O(\|\delta\|_{op})  \]
\[ \langle \frac{1}{d} I, e^{\delta} Z e^{\delta} - Z \rangle \leq \frac{O(\|\delta\|_{op})}{\sqrt{d}} \|Z\|_{F}   \]
\[ \|e^{\delta} Z e^{\delta}\|_{F}^{2} \leq (1 + O(\|\delta\|_{op})) \|Z\|_{F}^{2}   \]
\end{lemma}
\begin{proof}
Let $\delta := e^{\delta'} - I$ and note $\|\delta\|_{op} \approx \|\delta'\|_{op}$ for small $\delta'$. So we bound
\[ Tr[e^{2\delta'} - I] = \langle I, e^{2\delta'} - I \rangle \leq d \|2\delta + \delta^{2}\|_{op} \leq d O(\|\delta\|_{op})   \]
For the second line we use Cauchy Schwarz $\|Z\|_{1} \leq \sqrt{d} \|Z\|_{F}$
\[ \langle I, (I + \delta) Z (I + \delta') - Z \rangle = \langle 2 \delta + \delta^{2}, Z \rangle \leq O(\|\delta\|_{op}) \|X\|_{1} \leq O(\|\delta\|_{op}) \sqrt{d} \|Z\|_{F}  \]
The second line is similar but simpler using $\|AB\|_{F} \leq \|A\|_{op} \|B\|_{F}$
\end{proof}

So using the above lemma, we have bounds for $\langle Z, I_{a} \rangle = \langle Y, I_{b} \rangle = 0$:
\[ \mu \leq \frac{\|Z\|_{F} \|Y\|_{F}}{\sqrt{d_{a} d_{b}}} O(\|\delta_{a}\|_{op} \|\delta_{b}\|_{op}) \prod_{c} (1 + O(\|\delta_{c}\|_{op})  \]
\[ \log \E \exp \theta (... - \mu) \leq \theta^{2} \frac{\|Z\|_{F}^{2} \|Y\|_{F}^{2}}{2TD} \prod_{c} (1 + O(\|\delta_{c}\|_{op}))  \]
\[ \forall \theta < \left( \frac{\|Z\|_{op}\|Y\|_{op}}{TD} \right)^{-1} \prod_{c} (1 - O(\|\delta_{c}\|_{op}))  \]
So basically, as long as $\|\delta\|_{op} \ll 1$ everything is of the same order as in the unperturbed case, and therefore if we run a net on all parts simultaneously (of size $\exp(\sum_{a} d_{a}^{2})$) we get roughly the same probabilistic bounds as the start.

This creates a bottleneck though as in general the inequality $\|X\|_{op} \leq \|X\|_{F}$ could be tight, so in order to guarantee a small enough perturbation we can only move in the ball $\|X\|_{F} \ll 1$. This is why we wrote out the conditions required for $\|\nabla_{a}\|_{F} \ll \frac{1}{d_{a}}$, as therefore we can assume $\|Z_{a}\|_{F} \ll 1$ and require $\lambda \approx 1$.

\begin{remark}
Here is where we definitely would like a KLR style analysis to exploit the fact that we actually have robustness in $\|\cdot\|_{op}$.
\end{remark}


\subsection{Deterministic Robust Proof of Expansion}

We follow Michael's write-up in our setting. We can assume we have sufficiently strong expansion initially and we would like to show that any scaling ($e^{Y} \cdot g$ with $\forall a: \|Y_{a}\|_{F} \ll 1$) only changes the Hessian by a small amount.
\begin{fact}
Recall the formula for Hessian at $v$:
\[ \langle X, \nabla_{v}^{2}, X \rangle = \langle \frac{v v^{*}}{\|v\|_{2}^{2}}, \left( \sum_{a} X_{a} \otimes I_{\overline{a}}  \right)^{2} \rangle - \left( \sum_{a} \langle \frac{v v^{*}}{\|v\|_{2}^{2}}, X_{a} \otimes I_{\overline{a}} \rangle  \right)^{2}  \]
\end{fact}

Initially, we have expansion of the form
\[ \langle X, \nabla^{2}, X \rangle \geq (1-\epsilon) \sum_{a} \frac{\|X_{a}\|_{F}^{2}}{d_{a}} - \lambda \sum_{a \neq b} \frac{\|X_{a}\|_{F} \|X_{b}\|_{F}}{\sqrt{d_{a} d_{b}}} \geq (1-\epsilon') \sum_{a} \frac{\|X_{a}\|_{F}^{2}}{d_{a}}   \]
So for how large of a perturbation can we show that every quadratic form changes $<< \sum_{a} \frac{\|X_{a}\|_{F}^{2}}{d_{a}} =: \|X\|^{2}$?

\begin{lemma}
\[ \|\sum_{a} X_{a} \otimes I_{\overline{a}}\|_{op} \leq \sum_{a} \|X_{a}\|_{op} \leq \sum_{a} \|X_{a}\|_{F} \leq \sqrt{ \left( \sum_{a} d_{a} \right) \left( \sum_{a} \frac{\|X_{a}\|_{F}^{2}}{d_{a}}  \right)} \]
All inequalities can be tight, say at $X_{a} = \sqrt{d_{a}} E_{11}$. In Michael's notation, for standard tensor rep
\[ \sup_{X} \frac{\|\Pi(X)\|_{op}}{\sum_{a} \|X_{a}\|_{F}} = k  \text{  ;  } \hspace{7mm}
\sup_{X} \frac{\|\Pi(X)\|_{op}^{2}}{\|X\|^{2}} = \sum_{a} d_{a}  \]
\end{lemma}

\begin{theorem}
For $\|v\|_{2} = 1$ and $w := e^{Y} \cdot v / \|e^{Y} \cdot v\|_{2}$ with perturbation $\|\sum_{a} Y_{a} \otimes I_{\overline{a}}\|_{op} = \|\Pi(Y)\|_{op} \ll 1$:
\[ |\langle X, \nabla_{w}^{2}, X \rangle - \langle X, \nabla_{v}^{2}, X \rangle| \leq O(\sum_{a} d_{a}) \|X\|^{2} \|\Pi(Y)\|_{op}   \]
\end{theorem}
\begin{proof}
Assume wlog $\sum_{a} \frac{\|X_{a}\|_{F}^{2}}{d_{a}} = 1$:
\[ |\langle X, \nabla_{v}^{2}, X \rangle - \langle X, \nabla_{w}^{2}, X \rangle| \]
\[ \leq |\langle w w^{*} - v v^{*}, \left( \sum_{a} X_{a} \right)^{2} \rangle| + |\langle w w^{*}, \sum_{a} X_{a} \rangle^{2} - \langle v v^{*}, \sum_{a} X_{a} \rangle^{2}| \]
\[ = |\langle w w^{*} - v v^{*}, \Pi(X)^{2} \rangle| + |\langle w w^{*} + v v^{*}, \Pi(X) \rangle \langle w w^{*} - v v^{*}, \Pi(X) \rangle|    \]
\[ \leq \|w w^{*} - v v^{*}\|_{1} \|\Pi(X)\|_{op}^{2} + 2 \|\Pi(X)\|_{op} \|w w^{*} - v v^{*}\|_{1} \|\Pi(X)\|_{op}  \]
\[ = 3 (\sum_{a} d_{a}) \|w w - v v^{*}\|_{1}   \]
Now we just have to bound perturbations of rank one matrices:
\[ \|w w^{*} - v v^{*}\|_{1} \leq \sqrt{2} \|w w^{*} - v v^{*}\|_{F} \leq 2 \sqrt{2} \|w - v\|_{2}  \]
Finally we can just bound the action of $Y$:
\[ \|w - v\|_{2} \leq \|\frac{e^{Y} \cdot v}{\|e^{Y} \cdot v\|_{2}} - e^Y \cdot v\|_{2} + \|e^{Y} \cdot v - v\|_{2}  \]
\[ \leq |1 - \|e^{Y} \cdot v\|_{2}| \|w\|_{2} + \|e^{Y} - I\|_{op} \|v\|_{2} \leq 2 \|e^{Y} - I\|_{op} \leq O(1) \|\Pi(Y)\|_{op} \]
\end{proof}

\begin{claim}
For unit vectors $v,w: \|v v^{*} - w w^{*}\|_{F}^{2} \leq 2 \|v - w\|_{2}^{2}$
\end{claim}
\begin{proof}
Assume wlog $v = e_{1}, w = (x,y) \in \R^{2}$ and note
\[ v v^{*} - w w^{*} = \begin{pmatrix} 1 - x^{2} & -xy \\ -xy & -y^{2} \end{pmatrix}  \]
has $Tr = 1 - x^{2} - y^{2} = 0$ and $\det = (1-x^{2})(-y^{2}) - x^{2} y^{2} = - y^{2}$; so the eigenvalues are $\pm y$.
\[ \implies \|v v^{*} - w w^{*}\|_{F}^{2} = 2 y^{2} = 2(1-x^{2}) \]
\[ \|v - w\|_{2}^{2} = (1-x)^{2} + y^{2} = 2(1-x) \]
Finally note $1 - x^{2}$ is concave and $2(1-x)$ is a tangent line at $x=1$.
\end{proof}

Recall the sufficient condition for $\lambda = \Omega(1)$-expansion
\[ \sum_{a} d_{a} \|\nabla_{a}\|_{F}^{2} \lesssim \sum_{a} d_{a}^{-1} \|Y_{a}\|_{F}^{2}  \]
So we require $\langle X, \nabla_{e^{Y}}, X \rangle \gtrsim \|X\|^{2}$. We can get this bound for all $\|\Pi(Y)\|_{op} \ll (\sum_{a} d_{a})^{-1}$. Therefore we can get a proper perturbation bound for $\forall a: \|Y_{a}\|_{F} \ll (\sum_{a} d_{a})^{-1}$ which implies $\|Y\|^{2} \ll (\max_{a} d_{a})^{-3}$, so we require $\forall a: \|\nabla_{a}\|_{F} \ll (\max_{a} d_{a})^{-2}$. This is worse than the requirement from the larger net argument above.

\CF{cole agrees - needs $D > d_a^5!$}

\subsection{Better Deterministic Robustness}
\begin{lemma}\label{lem:block-perturbation-appendix}
For perturbation $v \to \otimes_{a} e^{\delta_{a}} \cdot v =: w$ where $\forall a: \|\delta_{a}\|_{op} \ll 1$, and let $\{\sigma_{1}^{ab}, \sigma_{2}^{ab}\}$ be the matrix $\|\cdot\|_{F} \to \|\cdot\|_{F}$ norm and matrix norm on subspace $\perp$ to $(I,I)$ for each bipartite part respectively:
\[ \forall a,b: \sigma_{2}^{ab}(w w^{*}) - \sigma_{2}^{ab}(v v^{*}) \leq O \left( \sum_{a} \|\delta_{a}\|_{op}  \right) \sigma_{1}^{ab}(v v^{*})   \]
The same is true for the diagonal blocks.
\end{lemma}
\begin{proof}
To lower bound the diagonal block, we just need a spectral lower bound on $\{Q_{a}\}$, since $\langle vec(X), M^{a} (vec(X)) \rangle := \langle Q_{a}, X^{2} \rangle$.
\[ \| e^{\delta_{a}} Q_{a} e^{\delta_{a}} - Q_{a}\|_{op} = \leq O(\|\delta_{a}\|_{op}) \|Q_{a}\|_{op}   \]
Now we address a perturbation on $b \neq a$. For a spectral lower bound, we choose test $Z \succeq 0$ and let $\delta := e^{2\delta_{b}} - I$:
\[ \langle e^{\delta_{b}} v v^{*} e^{\delta_{b}} - v v^{*}, I_{\overline{a}} \otimes Z_{a} \rangle
= \langle v v^{*}, \delta \otimes Z \rangle = \langle Z, V^{*} \delta V \rangle   \]
Here $V \in \R^{d_{b} \times d_{a}}$ is the matricized version of $v$. But now since $Z \succeq 0$, the argument is clear
\[ \leq \langle Z, V^{*} |\delta| V \rangle \leq \|\delta\|_{op} \langle Z, V^{*} I V \rangle = \|\delta\|_{op} \langle v v^{*}, I_{\overline{a}} \otimes Z \rangle     \]

The argument for the off-diagonal blocks is similar. We first argue the change of $\sigma^{ab}$ is small under perturbations where $\forall c \neq a,b: \delta_{c} = 0$. Let $M^{ab}$ be the matrix versions of the bipartite operators:
\[ \langle vec(Y), M_{v}^{ab}(vec(Z)) \rangle := \langle v v^{*}, I_{\overline{ab}} \otimes Z \otimes Y \rangle \]
\[ \langle vec(Y), M_{w}^{ab}(vec(Z)) \rangle := \langle w w^{*}, I_{\overline{ab}} \otimes Z \otimes Y \rangle\]

\[ \implies M_{w} = (e^{\delta_{b}} \otimes e^{\delta_{b}}) M_{v} (e^{\delta_{a}} \otimes e^{\delta_{a}})   \]
\[ \implies \|M_{w} - M_{v}\|_{op} \leq O(\|\delta_{a}\|_{op} + \|\delta_{b}\|_{op}) \|M_{v}\|_{op}   \]
where in the last step we used $\delta \ll 1$.

The more difficult part of the argument to see (at least for me) is how $\sigma^{ab}$ changes if some other part $c \neq a,b$ is changed, i.e. $\forall d \neq c: \delta_{d} = 0$. First we define $\delta := e^{2 \delta_{c}} - I$, and test vectors $Z,Y$:
\[ \langle w w^{*} - v v^{*}, I_{\overline{ab}} \otimes Z \otimes Y \rangle = \langle v v^{*}, \delta \otimes Z \otimes Y \rangle  = \langle Z \otimes Y, V^{*} \delta V \rangle \]
Here $V \in \R^{d_{c} \times d_{a}d_{b}}$ is the matricized version of $v$, i.e. the $k$-th element of $ij$-th column is $(V_{ij})_{k} := v_{ijk}$. Now in order to use our operator norm bounds, we need to deal with cancelations, so we split into positive and negative parts $Z := Z_{+} - Z_{-}, Y := Y_{+} - Y_{-}$:
\[ |\langle Z \otimes Y, V^{*} \delta V \rangle| \leq |\langle Z_{\pm} \otimes Y_{\pm}, V^{*} \delta V \rangle |  \]
Now we analyze one of these terms (by abuse of notation $Z, Y \succ 0$):
\[ \leq \langle Z \otimes Y, V^{*} |\delta| V \rangle \leq \|\delta\|_{op} \langle Z \otimes Y, V^{*} V \rangle = \|\delta\|_{op} \langle v v^{*}, I_{\overline{ab}} \otimes Z \otimes Y \rangle   \]
Each of these terms we can bound by $\sigma^{ab}_{1} \|Z\|_{F} \|Y\|_{F}$. (Note we can save a $2$-factor on these four terms since they are decompositions of $Z,Y$). So by iterating this argument over all $c$, we get the desired bound.
\end{proof}

\begin{remark}
Instead of $F$, we could use any pair of dual norms and get the same result. In particular, we will use these results to bound the $1 \to 1$ and $op \to op$ norms of the channels. Explicitly, if we redefine $\sigma := \|\Phi\|_{p \to p}$ and $(p,q)$ are Holder conjuguates:
\[ \langle v v^{*}, \delta \otimes Z \otimes Y \rangle \leq \|\delta\|_{op} \langle v v^{*}, I_{\overline{ab}} \otimes Z \otimes Y \rangle \leq \sigma \|\delta\|_{op} \|Z_{\pm}\|_{p} \|Y_{\pm}\|_{q} \]
\[ \leq \sigma \|\delta\|_{op} (2\|Z_{+}\|_{p}^{p} + 2\|Z_{-}\|_{p}^{p})^{1/p} (2\|Y_{+}\|_{q}^{q} + 2\|Y_{-}\|_{q}^{q})^{1/q} = 2 \sigma \|\delta\|_{op} \|Z\|_{p} \|Y\|_{q}     \]
\end{remark}

\subsection{Conclusion}
At the end of the day we require $nD \gtrsim \max_{a} d_{a}^{3} \log D$ to get small enough $\nabla$ and robust expansion. Note this is $d \log D$ away from the existence/ uniqueness threshold for the solution.

\subsection{Ideas for KLR style analysis}
\begin{enumerate}
    \item The delocalization type arguments may be helpful since we know the movement is spread out.
    \item Follow gradient flow and show we have strong convergence for $\Omega(\log \max_{a} d_{a})$ time, after which the operator norm of the gradient can never be bigger than the initial one. Therefore if we have the first point, it is enough to cover the $O(1)$-size operator norm ball.
    \item Even if we get that perturbations come from $O(1)$-size operator norm ball, without some improved delocalization (or something) we would still need an extra $d$ factor.
    \item One important lemma in KLR is that the operator norm of moment map is roughly monotone decreasing over all time (where the fudge is of the order of the change in objective function $\|v\|_{2}^{2}$ over time). This is not true for $k=3$. But we have some extra structure e.g.: the error of a row and column are positively correlated.
    \item Let $\epsilon(t) := \max_{a} d_{a} \|\nabla_{a}(t)\|_{op}$. Maybe we can show
    \[ T_{\epsilon} := \inf \{ t \mid \epsilon(t) > 100 t \epsilon(0) \}  \]
    is large enough. In particular by the argument above $T_{\epsilon}$ cannot be larger than $\log d$. Maybe there's some contradiction whereby we can always increase $T_{\epsilon}$.
\end{enumerate}

Recall the gradient flow is
\[ \partial_{t} v(t) = \sum_{a} I_{\overline{a}} \otimes X_{a} \cdot v(t)   \]
where $X_{a} := d_{a} \mu_{a}(t)$. Then we have
\[ \partial_{t} \sum_{a} d_{a} \|\mu_{a}(t)\|_{F}^{2} = \sum_{a} d_{a}^{2} \langle Q^{a}, \mu_{a}^{2} \rangle + \sum_{a \neq b} d_{a} d_{b} \langle Q^{ab}, \mu_{a} \otimes \mu_{b} \rangle  \]
\[ \geq (1-\epsilon) \sum_{a} d_{a}^{2} \frac{\|\mu_{a}\|_{F}^{2}}{d_{a}} - \lambda \sum_{a \neq b} d_{a} d_{b} \frac{\|\mu_{a}\|_{F} \|\mu_{b}\|_{F}}{\sqrt{d_{a} d_{b}}}   \]
\[ \geq (1-\epsilon') \sum_{a} d_{a} \|\mu_{a}\|_{F}^{2}   \]


\subsection{Incoherence}

\begin{definition}
For two orthonormal bases $\{u_{i}\},\{v_{i}\} \subseteq \R^{d}$, the coherence between them is
\[ \gamma(U,V) := \sqrt{d} \max_{i,j} |\langle u_{i}, v_{j} \rangle|  \]
\end{definition}

Let $(\epsilon/d_{a},u)$ be the eigenpair associated with operator norm of $\mu_{a}$ and recall that under gradient flow we move in direction $X := -\sum_{a} e_{a} \otimes d_{a} \mu_{a}\}$
\[ -\partial_{t=0} \|\mu_{a}\|_{op} = -\langle e_{a} \otimes uu^{*}, \nabla^{2} X \rangle     \]
\[ = \langle \rho^{\{a\}}, d_{a} \mu_{a} u u^{*} \rangle + \sum_{b \neq a} \langle \rho^{\{a,b\}}, u u^{*} \otimes d_{b} \mu_{b} \rangle    \]
\[ = \frac{1}{d_{a}} (1+\eps) d_{a} \frac{1}{d_{a}} \eps + \sum_{b \neq a} d_{b} \langle \rho^{\{a,b\}}, u u^{*} \otimes \mu_{b} \rangle   \]

Let's focus on a single $(a,b)$ term, and denote $T : L(\R^{d_{b}}) \to L(\R^{d_{a}})$ as
\[ \langle T(Y_{b}), X_{a} \rangle := \langle \rho^{\{a,b\}}, X_{a} \otimes Y_{b} \rangle \]
So the term we are analyzing is $\langle T(\mu_{b}), u u^{*} \rangle$, which can be viewed as a diagonal of the matrix $T(\mu_{b})$. We have good bounds on this matrix:
\[ \langle I_{a}, T(\mu_{b}) \rangle = \langle \rho^{\{a,b\}}, I_{a} \otimes \mu_{b} \rangle = \|\mu_{b}\|_{F}^{2}  \]
Now define $Y := T(\mu_{b}) - \frac{\|\mu_{b}\|_{F}^{2}}{d_{a}} I_{a}$ to be projection orthogonal to the identity:
\[ \|Y\|_{F} = \sup_{X} \frac{ \langle Y,X \rangle }{\|X\|_{F}} = \sup_{X \perp I} \frac{ \langle Y, X \rangle }{\|X\|_{F}} = \sup_{X \perp I, \|X\|_{F} = 1} \langle \rho^{\{a,b\}}, X \otimes \mu_{b} \rangle \leq \frac{\lambda}{\sqrt{d_{a} d_{b}}} \|\mu_{b}\|_{F}     \]
Now if $Y = \sum_{i} y_{i} v_{i} v_{i}^{*}$ is the eigendecomposition, we have a bound on the variance of the eigenvalues. Let $\mu_{a} = \sum_{i} x_{i} u_{i} u_{i}^{*}$, and if we have a bound on the coherence of $U,V$, we can bound the competing term:
\[ d_{b} \langle \rho^{\{a,b\}}, u u^{*} \otimes \mu_{b} \rangle = d_{b} \langle \rho^{\{a,b\}}, \frac{1}{d_{a}} I_{a} \otimes \mu_{b} + (u u^{*} - \frac{1}{d_{a}} I_{a}) \otimes \mu_{b} \rangle    \]
\[ = \frac{d_{b} \|\mu_{b}\|_{F}^{2}}{d_{a}} + d_{b} \langle Y, u u^{*} \rangle
= \frac{d_{b} \|\mu_{b}\|_{F}^{2}}{d_{a}} + d_{b} \sum_{i} y_{i} \langle u, v_{i} \rangle^{2}    \]
\[ \leq \frac{d_{b} \|\mu_{b}\|_{F}^{2}}{d_{a}} + d_{b} \left( \sum_{i} y_{i}^{2} \langle u, v_{i} \rangle^{2}  \right)^{1/2} \left( \sum_{i} \langle u, v_{i} \rangle^{2} \right)^{1/2}   \]
\[ \leq \frac{d_{b} \|\mu_{b}\|_{F}^{2}}{d_{a}} + d_{b} \frac{\gamma(U,V)}{\sqrt{d_{a}}} \|Y\|_{F}
\leq \frac{d_{b} \|\mu_{b}\|_{F}^{2}}{d_{a}} + \lambda \gamma \frac{\sqrt{d_{b}} \|\mu_{b}\|_{F}}{d_{a}}    \]

So adding up all these terms, we get a competing force of
\[ \sum_{b \neq a} \frac{d_{b}}{d_{a}} \|\mu_{b}\|_{F}^{2} + \frac{\lambda \gamma}{d_{a}}  \sqrt{d_{b}} \|\mu_{b}\|_{F} < \frac{1}{d_{a}} \left[ \|\mu\|_{*}^{2} + \lambda \gamma \sqrt{k} \|\mu\|_{*}   \right]   \]
Now the $a$-th part is pushing $\approx \eps_{a}/d_{a} = \|\mu_{a}\|_{op}$. For comparison, if $\epsilon_{b} := d_{b} \|\mu_{b}\|_{op}$, then we have the bound
\[ \|\mu\|_{*}^{2} = \sum_{b} d_{b} \|\mu_{b}\|_{F}^{2} \leq \sum_{b} d_{b}^{2} \|\mu_{b}\|_{op}^{2} = \sum_{b} \epsilon_{b}^{2}   \]
So the above says if $\lambda \gamma \ll 1/k$, any eigenvalue that is on the same order as the average will be shrinking in magnitude.

\begin{fact}
Random unitaries are very incoherent:
\[ \max_{ij} \langle u_{i}, e_{j} \rangle^{2} \leq \frac{\log d}{d}    \]
with $1/poly(d)$ failure probability.
\end{fact}

\begin{claim}
The eigenbases of $\{\mu_{a}\}, \{T_{ba}(\mu_{b})\}$ are roughly as incoherent as random unitaries. This is true even after running gradient flow for some time.
\end{claim}

The problem remaining is that eigenvectors are not continuous objects, so a bound on incoherence at various discrete times does not imply a bound for all continuous times.

%\subsection{Separability}

%If the state $\rho$ is separable,





\subsection{Experiments}
\begin{enumerate}
    \item Check $\|\mu(t)\|_{op}$ is (close to) monotone decreasing
    \item Check $\int_{0}^{T} \|\mu(t)\|_{op} \gg \|\int_{0}^{T} \mu(t)\|_{op}$. I'm assumming this last term is roughly the operator norm of the scaling, but can check that too: what is the operator norm of log of the scaling at time $T$?
    \item Plot $d \|\mu(t)\|_{op}^{2} / \|\mu(t)\|_{F}^{2}$. Should be $\Tilde{O}(1)$ for some amount of time.
\end{enumerate}


\subsection{Operator Norm Monotonicity}

Our quantum setting is that we have $\{X_{1}, ..., X_{n}\}$ gaussian samples $X \sim \cN(0, \frac{1}{n} \otimes_{a} \frac{1}{d_{a}} I_{a})$ and we want to show the optimizers of the log-Likelihood function is "nearby". In the classical setting we have $k$-tensor with entries from the $\chi$-square distribution with mean $\frac{1}{nD}$ and $n$ degrees of freedom: $\chi(\frac{1}{D},n)$. For any given torus $T \subseteq G$, the Kempf-Ness function of our quantum input on this torus looks the same as a classical input.

\begin{claim}
If we condition on $T$ being the "minimizing torus" so that the optimizer of the log-likelihood function lies in this torus, then the input looks like the classical setting.
\end{claim}

By the above reduction, if we can prove a bound on the optimizer for a fixed torus, then we have shown the same bound in the quantum setting.

\begin{fact}
For $A \in \R^{n \times m}$, $\|A\|_{\infty \to \infty} \leq \max_{i \in n} \sum_{j \in [m]} |A_{ij}|$.
\end{fact}

We consider $A$ to be a bipartite marginal, so with entries $\chi(\frac{1}{d_{a} d_{b}}, \frac{nD}{d_{a} d_{b}})$. With high probability the row sums are $\in \frac{1 \pm \eps}{d_{a}}$ and the column sums are $\in \frac{1 \pm \eps}{d_{b}}$, so $\|A\|_{\infty \to \infty} \leq \frac{1+\eps}{d_{a}}$. We want to show that the gradient flow decreases the operator norm of the moment map:
\[ -\partial_{t} \langle \mu, e_{a} \otimes e_{i} \rangle = (\rho_{a})_{i} d_{a} (\mu_{a})_{i} + \sum_{b \neq a} \langle e_{i} , A^{ab} d_{b} \mu_{b} \rangle    \]
Here $(\rho_{a})_{i} \in \frac{1 \pm \eps}{d_{a}}$, and the first term has the same sign as $(\mu_{a})_{i}$ and so is pushing the correct way. We would like to show all other forces (from $b \neq a$) cannot push enough to go the wrong way.
\[ -\partial_{t} |(\mu_{a})_{i}| \geq |(\mu_{a})_{i}| d_{a} (\rho_{a})_{i} - \sum_{b \neq a} \|A^{ab}\|_{\infty \to \infty} \|d_{b} \mu_{b}\|_{\infty}      \]
Then if e.g. $\forall b \neq a: \|A^{ab}\|_{\infty \to \infty} \ll 1/k d_{a}$ and we consider $(a,i)$ to be the marginal with highest error, this error is decreasing exponentially.

Let $\langle y, \vec{1}_{b} \rangle = 0$:
\[ \E \sum_{j} y_{j} \chi(\frac{1}{d_{a} d_{b}}, \frac{nD}{d_{a} d_{b}}) = \sum_{j} y_{j} \frac{1}{d_{a} d_{b}} = 0   \]
\[ \log \E \exp \theta \sum_{j} y_{j} \chi(\frac{1}{d_{a} d_{b}}, \frac{nD}{d_{a} d_{b}}) = \sum_{j} \frac{-n D}{2d_{a} d_{b}} \log \left( 1 - \frac{2 \theta y_{j}}{n D} \right)    \]
\[ \lesssim \theta^{2} \frac{\|y\|_{2}^{2}}{n D d_{a} d_{b}} \hspace{5mm} \forall \theta \lesssim \left( \frac{\|y\|_{\infty}}{n D}  \right)^{-1}    \]
\[ \implies \Pr [|\cdot| \geq t] \leq \begin{cases}
\exp \left( - \frac{\Omega(n D d_{a} d_{b} t^{2})}{\|y\|_{2}^{2}} \right) & \forall t \lesssim \frac{\|y\|_{2}^{2}}{d_{a} d_{b} \|y\|_{\infty}}
\\ \exp \left( - \frac{ \Omega(n D t) }{\|y\|_{\infty}} \right) & o.w.  \end{cases}   \]
To bound $\|A\|_{\infty \to \infty} \ll 1/k d_{a}$, we choose $t \ll \frac{\|y\|_{\infty}}{k d_{a}}$ which is in the second case. So we get probability $\leq \exp( - \frac{n D}{k d_{a}} )$. If we run a net over $\R^{d_{b}} \perp \vec{1}_{b}$ of size $\exp(\Tilde{O}(d_{b}))$ and union bound over all rows, we get $\|A\|_{\infty \to \infty} \ll 1/k d_{a}$ whp whenever $n D \gg k d_{a}( d_{b} + \log d_{a} ) $.

To make this robust to perturbations we could either give a deterministic bound on how a perturbation affects $\|A\|_{\infty \to \infty}$; or we could run the net over all perturbations near $I$ of size $\exp( \sum_{a} d_{a} )$. Either way we should get that $\|\mu\|_{op}$ is decreasing for all time when $nD \sim \Tilde{O}_{k}(\max_{ab} d_{a} d_{b} )$


\subsection{A silly trick}

The same idea that showed monotonicity of the operator norm for operator scaling can be used to very slightly improve the required expansion for $k$-tensors. In the proper norm, our gradient direction will be $\{d_{a} \mu^{a}\}$, and so to use robustness we need a bound on perturbation size $\sum_{a} d_{a} \|\mu^{a}\|_{op} =: \sum_{a} \eps_{a}$ for all time. Let $\eps := \max_{a} \eps_{a}$. In the proper norm, we have weak two sided bounds at any time:
\[ \eps^{2} = d_{a}^{2} \|\mu_{a}\|_{op}^{2} \leq d_{a}^{2} \|\mu^{a}\|_{F}^{2} \leq d_{a} \left( \sum_{b} d_{b} \|\mu^{b}\|_{F}^{2} \right) \leq d_{a} \left( \sum_{b} \eps_{b}^{2} \right) \leq d_{a} k \eps^{2}   \]

Now recall the change in operator norm under negative gradient flow:
\[ -\partial_{t} \eps := - d_{a} \partial_{t} \langle u u^{*}, \mu^{a} \rangle = d_{a} \left(\langle \rho^{a}, (u u^{*}) (d_{a} \mu^{a}) \rangle + \sum_{b \neq a} \langle \rho^{ab}, u u^{*} \otimes d_{b} \mu^{b} \rangle \right)   \]
\[ \geq (1 \pm \eps) \left( \eps_{a} - \sum_{b \neq a} \eps_{b} \right) \geq (1 \pm \eps) (k-2) \eps    \]
Here we've used that $\eps_{a}$ is the biggest. To see the transition for the first to second line, recall $\|d_{b} \mu^{b}\|_{op} = \eps_{b} \leq \eps$, and for any basis $\{v_{i}\} \subseteq \R^{d_{b}}$ we have
\[ \langle \rho^{ab}, u u^{*} \otimes d_{b} \mu^{b} \rangle \leq \langle \rho^{ab}, u u^{*} \otimes \eps_{b} I_{b} \rangle = \eps_{b} \langle \rho^{a}, u u^{*} \rangle = \eps_{b} \frac{1 \pm \eps}{d_{a}}    \]
Here we've used $\pm$ depending on whether $\eps$ comes from a largest or a smallest eigenvalue. So what the above inequalities show is $\partial_{t} (\log \eps) \leq (1+\eps) (k-2) \approxeq (k-2)$. From here on I'll be ignoring low order terms. I'll mention later that they add a small error to the main theorem. Therefore we assume we initially have $1-o(1) \approxeq 1$ strong convergence at time $0$ and further that this is true up to another $o(1)$ term while our perturbation size is $o(1)$. 

\begin{claim}
$\forall t: \eps(t) \lesssim d^{\frac{(k-2)}{1 + 2(k-2)}} \eps(0)$
\end{claim}
\begin{proof}
Let $T$ be the first time when $\eps(t) \geq d^{(k-2) c} \eps(0)$: 
\[ (\log \eps)' \leq k-2 \implies T \geq \frac{\log \eps(T) - \log \eps(0)}{k-2} = c \log d  \]
On the other hand let $T^{*}$ be the last time when $\eps(T^{*}) \leq o(1)$, so up to this time we have very strong convergence:
\[ \forall t \leq T^{*}: \sum_{a} d_{a} \|\mu^{a}\|_{F}^{2} (t) \leq e^{-t} \sum_{a} d_{a} \|\mu^{a}\|_{F}^{2} (0) \leq e^{-t} k \eps(0)^{2}    \]
In particular if $T^{*} > \log k d^{1 - 2(k-2)c}$:
\[ \eps(T^{*})^{2} \leq e^{-T^{*}} d k \eps(0)^{2} < d^{2(k-2)c} \eps(0)^{2} = \eps(T)^{2}    \]
This would be a contradiction if $T \geq T^{*}$, so matching terms we get
\[ c \log d = \log k + (1 - 2(k-2)c) \log d \iff c = \frac{1}{1 + 2(k-2)}     \]
Again we ignore the lower order $\log k$ term. So if we have strong convergence up to time $T^{*}$ then the claim is proven. 
\end{proof}

The above claim shows that if $d^{\frac{(k-2)}{1 + 2(k-2)}} \eps < o(1)$, and we have $1-o(1)$ at time $0$, then in fact we have $1-o(1)$ strong convergence at all time and we remain in an $d^{\frac{(k-2)}{1 + 2(k-2)}} \eps < o(1)$ sized operator norm ball the whole time. Therefore it suffices to take $d^{3 - \frac{1}{1 + 2(k-2)}}$ samples to get strong convergence. Forgive me the $k$'s and this is our savings on samples. 


\subsection{The Final Concentration}
Above we were trying to bound competing forces on the largest eigenvalue of $\mu$ for all time to make sure our scaling remains inside a small $\|\cdot\|_{op}$ ball. \AR{Here we will show it is enough to have small forces initially. }

\AR{Correction: the following line of argument isn't really good enough. This naive way of bounding the competing forces means that our $\|\mu\|_{op}$ jumps into the worst regime (exponentially increasing) within a constant amount of time. This is because our bound on the force is proportional to the size of the perturbation. }

If our input at time $t$ is scaled by $e^{\delta}$, then we have the following rough bound on our errors:
\[ \|\mu_{b}(t) - \mu_{b}(0)\|_{op} = \sup_{v} \langle (e^{\delta} - I) \cdot \rho^{ab}, I_{a} \otimes v v^{*} \rangle \lesssim \delta \|T^{*}\|_{op_{a} \to op_{b}} \|I_{a}\|_{op} \|v v^{*} \|_{1} \leq \delta \|\rho^{b}\|_{op}    \]

\[ \|T(\mu_{b})(t)\|_{op} - \|T(\mu_{b})(0)\|_{op} \leq \|T(\mu_{b})(t) - T(\mu_{b})(0)\|_{op}  = \sup_{u} \langle u u^{*}, T(\mu_{b})(t) - T(\mu_{b})(0) \rangle  \]
Let $\mu_{b}(t) := \mu_{b}(0) + Y =: \mu_{b} + Y$
\[ \|T^{t}(\mu_{b}(t)) - T^{0}(\mu_{b}(0))\|_{op} \leq \|T^{t}(\mu_{b} + Y) - T^{t}(\mu_{b})\|_{op} + \|T^{t}(\mu_{b}) - T^{0}(\mu_{b})\|_{op}     \]
\[ \leq \|T^{t}\|_{op_{b} \to op_{a}} \|Y\|_{op} + \delta \|T^{0}\|_{op_{b} \to op_{b}} \|\mu_{b}\|_{op}    \]
\[ \leq \delta \|\rho^{a}(t)\|_{op} \|\rho_{b}\|_{op} + \delta \|\rho_{a}\|_{op} \|\mu_{b}\|_{op}   \]
Assuming $\|\mu_{b}\|_{op} \ll \delta/d_{b}$, this is on the order of $\delta/ d_{a} d_{b}$. Recall the formula for the top eigenvalue change:
\[ -\partial_{t} d_{a} \|\mu_{a}\|_{op} = \|d_{a} \rho^{a}\|_{op} \|d_{a} \mu^{a}\|_{op} + \sum_{b \neq a} d_{a} d_{b} \langle \rho^{\{a,b\}}, u u^{*} \otimes \mu_{b} \rangle     \]
\[ \geq  (1 \pm d_{a} \|\mu_{a}\|_{op}) \|d_{a} \mu^{a}\|_{op} -  \sum_{b \neq a} d_{a} d_{b} \|T_{b \to a}(\mu_{b})\|_{op}   \]
\[ \geq (1 \pm \epsilon_{a}) \epsilon_{a} - (k-1) \delta  \]
This is the competing force at time $t$, assuming it is very small at time $0$. Therefore we have that any (normalized) eigenvalue of $\mu$ that is greater than $(k-1) \delta$ at time $t$ is decreasing in absolute value. It should be enough to have $\delta \ll 1/k \log D$, and then in total the scaling should be of operator norm $\ll 1$, so we will have convergence. 







\section{Old stuff}

\subsection{Different Inner Product}

\begin{definition}
For desired marginals $\{R_{a}^{2}\}_{a \in [k]}$ (assume for now $R$ are Hermitian though we can pick different square roots if required), define inner product
\[ \langle Z, Y \rangle_{R} := \sum_{a} \langle R_{a} Z R_{a}^{*}, Y \rangle \]
\[ \|Z\|_{R}^{2} := \langle Z, Z \rangle_{R} = \sum_{a} \|R_{a} Z_{a}\|_{F}^{2}   \]
\end{definition}

We restate the projective likelihood function and define gradient and Hessian in this metric:

\begin{definition}
\[ f_{\vec X}(\Theta_1, \dots, \Theta_n) = \log \left\langle \sum_{i \in [n]} X_{i} X_{i}^{*},  \bigotimes_{a \in [k]} \Theta_a \right\rangle - \sum_{a \in [k]} \frac{1}{d_a} \log\det \Theta_a  \]
Also $\rho := \sum_{i} X_{i} X_{i}^{*}$ and $\{\rho^{S}\}_{S \subseteq [k]}$ are marginals.
\end{definition}

\begin{fact}
\[ (\nabla f(I))_{a} = d_{a} \rho^{\{a\}} - I_{a}  \]
\end{fact}
\begin{proof}
We can define $\nabla f$ dually as $\forall Z: \langle \nabla f(I), Z \rangle_{R} := \partial_{t=0} f(e^{tZ})$
\[ \partial_{t=0} f(e^{t Z_{a}}) = \partial_{t=0} \langle \rho, I_{\overline{a}} \otimes e^{tZ_{a}}  \rangle - \partial_{t=0} \frac{1}{d_{a}} \log\det e^{t Z_{a}}  \]
\[ = \left\langle \rho^{\{a\}} - \frac{1}{d_{a}} I_{a}, Z_{a} e^{t Z_{a}} \right\rangle|_{t=0} = \left\langle R_{a}^{-1} \left( \rho^{\{a\}} - \frac{1}{d_{a}} I_{a} \right) R_{a}^{-1}, Z_{a} \right\rangle_{R}   \]

Similarly we define the Hessian as
\[ \partial_{s=t=0} f(e^{tZ_{a} + sY_{b}}) = \langle \rho, \{ I_{\overline{a}} \otimes Z_{a}, I_{\overline{b}} \otimes Y_{b} \} \rangle   \]
\[ \implies (\nabla^{2} f(I))_{aa} = \langle R_{a}^{-1} \rho^{\{a\}} R_{a}^{-1}, \{Z, Y\} \rangle_{R}    \]
\[ \implies (\nabla^{2} f(I))_{ab} = \langle \rho^{\{a,b\}}, Z \otimes Y \rangle   \]
\end{proof}

\AR{Not sure how to define Hessian. I think I'd like the Hessian to be
\[  \sum_{a} E_{aa} \otimes (1 \pm \epsilon) I + \sum_{a \neq b} E_{ab} \otimes \pm \lambda  \]
for some small $\epsilon,\lambda$. Then the Hessian will be $1-\epsilon - (k-1) \lambda$-strongly convex. }



\begin{lemma} [Restatement of \cref{lem:convex-ball}]
Let $f$ be geodesically convex everywhere. All the below quantities are wrt metric $\langle \cdot, \cdot \rangle_{R}$. Assume $f$ and $\lambda$-strongly geodesically convex ball of radius $\kappa$ about $I$; further assume the geodesic gradient satisfies $\|\nabla f(I)\|_{R} = \eps < \lambda \kappa$. Then there is an optimizer within an $\eps/\lambda$-ball.
\end{lemma}
\begin{proof} [Proof of \cref{lem:convex-ball}]
The proof is exactly the same except the following:
\[ g'(0) = \langle \nabla f(I), Z \rangle_{R} \geq - \|\nabla f(I)\|_{R} \|Z\|_{R} \geq - \eps     \]
\end{proof}

\begin{remark}
Note the perturbation lemma then gives the following strategy. By Cole's lemma, we have that $c \|\nabla f(I)\|_{R} \geq \|\nabla f(I)\|_{op}$. If we can say the same thing for the optimizer $Z$, then it is enough for $\lambda \kappa \geq \Omega(1/c) > \eps$ and we can improve sample complexity to $nD > c \max_{a} d_{a}^{2}$.

A similar thing is true if we can show the above inequality for the gradient flow for $\log \max_{a} d_{a}$ time.
\end{remark}

\begin{lemma}
$\lambda$-strong convexity is a sufficient condition for fast convergence of the gradient flow:
\[ - \partial_{t=0} \|\nabla f(e^{-t\nabla f(I)})\|_{R}^{2} = -\partial_{t=0}^{2} f(e^{-t\nabla f(I)}) = \langle \nabla^{2} f, \nabla f \otimes \nabla f \rangle \geq \lambda \|\nabla f\|_{R}^{2}    \]
\AR{Not sure how to write the third term above, the inner product with Hessian}
\end{lemma}

\subsection{Old proof of \cref{thm:tensor-convexity}}


\begin{proof} [Proof of \cref{thm:tensor-convexity}]
Take any quadratic form of the Hessian for $\{Z_{a} \perp I_{a}\}$:
\[ \partial_{t=0}^{2} f(e^{t Z}) = \sum_{a} \langle Q^{a}, Z_{a}^{2} \rangle + \sum_{a \neq b} \langle Q^{ab}, Z_{a} \otimes Z_{b} \rangle   \]
\[ \geq \sum_{a} \lambda_{\min}(Q^{a}) \|Z_{a}\|_{F}^{2} - \sum_{a \neq b} \|Q^{ab} (I-P_{ab})\|_{op} \|Z_{a}\|_{F} \|Z_{b}\|_{F}    \]
Now we can use our high probability bounds derived above:
\[ \forall a: Q^{a} \succeq \frac{1-\epsilon}{d_{a}} I_{a}; \hspace{10mm}
\forall a \neq b: \|Q^{ab} (I-P_{ab})\|_{op} < \frac{\lambda}{\sqrt{d_{a} d_{b}}}   \]
\[ \implies \partial_{t=0}^{2} f(e^{t Z}) \geq \sum_{a} \frac{1-\eps}{d_{a}} \|Z_{a}\|_{F}^{2} - \sum_{a \neq b} \frac{\lambda}{\sqrt{d_{a} d_{b}}} \|Z_{a}\|_{F} \|Z_{b}\|_{F}  \]
\[ \geq \sum_{a} \frac{1-\eps+\lambda}{d_{a}} \|Z_{a}\|_{F}^{2} - \lambda \left( \sum_{a} \frac{1}{\sqrt{d_{a}}} \|Z_{a}\|_{F}   \right)^{2}   \]
\[ \geq \sum_{a} \frac{1-\eps+\lambda}{d_{a}} \|Z_{a}\|_{F}^{2} - k\lambda \sum_{a} \frac{1}{d_{a}} \|Z_{a}\|_{F}^{2} \]
\[ = (1-\eps-(k-1)\lambda) \|Z\|^{2}    \]
Choosing $\eps,\lambda$ small enough gives the theorem.
\end{proof}

\begin{proof}[Proof: \CF{Akshay's conceptual proof of \cref{thm:tensor-convexity}}]
We can in fact show that $\nabla^{2}$ is well-conditioned using the following:
\[ -\begin{pmatrix} 1 & 0 \\ 0 & 1 \end{pmatrix}
\preceq \begin{pmatrix} 0 & 1 \\ 1 & 0 \end{pmatrix}
\preceq \begin{pmatrix} 1 & 0 \\ 0 & 1 \end{pmatrix}
\]
%We can rewrite the Hessian using shorthand $\{M_{a}\}$ for the diagonal blocks and $\{M_{ab}\}$ for off-diagonal blocks: \CF{why not just write $\nabla^2_{ab}f$?} \CF{Also, the matrices $E_{aa} \otimes M_a$ are not all of the same size} 
\\ \AR{Ya $\nabla_{ab}$ notation is fine, just needed something that was a matrix of the right dimensions, so shorthand M was to avoid weird things with $\rho$}
\\ \AR{It's fine if they're of different sizes, we enumerate the basis of the whole space as $\cup_{a} e_{a} \otimes \{e_{i \in [d_{a}]}\}$ }\\
\CF{$E_{aa} \otimes \nabla^2_{aa}$ is $k d_a^2 \times k d_a^2$ dimensional. So how does this make sense? Maybe needs to be updated along the lines of the next proof.}
\[ \nabla^{2} f = \sum_{a} E_{aa} \otimes \nabla^{2}_{aa} + \sum_{a \neq b} E_{ab} \otimes \nabla^{2}_{ab}  \]
Now we can again use the high-probability bounds derived above: \TODO{actually cref them}
\begin{align}\nabla^{2}_{aa} \in \frac{1 \pm \eps}{d_{a}}; \hspace{5mm} \forall a \neq b: \|\nabla^{2}_{ab}\|_{op} \leq \frac{\lambda}{\sqrt{d_{a} d_{b}}} \label{eq:expansion-thing}  \end{align}
\[ \nabla^{2} \preceq \sum_{a} E_{aa} \otimes \left( \frac{1+\eps}{d_{a}} I_{a} \right) + \sum_{a < b} E_{aa} \otimes \left( \frac{\lambda}{d_{a}} I_{a} \right) + E_{bb} \otimes \left( \frac{\lambda}{d_{b}} I_{b} \right)    \]
\[ \preceq \sum_{a} E_{aa} \otimes \frac{1+\eps+(k-1)\lambda}{d_{a}} I_{a}  \]

The same sequence of inequalities can be reversed to show a lower bound. So in fact we can show the above bounds on blocks shows $1+O(\eps + k \lambda)$-condition number bound on the Hessian in norm $\|\cdot\|_{d}$. 
\end{proof}





%If $\langle \cdot , \cdot \rangle_P$ is the metric at a point $P$, then the new metric $\langle \cdot, \cdot \rangle_{R,P}$ will be given by
%$$ \langle X, Y \rangle_{R,P} = \langle R^{1/2} X R^{1/2}, R^{1/2} Y R^{1/2} \rangle_{P}.$$

%At the identity the metric takes the form $\langle X, Y \rangle_R = \tr \sqrt{R} X  \sqrt{R} Y $. In particular the length of the geodesic $t\mapsto \sqrt{P}e^{Xt}\sqrt{P}$ for $t \in [0,1]$ is $\langle X, X \rangle_R$.

\bibliographystyle{alpha}
\bibliography{refs}

\end{document}
