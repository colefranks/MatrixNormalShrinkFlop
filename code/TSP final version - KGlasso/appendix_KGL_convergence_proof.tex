\begin{IEEEproof}

\begin{enumerate}
	\item Let $L(\mathbf{x}^0)=L(\mathbf{x}_1^0,\dots,\mathbf{x}_k^0)$ be the set of all limit points of $(\mathbf{x}^m)_{m\geq 0}$ starting from $\mathbf{x}^0$. By joint continuity of the objective, we have $J_\lambda(\mathbf{x}^{m_j}) \to J_\lambda(\mathbf{x}^*)$ as $j\to\infty$.
	%By the continuity of $J_i$, we have $\sum_{i=1}^k{J_i(\mathbf{x}_i^{m_j})} \to \sum_{i=1}^k{J_i(\mathbf{x}_i^*)}$ as $j \to \infty$. Since $J_0(\cdot)$ is jointly continuous, $J_0(\mathbf{x}_1^{m_j},\dots,\mathbf{x}_k^{m_j}) \to J_0(\mathbf{x}_1^*,\dots,\mathbf{x}_k^*)$. By continuity of $\eta_i(\cdot)$, $\sum_{i=1}^k{\bar{\lambda}_i \eta_i(\mathbf{x}_i^{m_j})} \to \sum_{i=1}^k \bar{\lambda}_i \eta_i(\mathbf{x}_i^*)$. Thus, $J_\lambda(\mathbf{x}^{m_j}) \to J_\lambda(\mathbf{x}^*)$ as $j\to\infty$.
	Note that $\nabla J_0$ is uniformly continuous on bounded subsets of its domain. Define the sequence $\{(x_i^m)^o\}_m$ as
	\begin{align*}
		(\mathbf{x}_1^m)^\circ & := \nabla_{\mathbf{x}_1}J_0(\mathbf{x}_1^m,\mathbf{x}_2^m\dots,\mathbf{x}_k^m) \\
			&\quad - \nabla_{\mathbf{x}_1}J_0(\mathbf{x}_1^m,\mathbf{x}_2^{m-1}\dots,\mathbf{x}_k^{m-1}) \\
		(\mathbf{x}_2^m)^\circ & := \nabla_{\mathbf{x}_2}J_0(\mathbf{x}_1^m,\mathbf{x}_2^m\dots,\mathbf{x}_k^m) \\
			&\quad - \nabla_{\mathbf{x}_2}J_0(\mathbf{x}_1^m,\mathbf{x}_2^m,\mathbf{x}_{3}^{m-1}\dots,\mathbf{x}_k^{m-1}) \\
		&\vdots \\
		(\mathbf{x}_j^m)^\circ & := \nabla_{\mathbf{x}_j}J_0(\mathbf{x}_1^m,\mathbf{x}_2^m\dots,\mathbf{x}_k^m) \\
			&\quad - \nabla_{\mathbf{x}_j}J_0(\mathbf{x}_1^m,\dots,\mathbf{x}_j^m,\mathbf{x}_{j+1}^{m-1}\dots,\mathbf{x}_k^{m-1}) \\
		&\vdots \\
		(\mathbf{x}_k^m)^\circ &:= 0
	\end{align*}
	Then, it can be shown $((\mathbf{x}_1^m)^\circ,\dots,(\mathbf{x}_k^m)^\circ) \in \partial J_\lambda(\mathbf{x}_1^m,\dots,\mathbf{x}_k^m)$. This implies $((\mathbf{x}^{m_j})^{\circ}) \in \partial J_\lambda(\mathbf{x}^{m^j})$ \cite{TsiligkaridisTSP}. Since the subsequence $(\mathbf{x}^{m_j})_j$ is convergent, we have $(\mathbf{x}^{m_j})^\circ \to 0$ as $j \to \infty$ \cite{TsiligkaridisTSP}. Since $\partial J_\lambda(\mathbf{x}^{m_j})$ is closed (see Theorem 8.6 in \cite{VariationalAnalysis}) for all $j$, we conclude that $\mathbf{x}^* \in C_J$. Thus, $L(\mathbf{x}^0) \subseteq C_J$.
	
	%Now, it can be shown \cite{TsiligkaridisTSP} that $((\mathbf{x}^{m_j})^{\circ}) \in \partial J_\lambda(\mathbf{x}^{m^j})$. Since the subsequence $(\mathbf{x}^{m_j})_j$ is convergent, we have \cite{TsiligkaridisTSP} $(\mathbf{x}^{m_j})^\circ \to 0$ as $j \to \infty$. As a result, since $\partial J_\lambda(\mathbf{x}^{m_j})$ is closed (see Theorem 8.6 in \cite{VariationalAnalysis}) for all $j$, we conclude that $\mathbf{x}^* \in C_J$. Thus, $L(\mathbf{x}^0) \subseteq C_J$.
	
	%We have thus proved that limit points are critical points of the objective function.
	We can easily rule out convergence to saddle points. Theorem \ref{convergence_fixed_point} implies that $L(\mathbf{x}^0)$ is nonempty and singleton.
	
	\item Let $\mu(\cdot)$ denote the point-to-point mapping during one iteration step, i.e., $\mathbf{x}^{m+1}=\mu(\mathbf{x}^m)$. We show that if $\mathbf{x}^0 \notin C_J$, then $L(\mathbf{x}^0) \subseteq C_{J,min} \cup C_{J,saddle}$. The result then follows by using the proof of the first part. To this end, let $\mathbf{x}^{'}$ be a fixed point under $\mu$, i.e., $\mu(\mathbf{x}^{'})=\mathbf{x}^{'}$. Then, the subiteration steps of the algorithm yield $0\in \partial_{\mathbf{x}_i} J_\lambda(\mathbf{x}_1^{'},\dots,\mathbf{x}_k^{'})$ for $i=1,\dots,k$, which implies $0\in \partial J_\lambda(\mathbf{x}^{'})$, i.e., $\mathbf{x}^{'}\in C_J$. The contrapositive implies that if $\mathbf{x}\notin C_J$, then $J_\lambda(\mu(\mathbf{x}))<J_\lambda(\mathbf{x})$. A simple induction on the number of iterations then concludes the proof.
\end{enumerate}
	
\end{IEEEproof}
