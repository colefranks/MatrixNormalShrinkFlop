\begin{IEEEproof}
As in the proof of Thm. 1 in \cite{EstCovMatKron}, let $\bB_* = \frac{\tr(\bA_0\bA_{init}^{-1})}{p} \bB_0$ and $\bA_* = (\frac{\tr(\bA_0\bA_{init}^{-1})}{p})^{-1} \bA_0$. Note that Assumption 1 implies that $\nn \bB_* \nn_2 = \Theta(1)$ and $\nn \bA_* \nn_2=\Theta(1)$ as $p,f\to\infty$. For conciseness, the statement ``with probability $1-\frac{2}{n^2}$'' will be abbreviated as ``w.h.p.''-i.e., with high probability.

For concreteness, we first present the result for $k=2$ iterations. Then, we generalize the analysis to all finite flip-flop iterations by induction.

The growth assumptions in the theorem imply
\begin{equation} \label{FF_sufficient_condition}
	\max \left\{p,f, \frac{f^2}{p}, \left(\frac{\sqrt{pf} + f\sqrt{\frac{f}{p}} + p \sqrt{\frac{p}{f}}}{p+f}\right)^2 \right\} \log M \leq C'n
\end{equation}
for some constant $C'>0$ large enough \footnote{This constant is independent of $p,f,n$, but may depend on the constants in Assumption \ref{assumption_posdef_unif}.}. In fact, the growth assumption in the theorem statement can be relaxed to (\ref{FF_sufficient_condition}).

Define intermediate error matrices:
\begin{align*}
	\tilde{\bB}^0 &= \hat{\bB}(\bA_{init}) - \bB_* \\
	\tilde{\bA}^1 &= \hat{\bA}(\hat{\bB}(\bA_{init})) - \bA_*
%	\tilde{\bB}^2 &= \hat{\bB}(\hat{\bA}(\hat{\bB}(\bA_{init}))) - \bB_*
\end{align*}
Define $\bY_*=\bB_*^{-1}$ and $\bX_*=\bA_*^{-1}$. Also, define:
\begin{align*}
	\bY_1 &= \hat{\bB}(\bA_{init})^{-1} \\
	\bX_2 &= \hat{\bA}(\hat{\bB}(\bA_{init}))^{-1}
%	\bY_3 &= \hat{\bB}(\hat{\bA}(\hat{\bB}(\bA_{init})))^{-1}
\end{align*}
These inverses exist if $n\geq \max(\frac{p}{f},\frac{f}{p}) + 1$ (see \cite{LuZimmerman}). Define the error $\tilde{\bSigma}_{FF}(k)=\bSigma_{FF}(k) - \bSigma_0$ for $k \geq 2$. For notational simplicity, let $\bB_0^{max}:=\max_{k}[\bB_0]_{k,k}$ and $\bA_0^{max}:=\max_{i}[\bA_0]_{i,i}$, $\psi_{\tau} := \psi(\frac{1}{2+\tau})$, where $\psi(\cdot)$ is defined in Lemma \ref{lemma: large_dev_Ted}.

Lemma \ref{lemma: large_dev_Ted} implies that for
\begin{equation} \label{cond_0}
	n> \frac{8(2+\tau)^2}{\psi_{\tau}} \log M
\end{equation}
then with probability $1-\frac{2}{n^2}$, we have:
\begin{equation} \label{B_0_Frob1}
	\nn \tilde{\bB}^0 \nn_F \leq 2\sqrt{2\psi_\tau} \bB_0^{max} \nn\bA_{init}^{-1} \bA_0\nn_2  f \sqrt{ \frac{\log M}{np}}
\end{equation}

As in the proof of Thm. 1 in \cite{EstCovMatKron}, we vectorize the operations (\ref{A_update}) and (\ref{B_update}):
\begin{align*}
	\vec(\hat{\bA}(\bB)) &= \frac{1}{f} \hat{\bR}_A \vec(\bB^{-1}) \\
	\vec(\hat{\bB}(\bA)) &= \frac{1}{p} \hat{\bR}_B \vec(\bA^{-1})
\end{align*}
where $\hat{\bR}_A$ and $\hat{\bR}_B$ are permuted versions of the sample covariance matrix \cite{EstCovMatKron}.

Let $\epsilon'>1$. Note that from (\ref{B_0_Frob1}), for
\begin{equation} \label{cond_1}
	n\geq (\epsilon' 2\sqrt{2\psi_\tau} \bB_0^{max} \nn\bA_{init}^{-1} \bA_0\nn_2)^2 f^2 p^{-1} \log M
\end{equation}
with probability $1-\frac{2}{n^2}$,
\begin{align*}
	\lambda_{min} &(\hat{\bB}(\bA_{init})) = \lambda_{min}(\tilde{\bB}^0 + \bB_*) \geq \lambda_{min}(\bB_*) - \nn \tilde{\bB}^0 \nn_2 \\
		&\geq \lambda_{min}(\bB_*) - \nn \tilde{\bB}^0 \nn_F \geq \left( 1-\frac{1}{\epsilon'} \right) \lambda_{min}(\bB_*)
\end{align*}
Thus, w.h.p.,
\begin{align}
	\nn & \bY_1-\bY_* \nn_F = \nn \bY_1 (\hat{\bB}(\bA_{init})-\bB_*) \bY_* \nn_F \nonumber \\ 
		&\leq \nn \bY_1 \nn_2 \nn \bY_* \nn_2 \nn \tilde{\bB}^0 \nn_F = \frac{\nn \tilde{\bB}^0 \nn_F}{ \lambda_{min}(\bB_*) \lambda_{min}(\hat{\bB}(\bA_{init})) } \nonumber \\
		&\leq \left( 1-\frac{1}{\epsilon'} \right)^{-1} \nn\bY_*\nn_2^2 2\sqrt{2\psi_\tau} \bB_0^{max} \nn\bA_{init}^{-1} \bA_0\nn_2 \nonumber \\
		&\quad  \times fp^{-1/2} \sqrt{\frac{\log M}{n}} \label{bound_Frob_spec}
\end{align}

Using Lemma \ref{lemma: large_dev_Ted}, for 
\begin{equation} \label{cond_2}
	n>\frac{8(2+\tau)^2}{\psi_{\tau}} \log M
\end{equation}
then, w.h.p.,
\begin{align}
	\nn & \tilde{\bR}_A\nn_2 = \sup_{\nn \bv \nn_2=1} \nn \tilde{\bR}_A \bv \nn_2 \leq p \sup_{\nn \bv \nn_2 = 1} \nn \hat{\bR}_A \bv - \bR_A \bv \nn_\infty \nonumber \\
		&= pf \sup_{\nn \bv \nn_2 = 1} \nn \frac{1}{f} \hat{\bR}_A \bv - \frac{<\vec(\bB_0),\bv>}{f} \vec(\bA_0) \nn_\infty \nonumber \\
		&\leq 2\sqrt{2\psi_\tau} \bA_0^{max} \nn\bB_0\nn_2  p\sqrt{f} \sqrt{\frac{\log M}{n}} \label{bound_mixed_norm2}
\end{align}

Expanding $\tilde{\bA}^1$:
\begin{align}
	\vec(\tilde{\bA}^1) &= \frac{1}{f} \hat{\bR}_A \vec(\bY_1) - \vec(\bA_*) \nonumber \\
		%&= \frac{1}{f} (\bR_A+\tilde{\bR}_A) \vec(\bY_1-\bY_*)  + \frac{1}{f} \tilde{\bR}_A \vec(\bY_*) \nonumber  \\
		&= \frac{\tr(\bB_0(\bY_1-\bY_*))}{f} \vec(\bA_0) + \vec(\hat{\bA}(\bB_*)-\bA_*) \nonumber \\
		&\quad + \frac{1}{f} \tilde{\bR}_A \vec(\bY_1-\bY_*) \label{A_1_error2}
\end{align}
where we used $\bR_A = \vec(\bA_0)\vec(\bB_0^T)^T$ (see Eq. (91) from \cite{EstCovMatKron}).

Now, using the triangle inequality in (\ref{A_1_error2}), the bounds (\ref{bound_Frob_spec}) and (\ref{bound_mixed_norm2}), the Cauchy-Schwarz inequality, we obtain w.h.p. (under conditions (\ref{cond_0}),(\ref{cond_1}),(\ref{cond_2})), after some algebra:
\begin{align}
	\nn \tilde{\bA}^1 \nn_F &\leq \sqrt{\frac{p}{f}} \nn \bA_0\nn_2 \nn \bB_0\nn_2 \nn \bY_1-\bY_* \nn_F + p |\hat{\bA}(\bB_*)-\bA_*|_\infty \nonumber \\
		&\quad + \frac{1}{f} \nn\tilde{\bR}_A\nn_2 \nn \bY_1-\bY_* \nn_F  \nonumber \\
		&\leq \tilde{C}_1 (\sqrt{f}+pf^{-1/2}) \sqrt{\frac{\log M}{n}} + \tilde{C}_2 \sqrt{pf} \frac{\log M}{n} \label{tilde_A_1_Frob_bound}
\end{align}
where $\tilde{C}_1,\tilde{C}_2$ are absolute constants \cite{TsiligkaridisTSP}.
%\begin{align*}
%	\tilde{C}_1 &:= 2\sqrt{2\psi_\tau} \max\Big\{ \kappa(\bSigma_0) \left(1-\frac{1}{\epsilon'}\right)^{-1} \nn\bTheta_0\nn_2 \nn\bA_{init}\nn_2^2 \\
%		&\quad \times \bB_0^{max} \nn\bA_{init}^{-1}\bA_0\nn_2, \bA_0^{max} \left| \frac{p}{\tr(\bA_0\bA_{init}^{-1})} \right| \Big\}  \\
%	\tilde{C}_2 &:= \left(1-\frac{1}{\epsilon'}\right)^{-1} 8 \psi_\tau \bA_0^{max}\bB_0^{max} \nn\bB_0\nn_2 \nn\bA_{init}^{-1}\bA_0\nn_2 \\
%		&\quad \times \nn\bTheta_0\nn_2^2 \nn\bA_{init}^{-1}\nn_2^2
%\end{align*}

Let $c_1>0$. For
\begin{equation} \label{cond_3}
	n \geq (\frac{\tilde{C}_2}{\tilde{C}_1 c_1})^2 p \log M
\end{equation}
then, from (\ref{tilde_A_1_Frob_bound}), we have w.h.p.
\begin{equation} \label{tilde_A_1_Frob2}
	\nn \tilde{\bA}^1 \nn_F \leq \tilde{C}_1 (1+c_1) (\sqrt{f}+pf^{-1/2}) \sqrt{\frac{\log M}{n}}
\end{equation}

Using the permutation operator $\mathcal{R}(\cdot)$ defined in \cite{EstCovMatKron}:
\begin{align}
	& \vec(\mathcal{R}(\tilde{\bSigma}_{FF}(2))) = \vec(\vec(\tilde{\bA}^1)\vec(\bB_*)^T) \nonumber \\
		&\quad + \vec(\vec(\bA_*)\vec(\tilde{\bB}^0)^T) + \vec(\vec(\tilde{\bA}^1)\vec(\tilde{\bB^0})^T) \label{error_FF_2iter}
\end{align}
From (\ref{B_0_Frob1}),(\ref{tilde_A_1_Frob_bound}), (\ref{error_FF_2iter}) and $\vec(\tilde{\bSigma}_{FF}(2)) = \bP_R \vec(\mathcal{R}(\tilde{\bSigma}_{FF}(2)))$ \cite{EstCovMatKron}, under conditions (\ref{cond_0}),(\ref{cond_1}),(\ref{cond_2}) and (\ref{cond_3}), w.h.p.,
\begin{align}
	\nn &\tilde{\bSigma}_{FF}(2) \nn_F \leq \nn \tilde{\bA_1} \nn_F \nn \bB_*\nn_F \nonumber \\
		&\quad + \nn \bA_*\nn_F \nn \tilde{\bB}^0 \nn_F + \nn \tilde{\bA}^1\nn_F \nn \tilde{\bB}^0\nn_F \nonumber \\
		&\leq \tilde{C}_3 (p+2f) \sqrt{\frac{\log M}{n}} + \tilde{C}_4 (f\sqrt{f/p}+\sqrt{pf}) \frac{\log M}{n} \label{tilde_R_FF_2}
\end{align}
where $\tilde{C}_3$ and $\tilde{C}_4$ are constants \cite{TsiligkaridisTSP}.
%\begin{align*}
%	\tilde{C}_3 &:= \max \Big\{ \nn\bB_0\nn_2 \left|\frac{\tr(\bA_0\bA_{init}^{-1})}{p}\right| \tilde{C}_1(1+c_1), \\
%		&\quad 2\sqrt{2\psi_\tau} \kappa(\bA_0) \nn\bA_{init}\nn_2 \bB_0^{max} \nn\bA_{init}^{-1} \bA_0\nn_2  \Big\} \\
%	\tilde{C}_4 &:= 2\sqrt{2\psi_\tau} \bB_0^{max} \nn\bA_{init}^{-1} \bA_0\nn_2 \tilde{C}_1(1+c_1)
%\end{align*}

For
\begin{equation*}
	n \geq (\frac{\tilde{C}_4}{\tilde{C}_3 c_2})^2 \frac{(f\sqrt{f/p}+\sqrt{pf})^2}{(p+2f)^2} \log M
\end{equation*}
then, from (\ref{tilde_R_FF_2}) w.h.p.,
\begin{equation*}
	\nn \tilde{\bSigma}_{FF}(2)\nn_F \leq \tilde{C}_3(1+c_2) (p+2f) \sqrt{\frac{\log M}{n}}
\end{equation*}

The proof for $k=2$ iterations is complete. Using a simple induction, it follows that the rate (\ref{FF_rate_2}) holds for all $k$ finite.


Next, we show that the convergence rate in the precision matrix Frobenius error is on the same order as the covariance matrix error. Let $\bTheta_{FF}(2):=\bSigma_{FF}(2)^{-1}$. From (\ref{tilde_A_1_Frob2}), for
\begin{equation*}
	n> (\epsilon' \nn\bX_*\nn_2 \tilde{C}_1 (1+c_1))^2 (\sqrt{f}+pf^{-1/2})^2 \log M
\end{equation*}
then w.h.p.,
\begin{align}
	\nn \bX_2-\bX_*\nn_F &\leq (1-\frac{1}{\epsilon'})^{-1} \nn\bX_*\nn_2^2 \tilde{C}_1(1+c_1) \nonumber \\
		&\quad \times (\sqrt{f}+pf^{-1/2}) \sqrt{\frac{\log M}{n}} \label{X_2_Frob_error}
\end{align}
Using (\ref{bound_Frob_spec}) and (\ref{X_2_Frob_error}), we have w.h.p.,
\begin{align}
	\nn & \bTheta_{FF}(2)-\bTheta_0 \nn_F \leq \nn \bX_2-\bX_*\nn_F \nn \bY_*\nn_F \nonumber \\
		&\quad + \nn \bY_1-\bY_*\nn_F \nn \bX_*\nn_F + \nn \bX_2-\bX_*\nn_F \nn \bY_1-\bY_*\nn_F \nonumber \\
		&\leq \tilde{D}_1 (2f+p) \sqrt{\frac{\log M}{n}} + \tilde{D}_2 (f\sqrt{\frac{f}{p}}+\sqrt{pf}) \frac{\log M}{n} \label{bound_Theta}
\end{align}
where $\tilde{D}_1$ and $\tilde{D}_2$ are constants.
%\begin{align*}
%	\tilde{D}_1 &= \nn\bTheta_0\nn_2 \left(1-\frac{1}{\epsilon'}\right)^{-1} \max \Big\{\kappa(\bA_0) \nn\bA_{init}^{-1}\nn_2 \tilde{C}_1(1+c_1), \\
%		&\quad  2\sqrt{2\psi_\tau} \nn\bTheta_0\nn_2 \nn\bA_{init}\nn_2 \bB_0^{max} \nn\bA_{init}^{-1} \bA_0\nn_2 \Big\} \\
%	\tilde{D}_2 &= (1-\frac{1}{\epsilon'})^{-2} 2\sqrt{2\psi_\tau} \nn\bTheta_0\nn_2^2 \bB_0^{max} \nn\bA_{init}^{-1} \bA_0\nn_2 \tilde{C}_1 (1+c_1)
%\end{align*}

For
\begin{equation*}
	n>(\frac{\tilde{D}_2}{\tilde{D}_1 d'})^2 (\frac{f\sqrt{f/p}+\sqrt{pf}}{2f+p})^2 \log M
\end{equation*}
the bound (\ref{bound_Theta}) becomes w.h.p.,
\begin{equation*}
	\nn \bTheta_{FF}(2)-\bTheta_0 \nn_F \leq \tilde{D}_1 (1+d') (2f+p) \sqrt{\frac{\log M}{n}}
\end{equation*}
Thus, the same rate $O_P\left( \sqrt{\frac{(p^2+f^2)\log M}{n}} \right)$ holds for the precision matrix Frobenius error.



%Next, consider another FF iteration.
%
%Expanding $\tilde{\bB}^2$:
%\begin{align}
%	\vec(\tilde{\bB}^2) &= \frac{\tr(\bA_0(\bX_1-\bX_*))}{p} \vec(\bB_0) + \vec(\hat{\bB}(\bA_*)-\bB_*) \nonumber\\
%		&\quad + \frac{1}{p} \tilde{\bR}_B \vec(\bX_2-\bX_*) \label{B_2_error2}
%\end{align}
%Using the triangle inequality in (\ref{B_2_error2}) and similar bounds as in (\ref{tilde_A_1_Frob2}), we obtain
%\begin{equation} \label{tilde_B_2_Frob2}
%	\nn \tilde{\bB}^2 \nn_F = O_P\left( (fp^{-1/2}+\sqrt{p}) \sqrt{\frac{\log M}{n}} \right)
%\end{equation}
%
%
%Proceeding similarly as in the bound (\ref{bound_k_2}), we obtain from (\ref{tilde_A_1_Frob2}) and (\ref{tilde_B_2_Frob2}):
%\begin{align}
%	\nn \tilde{\bR}_{FF}(3) \nn_F &\leq \nn \tilde{\bA_1} \nn_F \nn \bB_*\nn_F + \nn \bA_*\nn_F \nn \tilde{\bB}^2 \nn_F \nonumber \\
%		&\quad + \nn \tilde{\bA}^1\nn_F \nn \tilde{\bB}^2\nn_F \nonumber \\
%		&= O_P\left( (p+f)\sqrt{\frac{\log M}{n}} \right) \label{bound_k_3}
%\end{align}
%The proof for $k=3$ iterations is complete. Using a simple induction, it follows that the rate (\ref{FF_rate_2}) holds for all $k$ finite.

\end{IEEEproof}
