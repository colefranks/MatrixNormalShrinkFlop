\documentclass[aos]{imsart}
\pdfoutput=1
\RequirePackage[english]{babel}
\RequirePackage[ascii]{inputenc}
\RequirePackage[T1]{fontenc}
\RequirePackage{microtype,amsthm,amsmath,amsfonts,amssymb,mathtools,braket,bm,xcolor,float}
\RequirePackage[authoryear]{natbib}  % \RequirePackage[numbers]{natbib}
\RequirePackage[bookmarksnumbered,colorlinks,citecolor=blue,urlcolor=blue,linkcolor=blue,anchorcolor=green,breaklinks=true]{hyperref}
\RequirePackage{graphicx}
\RequirePackage[capitalize]{cleveref}
\RequirePackage{silence}
\WarningFilter{latexfont}{Font shape}
\WarningFilter{latexfont}{Size substitutions}

\startlocaldefs
\newtheorem{theorem}{Theorem}[section]
\newtheorem{innercustomthm}{Theorem}
\newenvironment{customthm}[1]{\renewcommand\theinnercustomthm{#1}\innercustomthm}{\endinnercustomthm}
\newtheorem{corollary}[theorem]{Corollary}
\newtheorem{prop}[theorem]{Proposition}
% \newtheorem{obs}[theorem]{Observation}
\newtheorem{lemma}[theorem]{Lemma}
\newtheorem{fact}[theorem]{Fact}
% \newtheorem{claim}[theorem]{Claim}
\theoremstyle{definition}
\newtheorem*{definition}{Definition}
\floatstyle{boxed}\newfloat{Algorithm}{ht}{alg}
\crefname{Algorithm}{Algorithm}{Algorithms}
\numberwithin{equation}{section}
% \allowdisplaybreaks[4]
\urlstyle{same}

\DeclareMathOperator{\Lie}{Lie}
\DeclareMathOperator{\Lin}{L}
\DeclareMathOperator{\ope}{op}
\DeclareMathOperator{\poly}{poly}
\DeclareMathOperator{\op}{op}
\DeclareMathOperator{\vol}{vol}
\DeclareMathOperator{\cut}{cut}
\DeclareMathOperator{\ch}{ch}

\DeclareMathOperator{\Mat}{Mat}
\DeclareMathOperator{\GL}{GL}
\DeclareMathOperator{\tr}{Tr}
\DeclareMathOperator{\rk}{rk}
\DeclareMathOperator{\PD}{PD}
\DeclareMathOperator{\SSPD}{SPD}
\DeclareMathOperator{\vect}{vec}

\DeclarePairedDelimiter{\abs}{\lvert}{\rvert}
\DeclarePairedDelimiter{\norm}{\lVert}{\rVert}
\DeclarePairedDelimiter{\ip}{\langle}{\rangle}

\newcommand{\R}{{\mathbb{R}}}
\renewcommand{\P}{{\mathbb{P}}}
\newcommand{\C}{{\mathbb{C}}}
\renewcommand{\H}{{\mathbb{H}}}
\newcommand{\G}{{\mathbb{G}}}
\newcommand{\Q}{{\mathbb{Q}}}
\newcommand{\N}{{\mathbb{N}}}
\newcommand{\Z}{{\mathbb{Z}}}
\renewcommand{\S}{\mathbb{S}}
\newcommand{\otheta}{\overline{\Theta}}
\newcommand{\htheta}{\widehat{\Theta}}
\newcommand{\oZ}{\overline{Z}}
\newcommand{\ot}{\otimes}
\renewcommand{\vec}{\bm}
\newcommand{\E}{\mathbb{E}}
\newcommand{\eps}{\varepsilon}
\newcommand{\cN}{\mathcal{N}}
\newcommand{\FF}{\mathcal{F}}
\newcommand{\HH}{\mathcal{H}}
\newcommand{\GG}{\mathcal{G}}
\newcommand{\SL}{\operatorname{SL}}
\newcommand{\Herm}{\operatorname{Herm}}
\newcommand{\Sym}{\mathcal{S}}
\newcommand{\smallSym}{S}
\newcommand{\SPD}{\mathbb{P}}
\newcommand{\samp}{x}
\newcommand{\rv}{x}
\newcommand{\ef}{f}
\newcommand{\TT}{\mathcal{T}}
\newcommand{\PP}{\mathcal{P}}
% \newcommand{\RR}{\mathcal{R}}
% \newcommand{\CC}{\mathcal{C}}
\newcommand{\BB}{\mathcal{B}}
\newcommand{\II}{\mathcal{I}}
\newcommand{\MM}{\mathcal{M}}
\newcommand{\AP}{\mathcal{AP}}
\newcommand{\eqdef}{:=}
\newcommand{\maps}{\colon}
\newcommand{\email}[1]{\href{mailto:#1}{\texttt{#1}}}

\def\dmin{d_{\min}}
\def\dmax{d_{\max}}

\newcommand{\CF}[1]{{\color{purple}[CF: #1]}}
\newcommand{\AR}[1]{{\color{orange}[AR: #1]}}
\newcommand{\RMO}[1]{{\color{red}[RMO: #1]}}
\newcommand{\MW}[1]{{\color{red}[MW: #1]}}
\newcommand{\TODO}[1]{{\color{blue}[TODO: #1]}}


\newcommand{\mitn}{\footnotemark[6]}
\newcommand{\nyun}{\footnotemark[7]}

\endlocaldefs

\begin{document}
%=============================================================================
\begin{frontmatter}
\title{Near optimal sample complexity for matrix and tensor normal models via geodesic convexity}
\runtitle{Near optimal sample complexity for matrix and tensor normal models}
%=============================================================================
\begin{aug}
\author[A]{\fnms{Cole} \snm{Franks}\corref{}\ead[label=e1]{franks@mit.edu}},
\author[B]{\fnms{Rafael} \snm{Oliveira}\corref{}\ead[label=e2]{second@somewhere.com}},
\author[B]{\fnms{Akshay} \snm{Ramachandran}\corref{}\ead[label=e3]{third@somewhere.com}} \\ \and
\author[C]{\fnms{Michael} \snm{Walter}\corref{}\ead[label=e4]{m.walter@uva.nl}}
\runauthor{C.\ Franks, R.\ Oliveira, A.\ Ramachandran \and M.\ Walter}
\affiliation[A]{Massachusetts Institute of Technology} %, \printead{e1}
\affiliation[B]{University of Waterloo} %, \printead{e2,e3}}
\affiliation[C]{University of Amsterdam} %, \printead{e4}}
\end{aug}
%=============================================================================
\begin{abstract}
The matrix normal model, the family of Gaussian matrix-variate distributions whose covariance matrix is the Kronecker product of two lower dimensional factors, is frequently used to model matrix-variate data. The tensor normal model generalizes this family to Kronecker products of three or more factors. We study the estimation of the Kronecker factors of the covariance matrix in the matrix and tensor models. We show nonasymptotic bounds for the maximum likelihood estimator (MLE) for the factors in several natural metrics. In contrast to existing bounds, our results do not depend on the factors being well-conditioned. For the matrix normal model, all our bounds are minimax optimal up to logarithmic factors, and for the tensor normal model our bound for the largest factor and overall covariance matrix are minimax optimal provided there are enough samples for any estimator to obtain better than constant Frobenius error. In the same regimes as our sample complexity bounds, we show that an iterative procedure to compute the MLE known as the flip-flop algorithm converges linearly with high probability. Our main tool is geodesic convexity in the Fisher-Rao metric on the positive definite matrices. We also provide numerical evidence that a simple regularizer can improve performance in the undersampled regime.



\end{abstract}
%=============================================================================
\begin{keyword}[class=MSC2020]
\kwd[Primary ]{???}
\kwd{???}
\kwd[; secondary ]{???}
\end{keyword}

\begin{keyword}
\kwd{???}
\kwd{???}
\end{keyword}
\end{frontmatter}
%=============================================================================
%%%%%%%%%%%%%%%%%%%%%%%%%%%%%%%%%%%%%%%%%%%%%%
%% Please use \tableofcontents for articles %%
%% with 50 pages and more                   %%
%%%%%%%%%%%%%%%%%%%%%%%%%%%%%%%%%%%%%%%%%%%%%%
\tableofcontents
%=============================================================================

\TODO{
\begin{enumerate}
\item Check that the claims on information-theoretic tightness claimed after the bulleted list in the intro are true. 
\item Check that the transformations which preserve $D_F, D_{op}, d$, also preserve $KL, d_{TV}$, etc. 
\item Find citation for Fisher-Rao metric.
\item Fix $O$ in Pisier.
%\item Use $n$ instead of $T$ for number of samples.
%\item Fix the dimension issue in \cref{thm:tensor-convexity}.
%\item At some point we need to mention that the final step in flip-flop is to supply the normalizing constant.
%\item When not random, use lower case $x$.
%\item Comment that the geodesic distance is some variation on the Fischer-Rao distance.
\end{enumerate}
Polishing:
\begin{enumerate}
\item Akshay's note on simplifying the proof of \cref{thm:tensor-convexity}: \AR{Can rephrase explicitly in terms of $\nabla^{2} F$ and rank-one term. Then \cref{prop:gradient-bound} gives $1-\eps$ lower bound on each diagonal block of $\nabla^{2} F$, offdiagPisier gives $\lambda$ bound on each off-diagonal block of $\nabla^{2} F$, and \cref{prop:gradient-bound} once again gives $k \eps^{2}$ bound on whole rank-one term.  }
\item \AR{Can improve the strong convexity stuff by a factor of two by using the projection to traceless on both sides of the inner product}
\item \CF{for the readers' sake some justification is needed for why perturbing $\samp$ this is the same as considering the Hessian of our function at another point in $\SPD$. We'll probably have to discuss this earlier in the paper when we mention all the different perspectives for scaling, but for now a little reminder would help.}
\item Add some discussion for noise added to data.
\item Before \cref{lem:expansion-convexity}, add explanation of what the marginals of $\Phi^{ab}$ are in terms of $\rho^{a}$ etc.
\end{enumerate}
}


%=============================================================================
\section{Introduction}
%=============================================================================
Covariance matrix estimation is an important task in statistics, machine learning, and the empirical sciences.
We consider covariance estimation for matrix-variate and tensor-variate Gaussian data, that is, when individual data points are matrices or tensors. Matrix-variate data arises naturally in numerous applications like gene microarrays, spatio-temporal data, and brain imaging.
A significant challenge is that the dimensionality of these problems is frequently much higher than the number of samples, making estimation information-theoretically impossible without structural assumptions.

To remedy this issue, matrix-variate data is commonly assumed to follow the \emph{matrix normal distribution} \citep{dutilleul1999mle,werner2008estimation}.
Here the matrix follows a multivariate Gaussian distribution and the covariance between any two entries in the matrix is a product of an inter-row factor and an inter-column factor.
In spatio-temporal statistics this is referred to as a separable covariance structure.
Formally, if a matrix normal random variable~$X$ takes values in the~$d_1\times d_2$ matrices, then its covariance matrix $\Sigma$ is a $d_1d_2\times d_1 d_2$ matrix that is the Kronecker product~$\Sigma_1 \ot \Sigma_2$ of two positive-semidefinite matrices~$\Sigma_1$ and~$\Sigma_2$ of dimension~$d_1\times d_1$ and~$d_2\times d_2$, respectively.
This naturally extends to the \emph{tensor normal model}, where $X$ is a $k$-dimensional array, with covariance matrix equal to the Kronecker product of $k$ many positive semidefinite matrices~$\Sigma_1, \dots, \Sigma_k$.
In this paper we consider the estimation of $\Sigma_1, \dots, \Sigma_k$ from $n$ samples of a matrix or tensor normal random variable $X$.

Much research has been devoted to estimating the covariance matrix for the matrix and tensor normal models, but gaps in rigorous understanding remain.
\cite{dutilleul1999mle} and later \cite{werner2008estimation} proposed an iterative algorithm, known as the \emph{flip-flop algorithm}, to compute the maximum likelihood estimator (MLE).
In the latter work, the authors also showed that the MLE is consistent and asymptotically normal, and showed the same for the estimator obtained by terminating the flip-flop after three steps.
Here we will be interested in non-asymptotic rates.
Standard estimation of the covariance matrix $\Sigma$ by the sample covariance matrix yields a mean-squared Frobenius norm error of $(d_1 d_2)^2/n$ assuming $n \geq C d_1 d_2$.
The matrix normal model, however, has $\Theta(d_1^2 + d_2^2)$ parameters so it should be possible to do much better.
Assuming that the covariance factors have constant condition number and that $n$ is at least $\tilde{\Omega}(\max\{d_1,d_2\})$, \cite{tsiligkaridis2013convergence} showed that a three-step flip-flop estimator has mean-squared Frobenius error of $O((d_1^2 + d_2^2)/n)$ for the full matrix $\Sigma$; they did not state a bound for the individual factors $\Sigma_1,\Sigma_2$.
The same authors showed tighter rates which hold even for~$n\ll d_i$ for a penalized estimator under the additional assumption that the precision matrices $\Sigma_i^{-1}$ are sparse.
In the extremely undersampled regime, \cite{zhou2014gemini} demonstrated a single-step penalized estimator that converges even for a single matrix $(n=1)$ when the precision matrices have constant condition number, are highly sparse, and have bounded $\ell_1$ norm off the diagonal.
Simply setting $\Sigma_2 = I_{d_2}$ or $\Sigma_1 = I_{d_1}$, in which case the matrix normal model reduces to standard covariance estimation with $d_1 n$ (resp. $d_2 n$) samples, shows the necessity of additional assumptions like sparsity or well-conditionedness if $n < \max\{d_1/d_2, d_2/d_1\}$.
\cite{allen2010transposable} also considered penalized estimators for the purpose of missing data imputation.
For the tensor normal model, a natural generalization of the flip-flop algorithm has been proposed to compute the MLE \citep{mardia1993spatial,manceur2013maximum}, but its convergence was not proven.
Assuming bounded constant condition number of the true covariances and knowledge of initializers within constant Frobenius distance of the true precision matrices, \cite{sun2015nonconvex} propose an estimator with tight rates.
In both the matrix and tensor case, no estimator for the Kronecker factors has been proven to have tight rates without additional assumptions on the factors' structure.

Even characterizing the existence of the MLE for the matrix and tensor normal model has remained elusive until recently.
\cite{amendola2020invariant} recently noted that the matrix normal and tensor MLEs are equivalent to algebraic problems about a group action called the \emph{left-right action} and the \emph{tensor action}, respectively.
In the computer science literature these two problems are called \emph{operator} and \emph{tensor scaling}, respectively.
Independently from \cite{amendola2020invariant}, it was pointed out by \cite{FM20} that the Tyler's M estimator for elliptical distributions (which is the MLE for the matrix normal model under the additional promise that~$\Sigma_2$ is diagonal) is a special case of operator scaling.
Using the connection to the left-right action, exact sample size thresholds for the existence of the MLE were recently determined in \cite{derksen2020matrix} for the matrix normal model and subsequently for the tensor normal model in \cite{derksen2020tensor}.
In the context of operator scaling, \cite{gurvits2004classical} showed much earlier that the flip-flop algorithm converges to the matrix normal MLE whenever it exists.
Recently it was shown that the number of flip-flop steps to obtain a gradient of magnitude $\eps$ in the log-likelihood function for the tensor and matrix normal model is polynomial in the input size and~$1/\eps$ \citep{GGOW19,burgisser2017alternating,burgisser2019towards}.

%In the context of tensor scaling, it was shown earlier that the flip-flop algorithm converges to the tensor MLE whenever it exists \CF{cite tensor scaling}.

%In the current literature on matrix normal and tensor models, typically the estimators are assessed using Frobenius or spectral norm between the estimated parameter and the truth. However, neither of these metrics bound statistical distances of interest such as the total variation or Kullback-Leibler divergence between the true distribution and that corresponding to the estimated parameter, or the Fischer-Rao distance. These quantities are affinely invariant, meaning that for any invertible matrix $g$ we have $d(\Sigma, \Sigma') = d(g \Sigma g^T, g \Sigma' g^T)$. \CF{not exactly sure how to write this, but I want it to say that we get the right rates with no assumptions and we use more appropriate metrics. Amusingly, to get from these metrics TO the less useful metrics or vice versa, one needs the condition number assumptions. Also, it appears that the reason that the better metrics aren't being used is that the estimators didn't have the equivariance property.}

%-----------------------------------------------------------------------------
\subsection{Our contributions}
%-----------------------------------------------------------------------------
We take a geodesically convex optimization approach to provide rigorous nonasymptotic guarantees for the estimation of the precision matrices, without any assumptions on their structure. For the matrix normal model we provide high probability bounds on the estimator that are tight up to logarithmic factors. For the tensor normal model, our bounds are tight in the regime where it is information-theoretically possible to recover the factors to constant Frobenius error with high probabilty.

%our rates are tight in every regime up to logarithmic factors, and for the tensor normal model our rates our tight assuming it is information-theoretically possible to recover the factors up to constant Frobenius error with high probability.

In the current literature on matrix normal and tensor models, typically the estimators are assessed using Frobenius or spectral norm between the estimated parameter and the truth.
However, neither of these metrics bound statistical dissimilarity measures of interest such as the total variation or Kullback-Leibler divergence between the true distribution and that corresponding to the estimated parameter, or the Fisher-Rao distance.

Here we consider the \emph{relative Frobenius error} $D_F(A \Vert B) = \norm{I - B^{-1/2} A B^{-1/2}}_F$ of the precision matrices. It is natural to consider the relative Frobenius error rather because it is scale invariant, whereas the the Frobenius error is not. Moreover, the dissimilarity measures ~$D_F(\Theta_1, \Theta_2)$, total variation distance $d_{TV}(\mathcal{N}(0, \Theta_1^{-1}), \mathcal{N}(0, \Theta_2^{-1}))$, the square of the KL-divergence $d_{KL}(\mathcal{N}(0, \Theta_1^{-1}), \mathcal{N}(0, \Theta_2^{-1}))^2$, and the Fisher-Rao distance between $\Theta_1, \Theta_2$ all coincide ``locally.'' That is, if any of them is at most a small constant then they are all on the same order. Signicant work has been done on the estimation of precision and covariance matrices under such dissimilarity measures, e.g. the work on Ledoit and Wolf on Stein's loss (which is proportional to $d_{KL}$ for Gaussians) \cite{ledoit2018optimal}. To obtain the sharpest possible results, we also consider the \emph{relative spectral error} $D_{\op}(A \Vert B) = \norm{I - B^{-1/2} A B^{-1/2}}_{\op}$, which has been studied in the context of spectral graph sparsification.
%The relative spectral error converges to zero in the regimes where the relative Frobenius error does not, and implies the Frobenius error bounds by the relation $D_{F} \leq \sqrt{d} D_{op}$.
The dissimilarity measure $d_F(A||B)$ (resp. $d_{op}(A||B)$) can be related to the usual norm $\|A - B\|_F$, (resp. $\|A - B\|_{op}$) by a constant factor assuming the largest singular vlaues of $B, B^{-1}$ are bounded by a constant. Though we caution that $D_F$ and $D_{\op}$ are not truly metrics, we will call them distances because they approximately (or ``locally'') obey symmetry and the triangle inequality. See \cref{sec:rel-error} for a discussion of these properties.

Informally, our theoretical contributions are as follows:
\begin{enumerate}
\item Consider the matrix normal model for $d_1 \leq d_2$. We show that for some $n_0 =\widetilde{O}( d_2/d_1)$, if $n \geq n_0$ then the MLE for the precision matrices $\Theta_1, \Theta_2$ has error $\widetilde{O}(\sqrt{{d_1}/{nd_2}})$ for $\Theta_1$ and $O(\eps \sqrt{{d_2}/{nd_a}})$ for $\Theta_2$ in $D_{\op}$ with probability $1 - O(e^{ - \Omega ( d_1)})$.
%As a corollary the MSE in $D_F$ is $O( \max \{ \frac{d_1^2}{nd_2}, \frac{d_2^2}{nd_1} \} )$.
% \item For the matrix normal model, if $n \geq C (\max\{d_1^2,d_2^2\}/d_1 d_2) \log^2 d_1$, then the MLE for the precision matrices $\Theta_1, \Theta_2$ has MSE $O( \max\{ d_1^2, d_2^2 \}/ (n d_1 d_2) )$ in $D_{\op}$.
% As a corollary the MSE in $D_F$ is $O( \max \{ d_1^3, d_2^3 \} / (nd_1d_2) )$.
\item In the tensor normal model, for $k$ fixed we show that for some $n_0 = O( \max\{d_i^3\}/ \prod_i d_i)$, if $n \geq n_0$ then the MLE for the precision matrix $\Theta$ has error $O( \frac{d_{\max}}{\sqrt{n}} )$ in $D_F$ with probability $1 - (n D /\max\{d_i\})^{-\Omega(\min\{d_i\})}$. We also give bounds for growing $k$ and a tight bound for the error of the largest Kronecker factor $\Theta_i$.
\item Under the same sample requirements as above in each case, the flip-flop algorithm converges exponentially quickly to the MLE with high probability.
As a corollary, there is an algorithm to compute the MLE up to precision $\eps$ in $D_F$ or $D_{op}$ with expected runtime polynomial in the input size and~$\log\frac1\eps$.
\end{enumerate}
%\CF{Mention $d_{TV}$ again.}
%\MW{Check that what is written above is the same as the results advertised in \cref{sec:main results}. Right now it isn't.}

We now discuss the tightness of our results. Our first result is tight up to logarithmic factors in the sense that it is information-theoretically impossible to obtain an error bound that is smaller by a polylogarithmic factor and holds with constant probability. For $n$ a polylogarithmic factor smaller than $n_0$, it is information-theoretically impossible to obtain any finite bound independent of $\Theta$ with high probability. Similarly, for the second result, provided $n \geq n_0$ it is impossible to obtain an error bound that is smaller than ours by a factor tending to infinity that holds with constant probability. For $n \ll n_0$, no constant error bound on the $D_F$ error of the largest Kronecker factor can hold with constant probability. These results follow by reduction to known results on the Frobenius and operator error for covariance estimation; see \cref{sec:lower}.

For interesting cases of the tensor normal model such as $d\times d \times d$ tensors we just require that $n$ is at least a large constant.
For the matrix normal model, our first result removes the added constraint $n \geq C \max\{d_1,d_2\}$ in \cite{tsiligkaridis2013convergence}.
We leave extending the $D_{\op}$ bounds for the matrix normal model to the tensor normal model as an open problem.

To handle the undersampled case, we also introduce a regularized estimator that is much simpler to compute than the penalized regularizers introduced in \cite{tsiligkaridis2013convergence,sun2015nonconvex,zhou2014gemini}, and empirically has comparable to and sometimes better performance than existing regularizers.
Our regularizer has a Bayesian interpretation as coming from a Wishart prior for the covariance, and is closer in spirit to the shrinkage estimators considered in \cite{goes2020robust}.
\CF{Expand up on this; put it in the bulleted list also?}
\MW{Good idea.}

\CF{Then add some discussion of methods.}

%-----------------------------------------------------------------------------
\subsection{Outline}
%-----------------------------------------------------------------------------
\TODO{In Section xxx, we\dots}

%-----------------------------------------------------------------------------
\subsection{Notation}
%-----------------------------------------------------------------------------
We write $\Mat(d)$ for the space of $d\times d$ matrices, $\PD(d)$ for the convex cone of $d\times d$ positive definite matrices; $\GL(d)$ denotes the group of invertible $d\times d$ matrices.
For a matrix $A$, $\norm{A}_{\op}$ denotes the operator norm, $\norm{A}_F = (\tr A^T A)^{\frac12}$ the Frobenius norm, and $\braket{A,B} = \tr A^T B$ for the Hilbert-Schmidt inner product.
We extend these definitions to tuples $A=(A_0;A_1,\dots,A_k)$, where~$A_0\in\R$ and the $A_a$ for $a\in[k]$ are matrices and denote them by the same symbol, i.e., $\norm{A}_F = (\abs{A_0}^2 + \sum_{a=1}^k \norm{A_a}_F^2)^{1/2}$ and similarly for the inner product. For functions $f,g:S \to \R$ for any set $S$, we say $f = O(g)$ if there is a constant $C > 0$ independent of $x$ such that $f(x) \leq C g(x)$ for all $x \in S$, and similarly $f = \Omega(g)$ if there is a constant $c > 0$ independent of $x$ such that $f(x) \geq c g(x)$ for all $x \in S$. If $f = O(g)$ and $g = O(f)$ we write $f \asymp g$.
%If the constant $C, c$ depends on another parameter $k$ we write $O_k, \Omega_k$, respectively.
%\MW{Define $O, \Theta$ notation and say that for us this is always about universal constants.}

%=============================================================================
\section{Model and main results}\label{sec:main results}
%=============================================================================
In this section we define the matrix and tensor normal models and we state our main technical results.
\MW{I need to re-read this section once our definite results are in.}

%-----------------------------------------------------------------------------
\subsection{Matrix and tensor normal model}\label{subsec:model}
%-----------------------------------------------------------------------------
The tensor normal model, of which the matrix normal model is a particular case, is formally defined as follows.

\begin{definition}
For positive definite matrices $\Sigma_1,\dots,\Sigma_k$, we define the \emph{tensor normal model} as the centered multivariate Gaussian distribution with covariance matrix given by the Kronecker product $\Sigma = \Sigma_1 \ot \dots \ot \Sigma_k$.
For $k=2$, this is known as the \emph{matrix normal model}.
\end{definition}

\noindent
Note that if each $\Sigma_a$ is a $d_a\times d_a$ matrix then $\Sigma$ is a $D\times D$-matrix, where $D=d_1 \cdots d_k$.
Our goal is to estimate the $k$ Kronecker factors $\Sigma_1, \dots, \Sigma_k$ given access to $n$ i.i.d.\ random samples $x_1, \dots, x_n \in \R^D$ drawn from the model.

One may also think of each random sample $x_j$ as taking values in the set of $d_1 \times \dots \times d_k$ arrays of real numbers.
There are $k$ natural ways to ``flatten" $x_j$ to a matrix:
for example, we may think of it as a $d_1 \times d_2d_3\cdots{}d_k$ matrix whose column indexed by $(i_2,\dots, i_k)$ is the vector in $\R^{d_1}$ with $i_1^{\text{th}}$ entry equal to $(x_j)_{i_1, \dots, i_k}$.
In an analogous way we may flatten it to a $d_2 \times d_1d_3\cdots{}d_k$ matrix, and so on.
In the tensor normal model, the $d_2d_3\cdots{}d_k$ many columns are each distributed as a Gaussian random vector with covariance proportional to~$\Sigma_1$.
Similarly the columns of the $d_2 \times d_1d_3\cdots{}d_k$ flattening have covariance proportional to~$\Sigma_2$, and so on.
As such, the columns of the $a^{\text{th}}$ flattening can be used to estimate~$\Sigma_a$ up to a scalar.
However, doing so na\"ively (e.g.\ using the sample covariance matrix of the columns) can result in an estimator with very high variance.
This is because the columns of the flattenings are not independent.
In fact they may be so highly correlated that they effectively constitute only one random sample rather than $d_2\dots d_k$ many.
The MLE decorrelates the columns to obtain rates like those one would obtain if the columns were independent.

The MLE is easier to describe in terms of the precision matrices, the inverses of the covariance matrices.
Let~$\Theta$ denote the \emph{precision matrix}, i.e., $\Theta = \bigotimes_{a=1}^k \Theta_a$, where $\Theta_a = \Sigma_a^{-1}$.
Let~$\P$ denote the manifold of all such $\Theta$, i.e.
% As above, we can fix this by working with tuples of precision matrices with equal determinant:
\begin{align*}
  \P &= \{ \Theta_1 \ot \dots \ot \Theta_k \in \PD(d_1) \times \dots \times \PD(d_k) \}.
 \end{align*}
Given a tuple $x$ of samples $\samp_1,\dots,\samp_n\in\R^D$, the following function is proportional to the negative log-likelihood: % $\ell(\Theta|x) = \frac{n}2 \log \det \Theta - \frac12 \sum_{i=1}^n x_i^T \Theta x_i$, which we can rewrite as
\begin{align*}
  \ef_\samp(\Theta)
=  \frac{1}{nD}\sum_{i = 1}^n \samp_i^T \Theta \samp_i -  \frac{1}{D}\log\det\Theta.
\end{align*}
Though $\Theta_a$ are not identifiable, the above expression is nonetheless well-defined.
The \emph{maximum likelihood estimator (MLE)} for $\Theta$ is then
\begin{align}\label{eq:mle}
  \widehat{\Theta} := \underset{\Theta \in \P}{ \arg\min} f_x(\Theta)
\end{align}
whenever the minimizer exists and is unique.
We write $\widehat\Theta = \widehat\Theta(x)$ when we want to emphasize the dependence of the MLE on the samples~$x$, and we say $(\htheta_1, \dots, \htheta_k)$ is \emph{an} MLE for~$(\Theta_1, \dots, \Theta_k)$ if $\otimes_{a = 1}^k \htheta_a = \htheta$.
Note that $\P$ is not a convex domain under the Euclidean geometry on the $D\times D$ matrices.

%-----------------------------------------------------------------------------
\subsection{Results on the MLE}
%-----------------------------------------------------------------------------
We may now state our result for the tensor normal models precisely.
As mentioned in the introduction, we use the following natural distance measures.

\begin{definition}
For positive definite matrix $A, B$, define their \emph{relative Frobenius error} (or \emph{Mahalanobis distance}) as
\begin{align*}
  D_F(A \Vert B) = \norm{I - B^{-1/2} A B^{-1/2}}_F.
\end{align*}
Similarly, define the \emph{relative spectral error} as
\begin{align*}
  D_{\op}(A \Vert B) = \norm{I - B^{-1/2} A B^{-1/2}}_{\op}.
\end{align*}
\end{definition}

To state our results, and throughout this paper, we write $d_{\min} = \min_a d_a$, $d_{\max} = \max_a d_a$.
Recall also that $D = \prod_{i=1}^k d_a$.
\newcommand{\TensorFrob}[2]{%
There is a universal constant~$C>0$ such that the following holds.
Suppose
\begin{#1}#2
  1 \leq \eps^2 \leq \frac{nD}{C k^2 d_{\max}^2 \max\{k, d_{\max}\}}.
\end{#1}
Then the MLE~$\htheta = \htheta_1 \ot \cdots \ot \htheta_k$ for $n$ independent samples of the tensor normal model with precision matrix~$\Theta = \Theta_1 \ot \cdots \ot \Theta_k$ satisfies
\begin{align*}
  D_F(\htheta_a\Vert\Theta_a) &= O\left(k^{1/2} \frac{\sqrt{d_a} \, d_{\max}}{\sqrt{n D}} \eps\right) \quad\text{ for all } a\in[k], \\
  D_F(\htheta\Vert\Theta) &= O\left(k^{3/2} \frac{d_{\max}}{\sqrt{n}} \eps\right),
\end{align*}
where the Kronecker factors of $\htheta$ and $\Theta$ are chosen such that $\det\htheta_1 = \cdots = \det\htheta_k$ and $\det\Theta_1 = \cdots = \det\Theta_k$, with probability at least
\begin{align*}
  1 - k e^{-\Omega\bigl( \eps^2 d_{\max} \bigr)} - k^2 \left( \frac{\sqrt{nD}}{k d_{\max}} \right)^{-\Omega(d_{\min})},
\end{align*}}

\begin{theorem}[Tensor normal Frobenius error]\label{thm:tensor-frobenius}
\TensorFrob{equation}{\label{eq:eps sqr assm}}
\end{theorem}

The error for the precision matrix~$\Theta_a$ with $d_a = d_{\max}$ matches that of the MLE for the precision matrix of a single Gaussian with $D/d_{\max}$ samples, which is the special case when all the other Kronecker factors are the identity. 

For the matrix normal model $(k=2)$, we obtain a stronger result. In the following theorem we identify $\Theta_1, \Theta_2$ from $\Theta$ using the convention $\det \Theta_1 = 1$, and define the MLE's $\htheta_1, \htheta_2$ to be the minimizers of $f$ restricted to the subset $\{P \in \PD(d_1): \det P = 1\} \times \PD(d_2)$. 

\newcommand{\MatrixSpec}{%
There is a universal constant~$C>0$ with the following property.
Suppose $d_1 \leq d_2$ and $n \geq C \frac{d_2}{d_1} \max \{\log \frac{d_2}{d_1},  \frac{\log^2 d_1}{\eps^2}\}$. Then the MLE $\htheta = \htheta_1 \ot \htheta_2$ for $n$ independent samples from the matrix normal model with precision matrix $\Theta = \Theta_1 \ot \Theta_2$ satisfies
\begin{align*}
  D_{\op}(\widehat{\Theta}_1 \Vert \Theta_1) = O\left(\eps \sqrt{\frac{d_1}{nd_2}} \log d_1\right)
\quad\text{and}\quad
D_{\op}(\widehat{\Theta}_2 \Vert \Theta_2) = O\left(\eps \sqrt{\frac{d_2}{nd_1}}\right),
\end{align*}
with probability at least  $1 - O(e^{ - \Omega( d_1 \eps^2)})$. }

%$\det\htheta_1 = \det\htheta_2$ and $\det\Theta_1 = \det\Theta_2$, with probability at least  $1 - O(e^{ - \Omega( d_1 \eps^2)})$.}
\CF{I think the following needs to have an upper bound on $\eps$ as well.}
\begin{theorem}[Matrix normal spectral error]\label{thm:matrix-normal}
\MatrixSpec
\end{theorem}

%-----------------------------------------------------------------------------
\subsection{Flip-flop algorithm}
%-----------------------------------------------------------------------------
The MLE can be computed by a natural iterative procedure known as the \emph{flip-flop algorithm} \citep{dutilleul1999mle,gurvits2004classical}.

For simplicity, we describe it for the matrix normal model ($k=2$), so that the samples $\samp_i$ can be viewed as $d_1\times d_2$ matrices which we denote by $X_i$.
Initialize $\overline{\Theta}_1 = I_{d_1}$, $\overline{\Theta}_2 = I_{d_2}$, and choose a distance measure~$d$ and a tolerance $\eps > 0$.
\begin{enumerate}
\item Set $\overline{\Theta}_1 \leftarrow (\frac{1}{n d_2} \sum_{i = 1}^n X_i \overline{\Theta}_2 X_i^T)^{-1}.$
\item\label{it:sinkhorn second} Set $\Upsilon = \frac{1}{n d_1} \sum_{i = 1}^n X_i^T \overline{\Theta}_1 X_i$.
If $d_F( \Upsilon^{-1}|| \overline{\Theta}_2) > \eps$, set $\overline{\Theta}_2 \leftarrow \Upsilon^{-1}$ and return to Step 1.
\item Output $\overline{\Theta}_1, \overline{\Theta}_2$.
\end{enumerate}

We can motivate this procedure by noting that if in the first step we already $\overline{\Theta}_2 = \Theta_2$, then $\frac{1}{n d_2} \sum_{i = 1}^n X_i \overline{\Theta}_2 X_i^T$ is simply a sum of outer products of $nd_2$ many independent random vectors with covariance $\Sigma_1 = \Theta_1^{-1}$; as such the inverse is a good estimator for $\Theta_1$.
As we don't know $\Theta_2$, the flip-flop algorithm instead uses $\overline{\Theta}_2$ our current best guess.

For the general tensor normal model, in each step the flip flop algorithm chooses one of the dimensions $a \in [k]$ and uses the $a^\text{th}$ flattening of the samples~$x_i$ (which are just $X_i$ and $X_i^T$ in the matrix case) to update $\overline{\Theta}_a$.

%-----------------------------------------------------------------------------
\subsection{Results on the flip-flop algorithm}
%-----------------------------------------------------------------------------
Our next results show that the flip-flop algorithm can efficiently find the MLEs with high probability.
We first state our result for the general tensor normal model and then give an improved version for the matrix normal model.



\newcommand{\TensorFlop}{
	If $\htheta$ denotes the MLE estimator for $\Theta$, then provided $n = \Omega(k^2 \cdot \dmax^3/D)$, the flip-flop algorithm computes $\otheta$ with
	$$ D_F(\htheta_a \ || \ \otheta_a) \leq \eps $$
	in $O(k \dmax \log(\dmax/\eps))$ iterations with probability at least
	$$ 1 - k^2 \cdot \left( \dfrac{\sqrt{nD}}{k \dmax} \right)^{-\Omega(\dmin)} - 2k \cdot e^{- \Omega(nD/k \dmax^2)}.$$}

\begin{theorem}\label{thm:tensor-flipflop}
\TensorFlop
\end{theorem}

\newcommand{\MatrixFlop}{
Let $(\widehat{\Theta}_1,\widehat{\Theta}_2)$ denote the MLE for $(\Theta_1,\Theta_2)$. There exists a universal constant $\Gamma > 0$ such that when given 
$$n \geq \Gamma \cdot \dfrac{\dmax}{\dmin} \cdot \max\left\{ \log\left( \dfrac{\dmax}{\dmin} \right), \dfrac{\log^2 \dmin}{\varepsilon^2} \right\}$$ 
samples in the matrix normal model, the flip-flop algorithm computes $(\overline{\Theta}_1,\overline{\Theta}_2)$ with
\begin{align*}
  D_{op}(\overline{\Theta}_a, \widehat{\Theta}_a) \leq \eps
\end{align*}
for $a\in\{1,2\}$ in $O\left(\dmax \log(\dmax/\varepsilon) \right)$ iterations, with probability at least $1 - e^{- \Omega(\dmin \varepsilon^2)}$.
}

\begin{theorem}[Matrix flip-flop]\label{thm:matrix-flipflop}
\MatrixFlop\end{theorem}

One may wonder why in the above theorems we consider the distances for the individual tensor factors and not the covariance matrix itself, but tight bounds for the covariance matrix itself follow from the above bounds (apart from the logarithmic factor in the matrix normal case and a constant factor in general).


%=============================================================================
\section{Sample complexity for the tensor normal model}\label{sec:tensor-normal}
%=============================================================================
It was observed by \cite{wiesel2012geodesic} that the negative log-likelihood exhibits a certain variant of convexity known as \emph{geodesic convexity}.
In this section, we use geodesic convexity, following a strategy similar to \cite{FM20}, to prove \cref{thm:tensor-frobenius}.
Our improved result for the matrix normal model, \cref{thm:matrix-normal}, requires additional tools and will
be proved later in \cref{sec:matrix-normal}.

%-----------------------------------------------------------------------------
\subsection{Geometry and geodesic convexity}\label{subsec:geom}
%-----------------------------------------------------------------------------
We now discuss the geodesic convexity used here and outline the strategy for our proof.
We start by introducing a Riemannian metric on the manifold $\PD(D)$ of positive-definite $D\times D$ matrices.
Rather than simply considering the metric induced by the Euclidean metric on the symmetric matrices, we consider the metric whose geodesics starting at a point $\Theta \in \PD(D)$ are of the form $t \mapsto \Theta^{1/2} e^{Ht} \Theta^{1/2}$ for~$t \in \R$ and a symmetric matrix~$H$. % with $\norm H_F=D$.
% Accordingly, the geodesic distance between $\Theta, \Theta'$ is given by $\norm{\log(\Theta^{-1/2} \Theta' \Theta^{-1/2})}_F$.
This metric arises from the Hessian of the log-determinant \citep{bhatia2009positive} and also as the Fisher-Rao metric on centered Gaussians parametrized by their precision matrices \CF{cite}.
If $\Theta$ is positive definite and $A$ an invertible matrix then $A\Theta A^T$ is again in positive definite.
The transformation $\Theta \mapsto A\Theta A^T$ is an isometry, i.e., it preserves the geodesic distance.
Importantly, the statistical distances we use are also \emph{invariant} under such transformations:
\begin{align*}
  D_F(A \Theta A^T \Vert A \Theta' A^T) = D_F(\Theta \Vert \Theta')
\end{align*}
and likewise for the distance~$D_{\op}$.
This invariance is natural because changing a pair of precision matrices in this way does not change the statistical relationship between the corresponding Gaussians; in particular the total variation distance, Fisher-Rao, and Kullback-Leibler divergence are unchanged \CF{double check}.

As observed by \cite{wiesel2012geodesic}, the negative log-likelihood is convex as the precision matrix moves along these geodesics, and in particular for the tensor normal model it is convex along geodesics in $\P = \{ \Theta_1 \ot \dots \ot \Theta_k \in \PD(d_1) \times \dots \times \PD(d_k) \}$. This is because the geodesics in $\PD(D)$ between elements of the manifold $\P = \{ \Theta_1 \ot \dots \ot \Theta_k \in \PD(d_1) \times \dots \times \PD(d_k) \}$ remain in $\P$. That is, $\P$ is a \emph{totally geodesic submanifold} of $\PD(D)$.  The tangent space of $\P$ can be identified with the real vector space
\begin{align*}
  \H &= \{ (H_0, H_1,\dots,H_k) \;:\; H_0 \in \R \text{ and }H_a \text{ a symmetric traceless $d_a \times d_a$ matrix} \, \forall a \in [k]  \}.
\end{align*}
The direction $(1, 0, \dots, 0)$ changes $\Theta$ by an overall scalar, and tangent directions supported only in the $a^{th}$ component for $a \in [k]$ only change~$\Theta_a$, subject to its determinant staying fixed. We now define the geodesics precisely.

\begin{definition}[Geodesics and balls]
Let $P\in\P$.
The \emph{exponential map} $\exp_\Theta \colon \H \to \P$ at~$\Theta$ is defined by
%\begin{align*}\exp_\Theta(H) &=   e^{\frac{H_0}{k \sqrt{D}}}\cdot ( \Theta_1^{1/2} e^{\sqrt{\frac{d_1}{D}} H_1} \Theta_1^{1/2}) \ot  \dots \ot (\Theta_k^{1/2} e^{\sqrt{\frac{d_k}{D}} H_k} \Theta_k^{1/2}).\end{align*}
\begin{align*}
  \exp_\Theta(H) &= e^{H_0} \cdot ( \Theta_1^{1/2} e^{\sqrt{d_1} H_1} \Theta_1^{1/2}) \ot \cdots \ot (\Theta_k^{1/2} e^{\sqrt{d_k} H_k} \Theta_k^{1/2}).
\end{align*}
The \emph{geodesics} through $\Theta$ are the curves $t \mapsto \exp_\Theta(t H)$ for $t\in\R$ and $H\in\H$.
Up to reparameterization, there is a unique geodesic between any two points of~$\P$.
We take the convention that the geodesics have unit speed if $\norm{H}_F^2 = \abs{H_0}^2 + \sum_{a=1}^k \norm{H_a}_F^2 = 1$.
Then, the geodesic distance $d(\Theta,\Theta')$ between two points $\Theta$ and $\Theta'=\exp_\Theta(H)$ is equal to~$\norm H_F$, and the closed \emph{(geodesic) ball} of radius~$r>0$ about~$\Theta$ is given by
\begin{align*}
  B_r(\Theta) = \{ \exp_\Theta(H) : \norm H_F \leq r \},
\end{align*}
Such balls are \emph{geodesically convex} subsets of~$\P$, that is, if $\gamma(t)$ is a geodesic such that~$\gamma(0),\gamma(1) \in B_r(\Theta)$ then $\gamma(t) \in B_r(\Theta)$ for all $r\in[0,1]$.
% The geodesic distance squared between two points $P,Q\in\P$ is given by
% \begin{align*}
%   \sum_{a=1}^k \frac 1{d_a} \norm{\log(P_a^{-1/2} Q_a P_a^{-1/2})}_F^2.
% \end{align*}}
\end{definition}

Using our definition of geodesics, we obtain the following notion of geodesic convexity.

\begin{definition}[Geodesic convexity]
A twice differentiable function $f\colon \P \to \R$ is said to be \emph{geodesically convex} at $\Theta\in\P$ if $\partial^2_{t=0} f(\exp_\Theta(tH)) \geq 0$ for all~$H\in\H$.
It is called \emph{$\lambda$-strongly geodesically convex} at $\Theta$ for some $\lambda>0$ if $\partial^2_{t=0} f(\exp_P(tH)) \geq \lambda \norm H_F^2$ for all~$H\in\H$.

We note that for a geodesically convex domain $D \subseteq \P$, a function $f$ is (strongly) geodesically convex on~$D$ if, and only if, the function $t \mapsto f(\gamma(t))$ is (strongly) convex on~$[0,1]$ for any (unit-speed) geodesic $\gamma(t)$ with $\gamma(0),\gamma(1)\in D$.
In other words, geodesic convexity simply means convexity in the ordinary Euclidean sense when restricted to geodesics.
% A function $f\colon \P \to \R$ is said to be \emph{geodesically convex} at $P\in\P$ if the functions $t \mapsto f(\exp_P(tH))$ are convex in $t\in\R$ for all~$H \in \H$.
% Assuming $f$ is twice differentiable, this holds if, and only if, $\partial^2_t f(\exp_P(tH)) \geq 0$ for all~$H\in\H$.

% Similarly, $f$ is called \emph{$\lambda$-strongly geodesically convex} at $P$ for some $\lambda>0$ if the same is true for the functions $t \mapsto f(\exp_P(tH))$ for all~$H\in \H$.
% Assuming the function is twice differentiable, this holds if, and only if, $\partial^2_t f(\exp_P(tH)) \geq \lambda \norm H_F^2$ for all~$H\in\H$.
\end{definition}

The invariance properties described above for $\PD(D)$ are directly inherited.
The manifold~$\P$ carries a natural action by the group
\begin{align*}
  \G =  \{G_1 \ot \dots \ot G_1: \GL(d_1)\ot \times \dots \times \ot \GL(d_k)\}
\end{align*}
Namely, if $\Theta \in \P$ and $A \in \G$ then the $A \Theta A^T$ is in $\P$. Moreover, the mapping $\Theta \mapsto A\Theta A^T$ is an isometry of the Riemannian manifold $P$. As discussed above, it and preserves the statistical distances~$D_F$ and $D_{\op}$.

% We also note that \MW{It's not \dots} the MLE obeys a certain \emph{equivariance} property under such transformations.
% For all $P \in \P$, $A \in \G$, and samples $x=(x_1,\dots,x_n)$, the log-likelihood satisfies
% \begin{align*}
%   \ell_{Ax}(P) = \ell_x(A^T P A),
% \end{align*}
% where we write $A x = ((G_1 \ot \dots \ot G_k) x_1, \dots, (G_1 \ot \dots \ot G_k) x_n)$.
% Thus the MLE $\widehat P=\widehat P(x)$ satisfies
% \begin{align}\label{eq:equivariance}
%   \widehat P(\samp) = P^{1/2} \widehat P(P^{1/2} \samp) P^{1/2}
% \end{align}
% assuming either MLE exists and is unique.

% %-----------------------------------------------------------------------------
% \subsection{Notation}
% %-----------------------------------------------------------------------------
% \CF{some aspects of this seem awfully specific to the tensor normal model and maybe could wait until after we define it, or simply merge the two, i.e. "Notation and model"}
% \MW{I think I like the former best.}
% The letter $n$ will denote a number of samples, and $d_1\leq \dots \leq d_k$ will denote sorted dimensions, and we set $D:=\prod_{i = 1}^k d_i$. Let $\PD_d$ denote the positive definite $d\times d$ matrices with unit determinant, and $\PD_d^1$ the subset of $\PD$ with unit determinant. $\rv$ will denote the random tuple $(\rv_1, \dots, \rv_n)$ where $\rv_i \in \R^{D}$ are drawn i.i.d from the tensor normal model with precision matrix $\Theta \in \PD_D$, and $\samp$ will denote a deterministic tuple of tensors.

% Let $\smallSym_d$ denote the vector space of $d\times d$ real symmetric matrices, and $\smallSym^0_d$ the subspace of traceless matrices in $\smallSym_d$, i.e. the tangent space of $\PD_d^1$. Let~$\SL_d$ denote the group of $d\times d$ matrices with unit determinant.
% %Then, $A^T e^Z A \in \PD_d^1$ for any $A \in \SL_d$ and $Z\in\Sym_d^0$.
% % Any matrix in $\PD_d^1$ can be written as the matrix exponential of a matrix in $\Sym_d^0$.
% Let
% $$\SL = \prod_{a=1}^k \SL_{d_a}, \SPD = \prod_{a = 1}^k \PD_{d_a}^1, \Sym = \bigoplus_{a = 1}^k \smallSym_{d_a}^0.$$ For a $k$-tuple of $A = (A_1, \dots, A_k)$ of matrices, $\|A\|_F^2:=\sum_{i = 1}^k \|A_i\|_F^2$.
%  We denote by $AB=(A_1B_1,\dots,A_kB_k)$ and $e^Z=(e^{Z_1},\dots,e^{Z_k})$ the componentwise product and matrix exponential, respectively, of matrix tuples $A, B \in \SL$ and $Z\in\Sym$. $I$ will denote an identity matrix, and $I_{a}$ a $d_a\times d_a$ identity matrix. $\langle \cdot, \cdot \rangle$ denotes the standard inner product. $C, c$ denote large (resp. small) absolute constants that change line to line.


% \CF{further notation to be introduced; delete as is done}
% \begin{itemize}
% %\item Number of samples $n$, dimensions $d_1\leq \dots \leq d_k$. $D$ for product of these.
% %\item $X$ for the tensor random variable, $\samp_1, \dots, \samp_n$ for each , $\samp = (\samp_1, \dots, \samp_n)$ for the random tuple of samples. $\rho = \samp \samp^T/\|\samp\|_F^2$. Lower case $x$ for samples.

% %\item $\rv$ for the random tuple $(\rv_1, \dots, \rv_n)$, $x$ for the tuple of samples $(x_1, \dots, x_n)$ when no longer random. Think of $\rv$ as $D \times n$ matrix, $\rho = \rv\rv^T,xx^T$ etc.
% \item When $x_i$ is a matrix, which unfortunately does happen sometimes, we'll use $x_i^\dagger$ for the matrix transpose (open to suggestions on this one).
% %\item $\braket{\cdot,\cdot}_{\vec d}$ denote modified Hilbert-Schmidt inner products\MW{sadly the corresponding norms look like $\ell_p$ norms},
% \item $f_{\rv}$ for the function in \cref{dfn:function}, mostly drop $\rv$. $\langle \cdot, \cdot \rangle$ is the $\ell_2$ inner product of vectors

% %\item $\smallSym_d$ for $d \times d$ real symmetric (meh), $\PD_d$ for $d \times d$ real positive definite, $\smallSym_d^0$ for traceless symmetric, $\PD_d^1$ for $\det=1$ positive definite?
% \item $\Theta$ for big tensor product pd precision matrix, $\Theta_a$ for individual pd's. \CF{at some point $\Theta$ gets used for the tuple. We need to decide what to do about this.}
% %\item I'm going to call $\SL = \oplus \SL_{d_i}, \SPD = \oplus \PD_{d_i}^1, \Sym = \oplus \smallSym_{d_i}^0$. Explain somewhere how $\Sym$ is the tangent space of $\SPD$.
% %\MW{Suppressing the $^1$ and $^0$ is a bit confusing I think. Maybe $\operatorname{SPD}$ for $\SPD$ with $\det=1$? I still feel that $\Sym$ looks somewhat horrible (with or without subscript, but I am not sure what would be better).}
% %\item $G = \prod_{a=1}^k \SL_{d_a}$, $P = \prod_{a=1}^k \PD_{d_a}^1$, and $S = \bigoplus_{a=1}^k \Sym_{d_a}^0$.

% \item $\nabla, \nabla^2$ for Riemannian Hessians and gradients, $\nabla_a f$, $\nabla^2_{ab} f$ for components. $\nabla f$ means at the identity. $\|\nabla^2_{ab}f\|_{op}:=\|\nabla^2_{ab}f\|_{F\to F}$. \CF{as operator from $\Sym_{d}^0$ to self?}
% %\item $C, c$ large (resp. small) constants that change line to line.
% \item $\rho^{(a)}$, $\rho^{(ab)}$ for marginals.
% %\item $I$ for an identity matrix, $I_a$ for the $d_a \times d_a$ identity matrix
% \end{itemize}


%-----------------------------------------------------------------------------
\subsection{Sketch of proof}\label{subsec:proof-sketch}
%-----------------------------------------------------------------------------
%As discussed above, the MLE problem reduces to minimizing $\sum_{i=1}^n x_i^T ( \bigotimes_{a=1}^k P_a ) x_i$ over $P \in \P$.
%We take as our objective its logarithm, which is also geodesically convex.

%Then the maximum likelihood estimator is given by $\widehat\Theta_a = {\widehat P_0}^{1/k} \widehat P_a$. We write $\widehat P = \widehat P(x)$ and $\widehat\Theta = \widehat\Theta(x)$ when we want to emphasize the dependence of the MLE on the samples~$x$.



%\begin{align}\label{eq:tilde ell}  \tilde\ell_{\samp}(\Theta) = \frac1D\log\det \Theta - \log \sum_{i=1}^n \braket{\samp_i, \Theta \samp_i};\end{align}

% One may think of \cref{eq:f extension} as proportional, up to an additive constant, to the log-likelihood of $[\samp_1, \dots, \samp_n]$ under the tensor normal model.
% Here $[\samp_1, \dots, \samp_n]$ denotes the equivalence class of the tuple of samples $(x_1, \dots, x_n)$ in projective space.


With these definitions in place, we are able to state a proof plan, which is a Riemannian version of the standard approach using strong convexity.

\begin{enumerate}
\item\label{it:reduce}
\textbf{Reduce to identity:}
We can obtain $n$ independent samples from $\cN(0, \Theta^{-1})$ as $x'_i = \Theta^{-1/2} x_i$, where $x_1,\dots,x_n$ are distributed as $n$ independent samples from a standard Gaussian.
The MLE $\widehat{\Theta}(x')$ for the former is exactly $\Theta^{1/2} \widehat{\Theta}(x) \Theta^{1/2}$.
By invariance of the relative Frobenius error, $D_F(\widehat\Theta(x') \Vert \Theta) = D_F(\widehat\Theta(x) \Vert I_D)$; the same is true for $D_{\op}$.
This shows that to prove \cref{thm:tensor-frobenius} it is enough to consider the case that $\Theta = I_D$, i.e., the standard Gaussian.
\item\label{it:grad} \textbf{Bound the gradient:}
Show that the gradient $\nabla f_x(I_D)$ (defined below) is small with high probability.
\item\label{it:convexity} \textbf{Show strong convexity:}
Show that, with high probability, $f_x$ is $\Omega(1)$-strongly geodesically convex near $I$.
\end{enumerate}

These together imply the desired sample complexity bounds -- as in the Euclidean case, strong convexity in a suitably large ball about a point implies the optimizer cannot be far. Moreover, it happens that under alternating minimization $f_\samp$ obeys a descent lemma (similar to what is shown in~\cite{burgisser2017alternating}); as such the flip-flop algorithm must converge exponentially quickly by the strong geodesic convexity of~$f_\samp$.

To make this discussion more concrete, we now define the gradient and Hessian formally, and state the lemma that we will use to relate the gradient and strong convexity to the distance to the optimizer as in the plan above.

\begin{definition}[Gradient and Hessian]\label{def:hess grad}
Let $f\colon \P \to \R$ be a once or twice differentiable function and $\Theta \in \P$.
The \emph{(Riemannian) gradient}~$\nabla f(\Theta)$ is the unique element in $\H$ such that
\begin{align*}
  \braket{\nabla f(\Theta), H}
= \partial_{t=0} f(\exp_\Theta(tH))
\qquad \forall H \in \H.
\end{align*}
Similarly, the \emph{(Riemannian) Hessian}~$\nabla^2 f(\Theta)$ is the unique linear operator on~$\H$ such that
\begin{align*}
  \braket{H, \nabla^2 f(\Theta) K}
= \partial_{s=0} \partial_{t=0} f(\exp_{\Theta}(sH + tK))
\qquad \forall H,K \in \H.
\end{align*}
We abbreviate $\nabla f = \nabla f(I_D)$ and $\nabla^2 f = \nabla^2 f(I_D)$ for the gradient and Hessian, respectively, at the identity matrix, and we write $\nabla_a f$ and $\nabla^2_{ab}f$ for the components.
As block matrices,
\begin{align*}
  \nabla f = \left[\begin{array}{c} \nabla_0 f \\ \hline \nabla_1 f \\ \vdots \\ \nabla_k f \end{array}\right],
  \qquad
  \nabla^2 f = \left[\begin{array}{c|ccc}
  \nabla_{00}^2 f & \nabla_{01}^2 f \dots & \nabla_{0k}^2 f \\
    \hline\nabla_{10}^2 f & \nabla_{11}^2 f \dots & \nabla_{1k}^2 f \\
  \vdots  & \vdots & \ddots & \vdots \\
  \nabla_{k0}^2 f & \nabla_{k1}^2 f \dots & \nabla_{kk}^2 f \\
  \end{array}\right].
\end{align*}
Here, $\nabla_0 f \in \R$ and each $\nabla_a f(\Theta)$ is a $d_a \times d_a$ traceless symmetric matrix.
Similarly, for~$a, b \in [k]$ (i.e., for the blocks of the submatrix to the lower-right of the lines) the components $\nabla_{ab}^2f(\Theta)$ of the Hessian are linear operators from the space of traceless symmetric $d_b\times d_b$ matrices to the space of traceless symmetric $d_a \times d_a$ matrices.
Moreover, $\nabla_{a0}f$ is a linear operator from $\R$ to the space of traceless symmetric $d_a\times d_a$ matrices (hence can itself be viewed as such a matrix), $\nabla_{0a}f$ is the adjoint of this linear operator (which using the Hilbert-Schmidt inner product can be identified with the same matrix), and $\nabla^2_{00} f(\Theta)$ is a real number.

\end{definition}

We note that the Hessian is symmetric with respect to the inner product~$\braket{\cdot,\cdot}$ on $\H$.
Just like in the Euclidean case, the Hessian is convenient to characterize strong convexity.
Indeed, $\braket{H, \nabla^2 f(\Theta) H} = \partial^2_{t=0} f(\exp_{\Theta}(tH))$ for all $H\in \H$.
Thus, $f$~is geodesically convex if and only if the Hessian is positive semidefinite, that is, $\nabla^2 f(\Theta) \succeq 0$. % where$\succeq$ is the Loewner order.
Similarly, $f$ is $\lambda$-strongly geodesically convex if and only if~$\nabla^2 f(\Theta) \succeq \lambda I$, i.e., the Hessian is positive definite with eigenvalues larger than or equal to~$\lambda$.

We can now state and prove the following lemma, which shows that strong convexity in a ball about a point where the gradient is sufficiently small implies the optimizer cannot be far.

\begin{lemma}\label{lem:convex-ball}
Let $f\colon \P \to \R$ be geodesically convex.
Assume the gradient at some~$\Theta\in\P$ is bounded as $\norm{\nabla f(\Theta)}_F \leq \eps$, and that $f$ is $\lambda$-strongly geodesically convex in a ball $B_r(\Theta)$ of radius~$r > \frac{2\eps}\lambda$.
Then the sublevel set $\{\Upsilon \in \P : f(\Upsilon) \leq f(\Theta)\}$ is contained in the ball~$B_{2\eps/\lambda}(\Theta)$, $f$ has a unique minimizer $\smash{\htheta}$, this minimizer is contained in $B_{\eps/\lambda}(\Theta)$, and
\begin{align}\label{eq:minimum lower bound}
  f(\htheta) \geq f(\Theta) - \frac{\eps^2}{2 \lambda}.
\end{align}
\end{lemma}
\begin{proof}
We first show that the sublevel set of~$f(\Theta)$ is included in the ball of radius~$\frac{2\eps}\lambda$.
Consider $g(t) := f(\exp_\Theta(tH))$, where~$H\in\H$ is an arbitrary vector of unit norm~$\norm H_F = 1$.
Then, using the assumption on the gradient,
\begin{align}\label{eq:grad bound}
  g'(0)
= \partial_{t=0} f(\exp_\Theta(tH))
= \braket{\nabla f(\Theta), H}
\geq -\norm{\nabla f(\Theta)}_F \norm H_F
\geq -\eps.
\end{align}
Since $f$ is $\lambda$-strongly geodesically convex on $B_r(\Theta)$, we have $g''(t) \geq \lambda$ for all $\abs t\leq r$.
It follows that for all $0 \leq t \leq  r$ we have
\begin{align}\label{eq:g convex lower}
  g(t) \geq g(0) - \eps t + \frac12 \lambda t^2.
\end{align}
Plugging in $t = r$ yields
$g(r) \geq  % g(0) - \eps  r + \frac12 \lambda r^2 =
g(0) + \left( \frac{\lambda r}2 - \eps \right)  r
> g(0)$.
Since $g$ is convex due to the geodesic convexity of $f$, it follows that, for any~$t \geq  r$,
\begin{align*}
  g(0) < g( r) \leq \left( 1-\frac{ r}t \right) g(0) + \frac{ r}t g(t),
\end{align*}
hence
\begin{align*}
  f(\Theta) = g(0) < g(t) = f(\exp_\Theta(tH)).
\end{align*}
Thus the sublevel set of~$f(\Theta)$ is contained in the ball of radius~$r$ about~$\Theta$.
By replacing $r$ with any smaller~$r'>\frac{2\eps}\lambda$, we see that the sublevel set is in fact contained in the ball of radius~$\frac{2\eps}\lambda$.
In particular, $f$ has a minimum and any minimizer~$\smash{\htheta}$ is contained in this ball.
Moreover, as the right-hand side of \cref{eq:g convex lower} takes a minimum at $t=\frac\eps\lambda$, we have $g(t) \geq g(0) - \frac{\eps^2}{2\lambda}$ for all~$0\leq t\leq r$.
By definition of $g$, this implies \cref{eq:minimum lower bound}.


Next, we prove that any minimizer of~$f$ is necessarily contained in the ball of radius~$\frac\eps\lambda$.
To see this, take an arbitrary minimizer~$\htheta$ and write it in the form $\htheta = \exp_\Theta(TH)$, where~$H\in \H$ is a unit vector and~$T>0$.
As before, we consider the function $g(t) = f(\exp_\Theta(tH))$.
Then, using \cref{eq:grad bound}, the convexity of~$g(t)$ for all $t\in\R$ and the $\lambda$-strong convexity of~$g(t)$ for~$\abs t \leq  r$, we have
\begin{align*}
  0 = g'(T) = g'(0) + \int_0^T g''(t) \, dt \geq \lambda \min(T,  r) - \eps.
\end{align*}
If $T> r$ then we have a contradiction as $\lambda r - \eps > \lambda r/2 - \eps > 0$.
Therefore we must have~$T\leq r$ and hence $\lambda T - \eps \leq 0$, so $T \leq \frac\eps\lambda$.
Thus we have proved that any minimizer of $f$ is contained in the ball of radius~$\frac\eps\lambda$.

We still need to show that the minimizer is unique; that this follows from strong convexity is convex optimization ``folklore,'' but we include a proof nonetheless.
Indeed, suppose that $\htheta$ is a minimizer and let $H\in \H$ be arbitrary.
Consider $h(t) := f(\exp_{\htheta}(tH))$.
Then the function $h(t)$ is convex, has a minimum at $t=0$, and satisfies $h''(0) > 0$, since $f$ is $\lambda$-strongly geodesically convex near~$\htheta$, as $\htheta \in B_r(\Theta)$ by what we showed above.
It follows that $h(t) > h(0)$ for any~$t\neq0$.
Since $H$ was arbitrary, this shows that $f(\Upsilon) > f(\htheta)$ for any $\Upsilon\neq \htheta$.
\end{proof}

%-----------------------------------------------------------------------------
\subsection{Bounding the gradient}
%-----------------------------------------------------------------------------
Proceeding according to step~\ref{it:grad} of the plan outlined in \cref{subsec:proof-sketch}, we now compute the gradient of the objective function and bound it using basic matrix concentration results.

To calculate the gradient, we need a definition from linear algebra.
Recall that our data comes as an $n$-tuple $x=(x_1,\dots,x_n)$ of $k$-tensors. %, which we can also think of as an array of format~$d_1\times\dots\times d_k\times n$.
Let $\rho := \frac1{nD}\sum_i x_i x_i^T$ denote the matrix of ``second sample moments'' of the data.
Then we can rewrite the objective function as
\begin{align}\label{eq:obj via rho}
  f_x(\Theta) = \tr \rho \, \Theta - \frac1D \log \det \Theta.
\end{align}
We may also consider the ``second sample moments'' of a subset of the coordinates~$J \subseteq [k]$.
For this the following definition is useful.

\begin{definition}[Partial trace]
Let $\rho$ be an operator on $\R^{d_1} \ot \dots \ot \R^{d_k}$, and~$J \subseteq [k]$.
Define the \emph{partial trace} $\rho^{(J)}$ as the $d_J \times d_J$-matrix, $d_J = \prod_{a\in J} d_a$, that satisfies the property that
\begin{align}\label{eq:partial trace duality}
  \tr \rho^{(J)} H
= \tr \rho \, H_{(J)}
\end{align}
for any $d_J\times d_J$ matrix~$H$, where $H_{(J)}$ denotes the operator on $\R^{d_1} \ot \cdots \ot \R^{d_k}$ that acts as~$H$ on the tensor factors labeled by $J$ and as the identity on the rest.
This property uniquely determines $\rho^{(J)}$.
We write $\rho^{(a)}$ and $\rho^{(ab)}$ when $J=\{a\}$ and $J=\{a,b\}$, respectively.


If $\rho$ is positive definite then so is $\rho^{(J)}$.
Moreover, $(\rho^{(J)})^{(K)}$ for $K \subseteq J$, and $\tr \rho = \tr \rho^{(J)}$.
\end{definition}

Concretely, the partial trace $\rho^{(I)}$ can be calculated as follows:
Analogously to the discussion in \cref{subsec:model}, ``flatten'' the data~$x$ by regarding it as a $d_I \times N_I$~matrix~$x^{(I)}$, where $N_I = \frac{nD}{d_I}$;
then $\rho^{(I)} := \frac1{nD} x^{(I)} (x^{(I)})^T$.

The components of the gradient can be readily computed in terms of partial traces.

% To calculate the gradient, we need a definition from linear algebra.
% Recall that our data comes as an $n$-tuple $x=(x_1,\dots,x_n)$ of tensors. %, which we can also think of as an array of format~$d_1\times\dots\times d_k\times n$.
% Let $\rho := \frac1{nD}\sum_i x_i x_i^T$ denote the matrix of ``second sample moments'' of the data.
% Then we can rewrite the objective function as
% \begin{align}\label{eq:obj via rho}
%   f_x(\Theta) = \tr \rho \, \Theta - \frac1D \log \det \Theta.
% \end{align}
% We may also consider the ``second sample moments'' of a single coordinate.
% For this, analogously to the discussion in \cref{subsec:model}, we may for any $a\in[k]$ ``flatten''~$x$ by regarding it as a $d_a \times N_a$~matrix~$x^{(a)}$, where $N_a = \frac{nD}{d_a}$, and define $\rho^{(a)} := \frac1{nD} x^{(a)} (x^{(a)})^T$.
% It is not hard to verify that $\rho^{(a)}$ is a positive semidefinite $d_a\times d_a$ matrix.
% It satisfies the property
% \begin{align}\label{eq:partial trace duality}
%   \tr \rho^{(a)} H = \tr \rho H_{(a)}
% \quad\text{where}\quad
%   H_{(a)} := I_{d_1\cdots d_{a-1}} \ot H \ot I_{d_{a+1} \cdots d_k}
% \end{align}
% for any $d_a \times d_a$ matrix~$H$.
% Note that $\tr \rho^{(a)} = \tr \rho = \frac1{nD}\norm x_2^2$.
% In the language of linear algebra, the matrices $\rho^{(a)}$ are known as the \emph{partial traces} of~$\rho$ over all but one of the coordinates.
% \MW{It would for sure be nice to define the partial trace once in general for a subset $I \subseteq \{1,\dots,k\}$. Maybe do it here\dots or what do you think?}
% The gradient can be readily computed in terms of this data.

\begin{lemma}[Gradient]\label{lem:gradient}
Let $\rho = \frac{1}{nD} \sum_{i=1}^n \samp_i \samp_i^T $.
Then the components of the gradient~$\nabla f_x$ at the identity are given by
\begin{align*}
 \nabla_a \ef_{\samp} &= \sqrt{d_a}\left( \rho^{(a)} - \frac{\tr\rho}{d_a} I_{d_a}\right)
  \qquad \text{ for } a \in [k], \\
  \nabla_0 \ef_\samp &= \tr \rho - 1.
\end{align*}
\end{lemma}
\begin{proof}
For all $a\in[k]$ and any traceless symmetric $d_a\times d_a$ matrix~$H$, we have
\begin{align*}
\braket{\nabla_a f_x(I_D), H}
&= \partial_{t=0} f_x(e^{t\sqrt{d_a} H_{(a)}})
= \partial_{t=0} \tr \rho \, e^{t\sqrt{d_a} H_{(a)}} - \frac1D\log\det(e^{t\sqrt{d_a} H_{(a)}}) \\
&= \sqrt{d_a} \tr \rho \, H_{(a)}
= \sqrt{d_a} \tr \rho^{(a)} \, H
\end{align*}
using \cref{eq:obj via rho,eq:partial trace duality} and that $\tr H_{(a)} = \tr H = 0$.
Since $\nabla_a f_{\samp}$ is traceless and symmetric by definition, while $\rho^{(a)}$ is symmetric, this implies
\begin{align*}
  \nabla_a f_{\samp}
= \sqrt{d_a} \left( \rho^{(a)} - \frac{\tr \rho^{(a)}}{d_a} I_{d_a} \right)
= \sqrt{d_a} \left( \rho^{(a)} - \frac{\tr \rho}{d_a} I_{d_a} \right).
\end{align*}
Finally,
\[
  \nabla_0 f_x
= \partial_{t=0} \left( \tr \rho e^t - \frac1D \log \det(e^t I_D) \right)
% = \partial_{t=0} \left( \tr \rho e^t - \frac1D \log e^{tD} \right)
= \partial_{t=0} \left( \tr \rho e^t - t \right)
= \tr \rho - 1.
\]
\end{proof}

Having calculated the gradient of the objective function, we are ready to state our bound:

\begin{prop}[Gradient bound]\label{prop:gradient-bound}
Let $\rv = (\rv_1,\dots,\rv_n)$ consist of independent standard Gaussian random variables in~$\R^D$, % where $D=d_1\cdots{}d_k$,
and let $0<\eps<1$.
Suppose $n \geq \frac{d_{\max}^2}{D \eps^2}$.
Then, the following occurs with probability at least $1 - 2(k+1)e^{-\eps^2 \frac{nD}{8d_{\max}}}$:
\begin{align*}
  \norm{\nabla_a f_x}_{\op} &\leq \frac{9\eps}{\sqrt{d_a}} \qquad\text{ for all $a\in[k]$}, \\
  \abs{\nabla_0 f_x} &\leq \eps.
\intertext{In particular,}
  \norm{\nabla f_x}_F^2
% = \abs{\nabla_0 f_x}^2 + \sum_{a=1}^k \norm{\nabla_a f_x}_F^2
% \leq \abs{\nabla_0 f_x}^2 + \sum_{a=1}^k d_a \norm{\nabla_a f_x}_{\op}^2 \\
% \leq \eps^2 + \sum_{a=1}^k 81 \eps^2 =
&\leq (1 + 81 k) \eps^2
\leq 82 k \eps^2.
\end{align*}
\end{prop}

To prove this we will need a standard result in matrix concentration.
When the samples $x=(x_1,\dots,x_n)$ are independent standard Gaussians in $\R^D$, then $\rho^{(a)}$ is distributed as $\frac1{nD} Y Y^T$, where~$Y$ is a random $d_a \times N_a$ matrix with independent standard Gaussian entries, where~$N_a = \frac{nD}{d_a}$.
The following result bounds the singular values of such random matrices.

\begin{theorem}[Corollary 5.35 of \cite{vershynin2010introduction}]\label{cor:vershynin}
Let $Y \in \R^{d \times N}$ have independent standard Gaussian entries where $N\geq d$.
Then, for every $t > 0$, the following occurs with probability at least $1 - 2 e^{-t^2/2}$:
\begin{align*}
  \sqrt{N} - \sqrt{d} - t \leq \sigma_d(Y) \leq \sigma_1(Y) \leq \sqrt{N} + \sqrt{d} + t,
\end{align*}
where $\sigma_j$ denotes the $j$-th largest singular value.
\end{theorem}

We will also need to bound $\tr\rho = \frac1{nD} \norm x_2^2$.
Because $\norm x_2^2$ is simply a sum of $nD$ many $\chi$-squared random variables, the next proposition follows from standard concentration bounds.

\begin{prop}[Example~2.11 of \cite{W19}]\label{prp:xnorm}
Let $\rv = (\rv_1,\dots,\rv_n)$ consist of independent standard Gaussian random variables in~$\R^D$.
Then, for $0 < t < 1$, the following occurs with probability at least $1 - 2e^{-t^2 nD/8}$:
\begin{align*}
  (1 - t) nD \leq \norm{x}_2^2 \leq (1 + t) nD.
\end{align*}
\end{prop}

Equipped with these results we now prove our gradient bound.

\begin{proof}[Proof of \cref{prop:gradient-bound}]
For any fixed $a\in[k]$, recall that $\rho^{(a)}$ has the same distribution as~$\frac1{nD} YY^T$, where $Y$ is an $d_a\times N_a$-matrix with standard Gaussian entries.
By \cref{cor:vershynin}, we have the following bound with $\leq 2 e^{-t^2/2}$ failure probability:
\begin{align*}
  \sqrt{N_a} - \sqrt{d_a} - t \leq \sigma_d(Y) \leq \sigma_1(Y) \leq \sqrt{N_a} + \sqrt{d_a} + t.
\end{align*}
Let $t = \eps \sqrt{N_a}$.
Using that $0<\eps<1$ and $d_a \leq N_a \eps^2$ by assumption, this bound implies that
% \begin{align*}
%   \abs*{\frac{\sigma(Y)}{\sqrt{N_a}} - 1}
% \leq \frac{\sqrt{d_a} + t}{\sqrt{N_a}}
% = \frac{\sqrt{d_a} + \eps\sqrt{N_a}}{\sqrt{N_a}}
% \leq \frac{2\eps\sqrt{N_a}}{\sqrt{N_a}}
% = 2\eps \\
% \Rightarrow
%   \abs*{\frac{\sigma^2(Y)}{N_a} - 1}
% = \abs*{\frac{\sigma(Y)}{\sqrt{N_a}} - 1} \abs*{\frac{\sigma(Y)}{\sqrt{N_a}} + 1}
% \leq \abs*{\frac{\sigma(Y)}{\sqrt{N_a}} - 1} \abs*{2 + \abs[\big]{\frac{\sigma(Y)}{\sqrt{N_a}} - 1}}
% \leq 2\eps \left( 2 + 2\eps \right)
% = 4\eps + 4\eps^2 \leq 8\eps
% \end{align*}
% the eigenvalues of $YY^T$ are in $N_a(1\pm8\eps)$. It follows that
with the same failure probability, the eigenvalues of $\rho^{(a)}$ are in $\frac1{d_a}(1\pm8\eps)$.
On the other hand, by \cref{prp:xnorm}, we have that $\abs{\tr \rho - 1} \leq \eps$ with $\leq 2e^{-\eps^2 nD/8}$ failure probability.
Therefore, by the formulas in \cref{lem:gradient} and the union bound, we have
\begin{align*}
  \norm{\nabla_a f_x}_{\op}
% &=\sqrt{d_a} \norm*{\rho^{(a)} - \frac{\tr\rho}{d_a} I_{d_a}}_{\op} \\
% &=\sqrt{d_a} \norm*{\rho^{(a)} - \frac1{d_a} I_{d_a} + \frac{\tr\rho - 1}{d_a} I_{d_a}}_{\op} \\
&\leq\sqrt{d_a} \norm*{\rho^{(a)} - \frac1{d_a} I_{d_a}} + \frac{\abs{\tr\rho - 1}}{\sqrt{d_a}}
\leq\frac{9\eps}{\sqrt{d_a}} \qquad \text{for all } a\in[k], \\
  \abs{\nabla_0 f_x}
&=\abs{\tr \rho - 1} \leq \eps,
\end{align*}
with failure probability at most
$2e^{-\eps^2 \frac{nD}8} + \sum_{a=1}^k 2 e^{-\eps^2 \frac{N_a}2}
% = 2e^{-\eps^2 \frac{nD}8} + \sum_{a=1}^k 2 e^{-\eps^2 \frac{nD}{2d_a}}
\leq 2(k+1)e^{-\eps^2 \frac{nD}{8d_{\max}}}.$
\end{proof}

%-----------------------------------------------------------------------------
\subsection{Strong convexity}\label{subsec:strong-convex}
%-----------------------------------------------------------------------------
In this section, we prove our strong convexity result, \cref{thm:ball-convexity}, in order to carry out step~\ref{it:convexity} of the plan from \cref{subsec:proof-sketch}.
The theorem states that, with high probability, $f_x$ is strongly convex near the identity.
We will prove it by first establishing strong convexity \emph{at} the identity, \cref{thm:tensor-convexity}, using quantum expansion techniques, and then giving a bound on how the Hessian changes away from the origin, \cref{convexRobustness}.
We first assemble these results and then prove \cref{thm:ball-convexity} at the end of this subsection.

Similarly as for the gradient, we can compute the components of the Hessian in terms of partial traces, but now we also need to consider two coordinates at a time.

% Recall that given data $x=(x_1,\dots,x_n)$, we defined $\rho = \frac1{nD} \sum_{i=1}^n x_i x_i^T$.
% For any $a \neq b \in [k]$, we may ``flatten'' $x$ by thinking of it as a $d_a d_b \times \frac{nD}{d_ad_b}$ matrix $x^{(ab)}$, and define $\rho^{(ab)} := x^{(ab)} (x^{(ab)})^T$.
% Then $\rho^{(ab)}$ is a positive semidefinite $d_ad_b \times d_ad_b$ matrix, and we have
% \begin{align}\label{eq:two body partial trace duality}
%   \tr \rho^{(ab)} (H \ot K) = \tr \rho \, H_{(a)} K_{(b)}
% \end{align}
% for any $d_a\times d_a$ matrix $H$ and any $d_b\times d_b$ matrix $K$, with $H_{(a)}$, $K_{(b)}$ defined as in \cref{eq:partial trace duality}.
% The matrices $\rho^{(ab)}$ are the \emph{partial trace} of $\rho$ over all but two of the coordinates.
% By comparing \cref{eq:partial trace duality,eq:two body partial trace duality}, we see that this definition is consistent with the definitions of~$\rho^{(a)}$,~$\rho^{(b)}$ in the sense that $\tr \rho^{(ab)} (H \ot I_{d_b}) = \tr \rho^{(a)} H$ and $\tr \rho^{(ab)} (I_{d_a} \ot K) = \tr \rho^{(b)} K$ for all~$H,K$.

\begin{lemma}[Hessian]\label{lem:hessian}
Let $\rho = \frac{1}{nD}\sum_{i=1}^n \samp_i \samp_i^T$.
Then the components of the Hessian~$\nabla^2 f_{\samp}$ at the identity are given by
\begin{align*}
  \braket{H, (\nabla^2_{aa} f_x) H} &= d_a \tr \rho^{(a)} H^2 \\
  \braket{H, (\nabla^2_{ab} f_x) K} &= \sqrt{d_a d_b} \tr \rho^{(ab)} \left( H \ot K \right)
\end{align*}
for all $a\neq b\in[k]$ and traceless symmetric $d_a\times d_a$ matrices $Y$, $d_b\times d_b$ matrices~$Z$, and
\begin{align*}
  \nabla^2_{a0} f_x = \nabla^2_{a0} f_x = & =  \sqrt{d_a} \left( \rho^{(a)} - \frac{\tr \rho}{d_a} I_{d_a} \right) \\
  \nabla^2_{00} f_x &= \tr \rho.
\end{align*}
for all $a \in [k]$.
\end{lemma}
Again we caution the reader that $\nabla^2_{a0} f_x$ is a linear operator from the real numbers to the traceless symmetric matrices, which we identify with a traceless symmetric matrices, and similarly for its adjoint $\nabla^2_{0a} f_x$.
\begin{proof}
  Note that the Hessian of~$f_x$ coincides with the one of $\tr\rho\,\Theta$.
  This follows from \cref{eq:obj via rho}, since the Hessian of $\log\det\Theta$ vanishes identically.
  % Indeed, for any $H\in\H$,
  % \begin{align*}
  %   \log\det(\exp_\Theta(tH))
  % = \log\det(e^{tH_0} \Theta)
  % = \log \det(\Theta) + t H_0 D,
  % \end{align*}
  % which is an affine-linear function in $t\in\R$.
  Accordingly, we will compute the Hessian of~$\tr\rho\,\Theta$.
  For $a\in[k]$ and any traceless symmetric $d_a\times d_a$ matrix $H$, we have
  \begin{align*}
    \braket{H, (\nabla^2_{aa} f_x) H}
  = \partial_{s=0} \partial_{t=0} \tr \rho \, e^{(s+t) \sqrt{d_a} H_{(a)}}
  = d_a \tr \rho H_{(a)}^2
  = d_a \tr \rho^{(a)} H^2
  \end{align*}
  using \cref{eq:partial trace duality}.
  Similarly, for $a\neq b\in[k]$, any traceless symmetric $d_a\times d_a$ matrix $H$, and any traceless symmetric $d_b\times d_b$ matrix $K$, we find that
  \begin{align*}
    \braket{H, (\nabla^2_{ab} f_x) K}
  &= \partial_{s=0} \partial_{t=0} \tr \rho \, e^{s \sqrt{d_a} H_{(a)} + t \sqrt{d_b} K_{(b)}} \\
  &= \sqrt{d_a d_b} \tr \rho \, H_{(a)} K_{(b)}
  = \sqrt{d_a d_b} \tr \rho^{(ab)} \left( H \ot K \right)
  \end{align*}
  using \cref{eq:partial trace duality}.
  Next, for $a\in[k]$ and any traceless symmetric $d_a\times d_a$ matrix $H$, we have
  \begin{align*}
    \braket{H, \nabla^2_{a0} f_x}
  = \partial_{s=0} \partial_{t=0} \tr \rho \, e^{s\sqrt{d_a} H_{(a)} + t}
  = \sqrt{d_a} \tr \rho \, H_{(a)}
  = \sqrt{d_a} \tr \rho^{(a)} H,
  \end{align*}
  recalling that we identify $\nabla^2_{a0} f_x$ with a traceless symmetric $d_a\times d_a$ matrix;
  this shows that
  \begin{align*}
    \nabla^2_{a0} f_x = \sqrt{d_a} \left( \rho^{(a)} - \frac{\tr \rho}{d_a} I_{d_a} \right),
  \end{align*}
  and similarly for the transpose.
  Finally,
  \begin{align*}
    \nabla^2_{00} f_x
  = \partial_{s=0} \partial_{t=0} \tr \rho \, e^{s+t}
  = \tr \rho.
  \end{align*}
\end{proof}

The most interesting part of the Hessian are the off-diagonal blocks for $a\neq b\in[k]$, which up an overall factor $\sqrt{d_a d_b}$ can be seen as the restrictions of the linear maps
\begin{align}\label{eq:hessian channel}
  \Phi^{(ab)} \colon \Mat(d_b) \to \Mat(d_a), \quad \braket{H, \Phi^{(ab)}(K)} &= \tr \rho^{(ab)} \left( H \ot K \right)
\end{align}
to the traceless symmetric matrices.
% For general $\omega \in \PD(d_1d_2)$, a linear map of the form
% \begin{align*}
%   \Phi \colon \Mat(d_1) \to \Mat(d_2), \quad \braket{H, \Phi(K)} = \tr \omega (H \ot K)
% \end{align*}
% is known as a \emph{completely positive map}.
% If $\omega = \sum_{i=1}^N \vect(A_i) \vect(A_i)^T$ for $d_2\times d_1$ matrices~$A_1,\dots,A_N$, we have the following two equivalent ways to write the corresponding completely positive map, which we denote by $\Phi_A$:
% \begin{align}\label{eq:kraus}
%   \Phi_A(K) = \sum_{i=1}^N A_i K A_i^T
% \quad\text{or}\quad
%   \vect(\Phi_A(K)) = \sum_{i=1}^N (A_i \ot A_i) \vect(K).
% \end{align}
% Conversely, any linear map of this form is completely positive.
% The matrices $A_1,\dots,A_N$ are known as Kraus operators of~$\Phi_A$.
% We denote by $\Phi^*\colon\Mat(d_2)\to\Mat(d_1)$ the adjoint of a completely positive map~$\Phi$ with respect to the Hilbert-Schmidt inner product; this is again a completely positive map (with Kraus operators the transposes of the original Kraus operators).
This is a special case of a \emph{completely positive map}, which is a linear map of the form
\begin{align}\label{eq:def cp}
  \Phi_A \colon \Mat(d_b) \to \Mat(d_a), \quad \Phi_A(X) = \sum_{i=1}^N A_i X A_i^T
\end{align}
for $d_a\times d_b$ matrices $A_1,\dots,A_N$.
To see the connection, note that since $\rho^{(ab)}$ is positive semidefinite, it can be written in the form $\sum_{i=1}^N \vect(A_i) \vect(A_i)^T$; then $\Phi^{(ab)} = \Phi_A$ follows.
The matrices $A_1,\dots,A_N$ are known as \emph{Kraus operators}.
\Cref{eq:def cp} can also be written as
\begin{align}\label{eq:vec rep}
  \vect(\Phi_A(X)) = \sum_{i=1}^N (A_i \ot A_i) \vect(X).
\end{align}
We denote by $\Phi^*\colon\Mat(d_a)\to\Mat(d_b)$ the adjoint of a completely positive map~$\Phi$ with respect to the Hilbert-Schmidt inner product; this is again a completely positive map, with Kraus operators $A_1^T,\dots,A_N^T$.
% \begin{align*}
%   \braket{H, \Phi^{(ab)}(K)}
% = \tr \rho^{(ab)} \left( H \ot K \right)
% = \sum_{i=1}^N \vect(A_i)^T \left( H \ot K \right) \vect(A_i) \\
% = \sum_{i=1}^N \sum_{a,a',b,b'} \braket{a|A_i|b} \braket{a,b|H \ot K|a',b'} \braket{a'|A_i|b'} \\
% = \sum_{i=1}^N \sum_{a,a',b,b'} \braket{a|A_i|b} \braket{a|H|a'} \braket{b|K|b'} \braket{a'|A_i|b'} \\
% = \sum_{i=1}^N \sum_{a,a',b,b'} \braket{a'|H^T|a} \braket{a|A_i|b} \braket{b|K|b'} \braket{b'|A_i^T|a'} \\
% = \sum_{i=1}^N \tr H^T A_i K A_i^T
% = \braket{H, \sum_{i=1}^N A_i K A_i^T}
% \end{align*}
% and
% \begin{align*}
%   \vect(A_i K A_i^T)
% = \sum_{a,a'} \ket{a,a'} \braket{a|A_i K A_i^T|a'}
% = \sum_{a,a',b,b'} \ket{a,a'} \braket{a|A_i|b} \braket{b|K|b'} \braket{b'|A_i^T|a'} \\
% = \sum_{a,a',b,b'} \ket{a,a'} \braket{a|A_i|b} \braket{a'|A_i|b'} \braket{b|K|b'} \\
% = \sum_{a,a',b,b'} (A_i \ot A_i) \vect(K)
% \end{align*}
%
In our proof of strong convexity, we will show that strong convexity follows if the completely positive maps $\Phi^{(ab)}$ are good \emph{quantum expanders}.

\begin{definition}[Quantum expansion]
Let $\Phi\colon\Mat(d_b) \to \Mat(d_a)$ be a completely positive map.
Say $\Phi$ is \emph{$\eps$-doubly balanced} if
\begin{align}\label{eq:doubly balanced}
  \norm*{\frac{\Phi(I_{d_b})}{\tr \Phi(I_{d_b})} - \frac{I_{d_a}}{d_a}}_{\op} \leq \frac\eps{d_a}
\quad\text{and}\quad
  \norm*{\frac{\Phi^*(I_{d_a})}{\tr \Phi^*(I_{d_a})} - \frac{I_{d_b}}{d_b}}_{\op} \leq \frac\eps{d_b}.
\end{align}
Say $\Phi$ is an \emph{$(\eps, \lambda)$-quantum expander} if $\Phi$ is $\eps$-doubly balanced and
\begin{align}\label{eq:expansion}
  \norm{\Phi}_0 := \max_{\substack{H \text{ traceless symmetric} \\ \norm H_F=1}} \norm{\Phi(H)}_F
\leq \lambda \frac{\tr \Phi(I_{d_b})}{\sqrt{d_ad_b}}
\end{align}
A $(0, \lambda)$-quantum expander is called a \emph{$\lambda$-quantum expander}.
We note that all these definitions are invariant under rescaling $\Phi \mapsto c\Phi$ for $c>0$.
\end{definition}

Quantum expanders play an important role in quantum information theory and quantum computation \TODO{cite}.
There one typically takes $d_a=d_b$, so that \cref{eq:expansion} simplifies to~$\norm{\Phi}_0 \leq \lambda$.
For us, their purpose will be the following lemma allowing us to translate quantum expansion properties into strong convexity.

\begin{lemma}[Strong convexity from expansion]\label{lem:expansion-convexity}
If the completely positive maps $\Phi^{(ab)}$ defined in \cref{eq:hessian channel} are $(\eps,\lambda)$-quantum expanders for every $a\neq b\in[k]$, then
\begin{align*}
  \norm*{\frac{\nabla^2 f_x}{\tr \rho} - I_\H}_{\op}
\leq (k-1)\lambda + (1 + \sqrt k) \eps.
\end{align*}
Assuming $k\geq2$, the right-hand side is at most $k (\lambda + \eps)$.
\end{lemma}
\noindent
It suffices to verify the hypothesis for $a<b$.
Indeed, since $\tr \Phi(I_{d_a}) = \tr \Phi(I_{d_b})$, any $\Phi$ is an $(\eps,\lambda)$-quantum expander if and only if this is the case for the adjoint $\Phi^*$, but note that the adjoint of~$\Phi^{(ab)}$ is~$\Phi^{(ba)}$.
\begin{proof}
We wish to bound the operator norm of $M = \frac{\nabla^2 f_\samp}{\tr \rho} - I_\H$, which we consider as a block matrix as in \cref{def:hess grad}.
For this, we use the following basic estimate of the norm of a block matrix in terms of the norm of the matrix of blocks norms, i.e.,
\begin{align}\label{eq:baby norm bounds}
  \norm{M}_{\op} \leq \norm{m}_{\op},
\quad \text{ where } m=(\norm{M_{ab}}_{\op})_{a,b\in\{0,1,\dots,k\}}.
\end{align}
We first bound the individual block norms.
Recall from \cref{lem:hessian} that the off-diagonal blocks of the Hessian for $a \neq b\in[k]$ are given by $\nabla^2_{ab} f_x = \sqrt{d_a d_b} \Phi^{(ab)}$.
Thus, since $\Phi^{(ab)}$ is an $(\eps,\lambda)$-quantum expander, we have
\begin{align*}
  \norm{M_{ab}}_{\op}
= \frac{\norm{\nabla^2_{ab} f_x}_{\op}}{\tr\rho}
= \frac{\sqrt{d_a d_b}}{\tr \Phi^{(ab)}(I_{d_b})} \norm{\Phi^{(ab)}}_0
\leq \lambda,
\end{align*}
using that $\tr \Phi^{(ab)}(I_{d_b}) = \tr \rho$.
The remaining off-diagonal blocks can be bounded as
\begin{align*}
\norm{M_{a0}}
= \frac{\norm{\nabla^2_{a0} f_x}_{\op}}{\tr \rho}
&= \norm*{\sqrt{d_a} \left( \frac{\rho^{(a)}}{\tr \rho} - \frac{I_{d_a}}{d_a} \right)}_F
= \sqrt{d_a} \norm*{\frac{\Phi^{(ab)}(I_{d_b})}{\tr \Phi^{(ab)}(I_{d_b})} - \frac{I_{d_a}}{d_a}}_F \\
&\leq d_a \norm*{\frac{\Phi^{(ab)}(I_{d_b})}{\tr \Phi^{(ab)}(I_{d_b})} - \frac{I_{d_a}}{d_a}}_{\op}
\leq \eps,
\end{align*}
using that $\Phi^{(ab)}(I_{d_b}) = \rho^{(a)}$.
On the other hand, the diagonal blocks for $a\in[k]$ can be bounded by observing that, for any traceless Hermitian $H$,
\begin{align*}
  \abs{\braket{H, M_{aa} H}}
&= \abs*{\braket{H, \left( \frac{\nabla^2_{aa} f_x}{\tr \rho} - I_\H \right) H}}
= d_a \abs*{\tr \left( \frac{\rho^{(a)}}{\tr \rho} - \frac{I_{d_a}}{d_a} \right) H^2} \\
&\leq d_a \norm*{\frac{\rho^{(a)}}{\tr \rho} - \frac{I_{d_a}}{d_a}}_{\op} \norm{H}_F^2
\leq \eps \norm H_F^2,
\end{align*}
hence $\norm{M_{aa}}_{\op} \leq \eps$, while $\abs{M_{00}} = \abs{\frac{\nabla^2_{00} f_x}{\tr \rho} - 1} = 0$.
% Since $m$ is a symmetric matrix, it follows that
% \begin{align*}
%   \norm m_{\op} \leq \max_i \sum_j m_{ij} = \max \{ k \eps, \eps + (k-1) \lambda \}
% \end{align*}
To conclude the proof, decompose
\begin{align*}
  m
% = \left[\begin{array}{c|cccc}
%   0 & m_{01} & m_{02} & \cdots & m_{0k} \\
%   \hline
%   m_{10} & m_{11} & m_{12} & \cdots & m_{1k} \\
%   m_{20} & m_{21} & m_{22} & & m_{2k} \\
%   \vdots & \vdots & & \ddots & \hdots \\
%   m_{k0} & m_{k1} & m_{k2} & \cdots & m_{kk}
%   \end{array}\right]
= \left[\begin{array}{c|cccc}
  0 & 0 & 0 & \cdots & 0 \\
  \hline
  0 & 0 & m_{12} & \cdots & m_{1k} \\
  0 & m_{21} & 0 & & m_{2k} \\
  \vdots & \vdots & & \ddots & \vdots \\
  0 & m_{k1} & m_{k2} & \cdots & 0
  \end{array}\right]
+ \left[\begin{array}{c|cccc}
  0 & 0 & 0 & \cdots & 0 \\
  \hline
  0 & m_{11} & 0 & \cdots & 0 \\
  0 & 0 & m_{22} & & 0 \\
  \vdots & \vdots & & \ddots & \vdots \\
  0 & 0 & 0 & \cdots & m_{kk}
  \end{array}\right]
+ \left[\begin{array}{c|cccc}
  0 & m_{01} & m_{02} & \cdots & m_{0k} \\
  \hline
  m_{10} & 0 & 0 & \cdots & 0 \\
  m_{20} & 0 & 0 & & 0 \\
  \vdots & \vdots & & \ddots & \vdots \\
  m_{k0} & 0 & 0 & \cdots & 0
  \end{array}\right].
\end{align*}
The nonzero entries of the first matrix are bounded by $\lambda$, hence its operator norm is at most $(k-1)\lambda$.
The second matrix is diagonal with diagonal entries bounded by $\eps$, hence its operator norm is at most~$\eps$.
The third matrix has nonzero entries bounded by $\eps$, hence its operator norm is bounded by~$\sqrt k \eps$.
Using \cref{eq:baby norm bounds} we obtain the desired bound.
% \begin{align*}
%   \norm M_{\op} \leq \norm m_{\op} \leq (k-1)\lambda + (1 + \sqrt k) \eps.
% \end{align*}
\end{proof}

% We note that a slightly more complicated proof yields the improved estimate $\leq (k-1) \lambda + (\sqrt k + 1) \eps$.

% The proof of \cref{lem:expansion-convexity} uses a basic lemma that gives upper and lower bounds on a block matrix in terms of block diagonal matrices and the Loewner order, and a resulting norm bound.

% \begin{lemma}\label{lem:block-matrix}
% Let $M$ be a symmetric block matrix with blocks $M_{ij}$ of size~$d_i \times d_j$, where~$i,j\in[N]$.
% Then,
% \begin{align*}
%   \bigoplus_{i=1}^N \left(M_{ii} - I_{d_i} \cdot \sum_{j \neq i} \norm{M_{ij}}_{\op} \right)
% \preceq M
% \preceq \bigoplus_{i=1}^N \left(M_{ii} + I_{d_i} \cdot \sum_{j \neq i} \norm{M_{ij}}_{\op} \right)
% \end{align*}
% and hence \MW{Isn't the following all we need?}
% \begin{align*}
%   \norm{M}_{\op} \leq \max_i \sum_j \norm{M_{ij}}_{\op}.
% \end{align*}
% \end{lemma}
% \begin{proof}
% We use the inequality for block matrices
% \begin{align}\label{eq:psd-bound}
%   -\begin{bmatrix} A & 0 \\ 0 & B \end{bmatrix}
% \preceq \begin{bmatrix} 0 & K \\ K^{*} & 0 \end{bmatrix}
% \preceq \begin{bmatrix} A & 0 \\ 0 & B \end{bmatrix}
% \end{align}
% for any $A,B \succ 0$ such that $\norm{A^{-1/2} K B^{-1/2}}_{\op} \leq 1$, which may be proved by computing Schur complements.
% We first apply the upper bound in \cref{eq:psd-bound} to the upper-left two-by-two block matrix.
% Taking $A = I_{d_1} \cdot \norm{M_{12}}_{\op}$ and $B = I_{d_2} \cdot \norm{M_{12}}_{\op}$,
% we find that
% \begin{align*}
%   M \preceq \begin{bmatrix}
%   M_{11} + I_{d_1} \cdot \norm{M_{12}}_{\op} & 0 & \quad M_{13} & \cdots & M_{1N} \\
%   0 & M_{22} + I_{d_2} \cdot \norm{M_{12}}_{\op} & \quad M_{23} & \cdots & M_{2N} \\
%   M_{31} & M_{32} & \quad M_{33} & \cdots & M_{3N} \\
%   \vdots & \vdots & \vdots & \;\;\;\;\ddots & \vdots \\
%   M_{N1} & M_{N2} & \quad\; M_{N3} & \cdots & M_{NN}
%   \end{bmatrix}.
% \end{align*}
% By sequentially apply this inequality to all other $2\times 2$ principal block submatrices, we obtain the desired upper bound.
% The lower bound is proved completely analogously.
% \end{proof}
% \begin{proof}[Proof of \cref{lem:expansion-convexity}]
% We apply \cref{lem:block-matrix} with $M = \frac{\nabla^2 f_\samp}{\tr \rho} - I_\H$, which we consider as a block matrix as in \cref{def:hess grad}.
% Recall from \cref{lem:hessian} that the off-diagonal blocks of the Hessian for $a \neq b\in[k]$ are given by $\nabla^2_{ab} f_x = \sqrt{d_a d_b} \Phi^{(ab)}$.
% Thus, since $\Phi^{(ab)}$ is an $(\eps,\lambda)$-quantum expander, we have
% \begin{align*}
%   \norm{M_{ab}}_{\op}
% = \frac{\norm{\nabla^2_{ab} f_x}_{\op}}{\tr\rho}
% = \frac{\sqrt{d_a d_b}}{\tr \Phi^{(ab)}(I_{d_b})} \norm{\Phi^{(ab)}}_0
% \leq \lambda,
% \end{align*}
% using that $\tr \Phi^{(ab)}(I_{d_b}) = \tr \rho$.
% The remaining off-diagonal elements can be bounded as
% \begin{align*}
% \norm{M_{a0}}
% = \frac{\norm{\nabla^2_{a0} f_x}_{\op}}{\tr \rho}
% &= \norm*{\sqrt{d_a} \left( \frac{\rho^{(a)}}{\tr \rho} - \frac{I_{d_a}}{d_a} \right)}_F
% = \sqrt{d_a} \norm*{\frac{\Phi^{(ab)}(I_{d_b})}{\tr \Phi^{(ab)}(I_{d_b})} - \frac{I_{d_a}}{d_a}}_F \\
% &\leq d_a \norm*{\frac{\Phi^{(ab)}(I_{d_b})}{\tr \Phi^{(ab)}(I_{d_b})} - \frac{I_{d_a}}{d_a}}_{\op}
% \leq \eps,
% \end{align*}
% using that $\Phi^{(ab)}(I_{d_b}) = \rho^{(a)}$.
% On the other hand, the diagonal blocks for $a\in[k]$ can be bounded by observing that, for any traceless Hermitian $H$,
% \begin{align*}
%   \abs{\braket{H, M_{aa} H}}
% &= \abs*{\braket{H, \left( \frac{\nabla^2_{aa} f_x}{\tr \rho} - I_\H \right) H}}
% = d_a \abs*{\tr \left( \frac{\rho^{(a)}}{\tr \rho} - \frac{I_{d_a}}{d_a} \right) H^2} \\
% &\leq d_a \norm*{\frac{\rho^{(a)}}{\tr \rho} - \frac{I_{d_a}}{d_a}}_{\op} \norm{H}_F^2
% \leq \eps \norm H_F^2,
% \end{align*}
% hence $\norm{M_{aa}}_{\op} \leq \eps$, while $\abs{M_{00}} = \abs{\frac{\nabla^2_{00} f_x}{\tr \rho} - 1} = 0$.
% Applying the bound from \cref{lem:block-matrix}, we obtain
% \begin{align*}
%   \norm{M}_{\op} = \max \{ k \eps, \eps + (k-1) \lambda \} \leq k(\eps+\lambda).
% \end{align*}
% % We next bound $\nabla_{aa}^2 f_\samp$. Recall again from \cref{lem:hessian} that the diagonal blocks of the Hessian are given by \begin{align}
% %  \langle Y,  \left( \nabla^2_{aa} f_{\samp} \right) Y \rangle=  d_a \tr \rho^{(a)} Y^2 .\label{eq:on-diag-hess-ii}
% % \end{align}
% % Again, the $\eps$-balancedness of $\Phi^{ab}$ implies $ |\tr (d_a \rho^{(a)}  - (\tr \rho) I_a)Y^2| \leq \eps (\tr \rho) \tr Y^2$, or that the first term of \cref{eq:on-diag-hess-ii} is in $(1 \pm \eps) \|Y\|_F^2 (\tr \rho)$. Hence $\|\nabla_{aa}^2 f_\samp - (\tr \rho) I\|_{op} \leq \eps^2 + \eps$. Finally, $\nabla_{00}^2 f_\samp$ is exactly $\tr \rho$. Using \cref{lem:block-matrix}, we find that the submatrix of $\nabla^2 f$ excluding the $0$ row and column tells us that dominates the block diagonal matrix where the $aa$ block is at least $(1 - \eps) (\tr \rho) I - (k-1) \lambda (\tr \rho) I $ for $a \in [k]$. Furthermore, the 00 block is exactly $ \tr \rho$. The analogous upper bound on $\nabla^2 f_\samp$ follows similarly. We now wish to apply \cref{lem:block-matrix} one more time, this time with only two diagonal blocks, one of which is the 00 block. It remains to bound the operator norm of the off diagonal block of this matrix, which we call $M_{\text{off}}$, consisting of $\nabla_{a 0}^2 f$ vertically concatenated together for $a \in [k]$.

% % First we bound $\|\nabla_{a0} f\|_{op}$. This is simply $\sqrt{d_a}\sup_{\|Z\|_F = 1} |\tr Z \rho^{(a)}|$ where $Z$ ranges over traceless Hermitians. Using that $\rho^{(a)} = \Phi^{ab}(I_{d_b})$, we have $\|d_a \rho^{(a)} - (\tr \rho) I_{d_a}\|_{op} \leq \eps \tr \rho $ by quantum expansion. For traceless $Z$ we have
% % \begin{align}\tr Z \rho^{(a)} = \frac{1}{d_a} \tr Z (d_a \rho^{(a)} - (\tr \rho) I_{d_a}) \leq \frac{\tr \rho}{d_a} \eps \|Z\|_1 \leq \frac{\tr \rho}{\sqrt{d_a}} \eps \|Z\|_F,\label{eq:second-term-hess}\end{align}
% % thus $\|\nabla_{a0} f\|_{op} \leq \eps \tr \rho$. It follows that $\|M_{\text{off}}\|_{op} \leq \sqrt{k} \eps \tr \rho$. Applying \cref{lem:block-matrix} to these two blocks shows $\nabla^2 f_\samp$ dominates a block diagonal matrix where each block is dominates $(1 - \eps) (\tr \rho) I - (k-1) \lambda (\tr \rho) I  - \sqrt{k} \eps (\tr \rho) I.$
%  \end{proof}

We are concerned with $\Phi^{(ab)}$ that arise from random Gaussians.
Just like random graphs can give rise to good expanders, it is known that random completely positive maps, namely~$\Phi$ constructed by choosing Kraus operators at random from various well-behaved distributions, yield good quantum expanders.
When the Kraus operators are chosen to be standard Gaussian we have the following result:

\begin{theorem}[\cite{pisier2012grothendieck,P14}]\label{thm:hess-pisier}
Let $A_1,\dots,A_N$ be independent $d_a\times d_b$ random matrices with independent standard Gaussian entries.
Then, for every $t \geq 2$, with probability at least~$1 - t^{-\Omega(d_a + d_b)}$, the completely positive map $\Phi_A$, defined as in \cref{eq:def cp}, satisfies
\begin{align*}
  \norm{\Phi_A}_0 \leq O\left(t^{2} \sqrt N \left( d_a + d_b \right)\right).
\end{align*}
\end{theorem}

Pisier's actual result is slightly different.
As stated, \cref{thm:hess-pisier} is an easy consequence Theorem~16.6 in~\cite{pisier2012grothendieck}, together with a standard symmetrization trick (see, e.g., the proof of Lemma~4.1 in~\cite{P14}).
We present the details in \cref{sec:pisier}.

When the samples $x=(x_1,\dots,x_n)$ are independent standard Gaussians in $\R^D$,
% then~$\rho^{(ab)}$ is distributed as $\frac1{nD} ZZ^T$, where $Z$ is a random $d_a d_b \times \frac{nD}{d_ad_b}$ matrix with independent standard Gaussian entries, as follows from the discussion above \cref{eq:two body partial trace duality}.
% Therefore,
the random completely positive maps $\Phi^{(ab)}$ we are interested in have the same distribution as~$\frac1{nD}\Phi_A$, where the Kraus operators~$A_1,\dots,A_N$ are $d_a \times d_b$ matrices with independent standard Gaussian entries and~$N=\frac{nD}{d_ad_b}$.
Accordingly, strong convexity at the identity follows quite easily from \cref{thm:hess-pisier} once double balancedness can be controlled.
For the latter, observe that
\begin{align*}
  \norm*{\frac{\Phi^{(ab)}(I_{d_b})}{\tr \Phi^{(ab)}(I_{d_b})} - \frac{I_{d_a}}{d_a}}_{\op}
= \frac1{\tr\rho} \norm*{\rho^{(a)} - \frac{\tr \rho}{d_a} I_{d_a}}_{\op}
% = \frac1{\tr\rho} \frac1{\sqrt{d_a}} \sqrt{d_a} \norm*{\rho^{(a)} - \frac{\tr \rho}{d_a} I_{d_a}}_{\op}
% = \frac1{\tr\rho} \frac1{\sqrt{d_a}} \norm*{\nabla_a f_x}_{\op}
= \frac1{1 + \nabla_0 f_x} \frac1{\sqrt{d_a}} \norm*{\nabla_a f_x}_{\op},
\end{align*}
by \cref{lem:gradient}, and similarly for the adjoint.
Therefore, the completely positive maps $\Phi^{(ab)}$ are $\eps$-doubly balanced if and only if, for all $a\in[k]$,
\begin{align}\label{eq:balanced via grad}
  % \frac1{1 + \nabla_0 f_x} \frac1{\sqrt{d_a}} \norm*{\nabla_a f_x}_{\op} \leq \frac\eps{d_a}
  \sqrt{d_a} \norm*{\nabla_a f_x}_{\op} \leq \eps \tr \rho = \left( 1 + \nabla_0 f_x \right) \eps,
\end{align}
hence double balancedness can be controlled using the gradient bounds in \cref{prop:gradient-bound}.

We now state and prove our strong convexity result at the identity:

\begin{theorem}[Strong convexity at identity]\label{thm:tensor-convexity}
There is a universal constant $C>0$ such that the following holds.
Let $x = (x_1,\dots,x_n)$ be independent standard Gaussian random variables in~$\R^D$, where $n \geq C k \frac{d_{\max}^2}D$.
Then, with probability at least~$1 - k^2 ( \frac{\sqrt{nD}}{k d_{\max}} )^{-\Omega(d_{\min})}$,
\begin{align*}
  \norm{\nabla^2 f_x - I_\H}_{\op} \leq \frac14;
\end{align*}
in particular, $f_x$ is $\frac34$-strongly convex at the identity.
\end{theorem}
\begin{proof}
By \cref{lem:expansion-convexity}, it is enough to prove that with the desired probability all $\Phi^{(ab)}$ are $(\eps,\lambda):=(\frac1{40 k^{1/2}},\frac1{20k})$-quantum expanders for $a\neq b\in[k]$ and $\tr \rho \in (\frac78,\frac98)$.
Indeed, then
\begin{align*}
  \norm*{\nabla^2 f_x - I_\H}_{\op}
&\leq \norm*{\frac{\nabla^2 f_x}{\tr \rho} - I_\H}_{\op} \tr \rho + \abs{1 - \tr\rho} \\
&\leq \left( (k-1)\lambda + (1 + \sqrt k) \eps \right) \tr\rho + \abs{1 - \tr\rho}
% &\leq \left( \frac1{20} + \frac1{20} \right) \frac98 + \frac18
% = \frac1{10} \frac98 + \frac18
\leq \frac14.
\end{align*}
Firstly, $\tr \rho = \frac{1}{nD} \|X\|^2$ is in $(\frac78, \frac98)$ with failure probability $e^{-\Omega(nD)}$ by \cref{prp:xnorm}.

Next, we describe an event that implies the $\Phi^{(ab)}$ are all $\eps$-balanced for $\eps=\frac1{40k^{1/2}}$.
By \cref{eq:balanced via grad}, this is equivalent to the condition $\sqrt{d_a} \norm{\nabla_a f_{\rv}}_{\op} \leq \eps \tr \rho$ for all $a \in [k]$.
By \cref{prop:gradient-bound}, and assuming the bound $\tr \rho \geq \frac78$ from above, the latter occurs with failure probability at most~$k \smash{e^{-\Omega(\frac{nD}{k d_{\max}})}}$ provided $n \geq C k \smash{\frac{d_{\max}^2}D}$ for a universal constant~$C>0$.

Finally, we describe an event that ensures that $\norm{\Phi^{(ab)}}_{\op} \leq \lambda \smash{\frac{\tr\rho}{\sqrt{d_a d_b}}}$ for $\lambda=\frac1{20k}$ for any fixed~$a \neq b$, which is the other condition needed for quantum expansion.
Recall that each~$\Phi^{(ab)}$ is distributed as $\frac1{nD} \Phi_A$, where $A$ is a tuple of $\frac{nD}{d_ad_b}$ many $d_a \times d_b$ matrices with independent standard Gaussian entries.
Thus, taking $t^2 = O(\smash{\frac{\lambda \sqrt{nD}}{d_a + d_b}})$ and again assuming that $\tr\rho \geq \frac78$, we have $\norm{\Phi^{(ab)}}_{\op} \leq \lambda \frac{\tr\rho}{\sqrt{d_a d_b}}$
% Take t^2 = \frac1{X} \frac78 \frac{\lambda \sqrt{nD}}{d_a + d_b}, with $X$ the universal constant from Pisier's theorem
% \begin{align*}
%   \norm{\Phi^{(ab)}}_0
% \leq \frac1{nD} \norm{\Phi_A}_0
% \leq \frac1{nD} X t^{2} \sqrt N \left( d_a + d_b \right) \\
% \leq \frac78 \frac1{nD} \lambda \sqrt{nD} \sqrt N
% \leq \frac78\frac1{nD} \lambda \sqrt{nD} \sqrt{nD} \frac1{\sqrt{d_ad_b}}
% = \frac78\lambda \frac1{\sqrt{d_ad_b}}
% \leq \lambda \frac{\tr\rho}{\sqrt{d_ad_b}}
% \end{align*}
with failure probability at most~$\smash{( \frac{\sqrt{nD}}{k d_{\max}} )^{-\Omega(d_{\min})}}$.
% \begin{align*}
%   \left( \frac{\lambda \sqrt{nD}}{d_a + d_b} \right)^{-\Omega(d_a + d_b)}
% \leq \left( \frac{20 k (d_a + d_b)}{\sqrt{nD}} \right)^{\Omega(d_a + d_b)}
% \leq \left( \frac{40 k d_{\max}}{\sqrt{nD}} \right)^{\Omega(d_a + d_b)} \\
% \leq \left( \frac{40 k d_{\max}}{\sqrt{nD}} \right)^{\Omega(2d_{\min})} \\
% \lesssim \left( \frac{k d_{\max}}{\sqrt{nD}} \right)^{\Omega(d_{\min})}
% \end{align*}

By the union bound, we conclude that all $\Phi^{(ab)}$ for $a\neq b$ are $(\eps,\lambda)$-quantum expanders and that $\tr\rho \in (\frac78,\frac98)$, up to a failure probability of at most
\begin{align*}
  e^{-\Omega(nD)}
+ k \smash{e^{-\Omega\bigl(\frac{nD}{k d_{\max}}\bigr)}}
+ k^2 \left( \frac{\sqrt{nD}}{k d_{\max}} \right)^{-\Omega(d_{\min})}.
\end{align*}
The final term dominates
% \MW{I don't think we need the assumption on $n$ to see this.}
% Indeed, we have
% \begin{align*}
%   nD \geq \frac{nD}{k d_{\max}} \geq d_{\min} \log \frac{\sqrt{nD}}{k d_{\max}}.
% \end{align*}
% To see the latter inequality, note that it is nontrivial only when
% \begin{align*}
%   \sqrt{nD} \geq k d_{\max},
% \end{align*}
% in which case we have
% \begin{align*}
%   \frac{nD}{k d_{\max}}
% \geq \frac{k^2 d_{\max}^2}{k d_{\max}}
% = k d_{\max}
% \geq d_{\min}.
% \end{align*}
and so we obtain the desired bound on the failure probability.
\end{proof}

We now show our second strong convexity result, namely that if our function is strongly convex at the identity then it is also strongly convex in an operator norm ball about the identity.
The proof is given in \cref{app:robust}.

\begin{lemma}[Robustness]\label{convexRobustness}
There is a universal constant $\eps_0>0$ such that if $f_x$ is $\lambda$-strongly convex at~$I_D$, $\norm{\nabla_a f_x(I_D)}_{\op} \leq \eps_0/\sqrt{d_a}$ for all $a\in[k]$ and $\abs{\nabla_{0} f_{\samp}(I_{D})} \leq \eps_0$, then~$f_x$ is $(\lambda-O(\delta))$-strongly convex at any $\Theta\in\P$ such that $\delta := \norm{\log\Theta}_{\op} \leq \eps_0$.
\end{lemma}

Finally we obtain our strong convexity result near the identity.

\begin{theorem}[Strong convexity near identity]\label{thm:ball-convexity}
There are universal constants $C,c>0$ such that the following holds.
Let $x = (x_1,\dots,x_n)$ be independent standard Gaussian random variables in~$\R^D$, where $n \geq C k \frac{d_{\max}^2}D$.
Then, with probability at least~$1 - k^2 ( \frac{\sqrt{nD}}{k d_{\max}} )^{-\Omega(d_{\min})}$,
the function~$f_x$ is $\frac12$-strongly convex at any point $\Theta\in\P$ such that $\norm{\log\Theta}_{\op} \leq c$.
\end{theorem}
\begin{proof}
We can choose $C>0$ such that both \cref{prop:gradient-bound,thm:tensor-convexity} apply (the former with $\eps\leq\eps_0/9$, where $\eps_0$ is the universal constant from \cref{convexRobustness}).
Then the assumptions of \cref{convexRobustness} are satisfied for $\lambda=\frac34$ with failure probability at most
\begin{align*}
  2(k+1)e^{-\eps^2 \frac{nD}{8d_{\max}}} + k^2 \left( \frac{\sqrt{nD}}{k d_{\max}} \right)^{-\Omega(d_{\min})},
\end{align*}
where the latter term dominates, and there exists a constant $0<c\leq\eps_0$ such that $f$ is $\frac12$-strongly convex at any point $\Theta$ such that $\norm{\log\Theta}_{\op} \leq c$.
\end{proof}

While \cref{thm:ball-convexity} uses the operator norm to quantify closeness to the identity, we can easily translate it into a statement in terms of the geodesic distance on~$\P$.
Namely, under the same hypotheses it holds that~$f_x$ is $\frac12$-strongly convex on the geodesic ball~$B_r(I_D)$ of radius $r=\smash{\frac c{\sqrt{(k+1)d_{\max}}}}$.
% Indeed, if $\Theta = \exp_{I_D}(H)$, then
% \begin{align*}
%   \norm{\log\Theta}_{\op}
% % = \norm{H_0 I_D + \sum_{a=1}^k \sqrt{d_a} H_{(a)}}_{\op}
% \leq \abs{H_0} + \sum_{a=1}^k \sqrt{d_a} \norm{H_a}_{\op}
% \leq \sqrt{d_{\max}} \left( \abs{H_0} + \sum_{a=1}^k \norm{H_a}_{\op} \right) \\
% \leq \sqrt{d_{\max}} \left( \abs{H_0} + \sum_{a=1}^k \norm{H_a}_F \right)
% \leq \sqrt{d_{\max}} \sqrt{k+1} \norm{H}_F.
% \end{align*}

%-----------------------------------------------------------------------------
\subsection{Proof of Theorem~\ref{thm:tensor-frobenius}}
%-----------------------------------------------------------------------------
We are now ready to prove the main result of this section according to the plan outlined in \cref{subsec:proof-sketch}.
We restate the theorem for convenience.

\begin{customthm}{\ref{thm:tensor-frobenius}}[Tensor normal Frobenius error, restated]
\TensorFrob{equation*}{\tag{\ref{eq:eps sqr assm}}}
\end{customthm}
\begin{proof}
By step~\ref{it:reduce} in \cref{subsec:proof-sketch}, it is enough to prove the theorem assuming $\Theta = I_D$.
Assuming this, we now show that the minimizer of $f_\rv$ exists and is close to $\Theta = I_D$ with high probability.
Let~$c>0$ be the constants from \cref{thm:ball-convexity}.
For
\begin{equation}\label{eq:def delta}
  \delta = \frac{\sqrt{82 k} \, d_{\max} }{\sqrt{nD}} \eps,
\end{equation}
consider the following two events:
\begin{enumerate}
\item\label{it:grad-bd} $\norm{\nabla f_x}_F \leq \delta$.
\item\label{it:sc-ball} $f_\rv$ is $\frac12$-strongly convex on the geodesic ball $B_r(I_D)$ of radius $r=\smash{\frac c{\sqrt{(k+1) d_{\max}}}}$.
\end{enumerate}
By our assumption~\eqref{eq:eps sqr assm}, if we choose the constant~$C>0$ large enough, both \cref{prop:gradient-bound} (with the parameter $\eps$ in the proposition set to $\delta/\sqrt{ 82 k}$) and \cref{thm:ball-convexity} apply, with the former showing the first and the latter showing the second event. The desired success probability then follows from the union bound. By \cref{lem:convex-ball}, these two events, together with our assumption~\eqref{eq:eps sqr assm} (again for~$C>0$ large enough), imply the MLE~$\htheta$ exists, is unique, and has geodesic distance at most $2\delta$ from~$I_D$.

We now show that $\htheta$ satisfies the desired bounds.
Note that if $\htheta = \exp_{I_D}(H)$ for $H \in \H$, the equal-determinant Kronecker factors are given by $\htheta_a = \smash{e^{\frac{H_0}{d_a k}} e^{\sqrt{d_a} H_a}}$ for all~$a\in[k]$.
Thus,
\begin{align*}
  \norm{\log \htheta_a}_F^2
&= \norm{\frac{H_0}{d_a k} I_{d_a} + \sqrt{d_a} H_a}_F^2
= \frac{\abs{H_0}^2}{d_a k^2} + \norm{\sqrt{d_a} H_a}_F^2 \\
&\leq d_a \left( \abs{H_0}^2 + \norm{H_a}_F^2 \right)
\leq d_a \norm H_F^2
\leq 4 d_a \delta^2.
\end{align*}
Our assumption~\eqref{eq:eps sqr assm} implies (again choosing $C$ large enough) that $4 d_a \delta^2 \leq 1$, so we obtain
\begin{align*}
  D_F(\htheta_a\Vert I_a)
= \norm{I_a - \htheta_a}_F
\leq 2 \norm{\log\htheta_a}_F
\leq 4 \sqrt{d_a} \, \delta
= O\left( \sqrt k \, \frac{\sqrt{d_a} \, d_{\max} }{\sqrt{nD}} \eps \right),
\end{align*}
which establishes the first bound.
The second now follows easily by a telescoping sum:
\begin{align*}
  D_F(\htheta\Vert I_D)
&\leq \sum_{a=1}^k \norm{\htheta_1}_F \cdots \norm{\htheta_{a-1}}_F \norm{I_{d_a} - \htheta_a}_F \norm{I_{d_{a+1}}}_F \cdots \norm{I_{d_k}}_F \\
% &= \sum_{a=1}^k \sqrt{d_{a+1}\cdots d_k} \norm{\htheta_1}_F \cdots \norm{\htheta_{a-1}}_F \norm{I_{d_a} - \htheta_a}_F \\
&\leq 2 \sqrt{D} \sum_{a=1}^k (1 + 2\delta)^{a-1} \delta
\leq 2 k e^{2\delta k} \sqrt{D} \delta
= O\left( k^{3/2} \frac{d_{\max} }{\sqrt{n}} \eps \right).
\end{align*}
\end{proof}


%=============================================================================
\section{Improvements for the matrix normal model}\label{sec:matrix-normal}
%=============================================================================
We now prove \cref{thm:matrix-normal}, which improves over \cref{thm:tensor-frobenius} in the case of the matrix normal model ($k=2$).
Our results for the matrix normal model are stronger in that
\begin{enumerate}
\item the MLE is shown to be close to the truth \emph{in spectral norm} rather than the looser Frobenius norm,
\item the errors are \emph{tight for the individual factors}, and
\item the failure probability is \emph{inverse exponential} in the number of samples rather than inverse polynomial.
\end{enumerate}

The proof plan is similar to that in \cref{subsec:proof-sketch}, but to work directly with quantum expansion instead of translating into strong convexity.
Our main tool is a bound by \cite{KLR19} which uses the notion of a \emph{spectral gap}.

\begin{definition}[Spectral gap]
Let $\Phi\colon\Mat(d_b) \to \Mat(d_a)$ be a completely positive map.
Say $\Phi$ has \emph{spectral gap} $\gamma$ if
\begin{align}\label{eq:spectral-gap}
  \sigma_2(\Phi) \leq (1 - \gamma) \frac{\tr \Phi(I_{d_b})}{\sqrt{d_a d_b}}
\end{align}
where $\sigma_2$ denotes the second largest singular value of~$\Phi$.
\end{definition}

It is known that this notion is closely related to quantum expansion.

\begin{lemma}[Lemma~A.3 in \cite{FM20}]\label{lem:fm20}
There exists a universal constant $c>0$ with the following property.
If $\Phi$ is an $(\eps,\lambda)$-quantum expander and $\eps \leq c(1-\lambda)$, then~$\Phi$ has spectral gap~$1-\lambda-O(\eps)$.
\end{lemma}

We now state the bound of~\cite{KLR19} in our language.
Because~$k = 2$, the gradient and Hessian are completely described by the completely positive map~$\Phi^{(12)}$ (compare \cref{lem:gradient,lem:hessian} with \cref{eq:hessian channel}).

\begin{theorem}[Theorem 1.8, Proof of Theorem~3.22 in \cite{KLR19}]\label{thm:klr}
There is a universal constant $C>0$ such that the following holds.
If the completely positive map $\Phi^{(12)}$ defined in \cref{eq:hessian channel} is $\eps$-doubly balanced and has spectral gap~$\gamma$, where $\gamma^2 \geq C \eps \log d_{\min}$, then, restricted to $\SSPD(d_1) \times \SSPD(d_2)$, where $\SSPD(d)$ denotes the $d\times d$ positive definite matrices of unit determinant, the function~$f_x$ has a unique maximizer $(P_1,P_2)$ such that
\begin{align*}
  \max \{ \norm{P_1 - I_{d_1}}_{\op}, \norm{P_2 - I_{d_2}}_{\op} \} = O\Bigl(\frac{\eps \log d_{\min}}\gamma\Bigr).
\end{align*}
Moreover, $f_x(P_1, P_2) \geq (1 - \frac{4 \eps^2}{\lambda}) \frac{\|x\|^2}{nD}$. \MW{XX}
\end{theorem}

\begin{corollary}\label{cor:klr}
There is a universal constant $C>0$ such that the following holds. Suppose 
%$\|x\| = \sqrt{nD}$ and 
$0 \leq \eps, \gamma \leq 1$. Suppose the completely positive map $\Phi^{(12)}$ defined in \cref{eq:hessian channel} is $\eps$-doubly balanced and has spectral gap~$\gamma$, where $\gamma^2 \geq C \eps \log d_{\min}.$ Further suppose that $\|x\|^2/nD = 1 + \delta$ where $|\delta| \leq .5$.  Then the MLE's $\htheta_1, \htheta_2$ satisfy
\begin{align*}
  \max \{ \norm{\htheta_1 - I_{d_1}}_{\op}, \norm{\htheta_2 - I_{d_2}}_{\op} \} = O\Bigl(\delta + \frac{\eps \log d_{\min}}\gamma\Bigr).
\end{align*}
\end{corollary}
\begin{proof}
To compute the MLE, we reparametrize by $\Theta_1 = P_1$ and $\Theta_2 = \lambda P_2$ where $P_1,P_2 \in \SPD(d_1) \times \SPD(d_2)$. Plugging this reparametrization into $f_x$ shows $\lambda, P_1, P_2$ solve
$$\arg\max_{\lambda, P_1, P_2} \lambda f_x(P_1, P_2) - \log \lambda.$$
In particular, the MLE $\htheta_1, \htheta_2$ exists uniquely if $f_x$ restricted to $\SPD(d_1) \times \SPD(d_2)$ has unique minimizers $P_1, P_2$. Such unique maximizers exist by \cref{thm:klr}. Given $P_1, P_2$, solving the simple one-dimensional optimization problem for $\lambda$ yields $\lambda = 1/f_x(P_1, P_2)$. We have $\htheta = P_1$, and
$$ \|\htheta_2 - I_{d_2} \|_{op} = \|\lambda P_2  - I_{d_2} \|_{op} \leq \lambda \|P_2 - I_{d_2}\|_{op} + |1 - \lambda|.$$
By \cref{thm:klr}, $nD/\|x\|^2  \leq \lambda \leq (1 + 4 \eps^2/\gamma)^{-1} (nD/\|x\|^2).$ By our assumptions on $\gamma, \eps$, we have $\eps^2/\gamma \leq \eps/\gamma \leq \eps/\gamma^2$. Thus $\lambda = (1 + O(\eps^2/\gamma))(1+ O(\delta)) = (1 + O(\eps^2/\gamma +\delta)) = O(1)$ if $C$ is large enough, so $\lambda \|P_2 - I_{d_2}\|_{op} = O(\|P_2 - I_{d_2}\|_{op}) = O(\eps^2 \log d_{\min}/\gamma)$ by \cref{thm:klr}. $|1 - \lambda| = O(\eps^2/\gamma +\delta) = O(\eps^2 \log d_{\min}/\gamma)$. Combining these bounds completes the proof.\end{proof}

% the MLE $\htheta = \htheta_1 \ot \htheta_2$ for $\Theta=I_{d_1} \ot I_{d_2}$ satisfies
% where the Kronecker factors $\htheta_1$, $\htheta_2$ are chosen such that $\det\htheta_1=\det\htheta_2$.
%\CF{look ok? I believe we only ever need to apply this for vectors with the desired norm, so the assumption $\|x\| = \sqrt{nD}$ doesn't hurt and  makes things cleaner.}

\MW{Once this is done I will check the following bounds:}
{\color{red}\Cref{lem:fm20,thm:klr}, along with what we have shown so far, already implies a preliminary version of \cref{thm:matrix-normal}.
In proving \cref{thm:tensor-convexity}, we've already shown that $\Phi$ is an $(\eps \sqrt{\frac{d_{\max}}{n d_{\min}}}, 1 - \lambda)$-quantum expander with failure probability
\[ O(e^{ - \Omega( d_{\max} \eps^2)}) + \left( \frac{\sqrt{nD}}{d_{\min}} \right)^{ - \Omega(d_{\min})}. \]
By \cref{thm:klr}, we immediately have that with the above failure probability the MLEs satisfy
\begin{align*}
  D_{\op}(\Theta'_a \Vert \Theta_a) = O\left(\eps \sqrt{\frac{d_{\max}}{n d_{\min}}} \log d_{\min}\right),
\end{align*}
which matches \cref{thm:matrix-normal} for the larger Kronecker factor.}

One of the main results of this section is the following theorem, which shows that the expansion constant $\lambda$ of $\Phi$ can be made constant with \emph{exponentially small} failure probability, albeit with a worse constant $\lambda$.
Recall that for the matrix model, the samples $x_i$ can be viewed a $d_1 \times d_2$-matrices, which we denote by $X_i$.

\begin{theorem}\label{thm:operator-cheeger}
There are universal constants $C > 0$ and $0<\lambda<1$ such that the following holds.
Let $X=(X_1,\dots,X_n)$ be random $d_1 \times d_2$ matrices with independent standard Gaussian entries, where $n \geq C \frac{d_{\max}}{d_{\min}} \max\{\log \frac{d_{\max}}{d_{\min}}, \frac1{\eps^2} \}$.
Then, $\Phi_X$ is an $(\eps \sqrt{\frac{d_{\max}}{n d_{\min}}}, \lambda)$-quantum expander with probability at least $1 - e^{ - \Omega( d_{\min} \eps^2)}$.
\end{theorem}

We will prove \cref{thm:operator-cheeger} in \cref{app:cheeky} using techniques similar to \cite{FM20} using Cheeger's inequality.
This also improves our result on strong convexity (\cref{thm:ball-convexity}), which will be useful in the analysis of the flip-flop algorithm. Indeed, for $k = 2$, using \cref{thm:operator-cheeger} (in place of \cref{thm:hess-pisier}) with \cref{lem:expansion-convexity} in the proof of \cref{thm:tensor-convexity} improves the failure probability in \cref{thm:tensor-convexity} to $1 - e^{ - \Omega( d_{\min} \eps^2)}$. As in the proof of \cref{thm:ball-convexity}, combining this failure probability bound with \cref{convexRobustness} yields the next corollary. \CF{ok? I think instead of going into more detail, we'd restructure section 3 so we have a theorem like (expansion)->(convex in a ball). Otherwise things will get redundant}

\begin{corollary}\label{cor:matrix-convexity}
There are universal constants $C, c > 0$ and $\lambda\in(0,1)$ such that the following holds.
Let $x=(x_1,\dots,x_n)$ be independent standard Gaussian random variables in~$\R^{d_1d_2}$, where $n \geq C \frac{d_{\max}}{d_{\min}} \max\{\log \frac{d_{\max}}{d_{\min}}, \frac1{\eps^2} \}$.
Then, with probability at least $1 - e^{ - \Omega( d_{\min} \eps^2)}$, the function~$f_x$ is $(1-\lambda)$-strongly convex at any point $\Theta\in\P$ such that $\norm{\log\Theta}_{\op} \leq c$.
\end{corollary}



%-----------------------------------------------------------------------------
\subsection{Proof of Theorem~\ref{thm:matrix-normal}}
%-----------------------------------------------------------------------------
We now use \cref{thm:operator-cheeger} as well as some more refined concentration inequalities to prove \cref{thm:matrix-normal}.
The additional concentration is required to obtain the tighter bounds on the smaller Kronecker factor.
Throughout this section, we assume without loss of generality that $d_1 \leq d_2$.

The idea of the proof is to apply one step of the flip-flop algorithm to ``renormalize'' the samples such that the second partial trace is proportional to $I_{d_2}$.
This has the effect of making the second component of the gradient~$\nabla f_\samp$ equal to zero.
We will show that the first component still enjoys the same concentration exploited in \cref{prop:gradient-bound} even with the step of flip-flop -- thus the total gradient has become smaller, but only the second component of the estimate has changed.
Thus, intuitively, the total change in the first component will be small.
Using \cref{convexRobustness} to control the change induced in the minimal eigenvalue of the Hessian by the first step of the flip-flop and applying \cref{lem:convex-ball} results in a Frobenius error proportional to the new gradient after flip-flop, which gives us the tighter bound.
To obtain a relative spectral error bound, we employ a similar strategy but with \cref{thm:klr} instead of \cref{lem:convex-ball}.

We now discuss the concentration bound.
Let $X_1,\dots,X_n$ be random $d_1 \times d_2$ matrices with independent standard Gaussian entries.
% Note that
% \begin{align*}
%   \Phi^{(12)}(\cdot) = \frac1{nD} \sum_{i=1}^n X_i (\cdot) X_i^T.
% \end{align*}
Consider new random variables $Y_i$ obtained by right-multiplying $X_i$ by the square root of the result of one step of the flip-flop algorithm for the second, larger Kronecker factor.
That is:
\begin{align*}
  Y_i = X_i \left( \frac1{nd_1} \sum_{i=1}^n X_i^T X_i \right)^{-1/2}.
\end{align*}
The completely positive map $\Phi^{(12)}$ corresponding to the ``renormalized'' samples $Y_1,\dots,Y_n$ is $\frac1{nD} \Phi_Y$.
By construction, it satisfies
\begin{align}\label{eq:one-step}
  \frac1{n D} \Phi_Y(I_{d_2}) = \frac1{d_2} X_i \left( \sum_{i=1}^n X_i^T X_i \right)^{-1} X_i
\quad\text{and}\quad
  \frac1{n D} \Phi^*_Y(I_{d_1}) = \frac{I_{d_2}}{d_2}.
\end{align}
Note also that $\tr \Phi_Y(I_{d_2}) = \tr \Phi_Y^*(I_{d_1}) = \|Y\|^2 = nD$. Thus $\Phi_Y$ is $\delta$-doubly balanced if and only if $\|\frac{1}{nD} \Phi_Y(I_{d_2}) - \frac{1}{d_1} I_{d_1}\|_{op} \leq \frac{\delta}{d_1}$.
\MW{Isn't there a factor $1/d_1$ missing on the right?}\CF{still, or did I fix it?}

\begin{lemma}[Concentration after flip-flop]\label{lem:flipflop-concentration}
There is a universal constant $C>0$ such that the following holds.
Let $X_1,\dots,X_n$ be random $d_1 \times d_2$ matrices with independent standard Gaussian entries, where $d_1 \leq d_2$.
If $n \geq \frac{d_2}{d_1}$ and $\eps\geq C$, then for $\Phi_Y$ as in \cref{eq:one-step} we have, with probability at least $1 - \exp(- \Omega( \eps^2 d_{1}))$,
\begin{align*}
  \norm*{\frac1{n D} \Phi_Y(I_{d_2}) - \frac{I_{d_1}}{d_1}}_{\op} \leq \eps \sqrt{\frac1{nD}}.
\end{align*}
By the remarks preceding the lemma, this condition implies $\Phi_Y$ is $\eps \sqrt{\frac{d_1}{nd_2}}$-doubly balanced.
\end{lemma}

The proof of this lemma uses a result of \cite{hayden2006aspects} on the overlap of two random projections, combined with a standard net argument.
The details can be found in \cref{app:flipflop-concentration}.
With this bound in hand, we may now prove \cref{thm:matrix-normal}.


\begin{customthm}{\ref{thm:matrix-normal}}[Matrix normal spectral error, restated]
\MatrixSpec
\end{customthm}
\begin{proof}
\MW{I'll go through this once the above is sanity checked.}
Recall that we define $\htheta_1(\samp), \htheta_2(\samp)$ to be $\arg\min_{\det \htheta_1 = 1} f_\samp(\htheta_1 \ot \htheta_2)$. By the equivariance discussed in \cref{subsec:proof-sketch}, it is enough to prove \cref{thm:matrix-normal} under the assumption $\Theta_a = I_{d_a}$ for $a \in \{1,2\}$.

Consider the following events.
\begin{enumerate}
\item\label{it:expander} The operator $\Phi_{\rv}$ is a $(\eps  \sqrt{{d_2}/{n d_1}},1-\lambda)$-quantum expander.
\item\label{it:flop-balanced} The operator $\tilde{\Phi}$ is $\eps \sqrt{{d_1}/{n d_2}}$-balanced.
\item \label{it:norm} $|1 - \frac{\|X\|}{\sqrt{nD}}| \leq \frac{\eps}{\sqrt{d_2}}. $
\end{enumerate}
These events occur with failure probability $1 - O(e^{ - \Omega( d_1 \eps^2)}).$ Indeed, by the assumption
$$n \geq C (d_2/d_1) \max\{\log (d_2/d_1),  \eps^{-2}\} $$
\cref{it:expander} occurs with probability $1 - O(e^{ - \Omega( d_1 \eps^2)})$ by \cref{thm:operator-cheeger}. By \cref{lem:flipflop-concentration}, \cref{it:flop-balanced} occurs with probability $1 - O(e^{ - \Omega( d_1 \eps^2)}).$ Finally, \cref{it:norm} occurs with probability $1 - e^{- \Omega(n d_1 \eps^2)}$ by \cref{prp:xnorm}. By the union bound all the events occur with probability $1 - O(e^{ - \Omega( d_1 \eps^2)})$. Let $\samp$ satisfy all three properties.

We now use a lemma relating the quantum expansion of $\tilde{\Phi}$ and $\Phi_\samp$; this is analogous to \cref{convexRobustness} guaranteeing the strong convexity of $f$ in a neighborhood of the origin. Recall that $y = (\sqrt{ nd_1} I_{d_1} \ot \Phi_x^*(I_2)^{-1/2}) x:= (I_{d_1} \ot B)x$ and $\tilde{\Phi} = \Phi_y$. By Lemma 4.4 in \cite{FM20}, if $\kappa(B^2) - 1 \leq \delta < c$,  then $\Phi_y$ is an $(\eps \sqrt{d_2/n d_1} + O(\delta), 1 - \lambda + O(\delta))$-quantum expander. We have $\kappa(B^2) = \kappa(\|x\|^{-2} \Phi_\samp^*(I_{d_1})) = O(\|\|x\|^{-2} \Phi_\samp^*(I_{d_1}) - I_{d_2}\|_{op}) = O(\eps \sqrt{d_2/nd_1})$ by the balancedness of $\Phi_x$. Thus $\tilde{\Phi}$ is an $(\eps \sqrt{d_1 / n d_2}, 1 - \lambda/2)$-quantum expander provided $\eps \sqrt{d_2/nd_1}$ is small enough compared to $\lambda$.

By our choice of $n$, we have $\eps \sqrt{d_1/nd_2} \leq c\lambda^2\log d_1$ provided $\eps \leq c\lambda^2$. Thus \cref{thm:klr} applies to $\tilde{\Phi}$ and so the MLE $\htheta_1(y), \htheta_2(y)$ satisfy
\begin{gather*} \| \htheta_1(y) - I_{d_1}\|_{op}, \| \htheta_2(y) - I_{d_2}\|_{op} = O\left(\eps \sqrt{\frac{d_1}{n d_2}} \log d_1\right).\end{gather*}
By the equivariance of the MLE we have $\htheta_1(x) = \htheta_1(y)$ and $\htheta_2(x) = B \htheta_2 (y) B$. This immediately yields the bound $ D_{op}(\htheta_1(x) \rVert I_{d_1}) \leq \eps \sqrt{\frac{d_1}{n d_2}} \log d_1.$
To bound $D_{op}(\htheta_1(x) \rVert I_{d_1})$, use invariance of $D_{op}$ and the approximate triangle inequality (\cref{lem:triangle-ineq}) to write
$$D_{op}(\htheta_1(x) \rVert I_{d_2}) = D_{op}(\htheta_1(x) \rVert B^{-2}) \leq D_{op}(\htheta_1(x)\rVert I_{d_2}) + D_{op}(B^{-2}\rVert I_{d_2}).$$
We have already shown that the first term is $O(\eps \sqrt{\frac{d_1}{n d_2}}).$ Writing $B^{-2} = \frac{\|x\|^2}{nd_1 d_2} \frac{d_2 \Phi^*(I_{d_1})}{\|x\|^2}$, the second term is $O(\eps \sqrt{\frac{d_2}{n d_1}})$ by \cref{it:norm} and \cref{it:expander}, completing the proof.  By setting the constant $C$ in the statement of the theorem large enough compared to $1/\lambda, 1/c$, all the constraints will be satisfied.
\end{proof}










%=============================================================================
\section{Convergence of flip-flop algorithms}
%=============================================================================

In this section we prove that the flip-flop algorithms for the matrix and tensor normal models converge quickly to the MLE with high probability. 

Before we formally state the flip-flop algorithm with the proper parameters, we need the following lemma that proves that when the gradient of the log-likelihood function is small, then it means we are close to the MLE.

%\begin{lemma}[Inequality]\label{lem:inequality}
%	Let $0 < \delta < 1/8$ and $0 < x_1 , \ldots, x_n < e^{4 \delta}$ be such that $\prod_{i=1}^n x_i = 1$. Then, we have that
%	$$ \sum_{i=1}^n (x_i-1)^2 \leq 16 \delta n $$
%\end{lemma}
%
%\RMO{Maybe move the proof of this inequality to the appendix? It is very marginal}
%
%\begin{proof}
%	Without loss of generality, we can assume that
%	$$0 < x_1 \leq \cdots \leq x_\ell \leq 1 \leq x_{\ell+1} \leq \cdots \leq x_n < e^{4\delta}. $$
%	Then, we have:
%	$$ \sum_{i=1}^n (x_i-1)^2 \leq \sum_{i=1}^\ell 2 - 2x_i + \sum_{i = \ell+1}^n 64 \delta^2. $$
%	Using that $\prod_{i=1}^n x_i = 1$ and $x_j < e^{4 \delta}$ for $j > \ell$, we get
%	$$e^{-4 \delta n } < e^{-4 \delta(n-\ell)} < \prod_{i=1}^\ell x_i \leq \left( \dfrac{\sum_{i=1}^\ell x_i}{\ell} \right)^\ell $$
%	which implies that $\displaystyle \sum_{i=1}^\ell x_i > \ell \cdot e^{-4 \delta n/\ell } \geq \ell (1- 4 \delta n /\ell)$, where the last inequality follows from $e^t \geq 1 + t$ for $t \leq 0$.
%
%	Putting everything together, we get
%	\begin{align*}
%		\sum_{i=1}^n (x_i-1)^2 &\leq \sum_{i=1}^\ell 2 - 2x_i + \sum_{i = \ell+1}^n 64 \delta^2 \\
%		&< 2 \ell - 2\ell (1-4 \delta n/\ell) + 64 \delta^2 \cdot (n - \ell) \\
%		&< 8 \delta n + 64 \delta^2 n = 8\delta n (1 + 8 \delta) \leq 16 \delta n
%	\end{align*}
%\end{proof}

\begin{lemma}[Gradient and distance to MLE]\label{lem:tensor-distance-opt}
	Let $\htheta$ be the MLE for $\Theta$, $\delta, \lambda \in \R_+$ be  parameters such that
	$0 < \delta < \lambda \dmax^{-1/2}/2$ and $\otheta$ be a scaling which satisfies $\delta = \norm{\nabla f_x(\otheta)}_F$. If $f_x$ is $\lambda$-geodesic convex over a geodesic ball containing $\htheta$ and $\otheta$, then for each component, we have
%	$$ D_F(\htheta \ || \ \otheta) \leq \left( 16 \delta D \cdot \sum_{a=1}^k \sqrt{d_a} \right)^{1/2} $$
	$$ D_{op}(\htheta_a \ || \ \otheta_a) \leq \sqrt{d_a} \cdot \frac{4\delta}{\lambda}  \ \ \text{ and } \ \ D_F(\htheta_a \ || \ \otheta_a) \leq d_a^{3/2} \cdot \frac{4\delta}{\lambda}. $$
\end{lemma}

\begin{proof}
	Let $Z \in \Sym$ be such that $\htheta = \otheta^{1/2} \cdot \exp(\sqrt{\vec d} \cdot Z) \cdot \otheta^{1/2}$, and $\oZ = Z / \norm{Z}_F$.
	Since
	$$ D_F(\htheta_a \ || \ \otheta_a) = \norm{I_a - \otheta_a^{-1/2} \htheta_a \otheta_a^{-1/2}}_F = \norm{I_a - \exp(\sqrt{d_a} \cdot Z_a)}_F $$
	and similarly
	$$ D_{op}(\htheta_a \ || \ \otheta_a) = \norm{I_a - \exp(\sqrt{d_a} \cdot Z_a)}_{op} $$
	to prove a good bound on the distance it is enough to show that $\norm{Z}_F$ is small.
	We will achieve this by using the strong geodesic convexity of $f_x$.

	Since $f_x$ is $\lambda$-strongly geodesically convex over geodesic ball containing $\htheta$ and $\otheta$,	we have that the function $g(t) = f_x(\exp_{\otheta} (t \oZ) )$ is $\lambda$-strongly convex, $g(0) = \otheta$ and $g(\norm{Z}_F) = \htheta$, which implies:
	$$ g(\norm{Z}_F) \geq g(0) + g'(0) \cdot \norm{Z}_F + \lambda \cdot \frac{\norm{Z}_F^2}{2}. $$
	Since $\htheta$ is the MLE estimator, we have $g(\norm{Z}_F) \leq g(0)$, and by definition of
	$g$ and Cauchy-Schwarz, we have that
	$$ g'(0) = \langle \nabla f_x(\otheta), \oZ \rangle \geq - \norm{\nabla f_x(\otheta)}_F \norm{\oZ}_F = - \norm{\nabla f_x(\otheta)}_F. $$
	Putting all the above together, we get
	$$ \norm{Z}_F \leq \frac{2}{\lambda} \cdot \norm{\nabla f_x(\otheta)}_F. $$
	Setting $\delta = \norm{\nabla f_x(\otheta)}_F$, the above inequality implies that $- \frac{2\delta}{\lambda} I_a \preceq Z_a \preceq \frac{2\delta}{\lambda} I_a$ for each $a \in [k]$. This in turn, together with $\delta < \lambda \dmax^{-1/2}/2$, yields
	$$ \exp(\sqrt{d_a} Z_a) \preceq \exp\left(\sqrt{d_a} \cdot \frac{2\delta}{\lambda} \cdot I_a \right) 
	= e^{\sqrt{d_a} \cdot \frac{2\delta}{\lambda}} \cdot I_a 
	\preceq \left(  1 + \sqrt{d_a} \cdot \frac{4\delta}{\lambda} \right) \cdot I_a, $$
	And similarly, $\left(  1 - \sqrt{d_a} \cdot \frac{4\delta}{\lambda} \right) \cdot I_a \preceq \exp(\sqrt{d_a} Z_a)$.	
	
	Thus, our distances become:
%	$$ D_F(\htheta \ || \ \otheta) =  \norm{\exp(\sqrt{\vec d} Z) - I_D}_F \leq \left( 16 \delta D \cdot \sum_{a=1}^k \sqrt{d_a} \right)^{1/2}. $$
	$$ D_{op}(\htheta_a \ || \ \otheta_a) =  \norm{\exp(\sqrt{d_a} Z_a) - I_a}_{op} \leq \sqrt{d_a} \cdot \frac{4\delta}{\lambda}. $$
	and 
	$$ D_F(\htheta_a \ || \ \otheta_a) =  \norm{\exp(\sqrt{d_a} Z_a) - I_a}_F \leq d_a^{3/2} \cdot \frac{4\delta}{\lambda}. $$
\end{proof}


We can now describe the flip-flop algorithm.


\begin{Algorithm}
\textbf{Input}: Samples $\samp = (\samp_1, \ldots, \samp_n)$ where $\samp_i \in \R^D$ is sampled from a (unknown) centered normal distribution $\cN(0, \Sigma)$, where each entry of $\samp_i$ is encoded in binary, with bit size $\le b$. \\ Approximation parameter $0 < \delta < 1$, given in binary representation. \\[.3ex]

\textbf{Output}: $\otheta \in \SPD$ such that $\norm{\nabla_a f_x(\otheta)}_F \leq \delta$ for each $a \in [k]$ \\[.3ex]

\textbf{Algorithm}:
\begin{enumerate}
\item\label{it:flip-flop step 1} Set $\otheta_a = I_a$ for each $a \in [k]$. 
%$\delta = \dfrac{\eps^2}{16 k^2 \dmax^{3/2}}$. for tensor normal model

\vspace{5pt}

\item\label{it:flip-flop step 2} For $t=1,\dots,T = 24 k \cdot \log(1/\delta)$, repeat the following:
%we need T = \log(1/\epsilon) \cdot 1/\alpha \lambda

\vspace{5pt}

\begin{itemize}
\item Compute each component of the gradient $\nabla f_{\samp}(\otheta_1, \ldots, \otheta_k)$, denoting $\nabla_a := \nabla_a f_{\samp}(\otheta_1, \ldots, \otheta_k)$, and find the index $a \in [k]$ for which $\norm{\nabla_a}_F$ is largest.

\vspace{5pt}

\item
If $\norm{\nabla_a}_F < \delta$, output $\otheta$ and return.
% S\Bigl(\frac {I_{n_{k^{t}}}{n_{k^{t}}}} \big\Vert \rho^{t}_{k^{t}}\Bigr)$ is largest.

\vspace{5pt}

\item Otherwise, set $\otheta_a \leftarrow  \dfrac{1}{d_a} \cdot \otheta_a^{1/2} (\rho^{(a)})^{-1} \cdot \otheta_a^{1/2}$, where $\rho = \dfrac{1}{nD} \cdot  \otheta^{1/2} \left( \sum_{i=1}^n x_ix_i^\dagger \right) \otheta^{1/2}.$
\end{itemize}
\end{enumerate}
\caption{Generic flip-flop algorithm}\label{alg:flip-flop}
\end{Algorithm}

Before we analyze the convergence of the flip-flop algorithms for the tensor and matrix normal models, we discuss the straightforward generalization of convergence of general descent methods whenever the objective function is strongly geodesically convex.

The next lemma shows that any descent method which manages to decrease the value of the function with respect to the gradient...
The proof of the lemma is the same as the one from~\cite[Lemma 4.8]{FM20}.
\MW{How about moving this to right after \cref{lem:convex-ball}, since it's also g-convexity general nonsense? I might go ahead and move it (and proofread it as I do so).}

\RMO{Actually, let's leave it here, since my plan now, as you can see, is to do a generic analysis of flip-flop given geodesic convexity, and then plug-and play with the initial conditions. Then we already have a bunch of generic descent lemmas here anyways... But you tell me what you prefer :)  }

\begin{lemma}\label{lem:descent-sublevel-set}
	Let $f : \SPD \rightarrow \R$ be $\lambda$-strongly geodesically convex in a sublevel set containing $x_0$, $\norm{\nabla f(x_0)}_F^2 \leq \nu \leq 1$ and $\alpha > 0$ be a constant.
	If $\{x_t\}$ is a descent sequence which satisfies
	$$ f(x_{t+1}) \leq f(x_t) - \alpha \cdot \min\{1,  \norm{\nabla f(x_t)}^2_F \} $$
	then in $T$ iterations we must have an element $x_r$ with $r \leq T$ such that
	$$ \norm{\nabla f(x_r)}^2_F \leq \nu \cdot 2^{- T \alpha \lambda}.   $$
\end{lemma}

\begin{proof}
	Let $f^*$ be the minimum value of the function $f$ and let $S$ be the sublevel set of $f$ containing $x_0$ over which $f$ is $\lambda$-strongly geodesically convex. Since $\{x_t\}$ is a descent sequence, we know that each $x_t \in S$.

	Since $f$ is $\lambda$-strongly geodesically convex in $S$, we have
	$$ f^* \geq f(x) - \frac{1}{2\lambda} \cdot \norm{\nabla f(x)}_F^2 $$
	for any $x \in S$, and in particular for any $x_t$ in our descent sequence.

	If $\norm{\nabla f(x_t)}_F^2 \leq \varepsilon \leq 1$, then we will show that in $\ell \leq 1/\alpha \lambda$ steps we must have an element $x_{t+\ell}$ such that $\norm{\nabla f(x_{t + \ell})}_F^2 \leq \varepsilon/2$. This is enough to conclude the proof of the lemma, as with this claim we see that we half the squared norm of the gradient at every sequence of $1/\alpha \lambda$ steps.

	To see this, assume that $\norm{\nabla f(x_{t+\ell})}_F^2 \geq \varepsilon/2$ for $0 \leq \ell \leq m$. 
	Then, from our descent property we have
	$$ f(x_{t+a+1}) \leq f(x_{t+a}) - \alpha \cdot \norm{\nabla f(x_{t+a})}^2_F \leq f(x_{t+a}) - \alpha \cdot \varepsilon/2$$
	and therefore $f(x_{t + m}) \leq f(x_t) - m \cdot \alpha \cdot \varepsilon/2$.

	On the other hand, our assumption that $\norm{\nabla f(x_t)}_F^2 \leq \varepsilon$, together with strong geodesic convexity of $f$ and minimality of $f^*$ imply
	$$ f(x_t) - \frac{\varepsilon}{2\lambda} \leq f(x_t) - \frac{1}{2\lambda} \cdot \norm{\nabla f(x_t)}_F^2 \leq f^* \leq f(x_{t+m}) $$
	and therefore we have
	$$ f(x_t) - \frac{\varepsilon}{2\lambda} \leq f(x_{t + m}) \leq f(x_t) - m \cdot \alpha \cdot \varepsilon/2 $$
	which implies $m \leq \frac{1}{\alpha \lambda}$. This concludes our proof.
\end{proof}

The following proposition shows that after the first step of the flip-flop algorithm, the trace part of the log-likelihood remains unchanged by the flip-flop algorithm.
Thus, the proposition below proves that $\nabla_0 f_x(\Theta) = 0$ after the first iteration. Henceforth, we shall assume that the trace part of the log-likelihood function is 1.

\begin{prop}[Trace Invariance]\label{prop:trace-invariance}
	Let $\Theta$ and $\Upsilon$ be successive scalings produced by the flip-flop algorithm, and let 
	$$\rho_\Theta := \dfrac{1}{nD} \cdot \Theta^{1/2} \cdot \sum_{i=1}^n x_i x_i^\dagger \cdot \Theta^{1/2} \ \text{ and } \ \rho_\Upsilon := \dfrac{1}{nD} \cdot \Upsilon^{1/2} \cdot \sum_{i=1}^n x_i x_i^\dagger \cdot \Upsilon^{1/2}.$$ 
	If $\tr \rho_\Theta = 1$, then $\tr \rho_\Upsilon = 1$.
\end{prop}

\begin{proof}
	Since $\Upsilon$ is the successive scaling from $\Theta$, we have that
	there is $a \in [k]$ such that
	$$\Upsilon_a = \dfrac{1}{d_a} \cdot \Theta_a^{1/2} (\rho_\Theta^{(a)})^{-1} \Theta_a^{1/2} $$
	and $\Upsilon_b = \Theta_b$ for all $b \neq a$. Let $E_{(a)} = I_1 \otimes \cdots \otimes I_{a-1} \otimes (\rho_\Theta^{(a)})^{-1} \otimes I_{a+1} \otimes \cdots \otimes I_k$.
	Thus, we have:
	\begin{align*}
		\tr \rho_\Upsilon 
		&= \dfrac{1}{nD} \cdot \tr\left[ \Upsilon^{1/2} \sum_{i=1}^n x_i x_i^\dagger \cdot \Upsilon^{1/2} \right] 
		= \dfrac{1}{nD} \cdot \tr\left[ \Upsilon \sum_{i=1}^n x_i x_i^\dagger \right] \\
		&= \dfrac{1}{nD} \cdot \tr\left[ \dfrac{1}{d_a} \cdot \Theta^{1/2} E_{(a)} \Theta^{1/2} \cdot \sum_{i=1}^n x_i x_i^\dagger \right] 
		= \dfrac{1}{d_a} \cdot \tr\left[ E_{(a)} \rho_\Theta \right] \\
		&= \dfrac{1}{d_a} \cdot \tr\left[ (\rho_\Theta^{(a)})^{-1} \rho_\Theta^{(a)} \right] = 1
	\end{align*}
\end{proof}

\begin{lemma}[Descent Lemma]\label{lem:tensor-descent-lemma}
	If $\Theta, \Upsilon$ are successive scalings from the flip-flop algorithm, such that $\nabla_0 f(\Theta) = \nabla_0 f(\Upsilon) = 0$, then we have:
	$$ f_x(\Upsilon) \leq f_x(\Theta) - \dfrac{1}{6} \cdot \min\left\{ \dfrac{1}{\dmax} \ , \ \dfrac{\norm{\nabla f_x(\Theta)}_F^2}{k} \right\} $$
\end{lemma}

\begin{proof}
	Let
	$$\rho := \dfrac{1}{nD} \cdot  \Theta^{1/2} \cdot \left( \sum_{i=1}^n x_i x_i^\dagger \right) \cdot \Theta^{1/2}.$$
	Additionally, let $a \in [k]$ be such that $\nabla_a := \nabla_a f_x(\Theta)$ has largest norm.
	As $\Upsilon$ is the successive scaling, we have that $\Upsilon_b = \Theta_b$ when $b \neq a$ and
	$$ \Upsilon_a = \dfrac{1}{d_a} \cdot \Theta_a^{1/2} \cdot (\rho^{(a)})^{-1} \cdot \Theta_a^{1/2}. $$
	In particular, the above means that we can write $\Upsilon$ in the following way:
	$$ \Upsilon = \Theta^{1/2} \cdot  E_{(a)} \cdot \Theta^{1/2} $$
	where $E_{(a)} = I_1 \otimes \cdots \otimes I_{a-1} \otimes \left( \dfrac{1}{d_a} \cdot (\rho^{(a)})^{-1} \right) \otimes I_{a+1} \otimes \cdots \otimes I_k$.
	Hence, we have:
	\begin{align*}
		f_x(\Upsilon) &= \dfrac{1}{nD} \sum_{i=1}^n \langle x_i , \Upsilon x_i \rangle - \dfrac{1}{D} \log \det(\Upsilon) \\
		&= \dfrac{1}{nD} \tr\left[ \Upsilon \cdot  \sum_{i=1}^n x_i x_i^\dagger  \right] - \dfrac{1}{D} \log \det(\Upsilon) \\
		&= \dfrac{1}{nD} \tr\left[ \Theta^{1/2} \cdot  E_{(a)} \cdot \Theta^{1/2}  \cdot  \sum_{i=1}^n x_i x_i^\dagger  \right] - \dfrac{1}{D} \log \det(\Theta^{1/2} \cdot  E_{(a)} \cdot \Theta^{1/2}) \\
		&= \tr\left[  E_{(a)} \cdot \rho  \right] - \dfrac{1}{D} \left( \log \det(E_{(a)}) + \log\det(\Theta) \right) \\
		&= \tr\left[ \dfrac{1}{d_a} \cdot (\rho^{(a)})^{-1} \cdot \rho^{(a)}  \right] - \dfrac{1}{D} \left( \log \det(E_{(a)}) + \log\det(\Theta) \right) \\
		&= f_x(\Theta) + \dfrac{1}{D} \cdot \log \det\left( E_{(a)}^{-1} \right) \\
		&= f_x(\Theta) + \frac{1}{d_a} \cdot  \log\det\left( d_a \cdot \rho^{(a)} \right)
	\end{align*}
	Where in the second to last equality we used \cref{lem:gradient} and the assumption $\nabla_0 f(\Theta) = 0$ to get $\tr \rho = 1$. 
	
	Lemma 5.1 from~\cite{GGOW19} states that for any $d$-dimensional PSD matrix $Z$ of trace $d$, the following inequality holds:
	$$ \log\det(Z) \leq \max\left\{- \dfrac{\norm{Z - I_d}_F^2}{6}, - \dfrac{1}{6} \right\}. $$
	Since $\tr \rho^{(a)} = \tr \rho = 1$, if $\norm{d_a \rho^{(a)} - I_a}_F \leq 1$ we obtain that:
	\begin{align*}
		\frac{1}{d_a} \cdot  \log\det\left(d_a \cdot \rho^{(a)} \right) &\leq - \dfrac{\norm{d_a \cdot \rho^{(a)} - I_a}_F^2}{6 d_a} \\
		&= - \dfrac{d_a \cdot \norm{ \rho^{(a)} - \frac{\tr \rho}{d_a} I_a}_F^2}{6} = - \dfrac{\norm{\nabla_a}_F^2}{6} \leq - \dfrac{\norm{\nabla f_x(\Theta)}_F^2}{6k}
	\end{align*}
	If $\norm{d_a \cdot \rho^{(a)} - I_a}_F > 1$, we have
	\begin{align*}
		\frac{1}{d_a} \cdot  \log\det(d_a \cdot \rho^{(a)}) &\leq - \dfrac{1}{6 d_a} \leq - \dfrac{1}{6 \dmax}
	\end{align*}
\end{proof}

We now have all the lemmas we need to prove that, given appropriate initial conditions on the input samples, the flip-flop algorithm will converge quickly to the MLE.

\begin{lemma}[Fast Convergence from Initial Conditions]\label{lem:fast-convergence-initial-generic}
	Let $x_1, \ldots, x_n \in \R^D$ be samples satisfying the following conditions:
	\begin{enumerate}
		\item $\norm{\nabla f_x (I_D)}_F < \dfrac{r \lambda}{2} \leq 1$
		\item $f_x$ is $\lambda$-strongly geodesically convex at $B_r(I_D)$
	\end{enumerate}
	Then, in $T = \dfrac{12k\dmax}{\lambda} \cdot \log\left( \dfrac{4 \dmax^{3/2}}{\eps \lambda} \right)$ iterations, \cref{alg:flip-flop} outputs an estimator $\otheta$ such that $\forall \ a \in [k]$
	$$ D_F(\htheta_a \ || \ \otheta_a) \leq \eps $$
\end{lemma}

\begin{proof}
	The initial conditions above imply that \cref{lem:convex-ball} applies, and therefore we have that the sublevel set $\{ \Upsilon \ \mid \ f(\Upsilon) \leq f(I_D)  \}$ is contained in the ball $B_r(I_D)$. 

	In particular, the above condition on the sublevel set implies that \cref{lem:tensor-descent-lemma} applies, and thus we have that each step of the flip-flop algorithm will decrease the value of the objective function in accordance with the requirements of \cref{lem:descent-sublevel-set}, with parameter $\alpha = 1/6k \dmax$.

	Thus, after $T = \dfrac{12k\dmax}{\lambda} \cdot \log(1/\delta)$ steps, \cref{lem:descent-sublevel-set} guarantees us that we will encounter a scaling $\otheta$ such that
	$$ \norm{\nabla f_x(\otheta)}_F \leq \delta.$$
	Setting $\delta = \dfrac{\lambda \eps}{4 \dmax^{3/2}}$, when we find such a scaling with gradient $\leq \delta$, \cref{lem:tensor-distance-opt} implies that for each $a \in [k]$, the component-wise distance from $\otheta$ to $\htheta$ is bounded by
	$$ D_F(\htheta_a \ || \ \otheta_a) \leq \eps. $$
	This in particular implies that $D_{op}(\htheta_a \ || \ \otheta_a) \leq \eps$
\end{proof}


%-----------------------------------------------------------------------------
\subsection{Tensor flip-flop: algorithm and convergence}
%-----------------------------------------------------------------------------

We are now ready to state the tensor flip-flop algorithm and prove its fast convergence to the MLE.

\begin{Algorithm}
\textbf{Input}: Samples $\samp = (\samp_1, \ldots, \samp_n)$ where $\samp_i \in \R^D$ is sampled from a (unknown) centered normal distribution $\cN(0, \Sigma)$, where each entry of $\samp_i$ is encoded in binary, with bit size $\le b$. \\ Approximation parameter $0 < \eps < 1$, given in binary representation. \\[.3ex]

\textbf{Output}: $\otheta \in \SPD$ such that $D_F(\htheta_a \ || \ \otheta_a) < \eps$, for each $a \in [k]$, where $\htheta$ is the MLE for the precision matrix of $\Sigma$. \\[.3ex]

\textbf{Algorithm}:
\begin{enumerate}
\item\label{it:flip-flop step 1} Set $\otheta_a = I_a$ for each $a \in [k]$, and 
$\delta = \dfrac{\eps}{8 \dmax^{3/2}}$. 

\vspace{5pt}

\item\label{it:flip-flop step 2} For $t=1,\dots,T = 24 k \dmax \cdot \log(1/\delta)$, repeat the following:

\vspace{5pt}

\begin{itemize}
\item Compute each component of the gradient $\nabla f_{\samp}(\otheta_1, \ldots, \otheta_k)$, denoting $\nabla_a := \nabla_a f_{\samp}(\otheta_1, \ldots, \otheta_k)$, and find the index $a \in [k]$ for which $\norm{\nabla_a}_F$ is largest.

\vspace{5pt}

\item
If $\norm{\nabla_a}_F < \delta$, output $\otheta$ and return.
% S\Bigl(\frac {I_{n_{k^{t}}}{n_{k^{t}}}} \big\Vert \rho^{t}_{k^{t}}\Bigr)$ is largest.

\vspace{5pt}

\item Otherwise, set $\otheta_a \leftarrow  \dfrac{1}{d_a} \cdot \otheta_a^{1/2} (\rho^{(a)})^{-1} \cdot \otheta_a^{1/2}$, where $\rho = \dfrac{1}{nD} \cdot \otheta^{1/2} \left( \sum_{i=1}^n x_ix_i^\dagger \right) \otheta^{1/2}.$
\end{itemize}
\end{enumerate}
\caption{Tensor flip-flop algorithm}\label{alg:tensor-flip-flop}
\end{Algorithm}



\begin{lemma}[Initial Conditions]\label{lem:tensor-initial-conditions}
	There exist absolute constants $\Gamma, \gamma > 0$ such that the following holds.
	When the number of samples $n \geq \Gamma \cdot k^2 \cdot \dmax^3/D$, with probability at least $1 - k^2 \cdot \left( \dfrac{\sqrt{nD}}{k \dmax} \right)^{-\Omega(\dmin)} - 2k \cdot e^{- \Omega(nD/k \dmax^2)}$ we have that the following conditions hold:
	\begin{enumerate}
		\item $\norm{\nabla f_x(I)}_F < \dfrac{\gamma}{4 \sqrt{(k+1)\dmax}}$ 
		\item $f_x$ is $\frac{1}{2}$-strongly geodesically convex at 
		$B_r(I_D)$, where $r = \dfrac{\gamma}{\sqrt{(k+1) \dmax}}$
	\end{enumerate}
\end{lemma}

\begin{proof}
	The lemma follows from the observation that \cref{prop:gradient-bound} implies condition 1, and \cref{thm:ball-convexity} implies condition 2. So all we need to do is to check the parameters.

	By \cref{thm:ball-convexity}, if we set $\gamma = c$ and if the number of samples $n \geq C k \dfrac{d_{\max}^2}{D}$, where $c, C > 0$ are the constants from \cref{thm:ball-convexity}, then with probability at most
	$$k^2 \cdot \left( \dfrac{\sqrt{nD}}{k \dmax} \right)^{-\Omega(\dmin)}$$
	the second condition fails to hold.

	By \cref{prop:gradient-bound} with parameter 
	$\varepsilon = \frac{\gamma}{100 k \dmax^{1/2}}$, 
	if the number of samples satisfies 
	$n \geq \dfrac{10^4 k^2 \dmax^3}{\gamma^2 \cdot D}$ then with probability at most
	$$2 k \cdot \exp\left(- \frac{n D \gamma^2}{128 (k+1) \dmax^2}\right) = 2k \cdot e^{- \Omega(nD/k \dmax^2)}$$
	the first condition will fail to hold.

	Letting $\Gamma = \max\{10^4/\gamma^2, C \}$, having $n \geq \Gamma k^2 \dmax^3/D$ samples gives a sample upper bound that holds for both situations above.
	Thus, by the union bound, with probability at most
	$$ k^2 \cdot \left( \dfrac{\sqrt{nD}}{k \dmax} \right)^{-\Omega(\dmin)} - 2k \cdot e^{- \Omega(nD/k \dmax^2)} $$
	one of the conditions 1 or 2 will fail to hold. This concludes our proof.
\end{proof}

\begin{customthm}{\ref{thm:tensor-flipflop}}[Tensor flip-flop convergence, restated]
\TensorFlop
\end{customthm}


\begin{proof}
	When the number of samples is $n = \Omega(k^2 \cdot \dmax^3/D)$, with probability 
	$$ 1 - k^2 \cdot \left( \dfrac{\sqrt{nD}}{k \dmax} \right)^{-\Omega(\dmin)} - 2k \cdot e^{- \Omega(nD/k \dmax^2)}$$ 
	we have that the hypothesis of \cref{lem:tensor-initial-conditions} applies, which implies that there exists a constant $\gamma > 0$ such that our objective function $f_x$ is $\frac{1}{2}$-strongly geodesically convex at a ball $B_r(I)$ for $r = \dfrac{\gamma}{\sqrt{(k+1)\dmax}}$ and $\norm{\nabla f_x(I)}_F \leq \dfrac{\gamma}{4 \sqrt{(k+1)\dmax}}$.
	
Thus, by \cref{lem:fast-convergence-initial-generic}, we have that in $T = 24k \dmax \cdot \log(8 \dmax^{3/2}/\eps)$ steps the flip-flop algorithm converges to an estimator such that $D_F(\htheta_a \ || \ \otheta_a) \leq \eps$ for all $a \in [k]$. Thus, \cref{alg:tensor-flip-flop} will output a good estimator.
\end{proof}

%-----------------------------------------------------------------------------
\subsection{Matrix flip-flop convergence}
%-----------------------------------------------------------------------------

We are now ready to state the matrix flip-flop algorithm and prove its fast convergence to the MLE. 
The proof strategy of this section is a bit different from the tensor normal model case, as now, the number of samples is not large enough to guarantee that the initial conditions from \cref{lem:fast-convergence-initial-generic} will apply with high probability.

However, we can proceed as in~\cite{FM20} and use the results from~\cite{KLR19} to show that the MLE is in a constant ball (in operator norm) around the identity. 
Then, using the fact that our log-likelihood is strongly geodesically convex around the MLE, we can use~\cite[Lemma 4.7]{FM20} to prove that any scaling with small gradient of the log-likelihood function will be close to the MLE.
Thus, from our initial samples, after we apply flip-flop for a large enough number of iterations, we will obtain a scaling where the log-likelihood function is strongly geodesically convex in its sublevel set around the minimizer, and thereby we can now apply \cref{lem:descent-sublevel-set} and obtain an $\varepsilon$-minimizer in $O(\log(1/\varepsilon))$ iterations.

 

\begin{Algorithm}
\textbf{Input}: Samples $\samp = (\samp_1, \ldots, \samp_n)$ where $\samp_i \in \R^{d_1 \times d_2}$ is sampled from a (unknown) centered normal distribution $\cN(0, \Sigma)$, where each entry of $\samp_i$ is encoded in binary, with bit size $\le b$. \\ 
Approximation parameter $\eps > 0$, given in binary representation. \\[.3ex]

\textbf{Output}: $\otheta \in \SPD$ s.th. $D_{\op}(\htheta_a \ || \  \otheta_a) < \eps$, for $a \in \{1,2\}$, where $\htheta$ is the MLE for the precision matrix of $\Sigma$. \\[.3ex]

\textbf{Algorithm}:
\begin{enumerate}
\item\label{it:flip-flop step 1 matrix} Let $\gamma, \lambda$ be the constants from \cref{lem:matrix-normal-initial-conditions}. Set $\otheta_a = I_a$ for each $a \in \{1,2\}$, and $\delta = \dfrac{\eps \lambda}{4 \sqrt{\dmax}}$.

\vspace{5pt}

\item\label{it:flip-flop step 2 matrix} For $t=1,\dots,T = 60 \dmax \cdot \left( \dfrac{16}{\lambda \gamma} \right)^2 + \dfrac{24 \dmax}{\lambda} \cdot \log(1/\delta)$, repeat the following:
%we need T = \log(1/\epsilon) \cdot 1/\alpha \lambda

\vspace{5pt}

\begin{itemize}
\item Compute each component of the gradient $\nabla f_{\samp}(\otheta_1, \otheta_2)$, denoting $\nabla_a := \nabla_a f_{\samp}(\otheta_1, \otheta_2)$, and find the index $a \in \{1,2\}$ for which $\norm{\nabla_a}_{op}$ is non-zero.

\vspace{5pt}

\item
If $\norm{\nabla_a}_{op} < \delta$, output $\otheta$ and return.
% S\Bigl(\frac {I_{n_{k^{t}}}{n_{k^{t}}}} \big\Vert \rho^{t}_{k^{t}}\Bigr)$ is largest.

\vspace{5pt}

\item Otherwise, set $\otheta_a \leftarrow \dfrac{1}{d_a} \cdot \otheta_a^{1/2} \cdot (\rho^{(a)})^{-1} \cdot \otheta_a^{1/2}$, where $\rho = \dfrac{1}{n d_1 d_2} \otheta^{1/2} \cdot \left( \sum_{i=1}^n x_i x_i^\dagger \right) \cdot \otheta^{1/2}$.
\end{itemize}
\end{enumerate}
\caption{Matrix flip-flop algorithm}\label{alg:flip-flop matrix}
\end{Algorithm}


\begin{lemma}[Initial Conditions Matrix Normal Model]\label{lem:matrix-normal-initial-conditions}
There exist absolute constants $\Gamma > 0$ and $\gamma, \lambda \in (0,1]$ such that the following holds.
	When the number of samples 
	$$n \geq \Gamma \cdot \dfrac{\dmax}{\dmin} \cdot \max\left\{ \log \dfrac{\dmax}{\dmin}, \ \dfrac{\log^2 \dmin}{\varepsilon^2}  \right\},$$ 
	with probability at least $1 - e^{- \Omega(\dmin \varepsilon^2)}$ we have that the following conditions hold:
	\begin{enumerate}
		\item $|\nabla_0 f_x(I)| \leq \frac{\gamma}{2}$
		\item The MLE's $\htheta_1, \htheta_2$ satisfy $\ \ \norm{\htheta_i - I_{d_i}}_{op} \leq \gamma/2, \ \ $ for $i \in \{1,2\}$
		\item $f_x$ is $\lambda$-strongly geodesically convex at any $\Theta \in \SPD$ such that $\norm{\log \Theta}_{op} \leq \gamma$.
	\end{enumerate}
\end{lemma}

\begin{proof}
We start by establishing condition 3, so that we may obtain the value of $\gamma$.

\noindent Let $C_1, c_1 > 0, \eta \in (0,1)$ be the universal constants from \cref{cor:matrix-convexity}, and set $\gamma = \min\{1, c_1\}$. 
If $\Gamma > C_1$, with probability 
$\geq 1 - e^{- \Omega(\dmin \varepsilon^2)}$ we have that $f_x$ is
$(1-\eta)$-strongly convex at $\Theta \in \SPD$ such that 
$\norm{\log(\Theta)}_{op} \leq \gamma$. Setting $\lambda = 1-\eta$ we obtain condition 3.


By \cref{prop:gradient-bound}, so long as $\Gamma \geq 4/\gamma$, with probability $\geq 1 - 6 e^{- \gamma^2 n \dmin/32}$ we have that 
$$ |\nabla_0 f_x(I)| \leq  \frac{\gamma}{2}. $$ 

To obtain condition 2, Let $C_2 > 0$ and $\nu \in (0,1)$ be the universal constants from \cref{thm:operator-cheeger}.
With the number of samples as above, so long as $\Gamma > C_2$, with probability $\geq 1 - e^{- \Omega(\dmin \varepsilon^2)}$ we obtain that 
$\Phi_x$ is an $\left(\varepsilon \sqrt{\dfrac{\dmin}{n \dmax}}, \nu \right)$-quantum expander. Note that with our number of samples, we have
$$ \varepsilon \sqrt{\dfrac{\dmin}{n \dmax}} \leq \dfrac{\varepsilon^2}{\sqrt{\Gamma} \cdot \log \dmin}.  $$
If $C_3, c_3 > 1$ are the universal constants from \cref{cor:klr}, so long as 
$\Gamma \geq \dfrac{4 C_3^2 c_3^2}{\gamma^2 (1-\nu)^4}$, we have that
$\Phi_x^{(12)}$ has spectral gap $1 - \nu$ and it satisfies
$$ C_3 \log \dmin \cdot \varepsilon \sqrt{\dfrac{\dmin}{n \dmax}} \leq C_3 \log \dmin \cdot  \dfrac{\varepsilon^2}{\sqrt{\Gamma} \cdot \log \dmin} \leq \dfrac{\varepsilon^2 \gamma (1-\nu)^2}{2} \leq (1-\nu)^2. $$

Thus, the hypotheses of \cref{cor:klr} apply and we have that the MLE's $\htheta_1, \htheta_2$ satisfy: 
$$ \norm{\htheta_i - I_{d_i}}_{op} \leq c_3 \cdot \varepsilon \sqrt{\dfrac{\dmin}{n \dmax}} \cdot \dfrac{\log \dmin}{1-\nu} \leq c_3 \cdot \dfrac{\varepsilon^2}{\sqrt{\Gamma} \cdot \log \dmin} \cdot \dfrac{\log \dmin}{1-\nu} \leq \dfrac{\gamma}{2}. $$

Thus, so long as $\Gamma \geq \max\left\{ C_1, C_2, \ \dfrac{4 C_3^2 c_3^2}{\gamma^2 (1-\nu)^4}  \right\}$, then all of the conditions above are satisfied on the required number of samples, and by a union bound all three events above happen with probability at least
$$ 1 - 6 e^{- \gamma^2 n \dmin/32} - e^{- \Omega(\dmin \varepsilon^2)} = 1 - e^{- \Omega(\dmin \varepsilon^2)}. $$
\end{proof}

To prove that flip-flop works once the initial conditions are satisfied, we need the following general lemma (\cite[Lemma 4.7]{FM20}) on strongly geodesically convex functions, which tells us that once our gradient is small then we must be inside a sublevel set of our function which is contained in a ball where our function is strongly convex. Though their lemma is stated for the manifold of positive definite matrices of determinant one, the proof works in an arbitrary Hadamard manifold (such as the manifold $\SPD$ considered here).

\begin{lemma}[\cite{FM20}]\label{lem:gradient-strong-convexity-fm}
Suppose that $f : \SPD \to \R$ is geodesically $\lambda$-strongly convex on the geodesic ball of radius $\kappa$ about $\theta$, and that $\nabla f(\theta) = 0$. 
If $\norm{\nabla f(\Upsilon)}_F < \lambda \kappa/8$, then $\Upsilon$ is contained in a sublevel set of $f$ on which $f$ is geodesically $\lambda$-strongly convex.
\end{lemma}

\RMO{Should I put your proof here, Cole? It is not quite stated in the same setting as ours, but it is so close...}

Now the objective is simply to show that the flip-flop algorithm reaches a scaling with small enough gradient relatively quickly, which is given by the following lemma, which follows the analysis given by~\cite{GGOW19}:

\begin{lemma}\label{lem:flip-flop-sinkhorn}
Given $\delta >0$ and samples $x = (x_1, \ldots, x_n)$, let $f^* := \min_{\otheta \in \SPD} f_x(\otheta)$ be the minimizer of the log-likelihood function. The flip-flop algorithm, in at most 
$$T = 24 \dmax \cdot  (f_x(I_D) - f^* + |\nabla_0 f_x(I)|) \cdot \dfrac{1}{\delta^2} \ \text{ iterations,}$$
reaches a scaling $\otheta$ such that $\norm{\nabla f_x(\otheta)}_F < \delta$.
\end{lemma} 
\begin{proof}
Let $\otheta^{(i)}$ be the scalings obtained by the flip-flop algorithm, where $\otheta^{(0)} = I_D$.
After we perform the first normalization step of the flip-flop algorithm, we obtain a scaling $\otheta^{(1)}$ such that $\nabla_0 f_x (\otheta^{(1)}) = 0$. 
By \cref{prop:trace-invariance}, after each subsequent step, that is $\otheta^{(i)}$ for $i \geq 2$, we maintain that $\nabla_0 f_x (\otheta^{(i)}) = 0$. 
Thus, \Cref{lem:tensor-descent-lemma} applies and we obtain that 
$$ f_x(\otheta^{(1)}) - f_x(\otheta^{(T)}) \geq \dfrac{1}{12 \dmax} \sum_{i=1}^{T-1} \min\{ 1 , \norm{\nabla f_x(\otheta^{(i)})}_F^2 \}. $$
Let $\alpha^2 := \dfrac{1}{nD} \sum_{i=1}^n x_i^\dagger x_i$. 
\Cref{lem:tensor-descent-lemma} and $\nabla_0 f_x(\alpha I) = 0$ imply $f_x(\alpha I_D) - f_x(\otheta^{(1)}) \geq 0$. 
Moreover, since $f_x(I_D) - f_x(\otheta^{(1)}) = f_x(I_D) - f_x(\alpha I_D) + f_x(\alpha I_D) - f_x(\otheta^{(1)})$, we have
$$ f_x(I_D) - f_x(\otheta^{(1)}) \geq f_x(I_D) - f_x(\alpha I_D) \geq - |\nabla_0 f_x(I)|. $$
The above inequalities imply that 
$$ f_x(I_D) - f^* \geq f_x(I_D) - f_x(\otheta) \geq - |\nabla_0 f_x(I)| + \dfrac{1}{12 \dmax} \sum_{i=1}^{T-1} \min\{ 1 , \norm{\nabla f_x(\otheta^{(i)})}_F^2 \}. $$
Thus, for some $1 \leq t \leq T = 24 \dmax (f_x(I_D) - f^* + |\nabla_0 f_x(I)|) \cdot \dfrac{1}{\delta^2}$ steps, we must reach a scaling $\otheta^{(t)}$ such that $\norm{\nabla f_x(\otheta^{(t)})}_F^2 < \delta^2$.
\end{proof}

We are now ready to prove our main theorem of this subsection: it says that as soon as the estimation in operator norm works, then the flip-flop converges exponentially fast to the minimizer.

\begin{customthm}{\ref{thm:matrix-flipflop}}[Matrix flip-flop convergence, restated]
\MatrixFlop
\end{customthm}

\begin{proof}
If we take $\Gamma$ to be the universal constant according to \cref{lem:matrix-normal-initial-conditions}, with probability at least $1 - e^{- \Omega(\dmin \varepsilon^2)}$ we have that the conditions of \cref{lem:matrix-normal-initial-conditions} are satisfied.
Thus, there exist constants $\lambda, \gamma \in (0, 1]$ such that:
\begin{enumerate}
	\item $|\nabla_0 f_x(I)| \leq \frac{\gamma}{2}$
	\item The MLE's $\htheta_1, \htheta_2$ satisfy $\ \ \norm{\htheta_i - I_{d_i}}_{op} \leq \gamma/2, \ \ $ for $i \in \{1,2\}$
	\item $f_x$ is $\lambda$-strongly geodesically convex at any $\Theta \in \SPD$ such that $\norm{\log \Theta}_{op} \leq \gamma$.
\end{enumerate}
In particular, we have that our function $f_x$ is lower bounded by 
$$f^* := f_x(\htheta_1, \htheta_2) = 1 - \dfrac{1}{D} \log\det(\htheta_1 \otimes \htheta_2) \geq 1 - \dfrac{1}{D} \log\det((1+\gamma/2) I_D) \geq - \gamma/2,  $$
and we also know that $f_x(I_D) = 1 + \nabla_0 f_x(I_D) \leq 1 + \gamma/2$.
Thus, by \cref{lem:flip-flop-sinkhorn}, we know that in at most 
$60 \dmax \cdot \left(\dfrac{16}{\lambda \gamma}\right)^2$ steps, we will find a scaling $\otheta$ with $\norm{\nabla f_x(\otheta)}_F^2 < \dfrac{\lambda \gamma}{16}$.

For the $\otheta$ we just obtained, \cref{lem:gradient-strong-convexity-fm} applies and therefore $f_x$ is $\lambda$-strongly convex in a ball that contains the sublevel set $\{ \Upsilon \ \mid \ f_x(\Upsilon) \leq f_x(\otheta) \}$. Henceforth, \cref{lem:descent-sublevel-set} applies with $\alpha = \dfrac{1}{12 \dmax}$ and in $T = \dfrac{24 \dmax}{\lambda} \log(1/\delta)$ steps we obtain a  $\otheta'$ such that $\norm{\nabla_x(\otheta')}_F \leq \delta$. Setting $\delta = \dfrac{\varepsilon \lambda}{4 \sqrt{\dmax}}$, \cref{lem:tensor-distance-opt} implies that $D_{op}(\htheta_a \ || \ \otheta_a') \leq \varepsilon$ for $a \in \{1, 2\}$.

In particular, \cref{alg:flip-flop matrix} correctly returns a scaling $\varepsilon$-close to the MLE. The number of iterations follows from the above.
\end{proof}

%=============================================================================
\section{Lower bounds}\label{sec:lower}
%=============================================================================
Here we discuss known lower bounds for estimating unstructured precision matrices (i.e., the case $k= 1$ of the tensor normal model). Afterwards we prove a new lower bound on the matrix normal model.
\subsection{Lower bounds for relative Frobenius and operator error}
Here we briefly recall and, for completeness, prove well-known lower bounds on the rates for estimating the precision matrix in the Frobenius and operator error from independent samples of a Gaussian. The lower bounds follow from Fano's method and the relationship between the Frobenius error and the relative entropy (which is proportional to Stein's loss). We then use these bounds to show bounds on the relative Frobenius and relative operator error. Informally, both bounds say that no estimator for a $d\times d$ precision matrix from $n$ samples can have accuracy smaller than $d^2/n$ (resp $d/n$) in Frobenius error or relative Frobenius error (resp. operator norm error or relative operator norm error) with probability more than $1/2$.
%\CF{define minimax rate}


\begin{prop}[Frobenius and operator error]\label{prp:standard-lower}
There is $c > 0$ such that the following holds. Let $X = X_{\Theta} \in \R^{dn}$ denote $n$ independent samples from a Gaussian with precision matrix $\Theta$. Consider any estimator $\htheta = \htheta(X)$ for the precision $\Theta$, and let $B\subset \PD(d)$ denote the ball about $I_d$ of radius $1/2$ in the operator norm.
\begin{enumerate}
\item Let $\delta^2 = c\min \left\{1,d^2/n\right\}$. Then
\begin{align}
\sup_{\Theta \in B} \Pr\left[ \| \htheta - \Theta\|_F \geq \delta\right] \geq \frac{1}{2}.\label{eq:frob-lower}
\end{align}
\item Let $\delta^2 = c\min \left\{1,d/n\right\}$. Then
 \begin{align}
\sup_{\Theta \in B} \Pr\left[ \| \htheta - \Theta\|_{op} \geq \delta\right] \geq \frac{1}{2}. \label{eq:op-lower}
\end{align}
\end{enumerate}
As a consequence, we have 
\begin{align*}\sup_{\Theta \in B}\E[\| \htheta - \Theta\|_F^2] &=\Omega\left( \min \left\{\frac{d^2}{n},1\right\}\right)\\
\text{ and } \sup_{\Theta \in B}\E[\| \htheta - \Theta\|_{op}^2] &= \Omega\left( \min \left\{\frac{d}{n},1\right\}\right).\end{align*}
\end{prop}
The proof uses Fano's inequality with mutual information bounded by relative entropy, as in \cite{yang1999information}.
\begin{lemma}[Fano's method]
Let $\{P_i: i \in [m]\}$ be a finite set of probability distributions over a set $\mathcal X$, and let $T: \mathcal X \to [m]$ be an estimator for $i$ from a sample of $P_i$. Then 
$$ \max_{i\in [m]} \Pr_{x \sim P_i}[T(x) \neq i] \geq 1 - \frac{ - \log 2 + \max_{i,j \in [m]} D_{KL}(P_i|| P_j)}{\log m}.$$
\end{lemma}

\begin{proof}[Proof of \cref{prp:standard-lower}]
We first prove \cref{eq:frob-lower}, the lower bound on estimation in the Frobenius norm. We begin by the standard reduction from estimation to testing; let $V_0$ be a $1$-packing of the Frobenius ball $B_F$ of radius $1$ in the $d\times d$ symmetric matrices, i.e. $B_F = \{A: A \text{ Symmetric}, \|A\|_F \leq 1\}$ of cardinality $m \geq 2^{d(d+1)/2}$, as guaranteed by the Gilbert-Varshamov bound. Let $0 \leq \delta \leq 1/2$, and let $V = I_d + \delta V_0 = \{I_d + \delta v: v \in V_0\}$. Write $V = \{\Theta_1, \dots, \Theta_m\}$. Note that $V$ is contained within the operator norm ball $B$.
Let $P_i$ to be the distribution of $X(\Theta_i)$ on $\R^{dn}$, i.e $\mathcal{N}(0, \Theta^{-1}_i)^{\ot n}$. Define the estimator $T$ by 
$$T(x) = \arg\min_{i \in [m]} \|\Theta_i - \htheta(x)\|_F,$$
so that $\Pr[T(X) = i] \geq \Pr[\|\htheta -  \Theta_i\|_F \leq \delta/2]$ because $V$ is $\delta$-separated. In order to apply the local Fano's method, we use the well-known fact that $D_{KL}(P_i|| P_j) = n D_{KL}(\mathcal{N}(0, \Theta_i^{-1}), \mathcal{N}(0, \Theta_j^{-1}) = O(nD_{F}(\Theta_j || \Theta_i)^2)$ when $\Theta_i^{-1}\Theta_j$ has eigenvalues uniformly bounded away from zero by \cref{prop:dissimilarities}. This condition on the eigenvalues holds because $I_d/2 \preceq \Theta_j, \Theta_j \preceq 3I_d/2$ for $i,j \in [m]$ by our assumption that $\delta \leq 1/2$. Moreover, for $i \in [m]$, we have $\|\Theta_i^{-1}\|_{op} \leq 2$ and so $D_F(\Theta_j|| \Theta_i) = O( \|\Theta_i - \Theta_j\|_F) = O(\delta).$ We now have $D_{KL}(P_i|| P_j) \leq Cn \delta^2$ for some absolute constant $C$. By Fano's lemma, 
 $$\min_{i \in [m]} \Pr_{x \sim P_i}[T(X) =  i] \leq \frac{ - \log 2 + C n \delta^2}{d(d+1)(\log 2)/2 }.$$
If $\delta^2 = c\min\{ \frac{d^2}{n}, 1\}$, the right-hand side is bounded by $\frac{1}{2}$ and the assumption $\delta \leq 1/2$ is satisfied provided $c$ is a small enough absolute constant. On the other hand, we showed $\Pr_{x \sim P_i}[T(X) =  i] \geq \Pr[\|\htheta -  \Theta_i\|_F \leq \delta/2]$. Thus, 
$$\min_{i \in [m]} \Pr[ \|\htheta - \Theta_i\|_F \leq \delta/2] \leq 1/2.$$
Becuse $V \subset B$, this proves \cref{eq:frob-lower}. 
To obtain \cref{eq:op-lower}, the lower bound in $\| \cdot \|_{op},$ instead start with a packing $V_0$ of the unit operator norm ball of cardinality $m \geq 2^{d(d+1)/2}$. We modify the proof by bounding $D_{KL}(P_i || P_j) = O(n \| \Theta_i - \Theta_j\|_F^2) = O(n d \|\Theta_i - \Theta_j\|_{op}^2) \leq C nd \delta^2.$ Proceeding as before, we find that for $\delta = c \min \{\frac{d}{n}, 1\}$, 
$$\min_{i \in [m]} \Pr[ \|\htheta - \Theta_i\|_{op} \leq \delta/2] \leq 1/2.$$
Again, we have $\Theta_1, \dots, \Theta_m \in B$, so \cref{eq:op-lower} follows. \end{proof}
We remark that the above proof shows the necessity of a scale-invariant dissimilarity measure to obtain error bounds that are independent of the ground truth precision matrix $\Theta$. Indeed, replacing the packing $V$ by $\kappa V$ for some $\kappa \to \infty$ shows that $\sup_{\Theta \in \kappa B} \Pr\left[ \| \htheta - \Theta\|_F \geq \kappa \delta \right] \geq \frac{1}{2}$. That is, no fixed bound can be obtained with probability $1/2$. 

We now use the result just obtained to prove bounds on the relative Frobenius and operator error. Because $I_d/2 \preceq \Theta \preceq 3I_d/2$ for $\Theta \in B$, the bounds $\|\Theta - \htheta\|_F \leq \|\Theta\|_{op} D_F(\htheta|| \Theta)$ and $\|\Theta - \htheta\|_{op} \leq \|\Theta\|_{op} D_{op}(\htheta|| \Theta)$ together with \cref{prp:standard-lower} imply the following corollary.
\begin{corollary}[Relative Frobenius and operator error]\label{prp:relative-lower}
There is $c > 0$ such that the following holds for $X, \htheta, B$ as in \cref{prp:standard-lower}.
\begin{enumerate}
\item Let $\delta^2 = c\min \left\{1,d^2/n\right\}$. Then
\begin{align}
\sup_{\Theta \in B} \Pr\left[ d_F(\htheta|| \Theta)  \geq \delta\right] \geq \frac{1}{2}.\label{eq:df-lower}
\end{align}
\item Let $\delta^2 = c\min \left\{1,d/n\right\}$. Then
 \begin{align}
\sup_{\Theta \in B} \Pr\left[ d_{op}(\htheta|| \Theta) \geq \delta\right] \geq \frac{1}{2}. \label{eq:dop-lower}
\end{align}
\end{enumerate}
As a consequence, we have 
\begin{align*}\sup_{\Theta \in B}\E[d_F(\htheta|| \Theta)^2] &=\Omega\left( \min \left\{\frac{d^2}{n},1\right\}\right)\\
\text{ and } \sup_{\Theta \in B}\E[d_{op}(\htheta|| \Theta)^2] &= \Omega\left( \min \left\{\frac{d}{n},1\right\}\right).\end{align*}

\end{corollary}





\subsection{Lower bounds for the matrix normal model}
One could hope that, for the purposes of estimating $\Theta_1$, having access to $X$ is like having access to $n D/d_1$ independent samples of a Gaussian with precision matrix $\Theta_1$. In particular, one could hope e.g. for an RMSE rate of $\sqrt{ d_1/ n d_2}$ for estimating $\Theta_1$ in spectral norm in the matrix model even if $d_2$ is very large. Here we show that, to the contrary, the rate cannot be better than $O(\sqrt{d_1/ n \min(n d_1, d_2)})$. 
%Here we show that our bounds for the matrix normal model are best possible for estimating the individual matrices $\Theta_a$. The bound for $\Theta_2$ is clearly best possible because it is what we would obtain even if $\Theta_1$ were known, but in fact even the bound on $\Theta_1$ is tight.

\begin{theorem}[Lower bound for matrix normal models]\label{thm:matrix-lower}
There is $c > 0$ such that the following holds. Let $\htheta_1$ be any estimator for $\Theta_1$ from a tuple $X$ of $n$ samples of the matrix normal model with precision matrices $\Theta_1, \Theta_2$. Let $B\subset \PD(d_1)$ denote the ball about $I_{d_1}$ of radius $1/2$ in the operator norm.
\begin{enumerate}
\item \label{it:frob-lower} Let $\delta^2 = c\min \left\{1,\frac{d_1^2}{n \min \{n d_1, d_2\}}\right\}$. Then
\begin{align}
\sup_{\Theta_1 \in B, \Theta_2 \in \PD(d_2)} \Pr\left[ d_F(\htheta_1|| \Theta_1)  \geq \delta\right] \geq \frac{1}{2}.
\end{align}
\item\label{it:op-lower} Let $\delta^2 = c\min \left\{1,\frac{d_1}{n \min \{n d_1, d_2\}}\right\}$. Then
 \begin{align}
\sup_{\Theta_1 \in B, \Theta_2 \in \PD(d_2)} \Pr\left[ d_{op}(\htheta_1|| \Theta_1) \geq \delta\right] \geq \frac{1}{2}. \label{eq:dop-lower-matrix-normal}
\end{align}
\end{enumerate}
As a consequence, we have 
\begin{align*}\sup_{\Theta_1 \in B, \Theta_2 \in \PD(d_2)}\E[d_F(\htheta_1|| \Theta_1)^2] &=\Omega\left( \min \left\{\frac{d_1^2}{n \min \{n d_1, d_2\}},1\right\}\right)\\
\text{ and } \sup_{\Theta_1 \in B, \Theta_2 \in \PD(d_2)}\E[d_{op}(\htheta_1|| \Theta_1)^2] &= \Omega\left( \min \left\{\frac{d_1}{n \min \{n d_1, d_2\}},1\right\}\right).\end{align*}
\end{theorem}

Intuitively, the above theorem holds because we can choose $\Sigma_2$ to zero out all but $nd_1$ columns of each $X_i$, which allows access to at most $n \cdot n d_1$ samples from a Gaussian with precision $\Theta_1$. However, this does not quite work because $\Sigma_2$ would not be invertible and hence the precision matrix $\Theta_2$ would not exist. We must instead choose $\Sigma_2$ to be approximately equal to a random low rank matrix. The resulting construction allows us to deduce the same lower bounds for estimating $\Theta_1$ as the Gaussian case with at most $n\min \{d_2, n d_1\}$ independent samples.



\begin{lemma}\label{lem:reduce-lower}Suppose $d_2 > n d_1$. Let $X$ denote a tuple of $n$ samples from the matrix normal model with precision matrices $\Theta_1, \Theta_2$. Let $Y$ be a tuple of $n\min\{nd_1, d_2\}$ Gaussians on $\R^{d_1}$ with precision matrix $\Theta_1$. Let $\widehat{\Theta}_1$ be any estimator for $\Theta_1$. For every $\delta > 0$, there is a distribution on $\Theta_2$ and an estimator $\tilde{\Theta}$ such that the distribution of $\widehat{\Theta}_1(X)$ and the distribution of $\tilde{\Theta}(Y)$ differ by at most $\delta$ in total variation distance. 
%In particular (see \cref{lem:dtv}), the minimax MSE for $\widehat{\Theta}_1$ can be no less than the minimax MSE for the precision matrix of a $d_1$ dimensional Gaussian with $nd_1$ samples; this holds as well for $\Theta_1$ to subsets such as sparse, etc. 
\end{lemma}
%In the proof, one finds that the estimator $\tilde{\Theta}$ uses additional randomness - but by Yao's theorem the minimax MSE is unchanged when estimators with additional randomness are allowed.

\begin{proof}
If $d_2 \leq nd_1$, then setting $\Theta_2 = I(d_2)$ shows that $\htheta_1$ has access to precisely $n d_2$ samples from a Gaussian $\R^{d_1}$ with precision matrix $\Theta_1$. Thus we may take $\tilde{\Theta} = \htheta_1$ in that case, completing the proof. The harder case is $d_2 > n d_1$. 

Let $B$ be any $d_2\times d_2$ matrix such that the last $d_2 - nd_1$ columns are zero. Given access to the tuple $X$ of $n$ samples $\sqrt{\Sigma_1} X_i B^T$, where $X_i$ are i.i.d Gaussian $d_1\times d_2$ matrices, clearly $\widehat{\Theta}_2$ has access to at most $n^2 d_1$ samples of the Gaussian on $\R^{d_1}$ with precision matrix $\Theta_1$ because $X_i B^T$ depends only on the first $d_1$ columns of each $X_i$.

However, we must supply \emph{invertible} $B$ in order for $\Theta_2 = (BB^T)^{-1}$ to exist. Choose the first $nd_1$ columns of $B$ to uniformly at random among the collections of $nd_1$ orthonormal vectors in $\R^{d_2}$. Let the remaining entries be i.i.d uniform in $[-\delta, \delta]$ (the precise distribution of the remaining entries will not matter as long as they are small). Let $Y_\delta:=(\sqrt{\Sigma_1} X_1 B^T, \dots, \sqrt{\Sigma_1} X_n B^T)$ denote the resulting random variable with $B$ and $X$ chosen independently. If $\delta = 0$, then, by the argument above, with access to the random variable $Y_\delta:=(\sqrt{\Sigma_1} X_1 B^T, \dots, \sqrt{\Sigma_1} X_n B^T)$ the estimator $\widehat{\Theta}_1$ has access to at most $n^2d_1$ samples of a Gaussian on $\R^{d_1}$ with precision matrix $\Theta_1$. We claim that as $\delta \to 0$, the distribution of $Y_\delta$ tends to that of $Y_0$ in total variation distance. Thus the distribution of $\widehat{\Theta}_1(Y_\delta)$ converges to that of $\widehat{\Theta}_1(Y_0)$ in total variation. Since $Y_0$ only depends on $n^2d_1$ samples to the Gaussian on $\R^{d_1}$ with precision matrix $\Theta_1$, which we call $Y$, defining $\tilde{\Theta}(Y) = \widehat{\Theta}_1(Y_0)$ proves the theorem. \CF{there's still a probability zero event that $B$ is not full rank; not an issue I think, but we should resolve.}

It remains to prove that $Y_\delta$ converges to $Y_0$ in total variation distance. First note that $Y_\delta = Y_0 + \delta Z$ where $Z_i = \sqrt{\Theta_1} X_i C^T$, where $C$ is a random matrix where the first columns are zero and the last $d_2 - n d_1$ columns have entries i.i.d uniform on $[-1, 1]$. Because of the zero patterns of $B$ and $C$ and the fact that the entries of $X$ are i.i.d., the random variables $Y_0$ and $Z$ are independent. If we can show that $Y_0$ has a density on $\R^{nd_1d_2}$, then $Y_0 + \delta Z$ converges to $Y_0$ in total variation because adding $\delta Z$ corresponds to convolving the density of $Y_0$, an $L_1$ function, by an approximate identity \CF{find reference}.

By invertibility of $\Sigma_1$, it is enough to show that $Y_0$ has a density when $\Sigma_1 = I(d_1)$. Now $X = (X_i B^T, \dots, X_n B^T)$. Consider the $nd_1$ random vectors in $ \R^{d_2}$ that are the rows of the matrices $BX_i^T$, for $i \in \{1, \dots, n\}$. Because $B$ is supported only in its first $nd_1$ columns, the joint distribution of these random vectors may be obtained by sampling $n d_1$ independent standard Gaussian vectors $v_j$ on $\R^{nd_1}$ and then multiplying them by the $d_2 \times nd_1$ matrix $B'$ that is the restriction of $B$ to its first $nd_1$ columns. We have chosen $B'$ such that it is an isometry into a uniformly random subspace of $\R^{d_2}$ of dimension $nd_1$. Thus $Bv_j/\|v_j\|$ are $nd_1$ many independent, random unit vectors in $\R^{d_2}$. As the $\|v_j\|$ are also independent, $B v_j$ are thus independent. Each marginal $Bv_i$ has a density; one may sample it by choosing a uniformly random vector and then choosing the length $\|v_i\|$, hence the density is a product density in spherical coordinates. The joint density of the $Bv_j$ is then the product density of the marginal densities. \end{proof}

The above lemma combined with \cref{prp:relative-lower} immediately implies \cref{thm:matrix-lower}. We remark that the below proof uses no properties about $d_F$; a lower bound on any error metric for estimating a Gaussian with $n \min \{n d_1, d_2\}$ samples will transfer to the matrix normal model. In particular, \cref{thm:matrix-lower} holds true when $d_F$ is replaced by the Frobenius error and $d_{op}$ replaced by the operator norm error.
\begin{proof}[Proof of \cref{thm:matrix-lower}] 
%We begin with the proofs of \cref{it:frob-lower,it:op-lower}, as they imply the rate lower bounds. If $nd_1 \geq d_1$, then the bound follows immediately from \CF{finish}

To show \cref{it:frob-lower}, let $\delta^2 \leq c\min \left\{1,\frac{d_1^2}{n \min \{n d_1, d_2\}}\right\}$. Let $\Theta_2$ be the distributed as in \cref{lem:reduce-lower} so that, as guaranteed by \cref{lem:reduce-lower} there is an estimator $\tilde{\Theta}$ with access to a tuple $Y$ of $n \min \{n d_1, d_2\}$ samples of a Gaussian on $\R^{d_1}$ with precision matrix $\Theta_1$ satisfying $d_{TV} (\htheta_1(X), \tilde{\Theta}(Y)) \leq \delta_0$. \cref{prp:relative-lower} implies \begin{align*}
\sup_{\Theta \in B} \Pr_Y\left[ d_F(\tilde{\Theta}|| \Theta_1)  \geq \delta\right] \geq \frac{1}{2}.
\end{align*}
On the other hand, the total variation distance bound implies 
\begin{align*}
\sup_{\Theta \in B, \Theta_2 \in \PD(d_2)} \Pr_{X} \left[ d_F(\htheta_1|| \Theta_1) \geq \delta \right] \geq \sup_{\Theta \in B} \Pr_{\Theta_2, X}\left[ d_F(\htheta_1|| \Theta_1)  \geq \delta\right] \geq \frac{1}{2} - \delta_0.
\end{align*}
Allowing $\delta_0 \to 0$ implies the theorem. The proof of \cref{it:op-lower} is similar.
\end{proof}



%=============================================================================
\section{Numerics and regularization}\label{sec:numerics}
%=============================================================================

In the undersampled regime, most effort so far has focused on the sparse case. Existing estimators, such as the Gemini estimator \cite{zhou2014gemini} and kGLASSO estimator \cite{tsiligkaridis2013convergence} work by adding a regularizer proportional to the $\ell_1$ norm of the precision matrices to encourage sparsity. We refer to these as GLASSO-type estimators.

To handle cases which have structure but are not necessarily sparse, we propose a simple regularizer. We show experimentally that, under a natural generative model for \emph{dense} covariance matrices, our regularized estimator outperforms the GLASSO-type estimators appearing in \cite{tsiligkaridis2013convergence,zhou2014gemini}. We also show that, even without regularization, the MLE can outputerform Gemini when $\Theta_2$ is ill-conditioned even when $\Theta_1$ is sparse. Moreover, we observe that our regularized estimator is significantly faster to compute than the GLASSO-type estimators. All the estimators require parameter tuning, so we compare throughout a plausible range of parameters for both estimators. We leave determination of the appropriate regularization parameters by cross-validation for future work.

 Our estimator is related to the shrinkage estimator as in \cite{goes2020robust} and the Frobenius penalty considered in \CF{tang}. Instead of the MLE, we consider the following penalized log-likelihood:

\begin{align*}
  \ell_{\samp}^\alpha(\Theta_1, \dots, \Theta_k)
% = \frac n2 \log \det(\Theta/2\pi) - \sum_{i=1}^n \frac12 \braket{\samp_i, \Theta \samp_i} ~
%  := \frac {n D} 2 \sum_{a = 1}^k \frac{1}{d_a} \log \det \Theta_a  - \frac12 \sum_{i=1}^n \braket{\samp_i, \left( \textstyle \bigotimes_{a=1}^k \Theta_a \right) \samp_i} + \alpha \prod_{i=1}^k \tr \Theta_a,
  := \ell_{\samp}(\Theta_1, \dots, \Theta_k) + \alpha \prod_{i=1}^k \tr \Theta_a,
\end{align*}
where $\alpha$ is a parameter which must be tuned. Set $\widehat{\Theta}^\alpha_x = \arg\min_{|\Theta|_1 =\dots =  |\Theta_k|} \ell^\alpha_\samp(\Theta)$.
Observe the following:
\begin{enumerate}
\item If $\alpha > 0$, then $\widehat{\Theta}^\alpha_x$ exists uniquely for every $x$.
\item $\widehat{\Theta}^\alpha_x$ is equal to $\widehat{\Theta}_{y}$ where $y$ is the tuple of $n + D$ vectors given by
$$(x_1, \dots, x_n, \alpha e_{1\dots 1}, \dots, \alpha e_{d_1d_2 \dots d_k}).$$ Here $e_{i_1 \dots i_k}$ denotes the standard basis vector in $\R^{d_1d_2 \dots d_k}$. Thus the flip-flop estimator applied to the tuple $y$ converges to $\widehat{\Theta}^\alpha_x$.
\item The function $\ell^\alpha_\samp$ is \CF{$\alpha?$}-geodesically strongly convex at the identity, and shares the same robustness properties as $\ell$. Thus for $\alpha$ large enough, flip-flop is guaranteed to converge linearly.
\item $\widehat{\Theta}^\alpha_\samp$ is the maximum a posteriori probability (MAP) estimator for $\Theta$ assuming $\Theta$ is drawn from the Wishart prior with $D + 1$ degrees of freedom and $V = I/\alpha$. That is, $\Theta_1, \dots, \Theta_k$ are selected proportional to $e^{- \alpha \tr \Theta_1 \ot \dots \ot \Theta_k}.$
\end{enumerate}
%As in the unpenalized case, we will often consider the scale-invariant function $f^\alpha_x$ given by


The flip-flop algorithm for this regularized estimator is nearly as simple as the unregularized flip-flop algorithm: see \cref{alg:reg-flip-flop}. Above we explained that it can be viewed as the flip-flop estimator for an augmented tuple of tensors $y$, but due to the special form of $y$ one need not look at the entirety of $y$ to perform the iterations. The only modification is that a certain multiple of the identity is added in each flip-flop step.  \CF{I need to check all the normalizations carefully; may want to define $\rho \leftarrow \rho/nD$ and $\alpha \leftarrow \alpha/nD$.}


\begin{Algorithm}
\textbf{Input}: Samples $\samp = (\samp_1, \ldots, \samp_n)$ where $\samp_i \in \R^D$ is sampled from a (unknown) centered normal distribution $\cN(0, \Sigma)$ and a real number $\eps > 0$.
% where each entry of $\samp_i$ is encoded in binary, with bit size $\le b$. Approximation parameter $\eps > 0$, given in binary representation.

\textbf{Output}: Regularized estimators $\otheta \in \SPD$.\\[.1ex]
% such that $\| \nabla \ell^\alpha_x (\otheta)\|_F < \eps$. \\[.3ex]

\textbf{Algorithm}:
\begin{enumerate}
\item\label{it:flip-flop step 1 reg} Set $\otheta_a = I_a$ for each $a \in [k]$, and define $\otheta:= \otimes_a \otheta_a$.
\item\label{it:flip-flop step 2 reg} Repeat the following:
%we need T = \log(1/\epsilon) \cdot 1/\alpha \lambda
\begin{itemize}
%\item Set $\otheta^1$
\item Define $\otheta^a:= \otheta_{(a)}^{-1} \otheta$, the tensor product in which the $i^{th}$ factor is $\otheta_i$ except for the $a^{th}$ which is $I_{d_a}$. Define $\Upsilon_a :=\tr_{[k] \setminus a} \otheta^a \rho$. Let $$a = \arg\max_{i \in [k]} d_i   \left\| \otheta_i^{1/2} \Upsilon_i \otheta_i^{1/2}  + \alpha (\prod_j \tr \Theta_j) \Theta_i - I_{d_i}/d_i\right\|_F^2,$$
i.e., the index $a$ for which $\|\nabla_a \ell^\alpha_{\samp}(\otheta)\|_F$ is largest. If $\|\nabla_a \ell^\alpha_\samp(\otheta)\|_F < \eps$, end loop and \textbf{return} $(\otheta_1, \dots, \otheta_a)$.
\item Set
$$\otheta_a = \frac{n D}{d_a}\left(\Upsilon_i + \alpha \left(\prod_{i \neq a} \tr \otheta_a\right) I_{d_a} \right)^{-1}.$$
% $Y_a := \tr_{[k] \setminus a} \otheta_1 \ot \dots \ot \otheta_{a-1} \ot I_{d_a} \ot \otheta_{a+1} \ot \dots \ot \otheta_{d},$
%\item If $\norm{\nabla_a}_F^2 < \delta^2/k$, output $\left( \bigotimes_{a =1}^k \otheta_a \right)^{-1}$ and return.
% S\Bigl(\frac {I_{n_{k^{t}}}{n_{k^{t}}}} \big\Vert \rho^{t}_{k^{t}}\Bigr)$ is largest.
%\item Otherwise, set $\otheta_a \leftarrow \det(\rho^{(a)})^{1/d_a} (\rho^{(a)})^{-1} \cdot \otheta_a$.
\end{itemize}
\end{enumerate}
\caption{Regularized flip-flop algorithm}\label{alg:reg-flip-flop}
\end{Algorithm}

\subsection{Spiked, dense covariances}

Here we compare the performance of our regularized estimator with Zhou's single step estimator (Gemini) for the matrix normal model assuming $\Sigma_1, \Sigma_2$ are dense, spiked matrices. More precisely, $\Sigma_1, \Sigma_2$ are each a random rank one matrix plus a small multiple of the identity matrix. \cref{fig:spiked} was generated by setting $d_1 = 25, d_2 = 50$ and $n = 1$, for each choice of regularization parameter independently generating 10 different pairs $\Sigma_1 \sim I_{d_1} + 10 v_1v_1^T$ and $\Sigma_2 \sim I_{d_2} + 10 v_2v_2^T$ where the $v_i$ are standard $d_i$-dimensional Gaussian vectors, and then computing the relative error (Geodesic distance from the estimator to the truth divided by the Geodesic distance from the identity to the truth) for samples drawn from this distribution 10 times and averaging all 100 errors. \CF{For simplicity, before computing the geodesic distance we normalize all matrices to be in $\SL(d_i)$}. Note that with a single sample ($n = 1$) the MLE does not exist. 

We observe that the Gemini estimator is never better than the ``trivial" estimator which simply always outputs the identity matrix. However, for a fairly broad range of regularizers, the Gemini does outperform the inverse of the the partial trace of the sample covariance matrix (the partial trace of the sample covariance matrix is an unbiased estimator of $\Sigma_1$). As expected, as the regularization parameter tends to zero Gemini becomes identical to the inverse of the the partial trace of the sample covariance matrix. \CF{explain that this is what Gemini for $\Theta_1$ is.} For a broad range of regularization parameters, our regularized estimator outperforms the trivial estimator.

\begin{figure}
\center\includegraphics[width=.7\textwidth]{"./code/zhou-comparison/GeoPlot"}
\caption{Average error in geodesic distance with $d_1 = 25, d_2 = 50, n = 1$ for spiked dense covariance matrices. ``Gemini" refers to Zhou's Gemini estimator, ``RegSink" refers to our regularized estimator, ``Simple" refers to the partial trace of the sample covariance matrix, and ``Trivial" refers to the estimator that always outputs the identity matrix. \CF{actually do 10 runs/draws; right now it is 5}}\label{fig:spiked}
\end{figure}
 
 \subsection{Sparse, ill-conditioned inputs} \CF{ACTUALLY DO THIS PART}

We compare the performance of the estimator with Zhou's single step estimator \cite{zhou2014gemini} for the matrix normal model. We find that when $\Theta_1$ is sparse and $\Theta_2$ is well-conditioned, Zhou's estimator obtains modest gains over the regularized Sinkhorn algorithm in Frobenius error. \CF{make a figure for this.} However, when $\Theta_2$ can be ill-conditioned, we find that the shrinkage estimator outperforms the lasso-type estimators. \CF{make a figure for this} Finally, we find that when $\Sigma_1$ and $\Sigma_2$ are sampled from ``spiked" models (each is equal to the identity plus a random, potentially dense, rank-one matrix), the regularized estimator performs substantially better than the lasso-type estimator. We also observe that the computational effort  






%=============================================================================
\section{Conclusion and open problems}
%=============================================================================
\TODO{$D_{op}$ for tensors?}


\begin{appendix}

%=============================================================================
\section{Pisier's proof of quantum expansion}\label{sec:pisier}
%=============================================================================
In this appendix we discuss the proof of \cref{thm:hess-pisier}.
Pisier's original theorem is proved in a slightly different language:

\begin{theorem}[Pisier]\label{thm:Pisier-expansion}
Let $A_1,\dots,A_N,A$ be independent $n \times m$ random matrices with independent standard Gaussian entries.
For any~$t \geq 2$, with probability at least $1 - t^{-\Omega(m+n)}$,
\begin{align*}
  \norm*{\left(\sum_{i=1}^N A_i \otimes A_i \right) \circ \Pi}_{\op}
  \leq O\left( t^2 \sqrt{N} \bigl( \E \norm{A}_{\op} \bigr)^2 \right),
\end{align*}
where $\Pi$ denotes the orthogonal projection onto the traceless subspace of $\R^m \ot \R^m$, that is, onto the orthogonal complement of $\vect(I_m)$.
\end{theorem}

We first explain how \cref{thm:Pisier-expansion}, along with the following estimate of the operator norm of a standard Gaussian $n \times m$ random matrix~$A$ (see Theorem~5.32 in \cite{vershynin2010introduction}),
\begin{equation}\label{eq:op norm upper bound}
  \E \norm{A}_{\op} \leq \sqrt{n} + \sqrt{m},
\end{equation}
implies \cref{thm:hess-pisier}.

\begin{proof}[Proof of \cref{thm:hess-pisier}]
Choose $n=d_a$ and $m=d_b$.
Observe that
\begin{align*}
  \norm{\Phi_A}_0
= \max_{\substack{H \text{ traceless symmetric} \\ \norm H_F=1}} \norm{\Phi(H)}_F
% = \max_{\substack{H \text{ traceless symmetric} \\ \norm H_F \leq 1}} \norm{\Phi(H)}_F
\leq %\max_{\substack{H \in \Mat(n,m) \\ \norm H_F \leq 1}} \norm{\Phi(\Pi(H))}_F =
\max_{\substack{H \in \Mat(n,m) \\ \norm H_F = 1}} \norm{\Phi(\Pi(H))}_F
= \norm{\Phi \circ \Pi}_{\op}.
\end{align*}
Here we identified $\Mat(m) \cong \R^m \ot \R^m$, so $\Pi$ identifies with the orthogonal projection onto the traceless matrices, and we used that $\norm{\Pi(H)}_F \leq \norm{H}_F$, since $\Pi$ is an orthogonal projection.
Using \cref{eq:vec rep,thm:Pisier-expansion}, it follows that
\begin{align*}
  \norm{\Phi_A}_0 \leq O\left( t^2 \sqrt{N} \bigl( \E \norm{A}_{\op} \bigr)^2 \right),
\end{align*}
with the desired probability.
Using \cref{eq:op norm upper bound}, we can bound the right-hand side operator norm,
\begin{align*}
  \bigl( \E \norm{Y}_{\op} \bigr)^2
= O\left( \left( \sqrt{n} + \sqrt{m} \right)^2 \right)
= O\left(n+m\right),
\end{align*}
which concludes the proof.
\end{proof}

In the remainder we discuss the proof of \cref{thm:Pisier-expansion}.
Our setting required the result on rectangular matrices with strong error bounds, so we follow the proof in \cite{pisier2012grothendieck} with these minor modifications and claim no originality.
The proof proceeds by a symmetrization trick, followed by the trace method.
We will first state a necessary concentration result and then give the proof of \cref{thm:Pisier-expansion}.

\begin{theorem}[Theorem~1.5 in \cite{P86}]\label{thm:banach conc}
Let $A$ be a centered Gaussian random variable that takes values in a separable Banach space with norm $\|\cdot\|$.
Then $\|A\|$ concentrates with parameter $\sigma^2 := \sup \{ \E \langle \xi, A \rangle^{2} \mid \|\xi\|_{*} \leq 1 \}$, where $\norm{\xi}_*$ denotes the dual norm.
That is:
\[ \forall t > 0 \colon \quad \Pr\Bigl( \abs[\big]{\norm A - \E \norm A} \geq t\Bigr) \leq 2 \exp\Bigl( - \frac{\Omega(t^2)}{\sigma^{2}}\Bigr).   \]
Another equivalent definition of sub-Gaussianity is (see Lemma~5.5 of \cite{vershynin2010introduction})
\begin{equation}\label{eq:conc via moments}
  \forall p \geq 2 \colon \quad (\E \|A\|^{p})^{\frac{1}{p}} \leq \E \|A\| + O \left( \sqrt{\frac{p}{\sigma^{2}}} \right).
\end{equation}
\end{theorem}

We calculate the sub-Gaussianity parameter for our setting below.

\begin{corollary}\label{l:opNormSubG}
Let $A$ be an $n \times m$ matrix with independent standard Gaussian entries.
Then $\norm{A}_{\op}$ is sub-Gaussian with parameter $\sigma^{2} = 1$.
\end{corollary}
\begin{proof}
Note that the dual norm is the trace norm~$\norm{\cdot}_1$, hence the concentration parameter can be estimated as
\begin{align*}
  \sigma^2
= \sup \left\{ \E \langle \xi, A \rangle^2 \;\mid\; \norm{\xi}_1 \leq 1 \right\}
= \sup \left\{ \norm\xi_F^2 \;\mid\; \norm{\xi}_1 \leq 1 \right\}
= 1,
\end{align*}
where we first used that $\braket{\xi,A}$ is distributed the same as $\norm\xi_F A_{11}$ by orthogonal invariance, and then that the trace norm dominates the Frobenius norm, with equality attained for example by $\xi = E_{11}$.
\end{proof}

We can now prove a lower bound that complements \cref{eq:op norm upper bound}.

\begin{lemma}\label{lem:op norm lower bound}
Let $n \geq m$ and $A$ be an $n \times m$ matrix with independent standard Gaussian entries. Then there is some universal constant $c > 0$ such that
\begin{equation*}
  \E \norm{A}_{\op} \geq c (\sqrt{n} + \sqrt{m}).
\end{equation*}
\end{lemma}
\begin{proof}
The moment definition of sub-Gaussianity from \cref{thm:banach conc} shows for $p=2$ and $\sigma^{2} = 1$ (by \cref{l:opNormSubG}) that
\[
  (\E \norm{A}_{\op}^{2})^{1/2} \leq \E \norm{A}_{\op} + C \sqrt{2}
\implies \E \norm{A}_{\op} \geq (\E \norm{A}_{\op}^{2})^{1/2} - C \sqrt{2},
\]
where $C$ the universal constant in \cref{eq:conc via moments}.
We can then lower bound the second moment
\[ \E \norm{A}_{\op}^{2} = \E \norm{A^{T} A}_{\op} \geq \norm{\E A^{T} A}_{\op} = n ,  \]
where the first step is by definition of $\norm{\cdot}_{\op}$, the second is by Jensen's inequality, and the third step is because $\E A^{T} A = n I_{m}$. Now if $n$ is large enough, i.e. $\sqrt{n} \geq 2 C$, then in total we can show
\[ \E \norm{A}_{\op} \geq \sqrt{n} - C\sqrt{2} \geq \left(1 - \frac{1}{\sqrt{2}} \right) \sqrt{n} \geq \frac{\sqrt{n} + \sqrt{m}}{10} , \]
where in the last step we used $n \geq m$. \AR{This is for small $n$, keep if you'd like} On the other hand, if $\sqrt{n} \leq 2 C$, then we can use the simpler bound
\[ \E \norm{A}_{\op} \geq \E |\langle E_{ij}, A \rangle| = \sqrt{\frac{2}{\pi}} \geq \frac{1}{C} \sqrt{ \frac{n}{2 \pi} } \geq \frac{\sqrt{n} + \sqrt{m}}{6 C} , \]
where we used that $\E |g| = \sqrt{2/\pi}$ for standard Gaussian $g$, and in the last step we used $n \geq m$. 

Therefore, since $C$ is some absolute constant from \cref{eq:conc via moments}, in both cases we have $\E \norm{A}_{\op} = \Omega(\sqrt{n} + \sqrt{m}) $. 

\AR{Third way, subsumes both}
\[ \E \norm{A}_{\op} \geq \E \norm{A e_{1}}_{2} \geq \frac{\sqrt{n}}{2} \geq \frac{\sqrt{m} + \sqrt{n}}{4}   ,  \]
where we used the lower bound on norm of standard Gaussian $g \in \R^{n}$ (see Section 3.1 of \cite{ENormLB}), and in the final step we used $n \geq m$. 
\end{proof}


We will also use the H\"older inequality for the \emph{Schatten $p$-norm} $\norm{A}_p = (\tr(A^TA)^{\frac p2})^{\frac1p}$, where $p\geq1$:
\begin{align}\label{eq:holder}
  \abs*{\tr \prod_{i=1}^{p} A_i} \leq \prod_{i=1}^{p} \norm{A_i}_p,
\end{align}

\begin{proof} [Proof of \cref{thm:Pisier-expansion}]
The operator we want to control has entries which are dependent in complicated ways.
We first begin with a standard symmetrization trick to linearize (compare the proof of Lemma~4.1 in~\cite{P14}).
A single entry of $A_i \otimes A_i$ is either a product $g g'$ of two independent standard Gaussians, or the square $g^2$ of a single standard Gaussian.
In expectation, we have $\E g g' = 0, \E g^{2} = 1$, and so the expected matrix is
\[ \E \left( \sum_{i=1}^N A_i \otimes A_i \right) = N \vect(I_n) \vect(I_m)^T. \]
Accordingly, after projection we have
\[ \E \left( \sum_{i=1}^N A_i \otimes A_i \right) \circ \Pi = 0. \]
Therefore we may add an independent copy:
Let $B_1,\dots,B_N$ be independent $n\times m$ random matrices with standard Gaussian entries, that are also independent from~$A_1,\dots,A_N$.
Then,
\begin{align*}
  \left( \sum_{i=1}^N A_i \otimes A_i \right) \circ \Pi
= \E_B \left( \sum_{i=1}^N A_i \otimes A_i - \sum_{i=1}^N B_i \otimes B_i \right) \circ \Pi
\end{align*}
and hence, for any $p\geq1$,
\begin{align*}
  \E_{A} \norm*{\left( \sum_{i=1}^N A_i \otimes A_i \right) \circ \Pi}_{\op}^p
\leq \E_{A,B} \norm*{\left( \sum_{i=1}^N A_i \otimes A_i - \sum_{i=1}^N B_i \otimes B_i \right) \circ \Pi}_{\op}^p
\end{align*}
by Jensen's inequality, as $\norm{\cdot}_{\op}^p$ is convex as the composition of the norm $\norm{\cdot}_{\op}$ with the convex and nondecreasing function $x \to x^{p}$.
Now note $(A_i,B_i)$ has the same distribution as $(\frac{A_i+B_i}{\sqrt2},\frac{A_i-B_i}{\sqrt2})$, so the right-hand side is equal to
\begin{align*}
&\quad \E \norm*{ \frac{1}{2} \left( \sum_{i=1}^N (A_i + B_i) \otimes (A_i + B_i) - \sum_{i=1}^N (A_i - B_i) \otimes(A_i - B_i) \right) \circ \Pi}_{\op}^p \\
&= \E \norm*{\left( \sum_{i=1}^N A_i \otimes B_i + \sum_{i=1}^N B_i \otimes A_i \right) \circ \Pi}_{\op}^p
% \leq 2 \E \norm*{\left( \sum_{i=1}^N A_i \otimes B_i \right) \circ \Pi}_{\op}^p
\leq 2^{p} \, \E \norm*{\sum_{i=1}^N A_i \otimes B_i}_{\op}^p
\end{align*}
Thus, we have proved that
\begin{align}\label{eq:symmetrization}
  \E \norm*{\left( \sum_{i=1}^N A_i \otimes A_i \right) \circ \Pi}_{\op}^p
\leq 2^{p} \, \E \norm*{\sum_{i=1}^N A_i \otimes B_i}_{\op}^p.
\end{align}
Note that we have lost the projection and removed the dependencies.
Next we use the trace method to bound the right-hand side of \cref{eq:symmetrization}.
That is, we approximate the operator norm by the Schatten $p$-norm for a large enough $p$ and control these Schatten norms using concentration of moments of Gaussians (compare the proof of Theorem~16.6 in~\cite{pisier2012grothendieck}).
For any $q\geq1$,
\begin{align*}
\E \norm*{\sum_{i=1}^N A_i \ot B_i}_{2q}^{2q}
% = \E \tr \left( \sum_{i=1}^N A_i \ot B_i \right)^{2q}
&= \E \tr \left( \sum_{i,j\in[N]} A_i^T A_j \ot B_i^T B_j \right)^q \\
&= \sum_{i, j \in [N]^q} \E \tr \left( A^T_{i_1} A_{j_1} \cdots A^T_{i_q} A_{j_q} \ot B^T_{i_1} B_{j_1} \cdots B^T_{i_q} B_{j_q} \right) \\
&= \sum_{i, j \in [N]^q} \E \tr \left( A^T_{i_1} A_{j_1} \cdots A^T_{i_q} A_{j_q} \right) \E \tr \left( B^T_{i_1} B_{j_1} \cdots B^T_{i_q} B_{j_q} \right)
\end{align*}
where we used the independence of $\{A_i\}$ and $\{B_i\}$ in the last step.
Now, the expectation of a monomial of independent standard Gaussian random variables is always nonnegative.
% Now, the expectation of a (non-empty) monomial of independent standard Gaussian random variables vanishes unless each variable appears an even number of times, in which case it is positive.
Thus the same is true for $\E \tr ( A^T_{i_1} A_{j_1} \cdots A^T_{i_q} A_{j_q} )$, so we can upper bound the sum term by term as
\begin{align*}
&\quad \sum_{i, j \in [N]^q} \E \tr \left( A^T_{i_1} A_{j_1} \cdots A^T_{i_q} A_{j_q} \right) \E \tr \left( B^T_{i_1} B_{j_1} \cdots B^T_{i_q} B_{j_q} \right) \\
&\leq \sum_{i, j \in [N]^q} \E \tr \left( A^T_{i_1} A_{j_1} \cdots A^T_{i_q} A_{j_q} \right) \E \left( \norm{ B_{i_1} }_{2q} \norm{B_{j_1}}_{2q} \cdots \norm{ B_{i_q} }_{2q} \norm{B_{j_q}}_{2q} \right) \\
&\leq \sum_{i, j \in [N]^q} \E \tr \left( A^T_{i_1} A_{j_1} \cdots A^T_{i_q} A_{j_q} \right) \E \left( \norm{ B_1 }_{2q}^{2q} \right) \\
&= \left( \E\norm{\sum_{i=1}^N A_i}_{2q}^{2q} \right) \left( \E \norm{ A }_{2q}^{2q} \right)
= N^q \left( \E\norm{A}_{2q}^{2q} \right)^2
\end{align*}
In the first step we used H\"older's inequality~\eqref{eq:holder} for the Schatten norm.
The second step holds since $\E\norm{B_i}_{2q}^k \leq (\E \norm{B_i}_{2q}^{2q})^{\frac k {2q}}$ by Jensen's inequality, so we can collect like terms together.
Next, we used that the~$B_i$ have the same distribution as $A$.
In the last step, we used that $\sum_i A_i$ has the same distribution as $\sqrt N A$.
Accordingly, we have proved
\begin{align}\label{eq:op vs 2q}
\E \norm*{\sum_{i=1}^N A_i \ot B_i}_{\op}^{2q}
\leq \E \norm*{\sum_{i=1}^N A_i \ot B_i}_{2q}^{2q}
\leq N^q \left( \E\norm{A}_{2q}^{2q} \right)^2
\leq N^q m^2 \Bigl( \E\norm{A}_{\op}^{2q} \Bigr)^2,
\end{align}
In the third inequality we used that $A \in \Mat(n,m)$ has rank $\leq m$, and therefore $\norm{A}_{2q}^{2q} \leq m \norm A_{\op}^{2q}$.
To bound the right-hand side, we wish to apply \cref{thm:banach conc} for the Banach space $\Mat(n,m)$ with the operator norm~$\norm{\cdot}_{\op}$.
Thus, using \cref{eq:symmetrization,eq:op vs 2q,thm:banach conc}, we obtain for $q\geq1$,
\begin{align*}
\E \norm*{\left( \sum_{i=1}^N A_i \otimes A_i \right) \circ \Pi}_{\op}^{2q}
% \leq 2 \E \norm*{\sum_{i=1}^N A_i \ot B_i}_{\op}^{2q}
\leq 2^{2q} N^q m^2 \Bigl( \E\norm{A}_{\op}^{2q} \Bigr)^2
\leq (4 N)^q m^2 \Bigl( \E\norm A_{\op} + C \sqrt{q} \Bigr)^{4q},
\end{align*}
where~$C>0$ is the universal constant from the big-$O$ notation in \cref{eq:conc via moments}.

\textcolor{blue}{OLD: Finally, we can use the Markov inequality to see that the event
\begin{align*}
  \norm*{\left(\sum_{i=1}^N A_i \otimes A_i \right) \circ \Pi}_{\op} \leq (2 t C')^2 \sqrt{4 N} \bigl( \E \norm{A}_{\op} \bigr)^2
\end{align*}
holds up to failure probability at most
\begin{align*}
  \frac{\E \norm*{\left(\sum_{i=1}^N A_i \otimes A_i \right) \circ \Pi}_{\op}^{2q}}{\left( (2 t)^2 \sqrt{4 N} \bigl( \E \norm{A}_{\op} \bigr)^2 \right)^{2q}}
\leq % \frac{2 N^q m^2 \Bigl( \E\norm A_{\op} + C \sqrt{q} \Bigr)^{4q}}{(2C')^{4q} t^{4q} N^q \bigl( \E \norm{A}_{\op} \bigr)^{4q}} =
  \frac{m^2 \Bigl( \E\norm A_{\op} + C \sqrt{q} \Bigr)^{4q}}{(2 t)^{4q} \bigl( \E \norm{A}_{\op} \bigr)^{4q}}.
\end{align*}
%\MW{I think this is buggy now. Also worry about $q\geq1$ (integer is not important I think).}
We would like simultaneously $q \geq 1$ and $C \sqrt{q} \leq \E \norm A_{\op}$, so that the numerator in the trace method can be bounded. Therefore assuming $n = \max\{m,n\}$ is large enough for \cref{lem:op norm lower bound} to hold, we can choose $q = c(\sqrt{m} + \sqrt{n})^{2}$ for some small universal constant $c$ so that $q = \Omega(m + n)$.
\MW{big O decision time}
Then, the failure probability is bounded as follows:
\begin{align*}
  \frac{m^2 \Bigl( \E\norm A_{\op} + C \sqrt{q} \Bigr)^{4q}}{(2 t)^{4q} \bigl( \E \norm{A}_{\op} \bigr)^{4q}}
\leq
% \frac{2 m^2 \Bigl( (1 + C') \E\norm A_{\op} \Bigr)^{4q}}{(C')^{4q} t^{4q} \bigl( \E \norm{A}_{\op} \bigr)^{4q}} =
m^2 \left( \frac{2 \E\norm A_{\op}}{2 t \E\norm A_{\op}} \right)^{4q}
\leq m^2 t^{-4q}
= t^{-\Omega(m+n)}
\end{align*}
In the final step we used that $q = \Omega(m + n)$, as well as the fact that $t \geq 2$, so the prefactor $m^{2}$ can be absorbed at the cost of slightly changing the constant in the exponent. }

\AR{Second Version} Finally, we can use the Markov inequality to see that, for some large universal constant $C' \geq 1$ that we choose later, the event
\begin{align*}
  \norm*{\left(\sum_{i=1}^N A_i \otimes A_i \right) \circ \Pi}_{\op} \leq (C' t)^2 \sqrt{4 N} \bigl( \E \norm{A}_{\op} \bigr)^2
\end{align*}
holds up to failure probability at most
\begin{align*}
  \frac{\E \norm*{\left(\sum_{i=1}^N A_i \otimes A_i \right) \circ \Pi}_{\op}^{2q}}{\left( (C' t)^2 \sqrt{4 N} \bigl( \E \norm{A}_{\op} \bigr)^2 \right)^{2q}}
\leq % \frac{2 N^q m^2 \Bigl( \E\norm A_{\op} + C \sqrt{q} \Bigr)^{4q}}{(2C')^{4q} t^{4q} N^q \bigl( \E \norm{A}_{\op} \bigr)^{4q}} =
  \frac{m^2 \Bigl( \E\norm A_{\op} + C \sqrt{q} \Bigr)^{4q}}{(C' t)^{4q} \bigl( \E \norm{A}_{\op} \bigr)^{4q}}.
\end{align*}
We choose $q = \max\{1, C^{-2} (\E \norm{A}_{\op})^{2} \}$, which by \cref{lem:op norm lower bound} is $\Omega(m + n)$. If $\E \norm{A}_{\op} \leq C$, and therefore $q = 1$, then we can bound the failure probability as
\begin{align*}
  \frac{m^2 \Bigl( \E\norm A_{\op} + C \sqrt{q} \Bigr)^{4q}}{(C' t)^{4q} \bigl( \E \norm{A}_{\op} \bigr)^{4q}}
\leq
% \frac{2 m^2 \Bigl( (1 + C') \E\norm A_{\op} \Bigr)^{4q}}{(C')^{4q} t^{4q} \bigl( \E \norm{A}_{\op} \bigr)^{4q}} =
m^2 \left( \frac{2 C}{C' t \E\norm A_{\op}} \right)^{4q}
\leq m^2 t^{-4q}
= t^{-\Omega(m+n)}
\end{align*}
The inequality follows by choosing $C'$ large enough, as $\E\norm A_{\op}$ is bounded below by a universal constant (according to the proof of \cref{lem:op norm lower bound}). In the final step we used that $q = \Omega(m + n)$, as well as the fact that $t \geq 2$, so the prefactor $m^{2}$ can be absorbed at the cost of slightly changing the constant in the exponent.

On the other hand if $C \sqrt{q} \leq \E \norm{A}_{\op}$, then we bound the failure probability
\begin{align*}
  \frac{m^2 \Bigl( \E\norm A_{\op} + C \sqrt{q} \Bigr)^{4q}}{(C' t)^{4q} \bigl( \E \norm{A}_{\op} \bigr)^{4q}}
\leq
% \frac{2 m^2 \Bigl( (1 + C') \E\norm A_{\op} \Bigr)^{4q}}{(C')^{4q} t^{4q} \bigl( \E \norm{A}_{\op} \bigr)^{4q}} =
m^2 \left( \frac{2 \E\norm A_{\op}}{C' t \E\norm A_{\op}} \right)^{4q}
\leq m^2 t^{-4q}
= t^{-\Omega(m+n)}
\end{align*}
Here the inequality follows by choosing $C' \geq 2$. In the final step we used that $q = \Omega(m + n)$, as well as the fact that $t \geq 2$, so the prefactor $m^{2}$ can be absorbed at the cost of slightly changing the constant in the exponent.
% This concludes the proof.
\end{proof}


%=============================================================================
\section{Proof of the robustness lemma}\label{app:robust}
%=============================================================================
We'll use an easy fact relating the exponential map and the operator norm.

\begin{fact} \label{f:expTaylor} For all symmetric $d\times d$ matrices $\delta $ such that $ \|\delta\|_{op} \leq 1$, we have
$$ \|e^{\delta} - I\|_{op} \leq 2 \|\delta\|_{op}.$$
\end{fact}

%by letting our input at the identity be represented by $x$ \CF{input at identity? use of "input" is in general imprecise}
\cref{thm:tensor-convexity} gives good bounds on the Hessian of $\samp$; so in order to bound the Hessian at perturbation $\samp' := \prod_{a} (e^{\delta_{a}})_{a} \samp$, it is enough to bound the difference in the Hessian.
%We will follow the structure presented in the proof of \cref{thm:tensor-convexity} by
Therefore we will show each block of the Hessian only changes a small amount under perturbation $x' := e^{\delta} x$ for $\delta \in \Sym$. In particular we will give bounds on each block under each component-wise perturbation $x' := (e^{\delta})_{a} x$ for $\delta_{a} \in \Mat(d_{a})$.
%Recall (c.f. \cref{lem:hessian}) that the $a^{th}$ diagonal block of the Hessian depends only on $\rho^a_{\samp}$. This motivates the next two lemmas quantifying the change of $\rho^{a}_{\samp}$ under perturbations. \CF{Akshay, will you use $\tr$ instead of $Tr$}


%Recall the definition of a quadratic form of the Hessian:
%\[ \langle Z, (\nabla^{2} f) Z \rangle = \langle Z, (\nabla^{2} F) Z \rangle + \langle Z, (\nabla^{2} f - \nabla^{2} F) Z \rangle     \]
%The second term is rank one, so the quadratic form is:
%\[ \langle Z, (\nabla^{2} f - \nabla^{2} F) Z \rangle = \left( \sum_{a} \sqrt{d_{a}} \langle \rho^{(a)}, Z_{a} \rangle   \right)^{2}       \]
The $00$ block is a scalar $\nabla^{2}_{00} f = Tr[\rho]$ and the terms involving the $0$ block are just a vector:
\[ \sum_{a} \langle z_{0}, (\nabla^{2}_{0a} f) Z_{a} \rangle = z_{0} \langle \rho, \sum_{a} \sqrt{d_{a}} Z_{a} \rangle       \]
The diagonal blocks involve only one-body marginals:
\[ \langle Z_{a}, (\nabla^{2}_{aa} f) Z_{a} \rangle = \langle d_{a} \rho^{(a)}, Z_{a}^{2} \rangle       \]
And the off-diagonal blocks involve two-body marginals:
\[ \langle Z_{a}, (\nabla^{2}_{ba} f) Z_{b} \rangle =  \langle \sqrt{d_{a} d_{b}} \rho^{(ab)}, Z_{a} \otimes Z_{b} \rangle   \]
Therefore in \cref{atoaaRobustness} and \cref{btoaaRobustness}, we will prove perturbation bounds on one-body marginals, and in \cref{btoabRobustness} we will prove perturbation bounds on two-body marginals. This will allow us to bound the change in the $0$ and diagonal blocks, and the off-diagonal blocks respectively. Then, following the structure of the proof of \cref{thm:tensor-convexity}, we will collect all terms to prove a total bound at the end of the section.


%By the above discussion then we will bound the difference of each under each component-wise perturbation. Note the terms involve $\{\rho^{(a)}\}, \{\rho^{(ab)}\}$, so we prove perturbations on marginals in the following lemmas.

\begin{lemma} \label{atoaaRobustness}
For input $\samp \in \R^{nD}$ and perturbation $\delta \in \Mat(d_{a})$ such that $\|\delta\|_{op} \leq \frac{1}{20}$, if we denote $\samp' := (e^{\delta})_{a} \samp$ then
\[ \|\rho_{\samp'}^{a} - \rho_{\samp}^{a}\|_{op} \leq 4.5 \|\delta\|_{op} \|\rho_{\samp}^{a}\|_{op}   . \]
\end{lemma}
\begin{proof} By definition, $\|\rho_{\samp'}^{(a)} - \rho_{\samp}^{(a)}\|_{op} = \sup_{\|Z\|_{1} \leq 1} \langle Z_{a}, \rho_{\samp'} - \rho_{\samp} \rangle $.


Choose $\eta$ such that  $\|(e^{\delta})_{a} \samp\|_{2}^{-1} = \|\samp\|_{2}^{-1} (1 + \eta)$; note that $|\eta| = O(\|\delta\|_{op})$ by \cref{f:expTaylor} provided $c$ is small enough. Letting $\delta' := (1+\eta)e^{\delta} - I_{a}$. Assuming without loss of generality that $\|Z\|_{1} = 1$, we have
\[ | \langle Z_{a}, (I+\delta')_a \rho_{\samp} (I+\delta')_a - \rho_{\samp} \rangle | \]
\[ \leq (2\|\delta'\|_{op} + \|\delta'\|_{op}^{2}) \|\rho^{(a)}\|_{op} \|Z\|_{1}    \]
from which the lemma follows since
\[ \|\delta\|_{op} \leq .05 \implies \|\delta'\|_{op} \leq |\eta| + (1+|\eta|)(\|\delta\|_{op} + \|\delta\|_{op}^{2}) \leq  2.1 \|\delta\|_{op} \]
\end{proof}

%\CF{ I think we should combine these lemmas into a single one with two items.}\AR{The proofs are different, and I like the similar structure for diagonal/off-diagonal blocks. It may clutter the statements more to combine. }
\begin{lemma} \label{btoaaRobustness}
For input $\samp \in \R^{nD}$ and perturbation $\delta \in \Mat(d_{b})$ such that $\|\delta\|_{op} \leq \frac{1}{20}$, if we denote $\samp' := (e^{\delta})_{a} \samp$ then for $b \neq a$:
\[ \|\rho_{\samp'}^{a} - \rho_{\samp}^{a}\|_{op} \leq 9.5 \|\delta\|_{op} \|\rho_{\samp}^{a}\|_{op}      \]
\end{lemma}
\begin{proof}
Choose $\eta$ such that $\|(e^{\delta})_{b} \samp\|_{2}^{-2} = (1+\eta) \|\samp\|_{2}^{-2}$; let $\delta' := (1+\eta)e^{2\delta} - I$. By by \cref{f:expTaylor} we have
\[ \|\delta\|_{op} \leq .1 \implies |\eta| \leq 2\|\delta\|_{op} + 4\|\delta\|_{op}^{2}, \hspace{3mm} \|\delta'\|_{op} \leq (2+|\eta|)(2\|\delta\|_{op} + 4\|\delta\|_{op}^{2}) \leq 9.24 \|\delta\|_{op} \]
We assume for now $Z \succeq 0$.
\begin{align*} | \langle Z_{a}, (1+\eta) (e^{\delta})_{b} \rho_{\samp} (e^{\delta})_{b}^{*} - \rho_{\samp} \rangle|
& = | \langle Z_{a} \otimes \delta'_{b}, \rho_{\samp} \rangle   |  \\
&\leq \langle Z \otimes |\delta'|, \rho_{\samp}^{(ab)} \rangle
\leq \|\delta'\|_{op} \langle Z, \rho_{\samp}^{(a)} \rangle
\end{align*}
Here in the first inequality we used that $\rho_{\samp} \succeq 0, Z \succeq 0$; and the last inequality was by definition of marginals.
In general we decompose $Z = Z_{+} - Z_{-}$ and use the above to show
\[ |\langle Z, \rho_{\samp'}^{(a)} - \rho_{\samp}^{(a)} \rangle| \leq \|\delta'\|_{op} (\|Z_{+}\|_{1} + \|Z_{-}\|_{1}) \|\rho_{\samp}^{(a)}\|_{op}     \]
The lemma follows by noting $\|Z\|_{1} = \|Z_{+}\|_{1} + \|Z_{-}\|_{1}$ and $\|\delta\|_{op} \leq c$:
\[ \|\delta'\|_{op} = \|(1+\eta) e^{2 \delta} - I\|_{op} \leq |\eta|(1 + 2 \|\delta\|_{op}) + 2\|\delta\|_{op} \leq O(\|\delta\|_{op})  \]
\end{proof}

This is already enough to prove a bound on the constant and diagonal terms:
%and rank one term $(\nabla^{2} f - \nabla^{2} F)$.

\begin{corollary} \label{diagRobustness}
For input $\samp \in \R^{nD}$ such that $\|d_{a} \rho_{\samp}^{(a)}\|_{op} \leq 1 + \frac{1}{20}$; and perturbation $\delta := \sum_{b} (\delta_{b} \in \Mat(d_{b}))_{b}$ such that $\|\delta\|_{op} = \sum_{b} \|\delta_{b}\|_{op} \leq \frac{1}{20}$; if we denote $\samp' := e^{\delta} \samp$, then we have:
\[ \|\nabla^{2}_{aa} f(e^{2\delta}) - \nabla^{2}_{aa} f(I)\|_{op} \leq 11 \|\delta\|_{op}     \]
\end{corollary}
%\CF{technically $f(e^{\delta})$ corresponds to $e^{\delta/2} \samp$}
\begin{proof}
Recall from the discussion after \cref{convexRobustness} that $\langle Y, (\nabla^{2}_{aa} f_{\samp}) Y \rangle = \langle d_{a} \rho_{\samp}^{(a)}, Y^{2} \rangle$. We treat the perturbation as the composition of $k$ perturbations;
\[ \samp_{(0)}:=\samp \to \samp_{(1)}:= (e^{\delta_{1}})_1 \samp_{(0)} \to ... \to \samp_{(k)}:=(e^{\delta_{k}})_{k} \samp_{(k-1)} = \samp'  \]
We can use \cref{atoaaRobustness} to handle $e^{\delta_{a}}$ and \cref{btoaaRobustness} for the rest:
\begin{align*}
 |\langle \rho_{\samp'}^{(a)} - \rho_{\samp}^{(a)}, Y^{2} \rangle|
 &\leq \sum_{j=1}^{k} |\langle \rho_{\samp_{(j)}}^{(a)} - \rho_{\samp_{(j-1)}}^{(a)}, Y^{2} \rangle| \underset{\cref{atoaaRobustness},\;\cref{btoaaRobustness}}{\leq} \sum_{j=1}^{k}  9.5 \|\delta_{j}\|_{op} \|\rho_{\samp_{(j-1)}}^{(a)}\|_{op} \|Y^{2}\|_{1} \\
& \leq \left( \prod_{j=1}^k (1+9.5 \|\delta_{j}\|_{op}) - 1 \right) \|\rho_{\samp}^{(a)}\|_{op} \|Y\|_{F}^{2} \\
&\leq 10 \|\delta\|_{op} \|\rho_{\samp}^{(a)}\|_{op} \|Y\|_{F}^{2}.   \end{align*}
The term in parenthesis is shown by induction, and in the inequality we used $\|\delta\|_{op} \leq \frac{1}{20}$. The final step follows by our initial condition on $\|d_{a} \rho_{\samp}^{(a)}\|_{op}$.
\end{proof}

\begin{corollary} \label{constantRobustness}
For input $\samp \in \R^{nD}$ such that $\tr[\rho_{\samp}] \leq 1 + \frac{1}{20}$ and $\|d_{a} \rho_{\samp}^{(a)}\|_{op} \leq 1 + \frac{1}{20}$; and perturbation $\delta := \sum_{a} (\delta_{a} \in \Mat(d_{a}))_{a}$ such that $\|\delta\|_{op} = \sum_{a} \|\delta_{a}\|_{op} \leq \frac{1}{20}$; if we denote $\samp' := e^{\delta} \samp$, then we have:
\[ |\nabla^{2}_{00} f_{\samp'} - \nabla^{2}_{00} f_{\samp}| \leq 5 \|\delta\|_{op}     \]
\[ \|\nabla^{2}_{0a} f_{\samp'} - \nabla^{2}_{0a} f_{\samp}\|_{op} \leq 11 \|\delta\|_{op}    \]
%\AR{Not really op norm here, just traceless part. Is there a notation for this?}
\end{corollary}
\begin{proof}
Recall that the $00$ block of the Hessian is just a scalar, so we can use the approximation for $e^{\delta}$ given above:
\[ |\tr[\rho_{\samp'} - \rho_{\samp}]| = |\langle \rho_{\samp}, e^{2\delta} - I \rangle| \leq |Tr[\rho_{\samp}]-1|) \|e^{2 \delta} - I\|_{op} \leq 5 \|\delta\|_{op}     \]
In the last step we used our bound on $\tr[\rho_{\samp}]$
The $0a$ block is a vector, so it is enough to bound the inner product with any $Z \perp I_{a}$:
\[ |\langle \rho_{\samp'}^{(a)} - \rho_{\samp}^{(a)}, \sqrt{d_{a}} Z \rangle| \leq \|\rho_{\samp'}^{(a)} - \rho_{\samp}^{(a)}\|_{op} \sqrt{d_{a}} \|Z\|_{1} \]
From here we can use the same iterative strategy as in the proof of \cref{diagRobustness}:
\[ \leq 10 \|\delta\|_{op} \|\rho^{(a)}_{\samp}\|_{op} d_{a} \|Z\|_{F} \leq 11 \|\delta\|_{op} \|Z\|_{F}   \]
In the final step we used our condition on $\|d_{a} \rho^{(a)}\|_{op} \leq 1 + \frac{1}{20}$.
\end{proof}

%\begin{corollary} \label{rankoneRobustness}
%For input $x \in \R^{nD}$ such that for all $a \in [k]$ such that $\|d_{a} \rho_{\samp}^{(a)} - \tr[\rho_{\samp}] I_{a}\|_{op} \leq \frac{1}{20}$; and perturbation $\delta := \sum_{a} (\delta_{a} \in \Mat(d_{a}))_{a}$ such that $\|\delta\|_{op} = \sum_{a} \|\delta_{a}\|_{op} \leq \frac{1}{20}$; if we denote $\samp' := e^{\delta} \samp$, then we have:
%\[ \|(\nabla^{2} f_{\samp'} - \nabla^{2} F_{\samp'}) - (\nabla^{2} f_{\samp} - \nabla^{2} F_{\samp})\|_{op} \leq 1.5 k \|\delta\|_{op}     \]
%\end{corollary}
%\begin{proof}
%Recall again from the discussion after $\cref{convexRobustness}$ that $\langle Z, (\nabla^{2} F - \nabla^{2} f) Z \rangle = \left\langle \rho, \sum_{a} \sqrt{d_{a}} (Z_{a})_{a}  \right\rangle^{2}$. We use the same iterative strategy as $\cref{diagRobustness}$:
%\[    \left\langle \rho_{\samp'}, \sum_{a} \sqrt{d_{a}} (Z_{a})_{a}  \right\rangle^{2} -  \left\langle \rho_{\samp}, \sum_{a} \sqrt{d_{a}} (Z_{a})_{a}  \right\rangle^{2}    \]
%\[ = \left\langle \rho_{\samp'} + \rho_{\samp}, \sum_{a} \sqrt{d_{a}} (Z_{a})_{a}  \right\rangle \left\langle \rho_{\samp'} - \rho_{\samp}, \sum_{a} \sqrt{d_{a}} (Z_{a})_{a}  \right\rangle  \]
%\[ = \left( \sum_{a} \langle (d_{a} \rho_{\samp'}^{(a)} - \tr[\rho_{\samp'}] I_{a}) + (d_{a} \rho_{\samp}^{(a)} - \tr[\rho_{\samp}] I_{a}) , d_{a}^{-1/2} Z_{a} \rangle \right) \left( \sum_{a} \sqrt{d_{a}} \langle \rho_{\samp'}^{(a)} - \rho_{\samp}^{(a)}, Z_{a} \rangle \right)     \]
%%\AR{Here I could use that $Z \perp I$ to get a constant factor improvement; need an assumption on $\nabla$; but it improves the overall constant by factor $\approx 2$}
%\[ \leq \left( \sum_{a} (2 + 10 \|\delta\|_{op}) \|d_{a} \rho_{\samp}^{(a)} - \tr[\rho_{\samp}] I_{a} \|_{op} \|d_{a}^{-1/2} Z_{a}\|_{1}   \right)
%\left( 10\|\delta\|_{op} \sum_{a} \sqrt{d_{a}} \| \rho_{\samp}^{(a)}\|_{op} \|Z_{a}\|_{1}   \right)    \]
%\[ \leq \left( \sum_{a} \frac{2 + 10 \|\delta\|_{op}}{20} \|Z_{a}\|_{F}  \right)
%\left( 10\|\delta\|_{op} \sum_{a} (1 + \frac{1}{20}) \|Z_{a}\|_{F} \right)
%\leq 1.5 k \|\delta\|_{op} \|Z\|^{2}      \]
%%\AR{What norm are we using on $Z$ as a whole? Should I keep around the $d_{a}$'s?} \CF{the norm is just the standard norm, there shouldn't be $d_a$'s}
%% The final step follows from the initial conditions on $\|d_{a} \rho_{a}\|_{op}, \|d_{b} \rho_{b}\|_{op}$.
%In the third line we used that $Z$ is traceless; in the last line we used our initial conditions on $\rho$; the last step was by Cauchy-Schwarz.
%\end{proof}


The off-diagonal blocks require the following lemma on bipartite marginals:

\begin{lemma} \label{btoabRobustness}
For input $\samp \in \R^{nD}$ and perturbation $\delta \in \Mat(d_{c})$ such that $\|\delta\|_{op} \leq \frac{1}{20}$; if we denote $\samp' := (e^{\delta})_{c} \samp$, then for $c \in \{a,b\}$ we have
\[ \sup_{Y \in \smallSym_{d_{a}}^{0}, Z \in \smallSym_{d_{b}}^{0}} \frac{\langle \rho_{\samp'}^{(ab)} - \rho_{\samp}^{(ab)}, Y \otimes Z \rangle}{\|Y\|_{F} \|Z\|_{F}} \leq 4.5 \|\delta\|_{op} \sup_{Y \in \smallSym_{d_{a}}, Z \in \smallSym_{d_{b}}} \frac{\langle \rho_{\samp}^{(ab)}, Y \otimes Z \rangle}{\|Y\|_{F} \|Z\|_{F}}        \]
Note that $\smallSym_{d}^{0}$ are traceless symmetric matrices, whereas $\smallSym_{d}$ are symmetric matrices.
%\[ \|\nabla^{2}_{ab} F_{\samp'} - \nabla^{2}_{ab} F_{\samp}\|_{0} \leq 4.5 \|\delta\|_{op} \|\nabla^{2}_{ab} F_{\samp}\|_{F \to F}    \]
\end{lemma}
\begin{proof}
%Recall from the discussion after $\cref{convexRobustness}$ that $\langle Y, (\nabla^{2}_{ab} F) Z \rangle = \sqrt{d_{a} d_{b}} \langle \rho^{(ab)}, Y \otimes Z \rangle$.
By taking adjoints, we can assume w.l.o.g. that $c = b$. Let $R : \Mat(d_{b}) \to \Mat(d_{b})$ be defined as $R(Z) := (1+\eta)^{2} e^{\delta} Z e^{\delta}$ for $\eta$ defined by our normalization $\|(e^{\delta})_{b} \samp\|_{2}^{-1} =: (1+\eta) \|\samp\|_{2}^{-1}$.
\[ |\langle \rho_{\samp'}^{(ab)} - \rho_{\samp}^{(ab)}, Y \otimes Z \rangle| = |\langle \rho_{\samp}^{(ab)}, Y \otimes (R(Z) - Z) \rangle|  \]
The subspace $\smallSym_{d_{b}}^{0}$ is not invariant under $R$, but we show $R \approx I$. Let $\delta' := (1+\eta) e^{\delta} - I$.
\[ \|R(Z) - Z\|_{F} \leq 2 \|\delta' Z\|_{F} + \|\delta' Z \delta'\|_{F} \leq (2 \|\delta'\|_{op} + \|\delta'\|_{op}^{2}) \|Z\|_{F}    \]
So we complete the proof using the fact that $Y,Z$ are traceless on the LHS of the inequality, and $\|\delta\|_{op} \leq \frac{1}{20}$ by the same calculation as in \cref{atoaaRobustness} using by \cref{f:expTaylor}.
\end{proof}

\begin{lemma} \label{ctoabRobustness}
For input $\samp \in \R^{nD}$ and perturbation $\delta \in \Mat(d_{c})$ such that $\|\delta\|_{op} \leq \frac{1}{20}$; if we denote $\samp' := (e^{\delta})_{c} \samp$, then for $c \not\in \{a,b\}$ we have
\[ \sup_{Y \in \smallSym_{d_{a}}^{0}, Z \in \smallSym_{d_{b}}^{0}} \frac{\langle \rho_{\samp'}^{(ab)} - \rho_{\samp}^{(ab)}, Y \otimes Z \rangle}{\|Y\|_{F} \|Z\|_{F}} \leq 19 \|\delta\|_{op} \sup_{Y \in \smallSym_{d_{a}}, Z \in \smallSym_{d_{b}}} \frac{\langle \rho_{\samp}^{(ab)}, Y \otimes Z \rangle}{\|Y\|_{F} \|Z\|_{F}}        \]
%\[ \|\nabla^{2}_{ab} F_{\samp'} - \nabla^{2}_{ab} F_{\samp}\|_{0} \leq 19 \|\delta\|_{op} \|\nabla^{2}_{ab} F_{\samp}\|_{F \to F}    \]
\end{lemma}
\begin{proof}
Define $\eta$ for normalization $\|(e^{\delta})_{c} \samp\|_{2}^{-1} =: (1+\eta) \|\samp\|_{2}^{-2}$, and let $\delta' := (1+\eta) e^{2 \delta} - I_{c}$. We will use a similar decomposition to lemma \cref{btoaaRobustness}, so first assume $Y,Z \succeq 0, \|Y\|_{F} = \|Z\|_{F} = 1$:
%\[ \frac{1}{\sqrt{d_{a} d_{b}} } \langle Y, (\nabla^{2}_{ab} F_{\samp'} - \nabla^{2}_{ab} F_{\samp}) Z \rangle = \langle \rho_{\samp}^{(abc)}, Y \otimes Z \otimes \delta' \rangle   \]
\[ |\langle \rho_{\samp}^{(abc)}, Y \otimes Z \otimes \delta' \rangle| \leq \langle \rho_{\samp}^{(abc)}, Y \otimes Z \otimes |\delta'| \rangle \leq \|\delta'\|_{op} \langle \rho_{\samp}^{(ab)}, Y \otimes Z \rangle   \]
Here we again used that $\rho_{\samp}^{(abc)} \succeq 0$. %We cannot bound this by $c_{0}$ as $Y \succeq 0$, but the RHS $c$ is sufficient.
To finish the lemma we decompose $Y = Y_{+} - Y_{-}, Z = Z_{+} - Z_{-}$ and bound
\[ |\langle \rho_{\samp'}^{(ab)} - \rho_{\samp}^{(ab)}, Y \otimes Z \rangle| \leq \left( \sup_{Y \in \smallSym_{d_{a}}, Z \in \smallSym_{d_{b}}} \frac{\langle \rho_{\samp}^{(ab)}, Y \otimes Z \rangle}{\|Y\|_{F} \|Z\|_{F}} \right) \|\delta'\|_{op} \sum_{s,t \in \{+,-\}} \|Y_{s}\|_{F} \|Z_{t}\|_{F}   \]
The summation we can bound by Cauchy Schwarz:
\[ \leq (2\|Y_{+}\|_{F}^{2} + 2\|Y_{-}\|_{F}^{2})^{1/2} (2\|Z_{+}\|_{F}^{2} + 2\|Z_{-}\|_{F}^{2})^{1/2} = 2 \|Y\|_{F} \|Z\|_{F}     \]
Using the the fact that the LHS is the $\sup$ over traceless matrices, as well as the same calculation from \cref{btoaaRobustness} using the condition $\|\delta\|_{op} \leq .05 \implies \|\delta'\| \leq 9.5 \|\delta\|_{op}$; we get the lemma.
\end{proof}

We need the following to translate to statements on the Hessian:

\begin{definition}
For operator $M : \Mat(d_{b}) \to \Mat(d_{a})$, we let $\|M\|_{0}$ denote the $F \to F$ norm of its restriction to the traceless subspaces $\smallSym^0_{d_b} \to \smallSym^0_{d_a}$
\end{definition}

\begin{lemma}[\cite{KLR19}] \label{inftyto2}
$\|\nabla^{2}_{ab} f_{\samp}\|_{F \to F}^{2} \leq \|d_{a} \rho_{\samp}^{(a)}\|_{op} \|d_{b} \rho_{\samp}^{(b)}\|_{op}$
\end{lemma}
%\begin{proof}This was already in KLR and we have two new proofs: one by convexity, and one by Riesz-Thorin. \AR{The proofs are in some other file, we can add it if we like}\end{proof}

\begin{corollary} \label{offdiagRobustness}
For input $\samp \in \R^{nD}$ such that $\|d_{a} \rho_{\samp}^{(a)}\|_{op}, \|d_{b} \rho_{\samp}^{(b)}\|_{op} \leq 1+\frac{1}{20}$; perturbation $\delta := \sum_{c} (\delta_{c} \in \Mat(d_{c}))_{c}$ with $\|\delta\|_{op} = \sum_{c} \|\delta_{c}\|_{op} \leq \frac{1}{20}$; if we denote $\samp' := e^{\delta} \samp$, then we have:
%\[ \frac{1}{\sqrt{d_{a} d_{b}}} \|\nabla^{2}_{ab} f(e^{2 \delta}) - \nabla^{2}_{ab} f(I)\|_{op} \leq 100 \|\delta\|_{op} \sqrt{\|\rho_{\samp}^{(a)}\|_{op} \|\rho_{\samp}^{(b)}\|_{op}}     \]
\[ \|\nabla^{2}_{ab} f_{\samp'} - \nabla^{2}_{ab} f_{\samp}\|_{0} \leq 21 \|\delta\|_{op}  \]
\end{corollary}
\begin{proof}
This is just a translation of \cref{btoabRobustness}, \cref{ctoabRobustness}:
\[ \sup_{Y \in \smallSym_{d_{a}}^{0}, Z \in \smallSym_{d_{b}}^{0}} \frac{\langle \rho_{\samp'}^{(ab)} - \rho_{\samp}^{(ab)}, Y \otimes Z \rangle}{\|Y\|_{F} \|Z\|_{F}} = \frac{\|\nabla^{2}_{ab} f_{\samp'} - \nabla^{2}_{ab} f_{\samp}\|_{0}}{\sqrt{d_{a} d_{b}} } \]
\[ \sup_{Y \in \smallSym_{d_{a}}, Z \in \smallSym_{d_{b}}} \frac{\langle \rho_{\samp}^{(ab)}, Y \otimes Z \rangle}{\|Y\|_{F} \|Z\|_{F}} = \frac{\|\nabla^{2}_{ab} f_{\samp}\|_{F \to F}}{\sqrt{d_{a} d_{b}} }       \]
Using the same iterative strategy as \cref{diagRobustness} we can show:
\[ |\langle Y, (\nabla^{2}_{ab} f_{\samp'} - \nabla^{2}_{ab} f_{\samp}) Z \rangle| \leq 20 \|\delta\|_{op} \|\nabla^{2}_{ab} f_{\samp}\|_{F \to F} \|Y\|_{F} \|Z\|_{F}    \]
We used \cref{btoabRobustness} for $\{a,b\}$ and \cref{ctoabRobustness} for the rest. The final step follows from \cref{inftyto2} stating $\|\nabla^{2}_{ab} f_{\samp}\|_{F \to F}^{2} \leq \|d_{a} \rho_{\samp}^{(a)}\|_{op} \|d_{a} \rho_{\samp}^{(a)}\|_{op}$; and the initial conditions on $\|d_{a} \rho_{a}\|_{op}, \|d_{b} \rho_{b}\|_{op} \leq 1 + \frac{1}{20}$.
\end{proof}

Now we can combine the above term-by-term bounds to bound the change in the Hessian.

\begin{proof} [Proof of \cref{convexRobustness}]
The above corollaries (\ref{constantRobustness},\ref{diagRobustness},\ref{offdiagRobustness}) require bounds on $\rho^{(a)}$, which are implied by the conditions on the gradient:
\[ \|d_{a} \rho^{(a)}\|_{op} \leq 1 + |Tr \rho - 1| + \|d_{a} \rho^{(a)} - (Tr \rho) I_{d_{a}} \|_{op}    \]
\[ = 1 + |\nabla_{0} f| + \|\sqrt{d_{a}} \nabla_{a} f\|_{op} \leq 1 + 2c     \]
Similarly $\tr \rho \leq 1 + |\nabla_{0} f|$, so choosing $c \leq \frac{1}{40}$ suffices for both.
Recall the definition of a quadratic form of the Hessian:
\[ \langle Z, (\nabla^{2} f) Z \rangle = z_{0} (\nabla^{2}_{00} f) z_{0} + 2 \sum_{a} \langle z_{0}, (\nabla^{2}_{0a} f) Z_{a} \rangle + \sum_{a} \langle Z_{a}, (\nabla^{2}_{aa} f) Z_{a} \rangle + \sum_{a \neq b} \langle Z_{a}, (\nabla^{2}_{ab} f) Z_{b} \rangle     \]
%\[ \langle Z, (\nabla^{2} f) Z \rangle = \langle Z, (\nabla^{2} F) Z \rangle + \langle Z, (\nabla^{2} f - \nabla^{2} F) Z \rangle     \]
%\[ = \sum_{a} \langle Z_{a}, (\nabla^{2}_{aa} F) Z_{a} \rangle + \sum_{a \neq b} \langle Z_{a}, (\nabla^{2}_{ab} F) Z_{b} \rangle - \left( \sum_{a} \sqrt{d_{a}} \langle \rho^{(a)}, Z_{a} \rangle  \right)^{2}       \]
Let $\{\samp' := e^{\delta} \samp\}$. Then by \cref{constantRobustness} we have a bound on the constant terms:
\[ | z_{0}^{2} (\nabla^{2}_{00} f_{\samp'} - \nabla^{2}_{00} f_{\samp} ) + 2 \sum_{a} \langle z_{0}, (\nabla^{2}_{0a} f_{\samp'} - \nabla^{2}_{0a} f_{\samp}) Z_{a} \rangle |      \]
\[ \leq 5 \|\delta\|_{op} z_{0}^{2} + (2 |z_{0}|) 12 \|\delta\|_{op} \sum_{a} \|Z_{a}\|_{F}
\leq \|\delta\|_{op} (17 k z_{0}^{2} + 12 \sum_{a} \|Z_{a}\|_{F}^{2})   \]
In the last step we used Young's inequality ($2ab \leq a^{2} + b^{2}$).
%and by $\cref{rankoneRobustness}$ we have a bound on the rank-one term.
%\[ \langle Z, (\nabla^2 f_{\samp'} - \nabla^{2} f_{\samp}) Z \rangle \leq \|\delta\|_{op} \left( 11 \sum_{a} \|Z_{a}\|_{F}^{2} + 21 \sum_{a \neq b} \|Z_{a}\|_{F} \|Z_{b}\|_{F} + 1.5 k \|Z\|^{2} \right)   \]
%\[ \leq (11 + 21(k-1) + 1.5 k) \|\delta\|_{op} \|Z\|^{2}    \]

By \cref{diagRobustness} we have a bound on the diagonal terms, and by \cref{offdiagRobustness} we have a bound on the off-diagonal terms:
\[ |\sum_{ab} \langle Z_{a}, (\nabla^{2}_{ab} f_{\samp'} - \nabla^{2}_{ab} f_{\samp} ) Z_{b} \rangle | \leq \|\delta\|_{op} \left( 11 \sum_{a} \|Z_{a}\|_{F}^{2} + 21 \sum_{a \neq b} \|Z_{a}\|_{F} \|Z_{b}\|_{F} \right)   \]
\[ \leq (11 + 21(k-1)) \|\delta\|_{op} \left( \sum_{a} \|Z_{a}\|_{F}^{2} \right)   \]
So combining all three terms we see:
\[ |\langle Z, (\nabla^{2} f_{\samp'} - \nabla^{2} f_{\samp} ) Z \rangle | \leq \|\delta\|_{op} \left( 17 k z_{0}^{2} + (12 + 11 + 21 (k-1)) \sum_{a} \|Z_{a}\|_{F}^{2} \right)    \]
\[ \leq 25 k \|\delta\|_{op} \left( z_{0}^{2} + \sum_{a} \|Z_{a}\|_{F}^{2} \right)    \]
Note that this also gives a spectral upper bound for $\nabla^{2} f_{\samp'}$.
%\AR{What norm are we using on $Z$ as a whole? Should I keep around the $d_{a}$'s? Also the constant is $50 k$ now, yay for better constants!}\CF{yay!}
\end{proof}

%=============================================================================
\section{The Cheeger constant of a random operator}\label{app:cheeky}
%=============================================================================

To prove \cref{thm:operator-cheeger}, we first define the Cheeger constant of an operator $\Phi:\Mat(d_1) \to \Mat(d_2)$. This is similar to a concept defined in \cite{H07}.
\begin{definition}
Let $\Phi : \Mat(d_1) \to \Mat(d_2)$ be a completely positive map. The Cheeger constant $\ch(\Phi)$ of the weighted bipartite graph associated to $B$ is given by
$$\ch(\Phi):=\min_{\Pi_1, \Pi_2: \vol(\Pi_1, \Pi_2) \leq \tr \Phi(I)} \phi(\Pi_1,\Pi_2)$$
where $\Pi_1: \C^{d_1} \to \C^{d_1}$ and $\Pi_1: \C^{d_2} \to \C^{d_2}$ are orthogonal projections that are not both zero and the \emph{conductance} $\phi$ of the cut $\Pi_1, \Pi_2$ is defined to be
$$\phi(\Pi_1,\Pi_2) := \frac{\cut(\Pi_1, \Pi_2)}{\vol(\Pi_1,\Pi_2)}$$
where
%$$ \vol(\Pi_1,\Pi_2):= \sum_{i \in T, j \in [d_2]} b_{ij} + \sum_{i \in [d_1], j \in S} b_{ij}\textrm{ and } \cut(S, T):= \sum_{i \not\in T, j  \in S} b_{ij} + \sum_{i \in T, j \not\in S} b_{ij}.$$
$$ \vol(\Pi_1,\Pi_2):=
\tr \Phi(\Pi_1) + \tr \Phi^*(\Pi_2)$$
and $$ \cut(\Pi_1, \Pi_2):= \tr \Pi_2 \Phi(I_{d_1} - \Pi_1) + \tr (I_{d_2} - \Pi_2) \Phi(\Pi_1).$$
\end{definition}

We now cite bound on expansion in terms of the Cheeger constant.
%In that work only the bound on $\|\Phi\|_0$ is shown explicitly, but the paper also shows that the second bound follows from the first.
%Recall the function $$f^{\Phi}:\samp \mapsto \frac{d_1}{d_2} \log\det(\Phi(\samp)) - \log\det (\samp).$$

\begin{lemma} [\cite{FM20}, Remark 5.5]\label{lem:op-cheeger} There exist absolute constants $c, C$ such if $\eps < c \ch(\Phi)^2$ and $\Phi$ is $\eps$-balanced, then $\Phi$ is an
$$ \left(\eps, \max\left\{1/2, 1 -  \ch(\Phi)^2 + C \frac{\eps}{\ch(\Phi)^2} \right\} \right)-\text{quantum expander}.$$
\end{lemma}
We proceed to bound the Cheeger constant of a random operator. The Cheeger constant of an operator is scale-invariant, so for convenience we let $\Phi$ have Kraus operators $\samp_1, \dots, \samp_n$, each drawn from $\cN(0,  I_{d_1} \ot I_{d_2}).$ Our main observation is the following.

\begin{lemma}\label{fact:chi} Let $\Pi_1:\C^{d_1} \to \C^{d_1}, \Pi_2: \C^{d_2} \to \C^{d_2}$ be orthogonal projections, of rank $r_1, r_2$, respectively. Then $\cut(\Pi_1, \Pi_2), \vol(\Pi_1, \Pi_2), \vol(I_{d_1}, I_{d_2})$ is jointly distributed as
$$ R_1, R_1 + 2R_2, 2R_1 + 2 R_2 + 2R_3$$ where
$R_1, R_2, R_3$ are independent $\chi^2$ random variables with $F_1:=n r_1(d_2 - r_2) + n r_2(d_1-r_1), F_2:= n r_1r_2, F_3:= n(d_1 - r_1)(d_2 - r_2)$ degrees of freedom, respectively.
\end{lemma}
\begin{proof} As the distribution of $\Phi$ is invariant under the action of unitaries, the distribution of $\cut(\Pi_1, \Pi_2), \vol(\Pi_1, \Pi_2)$ depends only on the rank of $\Pi_1, \Pi_2$. Thus we may compute in the case that $\Pi_1, \Pi_2$ are coordinate projections, in which case one may verify the fact straightforwardly.
\end{proof}


 We show a sufficient condition for the Cheeger constant being bounded away from $1$ that is amenable to the previous distributional description.
\begin{lemma}\label{lem:suff}
Let $r_1, r_2$ not both zero be the ranks of projections $\Pi_1: \C^{d_1} \to \C^{d_1}, \Pi_2: \C^{d_2} \to \C^{d_2}$, and let $F_1:= n r_1(d_2 - r_2) + n r_2(d_1-r_1)$ and $F_2:=n r_1r_2.$ If
\begin{itemize}
\item for all $\Pi_1, \Pi_2$ such that $F_2 \geq (4/9) n d_1 d_2$ we have
\begin{gather}\vol(\Pi_1, \Pi_2) \geq (1/2 - \delta) \vol(I_{d_1}, I_{d_2}),\label{eq:vol}\end{gather} and
\item for all $\Pi_1, \Pi_2$ such that $F_2 < (4/9) n d_1 d_2$, we have
\begin{gather} \vol(\Pi_1, \Pi_2) \leq (4/3 + \delta)(F_1 + 2 F_2) \textrm{ and } \cut(\Pi_1, \Pi_2) \geq (2/3 - \delta) F_1,\label{eq:cut} \end{gather}
\end{itemize}
then $\ch(\Phi) \geq 1/6 - O(\delta)$ for $\delta \leq c$.
\end{lemma}
\begin{proof} By the first assumption, it remains to show that $F_1/(F_1 + 2 F_2) \geq 1/3$ provided $F_2 < (4/9) n d_1 d_2$, or $r_1 r_2 < (4/9) d_1 d_2$. Indeed, if either $r_1 = 0$ or $r_2 = 0$, then $F_2 = 0$ and $F_1>0$ and the claim holds, else
\begin{align*}F_1/(F_1 + 2 F_2) &= \frac{r_1 d_2 + r_2 d_1 - 2 r_1 r_2}{r_1 d_2 + r_2 d_1}\\
 &= 1 -2 \sqrt\frac{ r_1 r_2}{d_1 d_2} \frac{1}{ \sqrt{ r_1 d_2/r_2 d_1} + \sqrt{r_2 d_1/ r_1 d_2}} \\
 &\geq 1 - \sqrt{4/9} = 1/3.
\end{align*}

In the last inequality we used that $a + a^{-1} \geq 2$ for all $a \in \R_+$ and that $r_1 r_2 < (4/9) d_1 d_2$. \end{proof}


Next we use this to show that for fixed $\Pi_1, \Pi_2$, with high probability the events in \cref{lem:suff} hold.
\begin{lemma}\label{lem:probabilities}
Let $r_1, r_2$ not both zero be the ranks of projections $\Pi_1: \C^{d_1} \to \C^{d_1}, \Pi_2: \C^{d_2} \to \C^{d_2}$, and let $F_1:= n r_1(d_2 - r_2) + n r_2(d_1-r_1)$ and $F_2 = n r_1 r_2$. Then
\begin{itemize}
\item if $F_2 \geq (4/9) n d_1 d_2$, then \cref{eq:vol} holds with $\delta = 0$ with probability at least $1 - e^{-\Omega( n d_1 d_2)}$.
\item else, \cref{eq:cut} holds with $\delta = 0$ with probability at least $1 - e^{-\Omega( F_1)}$.
\item Finally, $\vol(\Pi_1, \Pi_2) \geq \frac{1}{2}\tr \Phi(I_{d_1}) (d_1/d_2)$ with probability at least $1 - e^{- \Omega(n r_1 d_2 + n r_2 d_1)}$.
\end{itemize}
\end{lemma}


\begin{proof}
Recall from \cref{fact:chi} that, $\cut(\Pi_1, \Pi_2), \vol(\Pi_1, \Pi_2), \vol(I_{d_1}, I_{d_2})$ are jointly distributed as $R_1, R_1 + 2R_2, 2R_1 + 2R_2 + 2R_3$ for $R_1, R_2, R_3$ independent $\chi^2$ random variables with $F_1, F_2, F_3$ degrees of freedom, respectively. Thus it is enough to show that
\begin{itemize}
\item If $nr_1 r_2 \geq (4/9) n d_1 d_2$, then with probability $1 - e^{- \Omega( n d_1 d_2)}$ we have $R_2 > R_3$, and
\item if $nr_1 r_2 \leq (2/3) n d_1 d_2$, then with probability $1 - e^{- \Omega(F_1)}$ we have $R_1 \geq (2/3) F_1$ and $R_1 + 2R_2 \leq (4/3) (F_1 + 2 F_2),$
\item and with probability $1 - e^{- \Omega(F_1 + 2 F_2)}$, $R_1 + 2R_2 \geq (2/3) (F_1 + 2 F_2) = (2/3) n (r_1 d_2 + r_2 d_1)$ and $R_1 + R_2 + R_3 \leq (4/3)(F_1 + F_2 + F_3) = (4/3)n d_1 d_2$.
\end{itemize}
All three follow from standard results for concentration of $\chi^2$ random variables; see e.g. \cite{W19}. To prove the first item, first note that $F_1 + 2 F_2 \geq (4/3)(F_1 + F_2 + F_3)$, because
\begin{align*}
(F_1 + 2 F_2)/( F_1 + F_2 + F_3) &= \frac{r_1}{d_1} + \frac{r_2}{d_2}\\
 &= \sqrt{ \frac{r_1 r_2}{d_1 d_2}}\left( \sqrt{ \frac{r_1 d_2}{r_2 d_1}} + \sqrt{ \frac{r_2 d_1}{r_1 d_2}}\right) \geq (2/3) \cdot 2 \geq 4/3.
\end{align*}
In particular, $F_2 \geq (2/3)(F_2 + F_3)$. Thus, with probability $1 - e^{- c F_2}$, $R_2 \geq (5/9) (F_2 + F_3)$ and $R_2 + R_3 \leq (10/9) (F _2 + F_3),$ so $R_2 > R_3$ with probability $1 - e^{- c F_2} \geq 1 - e^{- c n d_1 d_2}$. The second and third items are straightforward.
\end{proof}

Finally, we show using an epsilon net that the Cheeger constant is large for \emph{all} projections.
\begin{lemma}[\cite{FM20}]\label{lem:net} There is a $\delta$-net $N$ of the rank $r$ orthogonal projections $\Pi: \C^d \to \C^d$ with $|N| = \exp(O(d r |\ln \delta|))$.
\end{lemma}
As a corollary, the number of pairs of projections $\Pi_1, \Pi_2$ of rank $r_1, r_2$ has a $\delta$-net of size on the order of $(r_1 d_1 + r_2 d_2) |\ln \delta|$.

\begin{lemma}[A net suffices]\label{lem:net-suffices}
Suppose $\|\Pi'_1 -\Pi_2\|_F, \|\Pi'_2 - \Pi_2\|_F \leq \delta$. Then
\begin{align*} |\cut(\Pi_1, \Pi_2) - \cut(\Pi'_1, \Pi'_2)| \leq4\delta \tr \Phi(I_{d_1})\\
\textrm{ and }|\vol(\Pi_1, \Pi_2) - \vol(\Pi'_1, \Pi'_2)| \leq 4\delta \tr \Phi(I_{d_1}).
\end{align*}
\end{lemma}
\begin{proof}
We first show the first inequality.
\begin{align*}|\cut(\Pi'_1, \Pi'_2) - \cut(\Pi_1, \Pi_2)| & \leq |\tr \Pi'_2 \Phi(I_{d_1} - \Pi'_1) - \tr \Pi_2 \Phi(I_{d_1} - \Pi_1)|\\
&  + |\tr (I_{d_2} - \Pi'_2) \Phi(\Pi'_2) - \tr (I_{d_2} - \Pi_2) \Phi(\Pi_2)|.
\end{align*}
We begin with the first term.
\begin{align*}&|\tr \Pi'_2 \Phi(I_{d_1} - \Pi'_1) - \tr \Pi_2 \Phi(I_{d_1} - \Pi_1)|\\
&= |\tr (\Pi'_2 - \Pi_2) \Phi(I_{d_1} - \Pi'_1) + \tr \Pi_2 \Phi(\Pi_1 - \Pi'_1)|\\
&\leq \delta\| \Phi(I_{d_1} - \Pi'_1)\|_F + \delta\| \tr \Phi^*(\Pi_2)\|_F\\
& \leq 2 \delta \tr \Phi(I_{d_1}).
\end{align*}
The second term follows by symmetry. The proof of the second inequality is similar.
\end{proof}

\begin{lemma}[Applying union bound]\label{lem:union}
Let $d_1 < d_2$. Suppose $n \geq C \frac{d_2}{d_1} \log (d_2/d_1)$. Then $\ch(\Phi) = \Omega(1)$ with failure probability $O(e^{- \Omega(n d_1)})$.
\end{lemma}
\begin{proof} Let $\delta' \leq c d_1/d_2$. Let $\cN(r_1, r_2)$ be a $\delta'$-net for the pairs of projections of rank $r_1, r_2$, respectively, with $|\cN(r_1, r_2)| = e^{O((d_1r_1 + d_2 r_2) \log(1/\delta'))}$, and $N = \bigcup_{r_1, r_2} \cN(r_1, r_2)$. We claim that it is enough to show that with probability $\exp( - c n d_1 )$, for all $r_1, r_2$ not both zero we have
\begin{enumerate}
\item \cref{eq:vol} holds with $\delta = 0$ for every $\Pi_1,\Pi_2 \in \cN(r_1, r_2)$ when $r_1 r_2 \geq (4/9) d_1 d_2$,
\item  and \cref{eq:cut} holds with $\delta =0$ for all $\Pi_1, \Pi_2 \in \cN(r_1, r_2)$ otherwise.
\item $\vol(\Pi_1, \Pi_2) \geq \tr \Phi(I) (d_1/d_2)$.
\end{enumerate}
Let us check that the hypotheses of \cref{lem:suff} with $\delta \leq c$ are implied by these three items; this will imply that conditioned on the three items we have $\ch(\Phi) \geq \Omega(1)$. Because every pair $(\Pi'_1,\Pi'_2)$ of projections of ranks $r_1,r_2$ is most $\delta$ far from some element $(\Pi_1, \Pi_2)$ of $\cN(r_1,r_2)$, then by \cref{lem:net-suffices} (and the inequality $\vol(\Pi_1, \Pi_2) \geq \tr \Phi(I)(d_1/d_2)$) we have
\begin{align*} (1 - 4 \delta'  \cdot d_2/d_1) \vol(\Pi_1, \Pi_2) \leq  \vol(\Pi_1', \Pi_2') \leq  (1 + 4 \delta'  \cdot d_2/d_1) \vol(\Pi_1, \Pi_2).\end{align*}
By assumption, $4 \delta' \cdot d_2/d_1 \leq c$. This shows \cref{eq:vol} holds with $\delta \leq c$ when $r_1 r_2 \geq (4/9) d_1 d_2$. It remains to show that \cref{eq:cut} holds otherwise. Firstly, when $r_1 r_2 < (4/9) d_1 d_2$ we have
\begin{gather} \vol(\Pi_1', \Pi_2') \leq (1 + c) \vol(\Pi_1, \Pi_2) \leq  (1 + c)(4/3)(F_1 + 2 F_2).\label{eq:not-net-9a}\end{gather}
  Next, observe that
$$  \cut(\Pi_1', \Pi_2') \geq \cut(\Pi_1, \Pi_2) - c \vol(\Pi_1, \Pi_2).$$
In the proof of \cref{lem:suff} it is shown that if $r_1 r_2 < (4/9) d_1 d_2$ then $F_1 \geq \frac{1}{3} (F_1 + 2 F_2)$, in which case
\begin{align}
\cut(\Pi_1', \Pi_2') &\geq \cut(\Pi_1, \Pi_2) - c \vol(\Pi_1, \Pi_2) \geq \nonumber\\
& \geq (2/3) F_1 -  c (4/3)(F_1 + 2 F_2) \geq (2/3 - c) F_1.\label{eq:not-net-9b}
\end{align}

Taken together, \cref{eq:not-net-9a,eq:not-net-9b} show that \cref{eq:cut} holds when $r_1 r_2 < (4/9) d_1 d_2$.

We must next show that the three conditions hold with the desired probability. We show that for fixed $r_1, r_2$, each item holds with probability at least $1 - e^{n (r_1 d_2 + r_2 d_1)}$. The sum of $e^{-\Omega(n (r_1d_2 + r_2 d_1))}$ over all $0 \leq r_1 \leq d_1, 0 \leq r_2 \leq d_2$ apart from $r_1 = r_2 = 0$ is $O(e^{- \Omega( n d_1)})$, so the conditions hold for all $r_1, r_2$ with the desired probability. Note that by our choice of $n$ we have $(d_1r_1 + d_2 r_2) \log(1/\delta') \leq c n (r_1d_2 + r_2 d_1)$ for $r_1, r_2$ not both zero.

We first bound the failure probability for the first item. By \cref{lem:probabilities}, if $r_1 r_2 \geq (4/9) d_1 d_2$ then \cref{eq:vol} holds for every $\Pi \in \cN(r_1, r_2)$ with probability
\begin{align*}
1 - |\cN(r_1, r_2)|e^{- \Omega( n d_1 d_2) } &= 1 - |\cN(r_1, r_2)| e^{ - \Omega(n (r_2d_1 + r_1d_2))}\\
&= 1 - e^{ - \Omega(n (r_2d_1 + r_1d_2))}.
\end{align*}

Next we bound the probability for the second item. By \cref{lem:probabilities}, \cref{eq:cut} holds for fixed $\Pi \in \cN(r_1, r_2)$ with probability $1 - e^{-\Omega( F_1)}$, but as in the proof of \cref{lem:suff} we have $F_1 \geq \frac{1}{3} (F_1 + 2 F_2)$ when $r_1 r_2 < (4/9) d_1 d_2$, so $F_1 = \Omega(n (r_1d_2 + r_2 d_1))$. Now, by the union bound and the lower bound on $n$, \cref{eq:cut} holds for every element of $\cN(r_1, r_2)$ with probability $1 - |\cN(r_1,r_2)| e^{-\Omega(n (r_1d_2 + r_2 d_1)} = 1 - e^{-\Omega(n (r_1d_2 + r_2 d_1)}$.


The third item holds with probability $1 - e^{-\Omega(n (r_1d_2 + r_2 d_1))}$ by \cref{lem:probabilities}, so by a similar application of the union bound and our choice of $n$ it holds for all elements of $\cN(r_1, r_2)$ with probability $1 - e^{-\Omega(n (r_1d_2 + r_2 d_1))}$. \end{proof}

\begin{proof}[Proof of \cref{thm:operator-cheeger}]
To prove \cref{thm:operator-cheeger}, we apply \cref{lem:op-cheeger} using \cref{prop:gradient-bound} to bound the balancedness of $\Phi$ and \cref{lem:union} to bound $\ch(\Phi)$. Indeed, $\|\nabla_a f\|_{op} \leq \eps_0$ for $a \in \{1,2\}$ if and only if $\Phi$ is $\eps_0$-balanced, so by \cref{prop:gradient-bound} the operator $\Phi$ is $\eps_0$-balanced with probability $1 -  e^{-\Omega(n d_1 \eps_0^2)} - e^{-\Omega(n d_2 \eps_0^2)} \geq 1 - 2e^{-\Omega(n d_1 \eps_0^2)}$ provided $n \geq C\eps_0^{-2} d_2/d_1 $. Setting $\eps_0 = \eps \sqrt{{d_2 }/{n d_1}}$ proves the balancedness claim. For the expansion, \cref{lem:union} shows $\ch(\Phi) = \Omega(1)$ with failure probability $O(e^{- \Omega(n d_1)}) = O(e^{- \Omega(d_2 \eps^2)})$. By \cref{lem:op-cheeger}, $\Phi$ is an $(\eps \sqrt{{d_2 }/{n d_1}}, 1 - \lambda)$-quantum expander for some absolute constant $\lambda$.
\end{proof}



%-----------------------------------------------------------------------------
\section{Proof of concentration for matrix normal model}\label{app:flipflop-concentration}
%-----------------------------------------------------------------------------

\begin{proof}[Proof of \cref{lem:flipflop-concentration}]
For convenience, we consider the differently normalized random variable $Z = Y/\sqrt{nd_1}$. Note that $Z$ satisfies
$Z_i = X_i \Phi_X^*(I_{d_1})^{-1/2} =  X_i (\sum_{i = 1}^n X_i^\dagger X_i)^{-1/2}.$ Thus we need to bound the random matrix
\begin{align} \sum_{i = 1}^n Z_i Z_i^{\dagger} - \frac{d_2}{d_1} I_{d_1} = \sum_{i = 1}^n X_i \left( \sum_{j}  X_i^{\dagger} X_i\right)^{-1} X_i^\dagger - \frac{d_2}{d_1} I_{d_1}.\label{eq:1marg}
\end{align}
Since we are interested in the spectral norm of \cref{eq:1marg}, we will consider the random variable $\langle \xi,  \sum_{i = 1}^n Z_i Z_i^{\dagger} \xi \rangle$ for a fixed unit vector $\xi \in \R^{d_1}$. We will show that this variable $\xi$ is highly concentrated, and apply a union bound over a net of the unit vectors. To show the concentration, we first cast $\langle \xi,  \sum_{i = 1}^n Z_i Z_i^{\dagger} \xi \rangle$ as the inner product between a random orthogonal projection and a fixed one.

 Considering $Z_i$ as a $d_1 \times d_2$ matrix, we can consider $Z$ as an $n d_1 \times d_2$ matrix by vertically concatenating the $Z_i$. By definition of the flip-flop step, $Z^\dagger Z = \sum Z_i^\dagger Z_i = I_{d_2}$, so the columns of $Z$ are an orthonormal basis of $\R^{n d_1}$. Here $\dagger$ denotes transpose for $nd_1 \times d_2$ matrices. In fact, the columns of $Z$ are a \emph{uniformly random} orthonormal basis of a $d_2$ dimensional subspace $\R^{n d_1}$; that is, they are a uniformly random element of the Steifel manifold. Thus, $Z Z^\dagger$ is a uniformly random rank $d_2$ orthonormal projection on $\R^{n d_1}$. We can now write
 $$ \langle \xi,  \sum_{i = 1}^n Z_i Z_i^{\dagger} \xi \rangle = \langle Z Z^\dagger,  \xi \xi^\dagger \ot I_{n} \rangle.$$
The matrix $\xi \xi^{*} \otimes I_{n}$ is a rank $n$ projection on $\R^{n d_1}$. We use the following result on the inner product of random projections.

\begin{theorem} [Lemma III.5 in \cite{hayden2006aspects}] Let $P$ be a uniformly random orthogonal projection of rank $a$ on $\R^{m}$ and let $Q$ be a fixed orthogonal projection of rank $b$ on $\R^{m}$. Then
\[ \Pr \left[ \langle P, Q \rangle \not\in (1 \pm \eps) \frac{ab}{m} \right] = e^{ - \Omega( ab \eps^{2} ) }.  \]
\end{theorem}
We may apply the above theorem with $Q = \xi \xi^{*} \otimes I_{n}$, $m = n d_1$, $a = d_2$, and $b = n$ to obtain
\begin{gather}\Pr\left[ \langle \xi,  \sum_{i = 1}^n Z_i Z_i^{\dagger} \xi \rangle \not\in (1 \pm \eps_0) \frac{d_2}{ d_1} \right] = e^{ - \Omega( n d_2 \eps_0^{2} ) }. \label{eq:fixed-concentration} \end{gather}
Next we apply a standard net argument for the unit vectors over $\R^{nd_1}$. We use the following lemma.
\begin{lemma}[Lemma 5.4 \cite{vershynin2010introduction}]\label{lem:versh-net} Let $A$ be a symmetric $d\times d$ matrix, and let $\mathcal{N}_\delta$ be an $\delta$-net of $\S^{d-1}$ for some $\delta \in [0,1)$. Then
$$\|A\|_{op} \leq (1 - 2 \delta)^{-1} \sup_{\xi \in \mathcal{N}_\delta} | \langle \xi, A \xi \rangle|.$$
\end{lemma}
We apply the above lemma with $A = \sum_{i = 1}^n Z_i Z_i^{\dagger} - \frac{d_2}{d_1} I_{d_1}$ and $d = d_1$.
Fix a net $\mathcal{N} = \mathcal{N}_\delta$ for $\delta = 1/4$; by standard packing bounds (e.g. Lemma 4.2 in \cite{vershynin2010introduction}) we may take $|\mathcal{N}| \leq 9^{d_1}$. By \cref{eq:fixed-concentration} and the union bound, with failure probability $9^{d_1} e^{- \Omega (n d_2 \eps_0^2)}$ we have that $|\langle \xi , A \xi \rangle| \leq \frac{d_2}{d_1} \eps_0$ for all $\xi \in \mathcal{N}$, and by \cref{lem:versh-net} this event implies $\|A\|_{op} \leq 2  \frac{d_2}{d_1} \eps_0$.

It remains to translate our bound on $A$ to our desired bound on $\Phi_Y$. Because $Z = Y/\sqrt{nd_1}$, $\Phi_Y = n d_1 \Phi_Z$. Thus $A = \frac{1}{nd_1}\Phi_Y(I_{d_2}) - \frac{d_2}{d_1} I_{d_1}$, so $\|\frac{1}{nD}\Phi_Y(I_{d_2}) - \frac{1}{d_1} I_{d_1}\|_{op} = \frac{1}{d_2} \|A\|_{op}$. Setting $\eps_0 = \eps \sqrt{\frac{d_1}{4n d_2}}$ shows $\|\frac{1}{nD}\Phi_Y(I_{d_2}) - \frac{1}{d_1} I_{d_1}\|_{op} \leq \frac{\eps}{\sqrt{n D}}$ with failure probability at most $9^{d_1} e^{- \Omega(d_1 \eps^2)}$, which is at most $e^{ - \Omega(d_1) \eps^2}$ provided $\eps$ is a large enough constant.\end{proof}


\section{Relative error metrics}\label{sec:rel-error}

In this section we discuss the properties of our relative error metrics $d_F, d_{op}$. First note that they can be related to the usual norms by following inequalities:
\begin{align*}\|B^{-1}\|_{op}^{-1}D_F(A\Vert B)&\leq \norm{A - B}_F \leq \norm{B}_{\op} \, D_F(A\Vert B)\\
\|B^{-1}\|_{op}^{-1} D_{op}(A\Vert B) &\leq \norm{A  - B}_{\op} \leq \norm{B}_{\op} \, D_{\op}(A \Vert B).
\end{align*}






Next we state the approximate triangle inequality, also called a \emph{local} triangle inequality in \cite{yang1999information}, and approximate symmetry for our relative error metrics.
\begin{lemma}\label{lem:triangle-ineq}
Let $A, B, C \in \PD(d)$. Let $D \in D_{op}, D_F$. Provided $D(A||B), D(B||C)$ are at most an absolute constant $c$, we have
\begin{align}D(A||C) &= O(D(A||B) + D(B||C)),\label{eq:tri}\\
D(B||A) &= O(D(A||B))\label{eq:sym} \text{, and }\\
D(A^{-1}||B^{-1}) &= O(D(A||B)).
\end{align}
\end{lemma}
For $D_{op}$, the result is \cite[Lemma C.1]{FM20}. For $D_F$, the result holds because because $D_F(A|| B) = \Theta(d(A, B))$ if either is at most some absolute constant (shown below). Because $d(A, B)$ is a metric, it automatically satisfies \cref{eq:tri,eq:sym}. Furthermore, $d(A,B) = d(A^{-1}, B^{-1})$ by direct calculation. We next consider the relationship between $D_F$ and other dissimilarity measures in the statistics literature.

\begin{prop}[Relationships between dissimilarity measures.]\label{prop:dissimilarities} There is a constant $c > 0$ such that the following holds. If any of $D_F(\Theta_1|| \Theta_2)$, $d_{TV}(\mathcal{N}(0, \Theta_1^{-1}), \mathcal{N}(0, \Theta_2^{-1})$, $d_{KL}(\mathcal{N}(0, \Theta_1^{-1}) || \mathcal{N}(0, \Theta_2^{-1})$ or the Fisher-Rao distance $d(\Theta_1, \Theta_2)$ is at most $c$, then 
$$ D_F(\Theta_1|| \Theta_2) \asymp  d_{TV}(\mathcal{N}(0, \Theta_1^{-1}), \mathcal{N}(0, \Theta_2^{-1})^2 \asymp d_{KL}(\mathcal{N}(0, \Theta_1^{-1}) || \mathcal{N}(0, \Theta_2^{-1}) \asymp d(\Theta_1, \Theta_2).$$
\end{prop}
\begin{proof}
For the relationship between $D_F(\Theta_1|| \Theta_2)^2$ and the Fisher-Rao distance $d(\Theta_1, \Theta_2)^2$, observe that the former is $\sum_{i = 1}^d (\lambda_i - 1)^2$ and the latter is $\sum_{i = 1}^d (\log \lambda_i)^2$ for the eigenvalues $\lambda_i$ of $\Theta_2^{-1} \Theta_1$. The relationship follows because in any fixed interval not containing $0$, $\lambda-1 \asymp \log \lambda$.

To relate $D_F$ to the total variation distance, we use the following bound from \cite{devroye2018total}:
$$.01  \leq \frac{d_{TV}(\mathcal{N}(0, \Theta_1^{-1}), \mathcal{N}(0, \Theta_2^{-1})}{D_F(\Theta_1 || \Theta_2)} \leq 1.5.$$ 
This implies that if either the numerator or denominator is a small enough constant, then they are on the same order. 

Next we reason for the relative entropy. If $D_{KL}(\mathcal{N}(0, \Theta_1^{-1}), \mathcal{N}(0, \Theta_2^{-1}) \leq c$ or $D_{op}(\Theta_2 || \Theta_1) \leq 1/2$, then we have 
$$D_{KL}(\mathcal{N}(0, \Theta_1^{-1}), \mathcal{N}(0, \Theta_2^{-1})  \asymp D_{F}(\Theta_2 || \Theta_1)^2.$$ This bound can be seen explicitly from the formula 
\begin{align*}D_{KL}(\mathcal{N}(0, \Theta_1^{-1}), \mathcal{N}(0, \Theta_2^{-1}) &= \frac{1}{2} \tr \Theta_1^{-1} \Theta_2 - \frac{1}{2}\log\det(\Theta_1^{-1} \Theta_2) - \frac{d}{2}\\
& = \frac{1}{2} \sum_{i =1}^d (\lambda_i -1 -  \log \lambda_i)
\end{align*}
 where $\lambda_i \in 1 \pm D_{op}(\Theta_2||\Theta_1)$ are the eigenvalues of $\Theta_1^{-1} \Theta_2$ and the fact that $\lambda - 1 -\log\lambda \asymp  (\lambda - 1)^2$ on $[1/2, 3/2]$. Choose $c$ small enough that $\frac{1}{2}(\lambda - 1 -\log\lambda) \leq c $ implies $\lambda \in [1/2, 3/2]$. Finally, there is some absolute constant $c$ such that $D_F(\Theta_1 || \Theta_2) \leq c$ or $d(\Theta_1, \Theta_2) \leq c$ then $D_F(\Theta_1 || \Theta_2) \asymp d(\Theta_1, \Theta_2)$.\end{proof}
\end{appendix}


%=============================================================================
\section*{Acknowledgements}
%=============================================================================
This work was supported in part by NWO Veni grant no.~680-47-459 and NWO grant OCENW.KLEIN.267.

%%%%%%%%%%%%%%%%%%%%%%%%%%%%%%%%%%%%%%%%%%%%%%
%% Supplementary Material, if any, should   %%
%% be provided in {supplement} environment  %%
%% with title and short description.        %%
%%%%%%%%%%%%%%%%%%%%%%%%%%%%%%%%%%%%%%%%%%%%%%
%\begin{supplement}
%\stitle{???}
%\sdescription{???.}
%\end{supplement}

\bibliographystyle{imsart-nameyear}
\bibliography{refs}

\end{document}
