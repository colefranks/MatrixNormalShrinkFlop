\documentclass{article}
\usepackage[utf8]{inputenc}
\usepackage{amsmath}
\usepackage{amsthm}
\usepackage{amssymb}
\usepackage{xcolor,cleveref}
\newtheorem{theorem}{Theorem}
\newtheorem{corollary}{Corollary}
\newtheorem{prop}{Proposition}
\newtheorem{lemma}[theorem]{Lemma}
\newtheorem{corol}[theorem]{Corollary}
\newtheorem{fact}[theorem]{Fact}
\newtheorem{claim}[theorem]{Claim}
\newtheorem{remark}{Remark}
\newtheorem{definition}{Definition}
\newtheorem{example}{Example}

\newcommand{\junk}[1]{}


\newcommand{\R}{{\mathbb{R}}}
\newcommand{\C}{{\mathbb{C}}}
\newcommand{\ot}{\otimes}
\newcommand{\Q}{{\mathbb{Q}}}
\newcommand{\N}{{\mathbb{N}}}
\newcommand{\Z}{{\mathbb{Z}}}
\renewcommand{\S}{\mathbb{S}}

\newcommand{\E}{\mathbb{E}}
\renewcommand{\Pr}{\mathbb{P}}

\newcommand\eps{\varepsilon}

\newcommand\FF{\mathcal{F}}
\newcommand\HH{\mathcal{H}}
\newcommand\GG{\mathcal{G}}
\newcommand\TT{\mathcal{T}}
\newcommand\PP{\mathcal{P}}
\newcommand\tr{\operatorname{Tr}}
\newcommand\RR{\mathcal{R}}
\newcommand\CC{\mathcal{C}}
\newcommand\BB{\mathcal{B}}
\newcommand\II{\mathcal{I}}
\renewcommand\AA{\mathcal{A}}
\newcommand{\MM}{\mathcal{M}}
\newcommand\AP{\mathcal{AP}}

\newcommand{\eqdef}{:=}
\newcommand{\maps}{\colon}
\newcommand{\ale}{\lesssim}
\newcommand{\age}{\gtrsim}
\newcommand{\CF}[1]{{\color{purple}[CF: #1]}}
\newcommand{\AR}[1]{{\color{green}[AR: #1]}}
\newcommand{\RMO}[1]{{\color{red}[RMO: #1]}}
\newcommand{\TODO}[1]{{\color{blue}[TODO: #1]}}


\title{Logarithmic sample complexity for dense tensor models}
\author{Cole Franks, Rafael Oliveira, Akshay Ramachandran}
\date{February 2020}

\begin{document}

\maketitle
\TODO{
\begin{enumerate}
\item Copy pasta stuff from main.
\item 
\end{enumerate}

}


\section{Introduction}
\CF{background, history; remark that we automatically know flip-flop works but our strong convexity will give efficient algorithms in this setting. } 

\section{Structured covariance estimation}
We study covariance estimation in the tensor normal model, in which $X$ is sampled according to $N(0, \Sigma = \bigotimes_{i = 1}^{k} \Sigma_i)$ for $\Sigma_i$ a positive-semidefinite $d_i\times d_i$ matrix.

\begin{definition}\CF{the MLE for tensor models}

\end{definition}

Our main theorem is the following:

\begin{theorem} The MLE for $\bigotimes_{i = 1}^{d_1} \Sigma_i$ from $n$ independent samples satisfies 
$$ d(\hat{\Sigma}, \Sigma) = O(\TODO{}). $$
with probability $\TODO{}$.
\CF{presumably there will be some dimension-dependent factors}.
\end{theorem}

\begin{proof}
\CF{cite the MLE paper and then our sections \cref{sec:g-convex,sec:scaling}.}
\end{proof}

\begin{theorem} If $\hat{\Sigma}$ denotes the MLE estimator for $\Sigma$, then provided $n \geq \TODO{}$ the flip-flop algorithm computes $\underline{\Sigma}$ 
$$ d(\underline{\Sigma}, \hat{\Sigma}) = O(\TODO{1/\sqrt{n}}) $$
in $O(\TODO{\log(1/\eps)})$ iterations with probability at least $\TODO{}$.
\end{theorem}

\begin{proof}
\CF{cite the MLE paper and then our sections \cref{sec:g-convex,sec:scaling}.}
\end{proof}



\section{Geodesic convexity}\label{sec:g-convex}
\TODO{Put these lemmas about how if you are geodesically convex on a ball, you get good condition number bounds/convergence rates for any descent method}

\section{Operator and tensor scaling}\label{sec:scaling}
\TODO{lower bound Hessian for operators and tensors for all different formats; we hope to get strong convexity with $\prod d_i /(d_1^2 + \dots + d_k^2)$ samples. I am concerned that a KLR19-style operator norm type theorem is needed to get $\tilde{O}$ of this, but we will do what we can with the Frobenius bounds for now; I'd expect to need at least $\max_i \sqrt{d_i}$ too many samples.}\\
\TODO{It would also be nice to have that tight example for the $\log$ in KLR19...}

\section{Noise}
\TODO{make sure things work under some error in the data}







%If $\langle \cdot , \cdot \rangle_P$ is the metric at a point $P$, then the new metric $\langle \cdot, \cdot \rangle_{R,P}$ will be given by 
%$$ \langle X, Y \rangle_{R,P} = \langle R^{1/2} X R^{1/2}, R^{1/2} Y R^{1/2} \rangle_{P}.$$

%At the identity the metric takes the form $\langle X, Y \rangle_R = \tr \sqrt{R} X  \sqrt{R} Y $. In particular the length of the geodesic $t\mapsto \sqrt{P}e^{Xt}\sqrt{P}$ for $t \in [0,1]$ is $\langle X, X \rangle_R$. 

\bibliographystyle{alpha}
\bibliography{refs}

\end{document}
