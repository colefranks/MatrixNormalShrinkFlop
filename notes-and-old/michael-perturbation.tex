\documentclass{article}
\usepackage[utf8]{inputenc}
\usepackage{amsmath}
\usepackage{amsthm,mathtools,braket}
\usepackage{amssymb,bm}
\usepackage{xcolor}
\usepackage[capitalize]{cleveref}
\newtheorem{theorem}{Theorem}
\newtheorem{corollary}{Corollary}
\newtheorem{prop}{Proposition}
\newtheorem{lemma}[theorem]{Lemma}
\newtheorem{corol}[theorem]{Corollary}
\newtheorem{fact}[theorem]{Fact}
\newtheorem{claim}[theorem]{Claim}
\newtheorem{remark}{Remark}
\newtheorem{definition}{Definition}
\newtheorem{example}{Example}

\DeclareMathOperator{\Lie}{Lie}
\DeclareMathOperator{\Lin}{L}
\DeclareMathOperator{\ope}{op}

\DeclarePairedDelimiter{\abs}{\lvert}{\rvert}
\DeclarePairedDelimiter{\norm}{\lVert}{\rVert}
\DeclarePairedDelimiter{\ip}{\langle}{\rangle}


\newcommand{\junk}[1]{}


\newcommand{\R}{{\mathbb{R}}}
\newcommand{\C}{{\mathbb{C}}}
\newcommand{\ot}{\otimes}
\newcommand{\mat}{\operatorname{Mat}}
\newcommand{\Q}{{\mathbb{Q}}}
\newcommand{\N}{{\mathbb{N}}}
\renewcommand{\vec}{\bm}
\newcommand{\Z}{{\mathbb{Z}}}
\renewcommand{\S}{\mathbb{S}}

\newcommand{\E}{\mathbb{E}}
\renewcommand{\Pr}{\mathbb{P}}

\newcommand\eps{\varepsilon}

\newcommand\FF{\mathcal{F}}
\newcommand\HH{\mathcal{H}}
\newcommand\GG{\mathcal{G}}
\newcommand\SL{\operatorname{SL}}
\newcommand\PD{\operatorname{PD}}
\newcommand\GL{\operatorname{GL}}
\newcommand\TT{\mathcal{T}}
\newcommand\PP{\mathcal{P}}
\newcommand\tr{\operatorname{Tr}}
\newcommand\RR{\mathcal{R}}
\newcommand\CC{\mathcal{C}}
\newcommand\BB{\mathcal{B}}
\newcommand\II{\mathcal{I}}
\renewcommand\AA{\mathcal{A}}
\newcommand{\MM}{\mathcal{M}}
\newcommand\AP{\mathcal{AP}}

\newcommand{\eqdef}{:=}
\newcommand{\maps}{\colon}
\newcommand{\ale}{\lesssim}
\newcommand{\age}{\gtrsim}
\newcommand{\CF}[1]{{\color{purple}[CF: #1]}}
\newcommand{\AR}[1]{{\color{green}[AR: #1]}}
\newcommand{\RMO}[1]{{\color{red}[RMO: #1]}}
\newcommand{\TODO}[1]{{\color{blue}[TODO: #1]}}


\title{Michael Perturbation Bound}
\author{CF, MW}
\date{February 2020}

\begin{document}

\maketitle


\section{Geodesic convexity}\label{sec:g-convex}
\TODO{Put these lemmas about how if you are geodesically convex on a ball, you get good condition number bounds/convergence rates for any descent method}

\subsection{Perturbation bound}
In applying what follows to the tensor action, we'll have
\begin{enumerate}
\item $G$ is the set of $k$-tuples of invertible matrices. 
\item $\pi$ is just the action $(g_1, \dots, g_k) \to g_1 \ot \dots \ot g_k$
\item $P$ is the set of $k$-tuples of PSD matrices.
\item $K$ is the set of $k$-tuples of unitaries.
\item $N(\pi)$ is $\sqrt{3}$.
\end{enumerate}

%Let's first just look at the Hessian at the identity. Since we are looking at the function $\log \langle v, P v \rangle$ over $P$ positive-definite, we want to compute$$ H_{I, v}(X, X) := \partial_t^2 \langle v, \pi(e^{tX}) v \rangle_{t = 0}$$ for $X = (X_1, \dots, X_k)$ a tuple of Hermitians. $e^{t(X)}$ is then a tuple of PSD's, and $\pi$ is just the action $(g_1, \dots, g_k) \to g_1 \ot \dots \ot g_3$. Let $\Pi$ denote the Lie algebra action, which here for e.g. $k = 3$ is $(X_1, \dots, X_3)\mapsto X_1 \ot I \ot I + I \ot X_2 \ot I + I \ot I \ot X_3. $ 



\subsubsection{Michael's write-up}
\CF{I'm literally copy-pasting from Michael's write-up of our calculation, so obviously we can't include this as-is.}\\
As usual, $G \subseteq \GL(n)$ is Zariski closed and $G = G^\dagger$, so that $K = G \cap U(n)$ is maximally compact and $G = K^{\mathbb C}$.
Let $P := \exp(i\Lie(K))$. Then $G/K \cong P$ via $g \mapsto g^\dagger g$.
Geodesics through $p\in P$ take the form $\gamma(t) = \sqrt p e^{Xt} \sqrt p$ for $X\in i\Lie(K)$.
We write $B_\delta(p) = \{ \sqrt p e^X \sqrt p : \norm{X}_F \leq \delta \}$ for the geodesic $\delta$-ball about~$p$.
Let $\pi\colon G \to \GL(V)$ be a representation of~$G$ with $K$-invariant inner product and $\Pi\colon\Lie(G)\to\Lin(V)$ the Lie algebra representation.
The \emph{weight norm} of $\pi$ is defined as
\begin{align*}
  N(\pi) := \norm{\Pi}_{F\to\ope} = \max \left\{ \norm{\Pi(X)}_{\ope} : X \in i\Lie(K), \norm{X}_F \leq 1 \right\}.
\end{align*}

We consider the problem of minimizing the Kempf-Ness function associated with a vector $0\neq v\in V$:
\begin{align*}
  f_v \colon P \to \R, \quad p \mapsto \log \ip{v, \pi(p) v}
\end{align*}
% Note that
% \begin{align*}
%   \partial_t f_v(\sqrt p e^{Xt} \sqrt p)
% = \partial_t \log \braket{\pi(\sqrt p) v, e^{\Pi(X)t} \pi(\sqrt p) v}
% = \frac {\braket{\pi(\sqrt p) v, \Pi(X) e^{\Pi(X)t} \pi(\sqrt p) v}} {\braket{\pi(\sqrt p) v, e^{\Pi(X)t} \pi(\sqrt p) v}}
% \end{align*}
% and
% \begin{align*}
%   \partial_t^2 f_v(\sqrt p e^{Xt} \sqrt p)
% % &= \partial_t \frac {\braket{\pi(\sqrt p) v, \Pi(X) e^{\Pi(X)t} \pi(\sqrt p) v}} {\braket{\pi(\sqrt p) v, e^{\Pi(X)t} \pi(\sqrt p) v}} \\
% &= \frac {\braket{\pi(\sqrt p) v, \Pi(X)^2 e^{\Pi(X)t} \pi(\sqrt p) v} \times \braket{\pi(\sqrt p) v, e^{\Pi(X)t} \pi(\sqrt p) v} - \braket{\pi(\sqrt p) v, \Pi(X) e^{\Pi(X)t} \pi(\sqrt p) v}^2} {\braket{\pi(\sqrt p) v, e^{\Pi(X)t} \pi(\sqrt p) v}^2}.
% \end{align*}
Its differential and geodesic Hessian are given by
\begin{align}
\label{eq:d}
  Df_v(p)[X]
= \partial_{t=0} f_v(\sqrt p e^{Xt} \sqrt p)
&= \braket{w, \Pi(X) w}, \\
\label{eq:d2}
  D^2f_v(p)[X,X]
= \partial^2_{t=0} f_v(\sqrt p e^{Xt} \sqrt p)
&= \braket{w, \Pi(X)^2 w} - \braket{w, \Pi(X) w}^2 \\
\nonumber
&= \braket{w, \bigl(\Pi(X) - \braket{w, \Pi(X) w}\bigr)^2 w} \geq 0
\end{align}
where $w = \frac {\pi(\sqrt p) v}{\norm{\pi(\sqrt p) v}}$.
We say that a function $f\colon P\to\R$ is \emph{$L$-geodesically smooth} at $p$ if
\begin{align*}
  D^2f(p)[X,X] \leq L \norm{X}_F^2
\end{align*}
for all $X\in i\Lie(K)$.
\Cref{eq:d2} shows that $f_v$ is $4N(\pi)^2$-smooth.
Finally, recall the moment map which assigns to any $v$ the gradient of $f_v$ at $p=I$:
\begin{align*}
  \mu\colon V\setminus\{0\}\to i\Lie(K), \quad \tr[\mu(v)X] = Df_v(I)[X].
\end{align*}

\paragraph{Strong convexity.}
We say that a function $f\colon P\to\RR$ is \emph{$\lambda$-geodesically strongly convex} at $p$ if
\begin{align*}
  D^2f_v(p)[X,X] \geq \lambda \norm{X}_F^2
\end{align*}
for all $X\in i\Lie(K)$.
We now show that for the function~$f_v$ this notion is stable.

\begin{lemma}\label{lem:hessian stable}
Let $0\neq v\in V$ such that $f_v$ is $\lambda$-geodesically strongly convex at~$q$ for some~$0<\lambda < 4N(\pi)^2$.
Then, $f_v$ is $\lambda/2$-geodesically strongly convex on $B_\delta(q)$, where $\delta = {c\lambda / N(\pi)^3}$ and $c = 1/(12\sqrt{2} e)$.
\end{lemma}
\begin{proof}
We assume without loss of generality that $\norm{v}=1$ and $q=I$.
% For the second claim, use that $f_v(\sqrt p e^H \sqrt p) = f_{\pi(\sqrt p) v}(e^H)$.
Let $p = e^H\in P$ and $w = \frac {\pi(\sqrt p) v}{\norm{\pi(\sqrt p) v}}$.
Then, using \cref{eq:d2}, we find that, for all $X\in i\Lie(K)$,
\begin{equation}\label{eq:hess first}
\begin{aligned}
&\quad \abs[\big]{D^2f_v(p)[X,X] - D^2f_v(I)[X,X]} \\
&\leq \abs[\big]{\braket{w, \Pi(X)^2 w} - \braket{v, \Pi(X)^2 v}} + \abs[\big]{\braket{w, \Pi(X) w}^2 - \braket{v, \Pi(X) v}^2} \\
&\leq \abs[\big]{\braket{w, \Pi(X)^2 w} - \braket{v, \Pi(X)^2 v}} + \abs[\big]{\braket{w, \Pi(X) w} - \braket{v, \Pi(X) v}}\abs[\big]{\braket{w, \Pi(X) w} + \braket{v, \Pi(X) v}} \\
&\leq \norm{ww^\dagger - vv^\dagger}_{\tr} \norm{\Pi(X)^2}_{\ope} + 2 \norm{ww^\dagger - vv^\dagger}_{\tr} \norm{\Pi(X)}_{\ope}^2 \\
&= 3 \norm{ww^\dagger - vv^\dagger}_{\tr} \norm{\Pi(X)}^2_{\ope} \\
&\leq 3 N(\pi)^2 \norm{ww^\dagger - vv^\dagger}_{\tr} \norm{X}_F^2
\end{aligned}
\end{equation}
In the second to last step, we used that $\Pi(X) = \Pi(X)^\dagger$ and $\norm{Y^\dagger Y}_{\ope} = \norm{Y}_{\ope}^2$ for any $Y\in\Lin(V)$.
Next,
\begin{align*}
  \norm{ww^\dagger - vv^\dagger}_{\tr}
&= \sqrt2 \norm{ww^\dagger - vv^\dagger}_F
\leq 2 \sqrt2 \norm{w - v}
\leq 2 \sqrt2 \bigl( \underbrace{\norm{w - \pi(\sqrt p) v}}_{=\abs[\big]{1 - \norm{\pi(\sqrt p) v}}} + \norm{\pi(\sqrt p) v - v} \bigr) \\
&\leq 4 \sqrt2 \norm{\pi(\sqrt p) - I}_{\ope}
= 4 \sqrt2 \norm{e^{\Pi(H)/2} - I}_{\ope}.
\end{align*}
This can be bounded using the fact that $\abs{e^x - 1} \leq e \abs x$ for $x\in\R$ with $\abs x \leq 1$, which translates immediately into a similar bound for the operator norm.
Thus, assuming $\norm{H}_F \leq 2 / N(\pi)$, which implies that $\norm{\Pi(H)/2}_{\ope} \leq N(\pi) \norm{H}_F / 2 \leq 1$, we have
\begin{align*}
  \norm{ww^\dagger - vv^\dagger}_{\tr}
\leq 4 \sqrt2 e \norm{\Pi(H)/2}_{\ope}
\leq 2 \sqrt2 e N(\pi) \norm{H}_F.
\end{align*}
If we plug this into \cref{eq:hess first} and use that $\delta \leq 2/N(\pi)$ then we find that, for all $p \in B_\delta(I)$,
%     c\lambda / N(\pi)^3 \leq 2/N(\pi)
% <=> \lambda \leq 24 sqrt(2) e N(\pi)^2
% <=  \lambda \leq 4 N(\pi)^2
\begin{align*}
  \abs[\big]{D^2f_v(p)[X,X] - D^2f_v(I)[X,X]}
% \leq 6\sqrt{2} e N(\pi)^3 \norm{H}_F \norm{X}_F^2
\leq 6\sqrt{2} e N(\pi)^3 \delta \norm{X}_F^2
\leq \frac\lambda2 \norm{X}_F^2.
\end{align*}
Therefore, if $f_v$ is $\lambda$-geodesically strongly convex at~$I$ then it is $\lambda/2$-geodesically strongly convex on~$B_\delta(I)$.
This concludes the proof.
\end{proof}


%One can calculate $$ H_{I, v}(X, X) = \langle w, \Pi(X)^2 w \rangle - \langle w, \Pi(X) w \rangle^2 $$ where $w = v/\|v\|$. We may calculate the Hessian $H_{P,v}$ using $H_{P, v} = H_{I, \sqrt{P}v}$. 







%If $\langle \cdot , \cdot \rangle_P$ is the metric at a point $P$, then the new metric $\langle \cdot, \cdot \rangle_{R,P}$ will be given by 
%$$ \langle X, Y \rangle_{R,P} = \langle R^{1/2} X R^{1/2}, R^{1/2} Y R^{1/2} \rangle_{P}.$$

%At the identity the metric takes the form $\langle X, Y \rangle_R = \tr \sqrt{R} X  \sqrt{R} Y $. In particular the length of the geodesic $t\mapsto \sqrt{P}e^{Xt}\sqrt{P}$ for $t \in [0,1]$ is $\langle X, X \rangle_R$. 

\bibliographystyle{alpha}
\bibliography{refs}

\end{document}
