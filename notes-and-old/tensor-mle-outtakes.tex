\documentclass{article}
\usepackage[utf8]{inputenc}
\usepackage{amsmath}
\usepackage{amsthm,mathtools,braket}
\usepackage{amssymb,bm, hyperref}
\usepackage{xcolor, float}
\usepackage[capitalize]{cleveref}

\newtheorem{theorem}{Theorem}
\newtheorem{corollary}[theorem]{Corollary}
\newtheorem{prop}[theorem]{Proposition}
\newtheorem{lemma}[theorem]{Lemma}
\newtheorem{corol}[theorem]{Corollary}
\newtheorem{fact}[theorem]{Fact}
\newtheorem{claim}[theorem]{Claim}
\newtheorem{remark}{Remark}
\newtheorem{definition}{Definition}
\newtheorem{example}{Example}
\floatstyle{boxed}\newfloat{Algorithm}{ht}{alg}


\newcommand\samp{x}
\newcommand\rv{X}
\DeclareMathOperator{\Lie}{Lie}
\DeclareMathOperator{\Lin}{L}
\DeclareMathOperator{\ope}{op}
\DeclareMathOperator{\poly}{poly}

\DeclarePairedDelimiter{\abs}{\lvert}{\rvert}
\DeclarePairedDelimiter{\norm}{\lVert}{\rVert}
\DeclarePairedDelimiter{\ip}{\langle}{\rangle}


\newcommand{\junk}[1]{}


\newcommand{\R}{{\mathbb{R}}}
\newcommand{\C}{{\mathbb{C}}}
\newcommand{\ot}{\otimes}
\newcommand{\mat}{\operatorname{Mat}}
\newcommand{\op}{\operatorname{op}}
\newcommand{\ch}{\operatorname{ch}}
\newcommand{\cut}{\operatorname{cut}}
\newcommand{\vol}{\operatorname{vol}}
\newcommand{\Q}{{\mathbb{Q}}}
\newcommand{\N}{{\mathbb{N}}}
\renewcommand{\vec}{\bm}
\newcommand{\Z}{{\mathbb{Z}}}
\renewcommand{\S}{\mathbb{S}}

\newcommand{\E}{\mathbb{E}}
\renewcommand{\Pr}{\mathbb{P}}

\newcommand\eps{\varepsilon}

\newcommand\cN{\mathcal{N}}
\newcommand\FF{\mathcal{F}}
\newcommand\HH{\mathcal{H}}
\newcommand\GG{\mathcal{G}}
\newcommand\SL{\operatorname{SL}}
\newcommand\PD{\operatorname{PD}}
\newcommand\Herm{\operatorname{Herm}}
\newcommand\Sym{\operatorname{Sym}}
\newcommand\GL{\operatorname{GL}}
\newcommand\TT{\mathcal{T}}
\newcommand\PP{\mathcal{P}}
\newcommand\tr{\operatorname{Tr}}
\newcommand\rk{\operatorname{rk}}
\newcommand\RR{\mathcal{R}}
\newcommand\CC{\mathcal{C}}
\newcommand\BB{\mathcal{B}}
\newcommand\II{\mathcal{I}}
\renewcommand\AA{\mathcal{A}}
\newcommand{\MM}{\mathcal{M}}
\newcommand\AP{\mathcal{AP}}

\newcommand{\eqdef}{:=}
\newcommand{\maps}{\colon}
\newcommand{\ale}{\lesssim}
\newcommand{\age}{\gtrsim}
\newcommand{\CF}[1]{{\color{purple}[CF: #1]}}
\newcommand{\AR}[1]{{\color{orange}[AR: #1]}}
\newcommand{\RMO}[1]{{\color{red}[RMO: #1]}}
\newcommand{\MW}[1]{{\color{red}[MW: #1]}}
\newcommand{\TODO}[1]{{\color{blue}[TODO: #1]}}

\crefname{Algorithm}{Algorithm}{Algorithms}


\title{Old tensor mle stuff}
\author{Cole Franks, Rafael Oliveira, Akshay Ramachandran, Michael Walter}
\date{February 2020}

\begin{document}

\maketitle
\tableofcontents


\section{Old stuff}

\subsection{Old lower bound lemma}

In the conclusion of the lemma we needed to use that convergence in total variation of some estimator $\widehat{\Theta}_n$ to another, $\widehat{\Theta}$, implied that the former has minimax error at least that of the latter in any dissimilarity measure. This holds by applying the next lemma to the random variables $X_n = d_*( \widehat{\Theta}_n, \Theta)$ and $Y = d_*(\widehat{\Theta}, \Theta)$ where $d_*$ represents any nonnegative function. For example, we could take $d_*$ to be the Frobenius, spectral, Fischer-Rao, Kullback-Leibler or Mahalanobis ``distances".
\CF{surely I can cite this next thing, I am just proving it for my own sanity}
\begin{lemma}\label{lem:dtv}
Suppose $X_n, Y$ are nonnegative random variables such that $X_n \to Y$ in $d_{TV}$. Then $$\limsup_{n\to \infty} \E X_n \geq \E Y.$$
\end{lemma}
\begin{proof}
%If $X_n \to Y$, then the random variables $\|X_n - I\|\to  \|Y - I\|$ in total variation. Does convergence in total variation implies convergence in means?

If the mean of $Y$ is bounded then we have Markov's inequality. Let $\eps > 0$; by the Dominated Convergence Theorem there is $\alpha$ large enough that $\E[Y 1_{Y \leq \alpha}] \geq \E[Y] - \eps$. Now we have
 $$\E[X_n] \geq \E[X_n 1_{X_n \leq \alpha} ] \to \E[ Y 1_{Y \leq \alpha}] \geq \E[Y] - \eps.% \geq (\E[Y| Y \leq \alpha] - \delta)(1 - \delta - \eps).
 $$
 as $n \to \infty$, where the limit is deduced by H\"older's inequality. Letting $\eps \to 0$ completes the proof.\end{proof}


\subsection{Different Inner Product}

\begin{definition}
For desired marginals $\{R_{a}^{2}\}_{a \in [k]}$ (assume for now $R$ are Hermitian though we can pick different square roots if required), define inner product
\[ \langle Z, Y \rangle_{R} := \sum_{a} \langle R_{a} Z R_{a}^{*}, Y \rangle \]
\[ \|Z\|_{R}^{2} := \langle Z, Z \rangle_{R} = \sum_{a} \|R_{a} Z_{a}\|_{F}^{2}   \]
\end{definition}

We restate the projective likelihood function and define gradient and Hessian in this metric:

\begin{definition}
\[ f_{\vec X}(\Theta_1, \dots, \Theta_n) = \log \left\langle \sum_{i \in [n]} X_{i} X_{i}^{*},  \bigotimes_{a \in [k]} \Theta_a \right\rangle - \sum_{a \in [k]} \frac{1}{d_a} \log\det \Theta_a  \]
Also $\rho := \sum_{i} X_{i} X_{i}^{*}$ and $\{\rho^{S}\}_{S \subseteq [k]}$ are marginals.
\end{definition}

\begin{fact}
\[ (\nabla f(I))_{a} = d_{a} \rho^{\{a\}} - I_{a}  \]
\end{fact}
\begin{proof}
We can define $\nabla f$ dually as $\forall Z: \langle \nabla f(I), Z \rangle_{R} := \partial_{t=0} f(e^{tZ})$
\[ \partial_{t=0} f(e^{t Z_{a}}) = \partial_{t=0} \langle \rho, I_{\overline{a}} \otimes e^{tZ_{a}}  \rangle - \partial_{t=0} \frac{1}{d_{a}} \log\det e^{t Z_{a}}  \]
\[ = \left\langle \rho^{\{a\}} - \frac{1}{d_{a}} I_{a}, Z_{a} e^{t Z_{a}} \right\rangle|_{t=0} = \left\langle R_{a}^{-1} \left( \rho^{\{a\}} - \frac{1}{d_{a}} I_{a} \right) R_{a}^{-1}, Z_{a} \right\rangle_{R}   \]

Similarly we define the Hessian as
\[ \partial_{s=t=0} f(e^{tZ_{a} + sY_{b}}) = \langle \rho, \{ I_{\overline{a}} \otimes Z_{a}, I_{\overline{b}} \otimes Y_{b} \} \rangle   \]
\[ \implies (\nabla^{2} f(I))_{aa} = \langle R_{a}^{-1} \rho^{\{a\}} R_{a}^{-1}, \{Z, Y\} \rangle_{R}    \]
\[ \implies (\nabla^{2} f(I))_{ab} = \langle \rho^{\{a,b\}}, Z \otimes Y \rangle   \]
\end{proof}

\AR{Not sure how to define Hessian. I think I'd like the Hessian to be
\[  \sum_{a} E_{aa} \otimes (1 \pm \epsilon) I + \sum_{a \neq b} E_{ab} \otimes \pm \lambda  \]
for some small $\epsilon,\lambda$. Then the Hessian will be $1-\epsilon - (k-1) \lambda$-strongly convex. }



\begin{lemma} [Restatement of \cref{lem:convex-ball}]
Let $f$ be geodesically convex everywhere. All the below quantities are wrt metric $\langle \cdot, \cdot \rangle_{R}$. Assume $f$ and $\lambda$-strongly geodesically convex ball of radius $\kappa$ about $I$; further assume the geodesic gradient satisfies $\|\nabla f(I)\|_{R} = \eps < \lambda \kappa$. Then there is an optimizer within an $\eps/\lambda$-ball.
\end{lemma}
\begin{proof} [Proof of \cref{lem:convex-ball}]
The proof is exactly the same except the following:
\[ g'(0) = \langle \nabla f(I), Z \rangle_{R} \geq - \|\nabla f(I)\|_{R} \|Z\|_{R} \geq - \eps     \]
\end{proof}

\begin{remark}
Note the perturbation lemma then gives the following strategy. By Cole's lemma, we have that $c \|\nabla f(I)\|_{R} \geq \|\nabla f(I)\|_{op}$. If we can say the same thing for the optimizer $Z$, then it is enough for $\lambda \kappa \geq \Omega(1/c) > \eps$ and we can improve sample complexity to $nD > c \max_{a} d_{a}^{2}$.

A similar thing is true if we can show the above inequality for the gradient flow for $\log \max_{a} d_{a}$ time.
\end{remark}

\begin{lemma}
$\lambda$-strong convexity is a sufficient condition for fast convergence of the gradient flow:
\[ - \partial_{t=0} \|\nabla f(e^{-t\nabla f(I)})\|_{R}^{2} = -\partial_{t=0}^{2} f(e^{-t\nabla f(I)}) = \langle \nabla^{2} f, \nabla f \otimes \nabla f \rangle \geq \lambda \|\nabla f\|_{R}^{2}    \]
\AR{Not sure how to write the third term above, the inner product with Hessian}
\end{lemma}

\subsection{Old proof of \cref{thm:tensor-convexity}}


\begin{proof} [Proof of \cref{thm:tensor-convexity}]
Take any quadratic form of the Hessian for $\{Z_{a} \perp I_{a}\}$:
\[ \partial_{t=0}^{2} f(e^{t Z}) = \sum_{a} \langle Q^{a}, Z_{a}^{2} \rangle + \sum_{a \neq b} \langle Q^{ab}, Z_{a} \otimes Z_{b} \rangle   \]
\[ \geq \sum_{a} \lambda_{\min}(Q^{a}) \|Z_{a}\|_{F}^{2} - \sum_{a \neq b} \|Q^{ab} (I-P_{ab})\|_{op} \|Z_{a}\|_{F} \|Z_{b}\|_{F}    \]
Now we can use our high probability bounds derived above:
\[ \forall a: Q^{a} \succeq \frac{1-\epsilon}{d_{a}} I_{a}; \hspace{10mm}
\forall a \neq b: \|Q^{ab} (I-P_{ab})\|_{op} < \frac{\lambda}{\sqrt{d_{a} d_{b}}}   \]
\[ \implies \partial_{t=0}^{2} f(e^{t Z}) \geq \sum_{a} \frac{1-\eps}{d_{a}} \|Z_{a}\|_{F}^{2} - \sum_{a \neq b} \frac{\lambda}{\sqrt{d_{a} d_{b}}} \|Z_{a}\|_{F} \|Z_{b}\|_{F}  \]
\[ \geq \sum_{a} \frac{1-\eps+\lambda}{d_{a}} \|Z_{a}\|_{F}^{2} - \lambda \left( \sum_{a} \frac{1}{\sqrt{d_{a}}} \|Z_{a}\|_{F}   \right)^{2}   \]
\[ \geq \sum_{a} \frac{1-\eps+\lambda}{d_{a}} \|Z_{a}\|_{F}^{2} - k\lambda \sum_{a} \frac{1}{d_{a}} \|Z_{a}\|_{F}^{2} \]
\[ = (1-\eps-(k-1)\lambda) \|Z\|^{2}    \]
Choosing $\eps,\lambda$ small enough gives the theorem.
\end{proof}

\begin{proof}[Proof: \CF{Akshay's conceptual proof of \cref{thm:tensor-convexity}}]
We can in fact show that $\nabla^{2}$ is well-conditioned using the following:
\[ -\begin{pmatrix} 1 & 0 \\ 0 & 1 \end{pmatrix}
\preceq \begin{pmatrix} 0 & 1 \\ 1 & 0 \end{pmatrix}
\preceq \begin{pmatrix} 1 & 0 \\ 0 & 1 \end{pmatrix}
\]
%We can rewrite the Hessian using shorthand $\{M_{a}\}$ for the diagonal blocks and $\{M_{ab}\}$ for off-diagonal blocks: \CF{why not just write $\nabla^2_{ab}f$?} \CF{Also, the matrices $E_{aa} \otimes M_a$ are not all of the same size} 
\\ \AR{Ya $\nabla_{ab}$ notation is fine, just needed something that was a matrix of the right dimensions, so shorthand M was to avoid weird things with $\rho$}
\\ \AR{It's fine if they're of different sizes, we enumerate the basis of the whole space as $\cup_{a} e_{a} \otimes \{e_{i \in [d_{a}]}\}$ }\\
\CF{$E_{aa} \otimes \nabla^2_{aa}$ is $k d_a^2 \times k d_a^2$ dimensional. So how does this make sense? Maybe needs to be updated along the lines of the next proof.}
\[ \nabla^{2} f = \sum_{a} E_{aa} \otimes \nabla^{2}_{aa} + \sum_{a \neq b} E_{ab} \otimes \nabla^{2}_{ab}  \]
Now we can again use the high-probability bounds derived above: \TODO{actually cref them}
\begin{align}\nabla^{2}_{aa} \in \frac{1 \pm \eps}{d_{a}}; \hspace{5mm} \forall a \neq b: \|\nabla^{2}_{ab}\|_{op} \leq \frac{\lambda}{\sqrt{d_{a} d_{b}}} \label{eq:expansion-thing}  \end{align}
\[ \nabla^{2} \preceq \sum_{a} E_{aa} \otimes \left( \frac{1+\eps}{d_{a}} I_{a} \right) + \sum_{a < b} E_{aa} \otimes \left( \frac{\lambda}{d_{a}} I_{a} \right) + E_{bb} \otimes \left( \frac{\lambda}{d_{b}} I_{b} \right)    \]
\[ \preceq \sum_{a} E_{aa} \otimes \frac{1+\eps+(k-1)\lambda}{d_{a}} I_{a}  \]

The same sequence of inequalities can be reversed to show a lower bound. So in fact we can show the above bounds on blocks shows $1+O(\eps + k \lambda)$-condition number bound on the Hessian in norm $\|\cdot\|_{d}$. 
\end{proof}


\section{Old gradient bounds}

\TODO{lower bound Hessian for operators and tensors for all different formats; we hope to get strong convexity with $\prod d_i /(d_1^2 + \dots + d_k^2)$ samples. I am concerned that a KLR19-style operator norm type theorem is needed to get $\tilde{O}$ of this, but we will do what we can with the Frobenius bounds for now; I'd expect to need at least $\max_i \sqrt{d_i}$ too many samples.}\\
\TODO{It would also be nice to have that tight example for the $\log$ in KLR19...}

We recall the moment map and Hessian calculations
\[ \partial_{t=0} f(e^{tX_{a}}) = \langle \nabla_{a}, X \rangle = \langle Q^{a} - sI_{a}, X \rangle   \]
\[ \partial_{t=0}^{2} f(e^{tX_{a}}) = \langle X, (\nabla^{2})_{aa} X \rangle = \langle Q^{a}, X^{2} \rangle  \]
\[ \partial_{s=0} \partial_{t=0} f(e^{tX_{a}} \otimes e^{sY_{b}}) = \langle Y, (\nabla^{2})_{ab} X \rangle = \langle Q^{ab}, X \otimes Y \rangle   \]

\section{Operator Scaling}
In this section we have $n$ samples of $X \sim \cN(0,\frac{1}{n} (\frac{1}{d_{1}} I_{1}) \otimes (\frac{1}{d_{2}} I_{2}))$. We will denote $D := d_{1} d_{2}$. In order to use the KLR analysis, we will show that the one-body marginals have low error in $\|\cdot\|_{op}$ and the whole operator is a sufficient expander at the start.

\subsection{Bernstein Proof of $\|\mu\|_{op}$}
This is proven using matrix concentration

\begin{theorem} [Bernstein]
Consider independent $\{X_{k}\}$ such that $\E X_{k} = 0$ and $\lambda_{max}(X_{k}) \leq R$ almost surely. Further let the variance be $\sigma^{2} := \|\sum_{k} \E X_{k}^{2} \|_{op}$.
\begin{eqnarray*} \Pr [ \lambda_{max} \left( \sum_{k} X_{k}  \right) \geq t ] & \leq & d \exp\left( - \frac{\Omega(t^{2})}{\sigma^{2} + t R} \right)
\\ & \leq & \begin{cases}
d \exp ( - \Omega(t^{2}/\sigma^{2}) ) & \text{if $t \leq \sigma^{2}/R$ }
\\ d \exp ( - \Omega(t/R) )           & \text{if $t \geq \sigma^{2}/R$}
\end{cases}
\end{eqnarray*}
\end{theorem}

In our setting, $Q^{a}$ is comprised of $N := \frac{TD}{d_{a}}$ copies of a rank one $g g^{*}$ where each Gaussian is $g \sim \mathcal{N}(0, N^{-1} \frac{1}{d_{a}} I_{a} ) = \mathcal{N}(0, \frac{1}{TD} I_{a}) $. We will drop subscripts for $d_{a}, I_{a}$ etc when they can be understood from context. Therefore we define $X := g g^{*} - \frac{1}{TD} I_{a}$ and note the following parameters:
\[ \lambda_{max}(X) = \|g\|_{2}^{2} - \frac{1}{TD} \hspace{10mm} \lambda_{min}(X) = - \frac{1}{TD}   \]
While $\|g\|_{2}$ is unbounded, we can threshold our distribution with a small loss in probability. Since we will be using $\chi^{2}$ distributions much from now on, we will do a quick exercise to prove our threshold bounds:

\begin{definition}
$\chi(\mu,d)$ denotes the $\chi^{2}$ distribution with mean $\mu$ and $d$ degrees of freedom. Explicitly $X \sim \chi(\mu,d) \implies X = \frac{\mu}{d} \sum_{i=1}^{d} g_{i}^{2}$
where $g \sim \mathcal{N}(0,1)$.
\end{definition}

\begin{lemma}
For $X \sim \chi(\mu,d)$ we have the following (explicit and approximate) formula for the MGF, $\forall \theta < \left(O(\frac{\mu}{d}) \right)^{-1}$:
\begin{eqnarray*} \log \E \exp(\theta X) & = & - \frac{d}{2} \log \left(1 - 2 \theta \frac{\mu}{d} \right)
\\ & \leq & \theta \mu + \theta^{2} \frac{O(\mu^{2})}{2 d}
\end{eqnarray*}
\end{lemma}

\begin{theorem} [Sub-exp variables]
The above MGF bound gives tail decay:
\[ \forall \theta < b^{-1}: \log \E \exp(\theta (X - \E X)) \leq \theta^{2} \frac{\sigma^{2}}{2} \]
\[ \implies  \Pr[X - \mu \geq t] \leq \begin{cases}
\exp( - \Omega(t^{2}/\sigma^{2}) ) & t \leq \sigma^{2}/b
\\ \exp( - \Omega(t/b) ) & t \geq \sigma^{2}/b
\end{cases}   \]
\end{theorem}

With these bounds in mind, note our variables $\|g\|_{2}^{2} \sim \chi(\frac{d_{a}}{TD},d_{a})$ so we have $\sigma^{2} = \frac{d}{(TD)^{2}}, b = \frac{1}{TD} \implies \sigma^{2}/b = \frac{d}{TD}$
\[ \Pr[ \exists k: \lambda_{max}(X_{k}) \geq M \sqrt{\log N}\frac{d}{TD} ] \leq \exp( - \Omega(M^{2}) )  \]
If we're happy with $1/poly$ failure probability we will take $M^{2} \sim \log D$, so in our matrix bound $R_{max} \leq \frac{d \log D}{TD}$

\[ \E X^{2} = \E (g g^{*})^{2} - \frac{1}{(TD)^{2}} I = \E \|g\|_{2}^{4} \hat{g} \hat{g}^{*} - \frac{1}{(TD)^{2}} \]
\[ = \frac{1}{(TD)^{2}} ( (3d + d(d-1)) \frac{1}{d} I - I ) = \frac{d+1}{(TD)^{2}} I   \]
Here $\hat{g} := g / \|g\|_{2}$ and the calculation is done by independence of $\|g\|_{2}, \hat{g}$. So we also have the variance parameter
\[ \sigma^{2} = N \|\E X^{2}\|_{op} = \frac{TD}{d} \frac{d+1}{(TD)^{2}} \sim \frac{1}{TD}  \]

\begin{corollary}
We have the following operator norm concentration
\[ \Pr[ \|Q^{a} - sI_{a}\|_{op} \geq t ] \leq d \exp \left( - \frac{\Omega(t^{2} TD)}{1 + t d_{a} \log D }  \right)  \]
Since we require $\|\cdot\|_{op}$ error $\ll \frac{1}{d_{a} \log D}$, if we are happy with $1/poly$ failure probability we require $TD \gg \max_{a} d_{a}^{2} \log^{3} D$.
\end{corollary}

\begin{remark} Note I'm using $\min_{a} d_{a} < \max_{a} d_{a} < D$ in a couple places so the $\log$ term may be slightly sharpened. But the exponent is tight as we require $TD > \max_{a} d_{a}^{2}$ samples for existence/ uniqueness of the solution.
\end{remark}


\subsection{Gaussian proof of $\|\mu\|_{op}$}
The above method of first thresholding the Gaussians then using Bernstein-style concentration on a bounded random matrix feels a bit square-peg round-hole - y. Turns out there are better results specifically for the case of Gaussian matrices. Recall again that in our setting $Q^{a}$ is the sum of $N := \frac{nD}{d_{a}}$ copies of $X X^{*}$ where $X \sim \mathcal{N}(0,\frac{1}{nD} I_{a})$. Note first the following fact which allows us to use these specialized inequalities

\begin{fact}
$\sum_{i=1}^{N} X_{i} X_{i}^{*} \equiv G G^{*} $ where $G := \{X_{1}, ..., X_{N}\}$.
This means if we denote $\{\lambda_{1}, ..., \lambda_{d}\}$ the spectrum of $\sum_{i=1}^{N} X_{i} X_{i}^{*}$, this is the same as $\{s_{1}^{2}, ..., s_{d}^{2}\}$ where $s_{j} := s_{j}(G)$ the $j$-th singular value. By Taylor expansion of $\sqrt{1+x}$ we have:
\[ \lambda_{1},\lambda_{d}(GG^{*}) \in \frac{1}{d_{a}} \left( 1 \pm \frac{1}{\log d_{a}} \right) \iff s_{1},s_{d}(G) \in  \frac{1}{\sqrt{d_{a}}} \left( 1 \pm \frac{1}{\log d_{a}} \right)  \]
\end{fact}

% https://arxiv.org/pdf/1011.3027.pdf

\begin{corollary} [Corollary 5.35]%\label{cor:vershynin}
Let $G_{d,N} \in \R^{d \times N}$ for $d < N$ have independent standard Gaussian entries. Then for $t \geq 0$, the following occurs with $\leq 2 \exp(-t^{2}/2)$ failure probability:
\[ \sqrt{N} - \sqrt{d} - t \leq s_{d}(G) \leq s_{1}(G) \leq \sqrt{N} + \sqrt{d} + t  \]
\end{corollary}

\begin{corollary}
If $nD \gtrsim d_{a}^{2} \log^{2} d_{a}$ then $\|Q^{a} - \frac{1}{d_{a}} I_{a} \|_{op} \ll \frac{1}{d_{a} \log d_{a}}$ with failure probability $\leq \exp( - \Omega(d_{a}))$
\end{corollary}
\begin{proof}
We have the following with $\leq 2 \exp(-t^{2}/2)$ failure probability:
\[ s_{1},s_{d}\left( \frac{1}{\sqrt{nD}} G_{d,N} \right) \in \frac{1}{\sqrt{nD}} \left( \sqrt{\frac{nD}{d_{a}}} \pm (\sqrt{d_{a}} + t) \right) = \frac{1}{\sqrt{d_{a}}}\left( 1 \pm \frac{d_{a} + t \sqrt{d_{a}}}{\sqrt{nD}}  \right)  \]
Choosing $t \sim \sqrt{d_{a}}$ and $nD \gtrsim d_{a}^{2} \log^{2} d_{a}$ gives the required bound.
\end{proof}


\section{Old robustness}

\begin{lemma}\label{lem:perturbation-old}
If $f$ is $\lambda$-strongly convex at $I$ and $\forall a: d_{a} \|(\nabla f)_{a}\|_{op} \leq \eps \ll 1/k$, then for $Z$ such that $\forall a: \|Z_{a}\|_{op} \leq \delta_{a} \ll 1/k$, the function $f$ at $e^{Z}$ is $\lambda - O(k \sum_{a} \delta_{a})$ strongly convex.
\end{lemma}

\begin{lemma}\label{lem:perturbation} \CF{what I think this lemma should say} Let $f = f_{\samp}$. There is a constant $c > 0$ such that if $f$ is $\lambda$-strongly convex at $I$ and that $\|(\nabla f)_{a}\|_{op} \leq \eps \leq c k ^{-1}$ for all $ a \in [k]$, then the function $f$ is 
$$ \lambda - O(k \sum \|Z_a\|_{op})$$
-strongly convex at $Z \in \PD$ provided $\|Z_a\|_{op} \leq c k^{-1}$ for all $a \in [k]$. \CF{just define operator norm on $\PD$?}

\end{lemma}




The bulk of the work goes towards an intermediate lemma showing that each block $\nabla^2_{ab} f$ of the Hessian changes fairly little on the operator norm ball.


\begin{lemma}\label{lem:block-perturbation-old}
For perturbation $v \to \otimes_{a} e^{\delta_{a}} \cdot v =: w$ where $\forall a: \|\delta_{a}\|_{op} \ll 1$, and let $\{\sigma_{1}^{ab}, \sigma_{2}^{ab}\}$ be the matrix norm $\|\cdot\|_{F} \to \|\cdot\|_{F}$ and matrix norm on subspace $\perp$ to $(I,I)$ for each bipartite part respectively:
\[ \forall a,b: \sigma_{2}^{ab}(w w^{*}) - \sigma_{2}^{ab}(v v^{*}) \leq O \left( \sum_{a} \|\delta_{a}\|_{op}  \right) \sigma_{1}^{ab}(v v^{*})   \]
The same is true for the diagonal blocks.
\end{lemma}

\CF{I think we are safe to just say $\|\nabla^2_{ab}f\|_{op}:=\|\nabla^2_{ab}f\|_{F\to F}$.}

\begin{lemma}\label{lem:block-perturbation}\CF{prev lemma in new notation; not married to the $\Pi$'s.} Let $\Pi$ denote the projection to the traceless matrices. There is a constant $c>0$ such that if $\|Z_a\|_{op} \leq c $ for all $a \in [k]$ we have
$$\|\nabla^2_{ab} f(Z) \circ \Pi\|_{op} - \|\nabla^2_{ab}f(I)  \circ \Pi \|_{op}  = O\left( \sum_{a \in [k]} \|Z_a\|_{op} \|\nabla^2_{ab}f\|_{op}\right)$$
for all $a,b \in [k]$.

\end{lemma}
\begin{proof}\CF{needs to be updated to new notation. At this point in the paper, there's no $M$.}
To lower bound the diagonal block, we just need a spectral lower bound on $\{\rho^{a}\}$, since $\langle vec(X), \nabla_{aa}^{2} (vec(X)) \rangle := \langle \rho^{a}, X^{2} \rangle$.
\[ \| e^{\delta_{a}} \rho^{a} e^{\delta_{a}} - Q_{a}\|_{op} \leq O(\|\delta_{a}\|_{op}) \|Q_{a}\|_{op}   \]
Now we address a perturbation on $b \neq a$. For a spectral lower bound, we choose test $Z \succeq 0$ and let $\delta := e^{2\delta_{b}} - I$:
\[ \langle e^{\delta_{b}} \rho e^{\delta_{b}} - \rho, I_{\overline{a}} \otimes Z_{a} \rangle
= \langle \rho, \delta \otimes Z \rangle = \langle Z, V^{*} \delta V \rangle   \]
Here $V \in \R^{d_{b} \times d_{a}}$ is the matricized version of $\rho$. But now since $Z \succeq 0$, the argument is clear
\[ \leq \langle Z, V^{*} |\delta| V \rangle \leq \|\delta\|_{op} \langle Z, V^{*} I V \rangle = \|\delta\|_{op} \langle \rho, I_{\overline{a}} \otimes Z \rangle \leq \|\delta\|_{op} \|\rho^{a}\|_{op} \|Z\|_{1}    \]

The argument for the off-diagonal blocks is similar. We first argue the change is small under perturbations just on those parts. 
\[ \langle vec(Y), M_{v}^{ab}(vec(Z)) \rangle := \langle v v^{*}, I_{\overline{ab}} \otimes Z \otimes Y \rangle \]
\[ \langle vec(Y), M_{w}^{ab}(vec(Z)) \rangle := \langle w w^{*}, I_{\overline{ab}} \otimes Z \otimes Y \rangle\]

\[ \implies M_{w} = (e^{\delta_{b}} \otimes e^{\delta_{b}}) M_{v} (e^{\delta_{a}} \otimes e^{\delta_{a}})   \]
\[ \implies \|M_{w} - M_{v}\|_{op} \leq O(\|\delta_{a}\|_{op} + \|\delta_{b}\|_{op}) \|M_{v}\|_{op}   \]
where in the last step we used $\delta \ll 1$.
\CF{comment things that we wouldn't want to accidentally leave in, as I have done in the next sentence}
The more difficult part of the argument to see \AR{at least for me} is the change caused be some other part $c \neq a,b$. First we define $\delta := e^{2 \delta_{c}} - I$, and test vectors $Z,Y$:
\[ \langle w w^{*} - v v^{*}, I_{\overline{ab}} \otimes Z \otimes Y \rangle = \langle v v^{*}, \delta \otimes Z \otimes Y \rangle  = \langle Z \otimes Y, V^{*} \delta V \rangle \]
Here $V \in \R^{d_{c} \times d_{a}d_{b}}$ is the matricized version of $v$, i.e. the $k$-th element of $ij$-th column is $(V_{ij})_{k} := v_{ijk}$. Now in order to use our operator norm bounds, we need to deal with cancelations, so we split into positive and negative parts $Z := Z_{+} - Z_{-}, Y := Y_{+} - Y_{-}$:
\[ |\langle Z \otimes Y, V^{*} \delta V \rangle| \leq |\langle Z_{\pm} \otimes Y_{\pm}, V^{*} \delta V \rangle |  \]
Now we analyze each of these terms:
\[ \leq |\langle Z_{\pm} \otimes Y_{\pm}, V^{*} |\delta| V \rangle| \leq \|\delta\|_{op} |\langle Z_{\pm} \otimes Y_{\pm}, V^{*} V \rangle| = \|\delta\|_{op} |\langle v v^{*}, I_{\overline{ab}} \otimes Z_{\pm} \otimes Y_{\pm} \rangle|   \]
Each of these terms we can bound by $\sigma^{ab}_{1} \|Z\|_{F} \|Y\|_{F}$. So by iterating this argument over all $c$, we get the desired bound.
\end{proof}

\begin{proof}[Proof of \cref{lem:perturbation}]
\[ \langle X, \nabla_{aa}^{2} X \rangle = \langle \rho^{a}, X^{2} \rangle \leq \|\rho^{a}\|_{op} \|X^{2}\|_{1} = \|\rho^{a}\|_{op} \|X\|_{F}^{2} \]
\AR{I am probably wrong on dimension factors here, but it's the right idea}
By the condition on the gradient \TODO{what condition? cref it}, we have that
\[ \forall a,b: \|\nabla_{ab}^{2}\|_{op}^{2} \leq \|\nabla_{aa}^{2}\|_{op} \|\nabla_{bb}^{2}\|_{op} = \|\rho^{a}\|_{op} \|\rho^{b}\|_{op} \leq \frac{1+\eps}{d_{a} d_{b}}   \]
We apply the perturbation lemma to each part sucessively, and if $\delta$ are small enough we can assume this bound holds in weaker form $1+\epsilon \leq 2$ for all iterations. The above lemma shows for each part and any test vectors
\[ \langle w w^{*} - v v^{*}, I_{\overline{ab}} \otimes \frac{Z}{\|Z\|_{F}} \otimes \frac{Y}{\|Y\|_{F}} \rangle \leq \frac{O( \sum_{a} \delta_{a} )}{\sqrt{d_{a} d_{b}}} =: \frac{\delta}{\sqrt{d_{a} d_{b}}}   \]
Here the suppressed constants are $\leq 7$. Therefore the difference between Hessians can be bounded
\[ |\langle Y, \nabla^{2} f(e^{Z}) - \nabla^{2} f(I), Y \rangle|  \leq \delta \left( \sum_{a} \frac{\|Y_{a}\|_{F}^{2}}{d_{a}} + \sum_{a \neq b} \frac{\|Y_{a}\|_{F} \|Y_{b}\|_{F}}{\sqrt{d_{a} d_{b}}} \right) \leq k \delta \|Y\|^{2}   \]
\end{proof}


After one more simple lemma, we will be ready to prove our second strong convexity result, \cref{lem:perturbation}.
\begin{lemma} [Lemma 3.6 in \TODO{cite KLR}; \CF{where is this used?}]
\AR{The amount we lose in robustness is related to the worst quadratic form in the whole space (not $\perp I$) since we have to break up into $\pm$ parts. }
\[ \|\nabla_{ab}^{2}\|_{F \to F}^{2} \leq \|\nabla_{aa}^{2}\|_{F \to F} \|\nabla_{bb}^{2}\|_{F \to F}   \]
\end{lemma}
\begin{proof}
\AR{New simple proof:} By convexity we know $\begin{pmatrix} \nabla_{aa}^{2} & \nabla_{ab}^{2} \\ \nabla_{ba}^{2} & \nabla_{bb}^{2}  \end{pmatrix} \succeq 0$. The result follows from e.g. Schur complements. 
\end{proof}




%If $\langle \cdot , \cdot \rangle_P$ is the metric at a point $P$, then the new metric $\langle \cdot, \cdot \rangle_{R,P}$ will be given by
%$$ \langle X, Y \rangle_{R,P} = \langle R^{1/2} X R^{1/2}, R^{1/2} Y R^{1/2} \rangle_{P}.$$

%At the identity the metric takes the form $\langle X, Y \rangle_R = \tr \sqrt{R} X  \sqrt{R} Y $. In particular the length of the geodesic $t\mapsto \sqrt{P}e^{Xt}\sqrt{P}$ for $t \in [0,1]$ is $\langle X, X \rangle_R$.

\bibliographystyle{alpha}
\bibliography{refs}

\end{document}
